\chapter*{Uvod}

Predmet je sestavljen iz dveh delov:
\begin{enumerate}
	\item{Teorija (diskretnih) grup in njih upodobitve,}
		\begin{itemize}
			\item{grupe, podgrupe, homomorfizmi, kvocientne grupe (strukturni izreki)}
			\item{reprezentacije kon\v cnih grup in kompaktnih grup}
		\end{itemize}
	\item{Gladke mnogoterosti in tenzorji}
		\begin{itemize}
			\item{tangentni sve\v zenj $\to$ kovariantni in kontravariantni tenzorji,}
			\item{diferencialne forme $\in$ multilinearna algebra}
			\item{integracija na mnogoterosti}
			\item{krovni prostori, fundamentalna grupa}
		\end{itemize}
\end{enumerate}

To skripto sem se odlo\v cil napisati, ker v je do sedaj nih\v ce se ni poizkusil, in ker je dobro
imeti tako gradivo na ra\v cunalniku. Matematiki operirajo z mnogimi, nam fizikom nenavadnimi
pojmi in te\v zko jim je slediti. Dobro je, \v ce lahko na ra\v cunalniku enostavno uporabi\v s
iskalnik besed ter se tako la\v zje navigira\v s preko tako zajetnega gradiva. To je meni glavna
motivacija.

Odlo\v cil sem se, da bom poleg tega dodal tudi neke vrste cheat-sheet za fizike, ki vsebuje pojme,
ki so fizikom grozljivi, pa vendar niso (npr. \emph{homomorfizem, endomorfizem, algebra\ldots}). Tam
je vse na enem mestu.

\begin{flushright}
Jo\v ze Zobec,\\[0.5cm]
Ribnica, \today
\end{flushright}
