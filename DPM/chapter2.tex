\chapter{Simetrije in upodobitve}

\begin{defin}
	Naj bo $X$ poljubna mno\v zica. \emph{Simetri\v cna} ali \emph{permutacijska mno\v zica} $X$ je mno\v zica vseh bijekcij
	$X \to X$.\vspace{2ex}

	\ni To imenujemo \emph{permutacije} ali \emph{simetrije} na $X$, oznaka za grupo je $S(X)$. Posebej \v ce je $X$ kon\v cna in
	$|X| = n$, ozna\v cimo $S(X)$ z $S_n$ in elemente $X$ ozna\v cimo z $\{1, 2, \ldots, n\}$. V tem primeru permutacije opi\v semo
	\[
		\begin{pmatrix}
			1 & 2 & 3 & \ldots & n \\
			\alpha(1) & \alpha(2) & \alpha(3) & \ldots & \alpha (n)
		\end{pmatrix},
	\]
	\v ce je $\alpha$ permutacija.
\end{defin}

\ni Posebne permutacije so cikli:

\begin{defin}
	$\alpha \in S_n$ je $r$-cikel, \v ce $\exists\ \overbrace{i_1, i_2, \ldots, i_r}^{\text{vsi razli\v cni}} \in \{1, 2, \ldots, n\}$, ki jih
	$\alpha$ premakne, ostale elemente pa pusti pri miru (oz. jih fiksira) in velja:
	\[
		\alpha (i_1) = i_2,\ \alpha(i_2) = i_3, \ldots,\ \alpha(i_{r-1}) = i_r,\ \alpha(i_r) = i_1.
	\]
	Tak $\alpha$ ponavadi zapi\v semo kot $\alpha = (i_1, i_2, i_3, \ldots, i_r)$.
\end{defin}

\begin{defin}
	Dva cikla, $\alpha$ in $\beta$ sta \emph{disjunktna}, \v ce premakneta razli\v cne elemente -- tj. $\beta$ lahko premika kve\v cjemu
	elemente, ki jih $\alpha$ fiksira in obratno.
\end{defin}

\paragraph{Vaja:}
\begin{enumerate}
	\item{$|S_n| = n!$ (o\v citno).}
	\item{\v Ce je $\alpha$ $r$-cikel, je $\red(\alpha) = r$:
		\paragraph{Dokaz:} $\red(\alpha) = k$, \v ce $\alpha^k = e$ in $k$ najmanj\v se tako \v stevilo. Jasno je, da je $\alpha^r = e$. Tak
		$r$ je najmanj\v si: za $k < r$, $\alpha^k$ preslika $i_1$ v $i_{1+k}$, kjer je $1 + k \in \{2, \ldots, r\}$, zato
		$i_{i+k} \neq i_1$. $\blacksquare$}
	\item{Naj bo $\alpha$ $r$-cikel in $d \in \mathbb{N}$. Potem $\alpha^d$ produkt $(d, r)$ disjunktnih ciklov dol\v zine $r/(d,r)$ (oznaka
		$(d, r)$ naj bi predstavljala najve\v cji skupni delitelj \v stevil $d$ in $r$).
		\paragraph{Dokaz:}
		\begin{itemize}
			\item{$G = \langle a \rangle$ je cikli\v cna grupa reda $r$.}
			\item{$H = \langle a^d \rangle$ je podgrupa v $G$}
		\end{itemize}
		$x = e^{2i\pi/12}$ generira $C_{12}$, $x^d$ generira podgrupo: $\langle x^d \rangle = \{x^{dk}\ |\ k \in \mathbb{Z}\}$
		\begin{itemize}
			\item[$\diamond$]{$\underline{d = 5:}$ $\langle x^5 \rangle = \{e, x^5, (x^5)^2, \ldots, (x^5)^12\}$ -- imamo 12 razli\v cnih elementov -- seveda,
				najve\v cji skupni delitelj grup je $1$, ker sta si 5 in 12 tuji \v stevili $\then \langle x^5 \rangle = \langle x \rangle = C_{12} \leq C_{12}$, saj
				je $12 / (12, 5) = 12/1 = 12$.}
			\item[$\diamond$]{$\underline{d = 3:}$ $\langle x^3 \rangle = \{e, x^3, (x^3)^2, (x^3)^3\} \cong C_4$, ker je $12 / (12, 3) = 12/3 = 4$.}
		\end{itemize} Odtod sklepamo, da \v ce $k = (d, r)$, potem je $H$ cikli\v cna podgrupa mo\v ci (reda) $r/k$. Dobiti moramo najmnanj\v si $n$, da je
		$(\alpha^d)^n  = e$. O\v citno je $(\alpha^d)^{r/k},\ \alpha^{dr/k} = (\alpha^r)^{d/k} = e$. To \v stevilo je najmanj\v se s to lastnostjo:
		\[
			\alpha^{dn} = e,\quad \text{upo\v stevamo, da je $\alpha$ reda $r$}
		\]

		\ni $\then r|dn$ ($r$ najmanj\v se \v stevilo, pri katerem $\alpha^r = e$). Ker je $k = (d,r) \then (r/k)|n \then n \geq r/k$.
		\paragraph{Sklep:}\v Ce je najve\v cja skupna mera \v stevil $d$ in $r$ enaka 1 $\big(\text{tj.}\ (d,r) = 1\big)$, $\alpha^d$ dolo\v ca isto grupo mo\v ci
		$r$ in je zato $r$-cikel. V primeru, da $k \neq 1$ trdimo, da $\alpha$ razpade na produkt $k$ disjunktnih ciklov.
		\qed}
\end{enumerate}

\paragraph{Opomba:} disjunktni permutaciji (cikla) komutirata.

\begin{trditev}
	Vsaka permutacija v $S_n\backslash\{e\}$ je produkt disjunktnih ciklov dol\v zine $\geq 2$. Ta produkt je do vrstnega reda faktorjev enoli\v cen.
\end{trditev}

\begin{posledica}
	Vsaka permutacija je produkt \emph{transpozicij}, tj. ciklov dol\v zine (reda) 2: $(a,b)$, $a \neq b$.

	\paragraph{Opomba:} Ta izrazitev \emph{ni} enoli\v cna. Enoli\v cna je le parnost \v stevila transpozicij).
\end{posledica}

\begin{defin}
	\emph{Predznak} ali \emph{parnost} permutacije, je \v stevilo, ki ga dobimo kot $(-1)^{\# \text{transpozicij}}$ v poljubni izrazitvi permutacije s transpozicijami.
	
	\paragraph{Oznaka:} $\alpha$ permutacija $\then$ predznak $\alpha\ \ldots\ \sign(\alpha) \in \{\pm 1\}$.
\end{defin}

\begin{trditev}
	$\sign : S_n \to C_2 = \{\pm 1\}$, je homomorfizem in jedro tega imenujemo \emph{alternirajo\v ca grupa}, $A_n$.
	\paragraph{Sledi:} $A_n \lhd S_n$, indeksa $2$.
\end{trditev}

\paragraph{Dokaz:}

$\alpha$ $r$-cikel, $\beta$ $p$-cikel $\then \sign(\alpha \beta) = (-1)^{r + p} = (-1)^r (-1)^p = \sign (\alpha) \sign (\beta)$. $\sign$ torej res homomorfizem in je
hkrati surjektiven, saj je znak $\forall$ transpozicije $-1$. $A_n = \ker (\sign)$ je edinka,
\[
	\quot{S_n}{A_n} = \{\pm 1\},
\]

\ni kar pa pomeni $[S_n : A_n] = 2$.
\qed

\paragraph{Ponovitev:}
\begin{itemize}
	\item{$X$ mno\v zica, $S(X) = \{\text{bijekcije} X \to X\}$}
	\item{$X = \{1, 2, \ldots, n\} \to S(X) = S_n$}
\end{itemize}

\begin{trditev}
	\paragraph{Caylejev izrek:}
	\begin{itemize}
		\item{$G$ poljubna grupa. Potem je $G$ izomorfna podgrupi simetri\v cne grupe $S(G)$.}
		\item{\v Ce je $G$ kon\v cna in $|G| = n$, potem je $G$ izomorna podgrupi v $S_n$.}
	\end{itemize}
\end{trditev}
\paragraph{Dokaz:} Za dano $G$ \v zelimo poiskati \ldots
\begin{itemize}
	\item{$S(G) = \{f: G \to G\ |\ f\ \text{bijekcija}\}$}
	\item{$G$ je grupa z operacijo $G\times G \to G$ in potem injektivni homomorfizem $G \to S(G)$.}
\end{itemize}

\ni Naj bo $L : G \to S(G),\ g \mapsto L_g$ (leva translacija). Za levo translacijo velja (str.~\pageref{translacija}):
\begin{itemize}
	\item{$L_g$ je bijekcija, njen inverz je $L_{g^{-1}}$}
	\item{$L$ je homomorfizem: $L(gh) = L_g \circ L_h$ \big(kompozitum v $S(G)$\big).}
	\item{$L$ je injektivna: \v ce je $L_g$ identiteta, je $L_g(e) = ge = e \then g = e\ \then\ \ker L = e$.}
\end{itemize}

\ni Slika $L(G) \in S(G)$ je torej izomorfna $G$.
\qed

\begin{posledica}
	Naj bo $G$ kon\v cna grupa, $|G| = n$, in $\F$ poljuben obseg. Tedaj je \emph{$G$ izomorfna podgrupi v $GL_n(\F)$} in
	$\exists$ homomorfizem $\varphi : G \to GL_n(\F)$ ki je injektiven.
\end{posledica}

\paragraph{Dokaz:}

Bolj na splo\v sno: naj bo $X$ poljubna kon\v cna mno\v zica in $\varphi: G \to S(X)$ homomorfizem. Ta homomorfizem lahko "`dvignemo"'
do homomorfizma v neko splo\v sno linearno grupo,
\begin{itemize}
	\item{Naj bo $V$ vektorski prostor nad $\F$ za bazo $X$, torej
		\begin{align*}
			X &= \{x_1, x2, \ldots, x_m\}, \\
			V &= \bigg\{\sum_{i = 1}^m \lambda_i x_i\ \bigg|\ \lambda_i \in \F\bigg\}.
		\end{align*}}
	\item{Grupo $S(X)$ lahko vlo\v zimo v $GL_{\F}(V)$, tako da vsakemu elementu 
		$\sigma \in S(X)$ priredimo linearno preslikavo, ki permutira vektorje v bazi
		\begin{align*}
			\sigma &\mapsto A_{\sigma};\\
			A_{\sigma} : V &\to V \\
			x_i &\mapsto \sigma(x_i) = x_{\sigma(i)}
		\end{align*}}
	\item{Matrika tak\v sne preslikave je \emph{permutacijska matrika}, ki je obrnljiva ($\det A_\sigma = \pm 1$) in jo dobimo
		z zamenjavo matrike identitete. Na ta na\v cin $S(X)$ postane podgrupa v $GL_{\F}(V)$ in $\varphi$ porodi
		\begin{equation*}
		\begin{matrix}
			\hat{\varphi} :\ &G                              & \longrightarrow & GL_{\F} (V) \\
			                 &\stackrel[\varphi]{}{\searrow} &                 & \nehookarrow \\
			                 &                               & S(X)            &
		\end{matrix}
		\end{equation*}
	}
\end{itemize}

\paragraph{Posebej:} \v Ce je $\varphi: G \to S(X)$ injektiven dobimo injektiven $\hat{\varphi}$. Za $X = G$ je $|X| = |G| = n$, zato je $V$ vektorski
prostor na $\F$ dimenzije $n$ in je $GL_\F (V) \equiv GL_n (\F)$.

\qed

\section{Teorija upodobitev in pridru\v zenih delovanj}

\begin{defin}
	Naj bo $G$ grupa in $X$ mno\v zica. Potem homomorfizem $\varphi : G \to S(X)$ imenujemo \emph{upodobitev} (\emph{reprezentacija}).
\end{defin}

\begin{defin}
	Naj bo $V$ vektorski prostor nad obsegom $\F$. Homomorfizem $\varphi : G \to GL_{\F}(V)$ imenujemo \emph{linearna upodobitev} $G$ na $V$.
\end{defin}

\paragraph{Opomba:}
\begin{itemize}
	\item{Iz dokaza zadnje posledice sledi, da poljubna upodobitev $G \to S(X)$ porodi linearno upodobitev $G \to GL_\F(V)$, kjer je $V$
		vektorski prostor z bazo $X$ nad obsegom $\F$.}
	\item{Upodobitvi $G \to S(G)$, oz. $G \to GL_\F(V)$, kjer je $V$ vektorski prostor z bazo $G$, re\v cemo \emph{regularna upodobitev} (tj. $X = G$).}
\end{itemize}

\ni Glavni rezultat teorije upodobitev kon\v cnih grup je, da {\bf linearna regularna upodobitev vsebuje vse linearne upodobitve}.

Poljubna upodobitev $\varphi : G \to S(X),\ g \mapsto \sigma$ porodi \emph{delovanje} grupe $G$ na mno\v zico $X$, tj. preslikavo
\begin{align*}
	\tilde{\varphi}: G \times X &\to X, \\
	(g, x) &\mapsto \varphi(g)(x) = \sigma (x),
\end{align*}

\ni z lastnostima:
\begin{itemize}
	\item{$\tilde{\varphi}(e, x) = \varphi(e) (x) = x$, ker je $\varphi$ homomorfizem, je $\varphi(e)$ spet identiteta.}
	\item{$\tilde{\varphi}\big(g, \tilde{\varphi}(h,x)\big) = \big[\varphi(g) \circ \varphi(h)\big] (x) = \varphi (gh)(x) = \tilde{\varphi}(gh,x)$}
\end{itemize}

\paragraph{Dogovor:} Kjer imamo opraviti le z enim delovanjem, izpustimo ime preslikave in pi\v semo kar
\[
	\tilde{\varphi}(g,x) = gx.
\]
V tem zapisu sta zgornji lastnosti
\begin{align*}
	\varphi(e) = \id \then ex &= x \\
		(gh)x &= g(hx)
\end{align*}
\ni izgledata kot definicija identitete in asociativnost.

\ni Velja tudi obratno: vsako delovanje $G$ na $X$ dolo\v ca upodobitev $G$ na $X$:
\begin{align*}
	\left.
	\begin{array}{rl}
		\text{\underline{Upodobitve}} & \\
		G \to S(X) & \\
		\text{homomorfizmi} &
	\end{array} \right\} \longleftrightarrow
	\left\{
	\begin{array}{rl}
		\text{\underline{Delovanja}} & \\
		G \times X \to X & \\
		ex = x &\\
		(gh)x = g(hx)
	\end{array}\right.
\end{align*}

\ni Analogno za linearne upodobitve dobimo korespondenco z delovanji, ki so pri fiksnem $g$ linearni.

\begin{align*}
	\left.
	\begin{array}{rl}
		G \to GL_\F(V) & \\
		\text{$V$ vekt. prostor} \\
		\text{homo.} &
	\end{array} \right\} \longleftrightarrow
	\left\{
	\begin{array}{rl}
		G \times V \to V& \\
		ex = x & \\
		(gh)x = g(hx)
		x \mapsto gx\ \text{je linearna}
	\end{array}\right.
\end{align*}

\begin{defin}
	Naj $G$ deluje na $X$ (imamo delovanje $G \times X \to X$).
	\begin{itemize}
		\item{\emph{Orbita} elementa $x \in X$ je $Gx = \{gx\ |\ g\in G\}$.}
		\item{\emph{Stabilizatorska podgrupa} elementa $x \in X$ je $G_x = \{g \in G\ |\ gx = x\}$.}
	\end{itemize}
\end{defin}

\begin{trditev}
	Naj $G$ deluje na $X$.
	\begin{enumerate}
		\item{Potem je $G_x \leq G, \forall x \in X$.}
		\item{Stabilizator poljubne druge to\v cke na orbiti $Gx$ je konjugiran stabilizatorju $x$.}
		\item{\v Ce je $X$ kon\v cen, je $|Gx| = [G:G_x]$}
	\end{enumerate}
\end{trditev}

\paragraph{Dokaz:}[\emph{Vaja}]
\begin{enumerate}
	\item{$G_x$ je res podgrupa:
		\begin{itemize}
			\item{$e \in G_x:\ ex = e \then e \in G_x$ po definiciji}
			\item{$g \in G_x \then g^{-1} \in G_x$:
				\begin{align*}
					gx &= x,\quad /\cdot g^{-1},\ \text{asociatovnost} \\
					x &= g^{-1}g x = g^{-1} x \then g^{-1} \in G_x \\
				\end{align*}}
			\item{produkt: $ghx = gx = gx = x$ (o\v citno, ker $gx = x$ in $hx = x$). $\blacksquare$}
		\end{itemize}}
	\item{Izberimo poljuben $y \in Gx$. Dokazujemo, da je $G_y$ konjugiran $G_x$, tj. $\exists a$, tako da: $G_y = a G_x a^{-1}$.
		Vemo: $y = gx$ za nek $g \in G$. Naj bo $h \in G_y: hy = y$,
		\begin{align*}
			\left.
			\begin{array}{rl}
			hgx &= gx\\
			g^{-1} hgx &= x\\
			g^{-1} h g &\in G_x
			\end{array} \right\} G_y \subseteq g G_x g^{-1}
		\end{align*}}
		Sedaj enako naredimo za $g^{-1}:\ x = g^{-1} y$, ostalo je enako. Od tam sledi, da je tudi $G_x \subseteq g^{-1} G_y g$.
		To je res natanko tedaj, ko $G_y = gG_x g^{-1}$. $\blacksquare$
	\item{$X$ kon\v cna, $Y = G_x \subseteq X$ tudi kon\v cna. $Y \subseteq X$ je invariantna za delovanje $G$, v smislu, da je
		$\forall g \in G$ in $\forall y \in Y: gy \in Y$.

		\ni Trdimo, da je $|Y|$ enaka \v st. odsekov $G_x$ v $G$. Glejmo preslikavo $f : G \to Y$, $g \mapsto gx$. $f$ je surjektivna  (po definiciji $Y$) in
		\[
			f(g) = f(h) \iff gx = hx \iff h^{-1}gx = x,\ h^{-1}g \in G_x,
		\] to pa je natanko tedaj, ko $h$ in $g$ dolo\v cata isti odsek $\then$ \v st. razli\v cnih to\v ck v $Y$ je enako \v st. razli\v cnih odsekov $G_x$
		v $G$. $|Y| = [G:G_x]$.}
\end{enumerate}
\qed

\section{Operacije na grupah in strukturni izreki}

\begin{defin}
	\emph{Direktni produkt grup} $G$ in $H$ je grupa $G \times H$ z operacijo mno\v zenja po komponentah:
	\[
		(g_1, h_1) \cdot (g_2, h_2) = (g_1 g_2, h_1 h_2).
	\]
	Identiteta je $(e, e)$, inverz elementa $(g, h)$ je $(g, h)^{-1} = (g^{-1}, h^{-1})$, asociativnost pa
	sledi iz asociativnosti v $G$ in $H$.
\end{defin}

\paragraph{Opombe:}
\begin{itemize}
	\item{V grupi $G\times H$ lahko gledamo $G$ in $H$ kot podgrupi, $G \equiv G\times\{e\}$ in $H \equiv \{e\}\times H$. Pri tej
		identifikaciji elementi podgrup $G$ in $H$ med seboj komutirajo.}
	\item{$(g,e) \cdot (e,h) = (ge, eh) = (g,h)$}
	\item{$(e,h) \cdot (g,e) = (eg,he) = (g, h)$}
	\item{Poleg tega sta $G$ in $H$ edinki v $G\times H$, njun presek je $G \cap H = \{e\} = (e,e)$.}
\end{itemize}

\ni Iz teh opa\v zanj dobimo
\begin{trditev}
	$G$, $H$, $K$ grupe, $H,K \lhd G$, $H \cap K = \{e\}$ in $HK = G$. Potem je $G = H \times K$.
\end{trditev}

\paragraph{Dokaz:} [\emph{Ideja}]
Najprej vidimo, da elementi iz $H$ komutirajo z elementi iz $K$ (saj $H \cap K = \{e\}$). Upo\v stevamo {\bf lemo:} $G$, $H$ abelovi $\then G\times H$ abelova.

\paragraph{Primer:}
\begin{enumerate}
	\item{$G = C_2 \times C_3 = ?$\vspace{0.5ex}
		Najprej lahko kar zapi\v semo grupne elemente:
		\begin{align*}
			C_2 &= \{1, -1\}, \\
			C_3 &= \{1, \exp(2i\pi/3), \exp(-2i\pi/3\}
		\end{align*}
		\[
			G = \big\{(1,1), \big(1,e^{2i\pi/3}\big), \big(1,e^{-2i\pi/3}\big), 
				(-1,1), \big(-1, e^{2i\pi/3}\big), \big(-1, e^{-2i\pi/3}\big)\big\}
		\]
		\ni Opazimo, da je $|G| = 6$. A je $G$ izomorfna kak\v sni znani grupi mo\v ci 6,
		npr. $C_6$ (v tem primeru, moramo najti element reda 6). \v Ce je $G$ cikli\v cna ima generator reda $6$
		in preslikava, ki preslika generator v $\exp(2i\pi/6)$ bo izomorfizem $G \to C_6$. Uganemo:
		\begin{align*}
			\big(-1, e^{2i\pi/3}\big)^1 &= \big(-1, e^{2i\pi/3}\big), \\
			\big(-1, e^{2i\pi/3}\big)^2 &= \big((-1)\cdot(-1), e^{2i\pi/3} e^{2i\pi/3}\big) = \big(1, e^{-2i\pi/3}\big), \\
			\big(-1, e^{2i\pi/3}\big)^3 &= (-1, 1), \\
			\big(-1, e^{2i\pi/3}\big)^4 &= \big(1, e^{2i\pi/3}\big), \\
			\big(-1, e^{2i\pi/3}\big)^5 &= \big(-1, e^{-2i\pi/3}\big), \\
			\big(-1, e^{2i\pi/3}\big)^6 &= (1,1)
		\end{align*}
		Vidimo: element $\big(1, \exp(-2i\pi/3)\big)$ generira celotno grupo $G$, in je reda 6 -- tj. $G \cong C_6$, z
		izomorfizmom:
		\begin{align*}
			f : G &\to C_6, \\
			f\Big(\big(-1, e^{2i\pi/3}\big)\Big) &= e^{2i\pi/6},\ \text{homomorfizem, torej,} \\
			f\Big(\big(-1, e^{2i\pi/3}\big)^k\Big) &= e^{2i\pi k/6}.
		\end{align*}
	}
	\item{
		$G = C_2 \times C_4$, ali je izomorfna kak\v sni znani grupi (npr. $C_8$, sode\v c po prej\v snjem primeru)?
		\[
			G = \{(1,1), (1, i), (1, -1), (1, -i), (-1, 1), (-1, i), (-1, -1), (-1, -i)\}
		\]
		Ne glede na to, s katerim elementom bomo posku\v sali, najve\v cji red elementa je $4$. Vendar pa ta grupa ni izomorfna $C_4$, saj
		nimamo bijektivnega homomorfizma, ki bi $8$ elementov preslikal v $4$. Imamo pa\v c abelovo grupo, ki je bolj komplicirana.
	}
	\item{
		$G = C_m \times C_n$, kjer sta $m$, $n$ tuji si \v stevili (njun najve\v cji skupni delitelj je 1).
		\paragraph{Vaja:} Poka\v zi $G \cong C_{m \cdot n}$. Bolj splo\v cno: za poljubni cikli\v cni grupi redov $m$ in $n$
		\[
			G = \langle a \rangle \times \langle b \rangle
		\]
		\ni Trdimo, da ima par $(a, b) \in G$ red $m \cdot n$. Velja $(a,b)^k = \big(a^k, b^k\big)$.
		\begin{itemize}
			\item{$(a, b)^{mn} = (e, e)$ je o\v citno.}
			\item{$mn$ je najmanj\v si tak eksponent v $\mathbb{N}$. Vemo $|G| = mn$. Red vsakega elementa deli $mn$. \v Ce je
				$(a, b)$ manj\v sega reda, kot $mn$, $\exists\ k\ |\ mn \backepsilon: (a,b)^k = (a^k, b^k) = (e, e)$.
				\[
					\left .
					\begin{array}{rl}
						a^k = e &\then m\ |\ k \\
						b^k = e &\then n\ |\ k
					\end{array}\right\} = k = mn
				\]}
		\end{itemize}
		\v Ce $m$ in $n$ nista tuja, je najve\v cji red enak najmanj\v semu skupnemu ve\v ckratniku, kar je manj od $|G| = mn \then$ ni cikli\v cna.
		\qed
	}
\end{enumerate}

\begin{defin}
	\begin{itemize}
		\item{$G$ grupa, $X$ je podmno\v zica v $G$, tj. $X \subseteq G$. $X$ \emph{generira} $G$, \v ce je $G = \langle X \rangle$ najmanj\v sa podgrupa
			v $G$, ki vsebuje $\langle X \rangle$.}
		\item{$G$ je \emph{kon\v cno generirana}, \v ce $\exists$ kon\v cna $X \subseteq G \backepsilon: G = \langle X \rangle$ (tj. \v stevilo generatorjev v $X$ je
			kon\v cno).}
		\item{$G$ je \emph{torzijska}, \v ce je $\forall$ element v $G$ kon\v cnega reda.}
		\item{Za pra\v stevilo $p \in \mathbb{P}$ je $G$ \emph{$p$-grupa}, \v ce je red vsakega elemnta v $G$ neka potenca $p$-ja.}
		\item{$G$ je \emph{prosta abelova grupa} \underline{ranga} $n \in \mathbb{N}$, \v ce je $G \cong \underbrace{\Z \times \Z \times \ldots \times \Z}_\text{$n$-krat} = \Z^n$.}
	\end{itemize}
\end{defin}

\begin{trditev}
	[\emph{Brez dokaza}.] Naj bo $A$ abelova grupa. Potem je mno\v zica vseh elementov kon\v cnega reda v $A$ torzijska podgrupa $T$. \v Ce je $A$ kon\v cno generirana, je
	$\quot{A}{T}$ prosta abelova grupa kon\v cnega ranga in
	\[
		A \cong T \times \quot{A}{T}
	\]
\end{trditev}

\begin{trditev}
	[{\em Brez dokaza}.] Naj bo $T$ kon\v cna abelova grupa. Potem $\exists$ zaporedje naravnih \v stevil $n_1$, $n_2$, $\ldots$, $n_k$, kjer
	\[
		n_k\ |\ n_{k-1}\ |\ n_{k-2}\ |\ \ldots\ |\ n_2\ |\ n_1
	\]
	in velja
	\[
		T \cong C_{n_1} \times C_{n_2} \times \ldots \times C_{n_k}.
	\]
\end{trditev}

\paragraph{Opomba:}
V primeru $C_2 \times C_4$ smo imeli primer tak\v snega zapisa kon\v cne abelove grupe $n_1 = 4$, $n_2 = 2$.

\begin{posledica}
	\v Ce je $T$ kon\v cna abelova grupa in $n = |T|$ nima kvadratnih faktorjev, je $T$ cikli\v cna mo\v ci $n$.
\end{posledica}

\paragraph{Dokaz:}
Po prej\v snjem je $T = C_{n_1} \times \ldots \times C_{n_k}$ \v Ce je $n_2 > 1$, potem je $n_2\ |\ n_1$ in $n_1 \cdot n_2\ |\ n$,
potem $n_2^2\ |\ n$, kar je v nasprotju s predpostavko. $\blacksquare$

\begin{posledica}
	\v Ce je $A$ kon\v cno generirana prosta abelova grupa in $B \leq A$ (podgrupa), je tudi $B$ prosta.
\end{posledica}

Iz vsega tega dobimo metodo za predstavitev poljubne kon\v cno generirane abelove grupe $A$:
\begin{itemize}
	\item{Naj bo $X = \{x_1, x_2,\ldots ,x_n\}$ kon\v cna mno\v zica generatorjev in $\Z^n$ prosta abelova grupa ranga $n$.
	\begin{align*}
		\{e\} \to \underbrace{\ker f}_{\cong \Z^m} \longrightarrow \Z^n &\stackrel{f}{\longrightarrow} A \to \{e\} \\
		\Z^m \in (k_1, k_2, \ldots, k_n) &\stackrel{f}{\mapsto}x_1^{k_1} x_2^{k_2} \ldots x_n^{k_n} \owns A,
	\end{align*}
	homomorfizem $f$ je surjektiven, ker so $x_i$ generatorji. Dobimo zgornje eksaktno zaporedje, ki pomeni $A \cong \quot{\Z^n}{\ker f}$. To
	da opis $A$ z generatorji in relacijami:
	\begin{equation}
		A = \big[\underbrace{x_1, x_2, \ldots, x_n}_\text{generatorji $A$}\ |\ \underbrace{r_1, r_2, \ldots, r_m}_\text{generatorji $\ker f$}\big]
	\end{equation}
	}
\end{itemize}

\paragraph{Opomba:}
Abelove grupe ve\v cinoma pi\v semo aditivno, v tem primeru direktni produkt zamenjamo z direktno vsoto, `$\oplus$'.

\begin{zgled}
	\begin{enumerate}
		\item{Ostanki po modulu $n$:
			\[
				C_n \cong \Z_n = \{0, 1, \ldots, n-1\},
			\]
			$\Z_n$ je cikli\v cna z generatorjem 1.
			\[
				\left.
				\begin{array}{rl}
					\underbrace{\ker f}_{= n\Z} \to \Z &\stackrel{f}{\to} \Z_n \\
					1 &\mapsto 1
				\end{array}
				\right\} \Z_n = \big[1\ |\ n\big],
			\]
			saj je `$1$' generator, in $n \cdot 1 = 0 \in \ker f$. \v Ce generator ozna\v cimo z $x$, potem je
			$\Z_n = \langle x \rangle$, relacija pa je $nx =0$, $\then \Z_n = \big[x\ |\ nx\big]$.
		}
		\item{Naj bo $A = [x, y\ |\ 2x + y,\ x - 2y]$. Zapi\v si $A$ kot vsoto (produkt) proste grupe in cikli\v cnih kon\v cnih grup. To pomeni nekako tole
			\[
				A = \quot{\Z_x \oplus \Z_y}{\langle 2x + y,\ x - 2y\rangle}
			\]
			Ideja je ta, da generatorja (bazo) tako zamenjamo, da bo imenovalec lep\v si. Potem bosta relaciji preprostej\v si, $\then$ relaciji oblike $au$, $bv$.
			Potem delamo linearno algebro nad $\Z$.
			\[
				\left.
				\begin{array}{rl}
					r_1 &= 2x + y \\
					r_2 &= x - 2
				\end{array}
				\right. \quad \Bigg|\ \text{Ideja:}\quad
				\begin{bmatrix}
					u \\ v
				\end{bmatrix} = A
				\begin{bmatrix}
					x \\ y
				\end{bmatrix},\ \text{tj. $u$, $v$ lin. komb. $x$ in $y$.}
			\]
			Matrika $A \in SL(2, \Z)$, tj. mora biti obrnljiva, z `$\det A = 1$' (sicer so dobre tudi matrike, ki imajo `$\det A  = -1$',
			vendar ugibamo, da zado\v s\v ca $\det A = 1$) in celimi koeficienti. Gotovo je ta matrika oblike
			\[
				\begin{bmatrix}
					1 & * \\ 0 & 1
				\end{bmatrix} \quad \text{ali} \quad \begin{bmatrix}
					1 & 0 \\ * & 1 \end{bmatrix}.
			\]	
			\pagebreak

			\ni To nam te ti dve mo\v znosti:
			\[
			\begin{bmatrix}
				1 & n \\ 0 & 1
			\end{bmatrix}
			\begin{bmatrix}
				x \\ y
			\end{bmatrix} =
			\begin{bmatrix}
				x + ny \\
				y
			\end{bmatrix}; \quad
			\begin{bmatrix}
				1 & 0 \\
				n & 1
			\end{bmatrix}
			\begin{bmatrix}
				x \\ y
			\end{bmatrix} =
			\begin{bmatrix}
				x \\ y + nx
			\end{bmatrix}.
			\]
			\ni Dovoljeni operaciji sta:
			\begin{align*}
				(x, y) &\mapsto (x + ny, y), \\
				(x, y) &\mapsto (x, y + nx).
			\end{align*}
			\ni Uporabimo lahko $(x, y) \to (x -2y, y) = (u_1, v_1)$. Relaciji tedaj izgledata
			\begin{align*}
				r_2 &= u_1, \\
				r_1 &= 2(2y + u_1) + y = 5y + 2u_1 = 5v_1 + 2u_1.
			\end{align*}
			$(u_1, v_1)$ ne moremo pretvoriti na ni\v c lep\v sega.
			\paragraph{Ampak:} $r_1$ in $r_2$ generirata podgrupo relacij, na teh operatorjih lahko uporabimo isti princip.
			\[
				(r_1, r_2) \to (r_1 - 2r_2, r_2),
			\]
			\[
				\left.
				\begin{array}{rl}
					r_2 &= u_1 \\
					r_1' &= 5v_1
				\end{array} \right\} A = \quot{\Z_{u_1} \oplus \Z_{u_2}}{\langle u_1, 5v_1\rangle}
			\]
			Ostanejo le ostanki po modulu $5$: generator $u_1$ smo \v cisto "`ubili"', generator $v_1$ pa nam da le tiste, ki so deljivi s $5$. To pomeni
			\[
				A = [u_1 ,v_1\ |\ u_1, 5v_1] = \Z_1 \oplus \Z_5 \cong \Z_5
			\]
		}
	\end{enumerate}
\end{zgled}

\section{Struktura kon\v cno generiranih abelovih grup}

$A$ abelova grupa. Potem
\[
	A \cong \overbrace{\Z^n}^\text{prosti del} \oplus \overbrace{\Z_{c_1} \oplus \Z_{c_2} \oplus \ldots \oplus \Z_{c_k}}^\text{torzijska podgrupa $T$},
\]

\ni $n$ = rang grupe $A$, $c_i$ so torzijski koeficienti in
\[
	c_i \geq 1, \forall i \in I, \quad c_1\ |\ c_2\ |\ \ldots\ |\ c_k,
\]

kanoni\v cna dekompozicija.

\v Ce je $A$ generirana z $x_1, \ldots, x_m$.

\[
	\{0\} \to \underbrace{K}_{\ker f} \to \stackrel[e_i]{}{\Z^m} \stackrel[\mapsto]{}{\to} \stackrel[x_i]{}{A} \to \{ 0\},
\]
\[
	e_i = (0,\ldots,0,\overbrace{1}^{n},0,\ldots,0).
\]

\ni \v Ce generatorje v $\Z^m$ poimenujemo $y_i = e_i$, potem je $K$ kot podgrupa v $\Z^m$ generirana z nekimi $r_j$, ki so linearne kombinacije $y_i$
\[
	r_j = \sum_{i = 1}^m a_{ji} y_i
\]

\ni Relacije lahko zapi\v semo v matriko relacij $R$

\[
	R = \big[a_{ji}\big] =
	\begin{bmatrix}
		a_{11} & a_{12} & \ldots & a_{1m} \\
		a_{21} & a_{22} & \ldots & a_{2m} \\
		\vdots &  & \ddots & & 
	\end{bmatrix}
\]

\ni Ideja za izra\v cuna kanoni\v cne dekompozicije $A$ je, da s spremembami nabora generatorjev matriko $R$ preoblikujemo tako,
da iz nje lahko od\v citamo torzijske koeficiente.

\begin{trditev}
	Naj bo $A$ abelova in $X = \{x_1, x_2, \ldots x_m\}$ nabor generatorjev za $A$. Potem je mno\v zica $Y = \{y_1, \ldots, y_m\}$
	tudi nabor generatorjev za $A$, \v ce $Y$ dobimo iz $X$ z zaporedjem operacij iz tegale nabora:
	\begin{enumerate}
		\item{Zamenjamo lahko dva generatorja:
			\[
				y_j = x_i,\ y_i = x_j, \quad y_k = x_k\ \forall k \neq i,j.
			\]}
		\item{En generator pomno\v zimo z $(-1)$:
			\[
				y_i = -x_i, \quad y_k = x_k \forall k \neq i.
			\]}
		\item{Poljubnemu generatorju pri\v stejemo linearno kombinacijo ostalih:
			\[
				y_i = x_i + \sum_{j \neq i} n_j x_j; \quad n_j \in \Z.
			\]}
	\end{enumerate}
\end{trditev}

\paragraph{Dokaz:}
O\v citno ima vsaka od zgornjih operacij inverz iste vrste, torej res sistem generatorjev preslika v sistem generatorjev
({\em opomba}: to je Gaussova eliminacija nad $\Z$). \qed

\paragraph{Algoritem:}
za izra\v cun KD (kanoni\v cne dekompozicije): Dani $R$ preoblikujemo v "`diagonalno"' matriko,

\[
	\underbrace{\begin{bmatrix}
		c_1 &     &        &        &   &        &   \\
		    & c_2 &        &        &   &        &   \\
		    &     & \ddots &        &   &        &   \\
		    &     &        & c_\ell &   &        &   \\
		    &     &        &        & 0 &        &   \\
		    &     &        &        &   & \ddots &   \\
		    &     &        &        &   &        & 0
	\end{bmatrix}}_{\text{Smithova normalna oblika}}, \quad c_1\ |\ c_2\ |\ \ldots\ |\ c_\ell,\quad c_i \geq 1,
		\quad m - \ell = \rang A.
\]

\ni Matrika $R$ je dimenzije $m \times m$, vendar je njen rang manj\v si za rang matrike $A$. \v Ce je $c_1 = 1$: za generator $x_1$
velja, da je tudi v podgrupi relacij $K$ -- v kvocuentu dobimo $\quot{\Z_{x_1}}{\Z_{x_1}}$ trivialno grupo.

\paragraph{Postopek:}
Na mesto $(1,1)$ postavimo najve\v cji skupni delitelj elementov matrike $R$. Ta postane na\v s $c_1$. Z njim potem najprej uni\v cimo
preostale elemente v prvi vrstici (s pri\v stevanjem prvega stolpca), nato z novo prvo vrstico uni\v cimo preostale elemente v
prvem stolpcu:
\[
	R \to
	\begin{bmatrix}
		c_1 & * & *  & * \\
		\cline{2-4}
		 * &\hspace{-3.2ex}\vline   &    &   \\
		 * &\hspace{-3.2ex}\vline   & R' &  \\
		 * &\hspace{-3.2ex}\vline   &    &
	\end{bmatrix} \to
	\begin{bmatrix}
		c_1 & 0 & 0  & 0 \\
		\cline{2-4}
		 0 &\hspace{-3.2ex}\vline   &    &   \\
		 0 &\hspace{-3.2ex}\vline   & R'' &  \\
		 0 &\hspace{-3.2ex}\vline   &    &
	\end{bmatrix}
\]
Nadaljujemo na matriki $R''$.
\paragraph{Opomba:} Delovanje algoritma da dokaz obstoja kanoni\v cne dekompozicije.

\paragraph{Primeri:}
\begin{itemize}
	\item{Izra\v cunaj KD grupe, ki je kvocient $\Z^3$ po podgrupi, generirani z generatorji $(3,6,-3)$, $(3,3,-6)$, $(3,3,3)$:
		\[
			R = \begin{bmatrix}
				3 & 6 & -3 \\
				3 & 3 & -6 \\
				3 & 3 & 3
			\end{bmatrix} \to \begin{bmatrix}
				3 & 0  & 0  \\
				3 & -3 & -3 \\
				3 & -3 & 6
			\end{bmatrix} \to \begin{bmatrix}
				3 & 0 & 0 \\
				\cline{2-3}
				0 &\hspace{-1ex}\vline -3 & -3 \\
				0 &\hspace{-1ex}\vline -3 & 6
			\end{bmatrix} \to \begin{bmatrix}
				3 & 0 & 0 \\
				0 & 3 & -3\\
				0 & 3 & 6
			\end{bmatrix} \to \begin{bmatrix}
				3 &   & \\
				  & 3 & \\
				  &   & 9
			\end{bmatrix};
		\]
		\[
			\then
			\left.
			\begin{array}{rl}
				c_1 &= 3\\
				c_2 &= 3\\
				c_3 &= 9
			\end{array}\right.,\quad \rang A = 0,
		\]
		$R$ je matrika relacij za grupo $\quot{\Z_x \oplus \Z_y \oplus \Z_z}{\langle3x, 3y, 9z \rangle} =
		[x,y,z\ |\ 3x, 3y, 9z]$.
	}
\end{itemize}
