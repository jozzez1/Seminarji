\part{Grupe}

\chapter{Osnovni pojmi}

\begin{defin}
	\emph{Grupa} $G$ je neprazna mno\v zica z binarno operacijo $\mu$, ki jo imenujemo
	mno\v zenje: $\mu: G\times G \to G$, za katero velja ($a, b, c \in G)$:
	\begin{enumerate}
		\item{operacija $\mu$ je asociativna: $\mu(\mu(a,b), c) = \mu(a, \mu(b, c))$.}
		\item{mno\v zica $G$ vsebuje element, ki je za operacijo $\mu$ identiteta -- $e$: $\mu (e, a) = a$.}
		\item{$\forall a \in G,\ \exists\ a^{-1} \in G$, ki je inverzni element za operacijo $\mu$: $\mu (a, a^{-1}) = e$.}
	\end{enumerate}
\end{defin}

\ni Z drugimi besedami, imamo mno\v zico, ki je zaprta za neko asociativno operacijo $\mu$. Po navadi ena\v cimo $\mu (a, b) \equiv ab$
in pi\v semo skraj\v sano.

\begin{defin}
	Grupa $G$ je \emph{komutativna} ali \emph{abelova}, \v ce za $\forall a,b \in G$ velja $ab = ba$. V tem primeru
	operacijo $\mu$ imenujemo se\v stevanje in zapi\v semo $a + b = b + a$.
\end{defin}

\paragraph{Opombe:}
\begin{itemize}
	\item{Mno\v zica z asociativno operacijo je \emph{polgrupa} (tj. nima nujno vseh inverzov ali identitete).}
	\item{Polgrupa z enoto (identiteto) je \emph{monoid}.}
	\item{Identiteto, $e$, obi\v cajno ozna\v cimo z 1.}
	\item{V primeru abelove grupe v\v casih operacijo ozna\v cimo z znakom `$+$' in identiteto z `$0$'.}
\end{itemize}

\ni Primeri iz znanih mno\v zic \v stevil:
\begin{itemize}
	\item{($\mathbb{N},+$), $+: \mathbb{N}\times\mathbb{N} \to \mathbb{N}$ je asociativna, vendar nima identitete
		(\v stevilo $0$ ni naravno \v stevilo) -- torej je polgrupa.}
	\item{($\mathbb{N}\cup\{0\}, +$) je monoid, saj nimamo inverzov. Do grupe dopolnimo tako: $(\mathbb{Z}, +)$.}
	\item{($\mathbb{N},\cdot$) je spet monoid. Do grupe manjkajo ulomki. Grupa je torej $(\mathbb{Q}, \cdot)$.}
\end{itemize}

\ni Opazimo, da imamo nad mno\v zico $\mathbb{Q}$ v resnici dve asociativni operaciji: mno\v zenje in se\v stevanje.

\begin{defin}
	\emph{Obseg} je (neprazna) mno\v zica $O$, ki ima med svojimi elementi dve asociativni operaciji: mno\v zenje, za
	katero je $O$ grupa; in se\v stevanje, za katero je $O$ abelova grupa, med operacijama pa velja distributivnostni zakon.
	Tj. $\forall a,b,c \in O,\ a (b + c) = ab + ac$.
\end{defin}

\ni Kadar imamo opravka z mno\v zico, ki je abelova grupa za operacijo se\v stevanja, za operacijo mno\v zenja pa le polgrupa,
govorimo o \emph{kolobarju}.

Ostali primeri grup:
\begin{itemize}
	\item{Za $\forall n \in \mathbb{N}$ je mno\v zica ostankov pri deljenju z $n$ kon\v cna grupa z $n$ elementi, operacija
		je `$+$', po modulu $n$. Ta grupa je $\mathbb{Z}_n = \{0, 1, 2, \ldots, n-1\};$ $(a,b) \mapsto a + b\ (\text{mod}\ n)$:
		\begin{itemize}
			\item{identiteta: 0,}
			\item{inverz: $a^{-1} = n - a$}
		\end{itemize}}
	\item{$S^1 =\mathbb{T}$ -- kro\v znica. $S^1 = \{z \in \mathbb{C}\ \big|\ |z| = 1\}$ je grupa za operacijo mno\v zenja kompleksnih \v stevil,
		to je gladka krivulja. Enotsko kro\v znico lahko parametri\v cno zapi\v semo tudi kot $\{e^{i\phi}\ |\ \phi \in \mathbb{R}\}$.}
	\item{$C_n = \{e^{2i\pi k/n}\ |\ k \in 0, 1, 2,\ldots,n-1\}$, oz. cikli\v cna grupa z $n$ elementi. Grupna operacija je
		mno\v zenje kompleksnih \v stevil. Dobimo jo kot grupo $n$-tih korenov. $C_n$ in $\mathbb{Z}_n$ sta algebraji\v cno ekvivalentna.}
\end{itemize}

\begin{defin}
	\v Ce sta $G$ in $H$ grupi, je preslikava $f: G \to H$ \emph{homomorfizem}, \v ce velja $f (g_1 \cdot g_2) = f(g_1) f(g_2)$. Bijektivni
	homomorfizem imenujemo \emph{izomorfizem}. $G$ in $H$ sta \emph{izomorfni}, \v ce med njima obstaja kak tak izomorfizem.
\end{defin}

\begin{zgled}
Trdimo, da sta $\mathbb{Z}_n$ in $C_n$ izomorfni. Poiskati moramo $f: \mathbb{Z}_n \to C_n$. Uganemo
\[
	f(k) \equiv e^{2i\pi k/n},
\]

\ni kar je o\v citno bijekcija. Pokazati moramo, da je \v se homomorfizem. V grupi $\mathbb{Z}_n$ je na\v sa operacija
grupnega mno\v zenja se\v stevanje po modulu $n$. Torej

\[
	f (k_1 \cdot k_2) = f\big(k_1 + k_2\ (\text{mod}\ n)\big) = e^{2i\pi (k_1 + k_2)/n} = e^{2i\pi k_1/n}e^{2i\pi
		k_2/n} = f(k_1) f(k_2).
\]
\qed

\end{zgled}

\pagebreak
\section{Grupe linearnih transformacij}

\begin{defin}
	Mno\v zica $V$ je \emph{vektorski prostor} nad obsegom $\mathbb{F} \in \{\mathbb{R}, \mathbb{C}, \mathbb{H}\}$, kadar imamo
	elementi operaciji se\v stevanja:
	\begin{align*}
		+: V\times V &\to V, \\
		(v, w) &\mapsto v + w,
	\end{align*}
	za katero je abelova grupa; in mno\v zenje s skalarji,
	\begin{align*}
		\cdot: \mathbb{F}\times V &\to V,\\
		(\lambda, v) &\mapsto \lambda v,
	\end{align*}
	za katero je asociativna in velja distributivnostni zakon. Pri mno\v zenju s skalarjem ponavadi ne pi\v smo pike.

	$\emph{Algebra}$ je vektorski prostor nad kolobarjem.
\end{defin}

\ni Naj bo $V$ vektorski prostor nad obsegom $\mathbb{F}$ in naj bo $L_{\mathbb{F}}(V)$ mno\v zica linearnih preslikav (veljata \emph{asociativnost}
in \emph{homogenost}):
\begin{align*}
	T: V &\to V,\\
	T(\lambda v + \mu w) &= \lambda T(v) + \mu T(w).
\end{align*}

\ni Linearne preslikave lahko mno\v zimo s skalarji:
\begin{align*}
	(S + T)v &= Sv + Tv, \\
	(\lambda S)v &= \lambda (Sv).
\end{align*}

\ni Opazimo, da je $L_{\mathbb{F}}(V)$ vektorski prostor nad $\mathbb{F}$. Poleg tega, pa lahko preslikave v $L_{\mathbb{F}}(V)$ \v se komponiramo,
\v cemur re\v cemo produkt:
\[
	(ST)(v) = S (T (v)).
\]

\ni Poraja se nam vpra\v sanje: ali je $L_{\mathbb{F}}(V)$ za mno\v zenje (komponiranje) grupa? Ni\v celna linearna preslikava gotovo ni obrnljiva,
$0: V \to V,\ v \mapsto 0$. Kaj pa ostale? Preslikava $T:\mathbb{R}^2 \to \mathbb{R}^2,\ (x,y) \mapsto (x, 0)$ ni obrnljiva.

Obrnljive so natanko \emph{bijektivne linearne preslikave}. Mno\v zico vseh teh ozna\v cimo z
\[
	GL_{\mathbb{F}}(V) = \{T \in L_{\mathbb{F}}(V)\ |\ T\ \text{ima inverz}\}.
\]

$GL_{\mathbb{F}}(V)$ imenujemo \emph{splo\v sna linearna grupa} -- grupa za komponiranje linearnih preslikav. Naj bo $\{v_1, v_2, \ldots, v_n\}$ baza
v prostoru $V$, nad obsegom $\mathbb{F}$. Potem lahko linearne preslikave predstavimo z matrikami
\[
	M_n (\mathbb{F}) = \{\text{matrike dimenzije $n\times n$, s koeficienti v $\mathbb{F}$}\}.
\]

\ni Potem $GL_{\mathbb{F}}(V)$ ustrezajo obrnljive matrike dimenzije $n\times n$ s koeficienti v $\mathbb{F}$, ki jih ozna\v cimo z

\[
	GL_{\mathbb{F}}(V) = \{A \in M_n (\mathbb{F})\ |\ \text{det}A \neq 0\}.
\]

V definiciji grupe nismo trdili dnoli\v cnosti inverzov in enote, pa te kljub temu velja. \v Se ve\v c, za grupo potrebujemo le "`polovico"' lastnosti.

\pagebreak

\begin{trditev}
	Naj bo $G$ mno\v zica z asociativno operacijo, za katero velja:
	\begin{itemize}
		\item{$\exists e \in G$, tako da $a\cdot e = a,\ \forall a \in G$,}
		\item{$\forall a \in G$, $\exists b \in G$, tako da $a\cdot b = e$.}
	\end{itemize}
	\ni Potem velja:
	\begin{itemize}
		\item[(1)]{\v Ce je $a\cdot a = a\ \Rightarrow a = e$.}
		\item[(2)]{$G$ je grupa (zgoraj velja \v se $ae = a$ in $ab = e$).}
		\item[(3)]{$e$ je en sam in $b$ je za $a$ enoli\v cno dolo\v cen.}
	\end{itemize}
\end{trditev}

\paragraph{Dokaz:}

Dokazali bomo cikli\v cno: $(1) \Rightarrow (3) \Rightarrow (2) \Rightarrow (1) \Rightarrow (3)$.

\ni {\bf (1)} Recimo, da je $a\cdot a = a$. Velja $\forall a$ $\exists b$, tako da $a\cdot b = e$.
\begin{align*}
	aa &= a\ / \cdot b \\
	(aa) b &= \underbrace{ab}_e \\
	a (ab) &= e \\
	\Rightarrow ae &= e
\end{align*}
To je o\v citno res lahko samo, kadar $a = e$.

\ni {\bf (3)} $e$ je en sam: recimo, da obstaja \v se $e'$ z istimi lastnostmi: $\Rightarrow a e' = a, \forall a\in G$.
Ker velja $\forall a \in G$, si za $a$ izberemo $a = e'$
\[
\Rightarrow\ a = e': e'\cdot e' = e' \Rightarrow e' = e,\ \text{po to\v cki (1).}
\]
Za dani $a$ je $b$ en sam: pa recimo, da je $ab = e$ in $ab' = e$.

\ni {\bf (2)} Naj za $a$ in $b$ velja $ab = e$. Videti \v zelimo
\[
ba = e:\ (ba)\cdot (ba) = b \underbrace{(ab)}_e a = \underbrace{(be)}_b a =b a.
\]

Po to\v cki (1) sledi, da je $(ba) = e$. Preveriti moramo \v se, da je $ea = a,\ \forall a\in G$:
\[
	\underbrace{e}_{ab} a = (ab)a= a(ba) = ae = a.
\] 

\ni {\bf (3)} Vrnimo se \v se nazaj k to\v cki (3) in poka\v zimo enoli\v cnost $b$:
\begin{align*}
	ab = e, &\quad ab' = e,\\
	ab' &= e\ / b\cdot\ \text{(mno\v zimo z leve)}\\
	\Rightarrow \underbrace{(ba)}_e b' &= b\\
	eb' &\Rightarrow b
\end{align*}
\qed

\begin{defin}
	Naj bo $G$ grupa. Podmno\v zica $H \subseteq G$ je \emph{podgrupa}, \v ce je $H$ skupaj z operacijo v $G$
	grupa. To ozna\v cimo s $H \leq G$.
\end{defin}

O\v citno za $H$ velja:

\begin{enumerate}
	\item{$e \in H$, identiteta,}
	\item{$a \in H$, potem je tudi $a^{-1} \in H$,}
	\item{$\forall a,b \in H$ je $ab \in H$.}
\end{enumerate}

\begin{trditev}
	Neprazna podmno\v zica $H \subseteq G$ je podgrupa, \v ce in samo \v ce:
	\begin{itemize}
		\item[(i)]{$e \in H$}
		\item[(ii)]{$\forall a \in H$ je $a^{-1} \in H$}
		\item[(iii)]{$\forall a,b \in H$ je $ab \in H$}
	\end{itemize}
\end{trditev}

\paragraph{Dokaz:} Res iz privzetkov in lastnosti mno\v zenja v $G$. $\blacksquare$

\begin{trditev}
	Neprazna podmno\v zica $H \subseteq G$ je podgrupa $\iff \forall a,b \in H$ velja $ab^{-1} \in H$.
\end{trditev}

\paragraph{Dokaz:}
\begin{itemize}
	\item[($\then$)]{ O\v citno.}
	\item[($\Leftarrow$)]{Tu si bomo pomagali s prej\v snjim izrekom:
		\begin{itemize}
			\item{za $b = a$ dobimo $a\cdot a^{-1} \in H$, kar zadosti pogoju (i).}
			\item{\v ce vzamemo $a = e$ in $b = a \then ab^{-1} = ea^{-1} = a^{-1} \in H$, kar zadosti pogoju (ii).}
			\item{$ab = a \cdot (b^{-1})^{-1} \in H$, kar zadosti pogoju (iii).}
		\end{itemize}}
\end{itemize}
\qed

\begin{posledica} Naj bo $G$ grupa; $\forall a \in G$ je mno\v zica
\[
	\langle a \rangle = \{a^n\ |\ n\in\mathbb{Z}\}
\]
podgrupa v $G$, ki jo imenujemo \emph{cikli\v cna grupa}, generirana z $a$. Pri tem $a^n$ pomeni:
\begin{align*}
	a^0 &\equiv e, \\
	a^1 &\equiv a, \\
	a^2 &\equiv a\cdot a, \\
	&\ldots \\
	a^n &\equiv a\cdot a^{n-1} = \underbrace{a\cdot a\cdot \ldots \cdot a}_{n\text{-krat}},\ n\in \mathbb{N}.
\end{align*}

\ni Element $a^{-1}$ imenujemo inverz $a$. Velja $a^{-n} \equiv (a^{-1})^n$.
\end{posledica}

\paragraph{Dokaz:}
Preveriti moramo, da je za $a^n,a^m \in \langle a \rangle \in$, tudi $a^n\cdot a^{-m} \in \langle a\rangle$.
\[
	a^n\cdot a^{-m} = a^{n-m}
\]

\ni Recimo, da je $n,m > 0$. Za $n \geq m$ velja:
\[
	a^n a^{-m} = a^n (a^{-1})^m = a^{n-m+m} (a^{-1})^m = a^{n-m} \underbrace{a^m\cdot(a^{-1})^m}_{e} = a^{n-m}.
\]
Za $n < m$ velja:
\[
	a^n a^{-m} = a^n(a^{-1})^m = \ldots = a^{n-m} = (a^{-1})^{m-n}.
\]
\qed

\paragraph{Primeri podgrup:}
\begin{enumerate}
	\item{$(S^1,\cdot) \leq (\mathbb{C}\backslash\{0\}, \cdot)$},
	\item{$(C_n,\cdot) \leq (S^1, \cdot)$},
	\item{$(\mathbb{Z}_n, +\ {}_{\text{mod}\ n}) \nleq (\mathbb{Z}, +)$, ker operaciji nista isti,}
	\item{$(n\mathbb{Z}, +) \leq (\mathbb{Z}, +)$, kjer $n\mathbb{Z} = \{nk\ |\ k \in \mathbb{Z}\}$, torej mno\v zica celih ve\v ckratnikov
		\v stevila $n$. Opazimo lahko, da je $n\mathbb{Z}$ cikli\v cna grupa, generirana z $n$: $kn$ pomeni
		\[
			\underbrace{n + n + n \ldots + n}_{k\text{-krat}},
		\] v multiplikativnem smislu, je to $n^k$. Aditivni inverz je $(-1)n$, kar multiplikativno pi\v semo $n^{-1}$.}
	\item{$\underbrace{SL_n(\mathbb{F})}_{\leq GL_n(\mathbb{F})} = \{A \in GL_n(\mathbb{F})\ |\ \text{det}A = 1\}$. Enostavno se lahko prepri\v camo, da je
		res podgrupa:
		\begin{itemize}
			\item{$I \in SL_n(\mathbb{F})$, $\text{det}I = 1$}
			\item{$A \in SL_n(\mathbb{F})$, $A^{-1}:\ \det (A^{-1}) = 1/\det A = 1\ \then A^{-1} \in SL_n(\mathbb{F})$}
			\item{$A,B \in SL_n(\mathbb{F})$, $\det (AB) = \det A\det B = 1 \then AB \in SL_n(\mathbb{F})$}
		\end{itemize}}
	\item{$O_n = \{A \in GL_n(\mathbb{R})\ |\ A^T A = I\} \leq GL_n (\mathbb{R})$}
	\item{$U_n = \{A \in GL_n(\mathbb{C})\ |\ A^* A = I\} \leq GL_n (\mathbb{C})$, v fiziki bi $A^*$ pisali kot $A^\dagger$.}
	\item{$SO_n = SL_n (\mathbb{R}) \cap O_n$}
	\item{$SU_n = SL_n (\mathbb{C}) \cap U_n$}
	\item{$Sp_n = \{A \in GL_n(\mathbb{H})\ |\ A^* A = I\} \leq GL_n (\mathbb{H})$\}, \emph{simplekti\v cna} grupa -- ohranja neko koli\v cino (v fiziki bi to bila energija). Mno\v zica
		$\mathbb{H}$ je prostor kvaternionov.}
\end{enumerate}

\begin{trditev}
	Naj bodo $H_i \leq G$, za $i \in I$ ($I$ je indeksna mno\v zica in se bo \v se velikokrat pojavljala). Potem je tudi
	\[
		\bigcap_{i\in I} H_i \leq G
	\]
\end{trditev}

\paragraph{Dokaz:} Sledi iz trditve o $ab^{-1}$\ldots O\v citno res. $\blacksquare$

\begin{posledica}
	$\forall X \subseteq G$ obstaja najmanj\v sa podgrupa v $G$, ki vsebuje $X$. Tej grupi re\v cemo podgrupa, generirana z $X$ in jo ozna\v cimo z $\langle X \rangle$.
\end{posledica}

\section{Homomorfizmi in izomorfizmi}

\begin{defin}
$G$, $H$ grupi. Preslikava $f : G \to H$ je \emph{homomorfizem}, \v ce je $f(a,b) = f(a) f(b)\ \forall a,b \in G$. Bijektivni homomorfizem imenujemo \emph{izomorfizem}:

\paragraph{Opomba:} \v Ce je $f : G \to H$ homomorfizem, potem je $f(e) = e$ in $f(a^{-1}) = \big(f(a)\big)^{-1}$ (torej enoto preslika v enoto in inverze preslika v inverze).
\end{defin}

\begin{defin}
	\emph{Endomorfizem} je homomorfizem, ki slika sam nase.

	\ni \emph{Avtomorfizem} je bijektivni endomorfizem, tj. izomorfizem, ki slika sam nase.
\end{defin}

\paragraph{Primeri:}
\begin{itemize}
	\item{$H, G$ grupi, $H \leq G$. \emph{Inkluzijska preslikava} $i: H \to G,\ h\mapsto h$ je homomorfizem. \label{inkluzija}}
	\item{$G$ grupa, $a \in G$.
		\begin{itemize}
			\item{\underline{\emph{Leva translacija} za $a$} je $L_a: G\to G,\ g \mapsto ag$. \label{translacija}}
			\item{\underline{\emph{Desna translacija} za $a$} je $R_a: G\to G,\ g \mapsto ga$.}
		\end{itemize}
		$L_a$ in $R_a$ sta homomorfizma le za $a = e$, tedaj je $L_a = R_a$, tj. identiteta. Za $a \neq e$ pa velja $L_a (e) \neq e$, $R_a(e) = a \neq e$.
		Ampak: $L_a$ in $R_a$ sta bijekciji, inverza sta $L_{a^{-1}}$ in $R_{a^{-1}}$.}
	\item{$f_a : G \to G,\ g \mapsto aga^{-1}$ je \emph{konjugacija} ali \emph{notranji avtomorfizem}.
		\paragraph{Dokaz:}
		\begin{itemize}
			\item{$f_a (gh) = agha^{-1} = ag\underbrace{a^{-1}a}_{e}ha^{-1} = f_a (g) \cdot f_a (h)$ -- res endomorfizem.}
			\item{$f_a = L_a \circ R_{a^{-1}}$, obe sta bijektivni, tj. je tudi njun kompozitum, $f_a$ bijektivna -- res avtomorfizem. $\blacksquare$}
		\end{itemize}}
	\item{$\exp: (\mathbb{R},+) \to \big((0,\infty), \cdot\big)$ je izomorizem (inverz je $\ln$, oz. $\log$).
		\paragraph{Dokaz:}
		\begin{itemize}
			\item{O\v citno bijektivna funkcija.}
			\item{Dokazati moramo, da je homomorfizem:
			\[
				\exp (x+y) = \exp(x) \cdot \exp(y)\quad \blacksquare
			\]}
		\end{itemize}}
	\item{Za $d,n \in \mathbb{N}$ je $f_d : C_n \to C_{nd},\ z \mapsto z^d$ (spomnimo se, da je $C_n = \{z \in \mathbb{C}\ |\ z^n = 1\}$ grupa
		$n$-tih korenov) homomorfizem.
		\paragraph{Dokaz:} Res slika v $C_{nd}$:
		\[
			z^n = 1;\ f(z) = z^d\ \then\ z^{nd} = (z^n)^d = 1^d = 1.
		\] To je o\v citno homomorizem, ki slika v $C_{nd}$, vendar $f_d$ ni surjektivna. $\blacksquare$}
	\item{Matri\v cna homomorfizma nad obsegom $\F$:
		\begin{itemize}
			\item{Determinanta za mno\v zenje matrik: $\det: (GL_n (\F, \cdot) \to (\F, +)$, $A \mapsto \det A$ obseg.
				\paragraph{Dokaz:} $\det (AB) = \det A \det B.\ \blacksquare$}
			\item{Sled za se\v stevanje matrik: $\tr : \big(M_n(\F), +\big)$, $A = \big[a_{ij}\big]^n_{i,j = 1} \mapsto \tr (A) = \sum_{i=1}^n a_{ii}$
				\paragraph{Dokaz:} $\tr(A + B) = \tr(A) + \tr(B)$. $\blacksquare$}
		\end{itemize}}
\end{itemize}

\ni Za la\v zjo pisavo bomo uvedli nekaj oznak.

\paragraph{Oznaka:} $G$ grupa, $A,B \subseteq G$ podmno\v zici.
\begin{itemize}
	\item{$AB \equiv \{ab\ |\ a \in A, b \in B\}$}
	\item{$Ab \equiv \{ab\ |\ a \in A\}$}
	\item{$aB \equiv \{ab\ |\ b \in B\}$}
\end{itemize}

\begin{defin}
	$G, H$ grupi, $H \leq G$. $H$ je \emph{edinka} v $G$ (\emph{normalna podgrupa}), \v ce $\forall a \in G$ velja
	\[
		a H a^{-1} \subseteq H.
	\]
	Oznaka je $H \lhd G$.
\end{defin}

\begin{lema}
	$H \leq G$ je edinka $\iff a H a^{-1} = H\ \forall a \in G$.
\end{lema}

\paragraph{Dokaz:}
\begin{itemize}
	\item[($\Leftarrow$)]{O\v citno, saj je $a H a^{-1} = H \leq H$.}
	\item[($\then$)]{Vemo: $aHa^{-1} \subseteq H,\ \forall a \in G$. Za poljubno izbrani $a \in G$ velja $aHa^{-1} \subseteq H$. Za
		svoj "`$a$"' izberem njegov inverz, tj. `$a^{-1}$', kar nam da $a^{-1}Ha \subseteq H$. To pomeni, $\forall h \in H, a^{-1} h a \in H$,
		torej $a^{-1} h a = k \in H \then h = aka^{-1} \in a H a^{-1} \then \forall h \in H$ je $h \in aHa^{-1} \then H\subseteq a H a^{-1}$.
		Vemo, da \v ce $A \subseteq B$ in $B \subseteq A \then A = B$, tj.
		\[
			a H a^{-1} = H.
		\]}
\end{itemize}
\qed

\pagebreak
\begin{trditev}
	Naj bo $f : G \to H$ homomorfizem grup. Potem velja:
	\begin{itemize}
		\item[(1)]{Slika $f$: $\im f = \{ f(g)\ |\ g\in G\} \leq H$.}
		\item[(2)]{Jedro $f$: $\ker f = \{ g \in G\ |\ f(g) = e\} \lhd G$.}
	\end{itemize}
\end{trditev}

\paragraph{Dokaz:}
\begin{itemize}
	\item[{\bf (1)}]{Poka\v zimo, da je $\im f$ res podgrupa v $H$:

		Za $f(g_1)$, $f(g_2) \in \im f$ moramo preveriti, da je $f(g_1) \big(f(g_2)\big)^{-1} \in \im f$.
	\[
		f(g_1) \big(f(g_2)\big)^{-1} = f (\underbrace{g_1 g_2^{-1}}_{\in G}) \in \im f \then\ \text{je res grupa.}
	\]}
	\item[{\bf (2)}]{Poka\v zimo, da je $\ker f$ res edinka v $G$:
		\begin{itemize}
			\item[$\bullet$]{Res grupa:
			\begin{itemize}
				\item[--]{$e \in \ker f$, ker je homomorfizem (identiteto slika v identiteto, tj. $e \in \ker f$).}
				\item[--]{$a \in \ker f\ \then f(a) = e \then f(a^{-1}) = e^{-1} = e \then a^{-1} \in \ker f$,}
				\item[--]{$a,b \in \ker f\ \then\ f(a) = e = f (b) \then f(ab) = f(a) f(b) = ee = e\ \then\ ab \in \ker f$.}
			\end{itemize}}
			\item[$\bullet$]{Res edinka:
				\[
					g\in \ker f\ \text{in}\ a \in G:\ f(aga^{-1}) = f(a) f(g) f(a^{-1}) = f(a) e f(a^{-1}) ) e\ \then aga^{-1} \in \ker f.
				\]}
		\end{itemize}
		}
\end{itemize}
\qed

\ni Na osnovi tega dobimo kanoni\v cno dekompozicijo homomorfizma $f : G \to H$ tako, da ga predstavimo kot kompozitum surjektivnega homomorfizma, izomorfizma
in injektivnega homomorfizma.

\ni To znamo narediti pri linearni algebri:
\begin{trditev}
	\v Ce sta $V, W$ vektorska prostora nad obsegom $\F$ in je $T : V\to W$ linearna preslikava, ozna\v cimo s $\ker T = \{v \in V\ |\ T(v) = 0\}$; in
	$\im T = \{T(v)\ |\ v\in V\}$, velja
	\[
		V\ \to \quot{V}{\ker T} \stackrel{\overline{T}}{\longrightarrow}\ \im T \hookrightarrow W
	\]
\end{trditev}

To "`klobaso"' interpretiramo kot
\begin{itemize}
	\item{$V \to \quot{V}{\ker T}$ je kvocientna projekcija: $v_1 \sim v_2$, \v ce je $(v_1 - v_2) \in \ker T$. Simbol `$\sim$' predstavlja ekvivalen\v cno relacijo.}
	\item{$\quot{V}{\ker T} \stackrel{\overline{T}}{\longrightarrow} \im T$ je preslikava inducirana s $T$, tj. slika enako, kot $T$.}
	\item{$\im T \hookrightarrow W$ je inkluzija (oz. inkluzijska preslikava, glej str.~\pageref{inkluzija}).}
\end{itemize}

Podobno bi radi naredili za grupe (tj. dobili kanoni\v cno dekompozicijo homomorfizma grup).

\pagebreak
\begin{defin}
	Naj bo $H \leq G$. \emph{Levi odsek} $H$ v $G$ je $aH$, $a \in H$, \emph{desni odsek} pa $Ha$, $a \in G$. Element $a$ je \emph{predstavnik} odseka.
\end{defin}

\paragraph{Opomba:} Odsek ima ve\v c predstavnikov. Kdaj je $aH = bH$ (kdaj je levi odsek za dva predstavnika enak)?
\begin{itemize}
	\item{\v Ce na desni v $H$ izberemo $e$, sledi $b\cdot e \in bH = aH$ ($\then b \in aH$).
		\[
			\then b = ah\ \text{za nek}\ h \in H \then a^{-1}b = h \in H.
		\]
		\v Ce $a$ in $b$ predstavljata isti odsek, potem je $ab^{-1} \in H$ (in ekvivalentno $b^{-1}a \in H\ \then$ v eno smer velja).}
	\item{\v Ce je $a^{-1}b = h \in H$, potem $b = ah$, zato je $bH = ahH$. Ker je $H$ grupa, je $hH \in H \then \overbrace{ah}^b H \subseteq aH$. Lahko tudi
		zamenjamo vlogi $a$ in $b$ $\then\ aH = bH$.}
	\item{$a \sim b \stackrel{\text{def}}{\iff} a^{-1}b \in H$ je ekvivalen\v cna relacija. $a \sim b$ v tem primeru pomeni $aH = bH$.}
\end{itemize}

\begin{trditev}
	Naj bo $H \leq G$. Potem sta odseka $aH$ in $bH$ bodisi disjunktna (presek je prazna mno\v zica), bodisi enaka. Slednje velja $\iff a^{-1}b \in H$.
\end{trditev}

\paragraph{Dokaz:}
\v Ce $aH$ in $bH$ nista disjunktna obstaja $c \in aH \cap bH \then c = ah = bk;\ h,k \in H$. Ker $H \owns hk^{-1} = a^{-1}b \then a^{-1}b \in H \then$ sta
enaka.
\[
	\then a^{-1} b \in H \then aH = bH\ \text{(dokazali v opombi)\footnote{Dokaz: \v ce imata 1 skupni element, sta enaka.}}.
\]
\qed

\begin{posledica}
	Po zadnji trditvi grupa $G$ razpade na disjunktne odseke po podgrupi $H$:
	\[
		G = a_1 H \cup a_2 H \cup \ldots \cup a_n H = \bigcup_{i \in I} a_i H,
	\]
	izberemo tako, da iz vsakega ekvivalen\v cnega razreda vzamemo enega predstavnika. Podobno lahko naredimo z desnimi odseki:
	\[
		G = Hb_1 \cup Hb_2 \cup \ldots \cup Hb_n = \bigcup_{i \in I} Hb_i
	\]
\end{posledica}

Ali je \v stevilo (medsebojno disjunktnih) levih odsekov enako \v stevilu (medsebojno disjunktnih) desnih odsekov? Ustrezajo\v ci levi in desni odseki
\emph{niso nujno enaki}!

\pagebreak
\begin{defin}
	\begin{itemize}
		\item{\v ce je $G$ kon\v cna grupa, in $H \leq G$, ozna\v cimo \v st. odsekov $H$ v $G$ kot $[G:H]$. Isto oznako uporabimo tudi, \v ce je
		$G$ neskon\v cna, $[G:H]$ pa je \v se vedno kon\v cno. To \v stevilo imenujemo \emph{indeks} grupe $H$ v $G$.}
		\item{\v Stevilo elementov grupe $G$ ozna\v cimo z $|G| \in \mathbb{N}\cup\{\infty\}$. To imenujemo \emph{mo\v c} ali \emph{red} grupe.}
		\item{Naj bo $a \in G$. Najman\v se \v stevilo $n \in \mathbb{N}$, za katerega je $a^n = e$, imenujemo \emph{red elementa} $a$. \v Ce tak $n$
		ne obstaja, je red $\infty$ (neskon\v cno).}
	\end{itemize}
\end{defin}

\begin{trditev}
\paragraph{Lagrangejev izrek:} $G$ kon\v cna grupa, $H \leq G$. Potem
\[
	[G:H] = \frac{|G|}{|H|}.
\]
Posebej sledi, da mo\v c podgrupe deli mo\v c grupe.
\end{trditev}

\paragraph{Dokaz:}
Odseki $H$ v $G$ dolo\v cajo (razcep) za $G$: $G = a_1 H\cup a_2 H \cup \ldots \cup a_n H$, vsi navedeni odseki so disjunktni.
\begin{itemize}
	\item{Da unija odsekov zastopa ves $G$, saj je poljuben $a \in G$ v odseku $aH$ (ker $e \in H$).}
	\item{V $a_i H$ je ravno $|H|$ elementov ($\forall i\in I$), saj je $L_{a_i}$ bijekcija.
		\[
			|G| = |a_1 H| + \ldots + |a_n H| = n \cdot |H| = [G:H] \cdot |H|.
		\]}
\end{itemize}
\qed

\begin{trditev}
	$G,H$ grupi, $H \leq G$. Levi odseki $H$ v $G$ so v bijektivni korsespondenci z desnimi odseki.
\end{trditev}

\paragraph{Dokaz:}
\begin{itemize}
	\item{I\v s\v cemo preslikavo, za katero bi radi pozneje pokazali, da je bijekcija:
	\begin{align*}
		F:\{\text{levi odseki}\} &\to \{\text{desni odseki}\}, \\
		aH &\to Ha^{-1}.
	\end{align*}
	Zakaj je dobro $F(aH) = Ha^{-1}$ vidimo, \v ce preverimo, kdaj sta desna odseka enaka: $Hb = Hc
	\iff b = hc \iff bc^{-1} \in H$, za leve je pa $a^{-1}b \in H\ \then$ na eni strani moramo dobiti
	inverz, da bosta pogoja enaka.}
	\item{Preverimo, da je $F$ dobro definirana (tj. da je relacija $F$ res preslikava): \v ce sta $aH$
	in $bH$ enaka leva odseka, jih mora $F$ preslikati v enaka desna,
	\begin{align*}
		aH = bH &\iff a^{-1}b \in H, \\
		Ha^{-1} = Hb^{-1} &\iff a^{-1}(b^{-1})^{-1} = a^{-1} b \in H,
	\end{align*}
	to pa pomeni
	\begin{itemize}	
		\item{Dva enaka slika v dva enaka $\then$ je funkcija.}
		\item{Dva razli\v cna slika v 2 razli\v cna -- $a^{-1}b \notin H \then a^{-1}(b^{-1})^{-1} =
			(b^{-1}a)^{-1} \notin H \then$ injekcija.}
		\item{Vsi so slike: vsak ima $a^{-1}$ inverz.}
	\end{itemize}}
	\item{O\v citno je bijekcija.}
\end{itemize}
\qed

\begin{posledica}
	\v Ce je $G$ kon\v cna grupa in $a \in G$, potem 
	\[
		\red(a)\ \big|\ |G|
	\] (red $a$ deli mo\v c grupe $G$). \v Se ve\v c, $\red(a)$ je mo\v c cikli\v cne podgrupe, generirane
	z $a$:
	\[
		\red(a) = |\langle a \rangle|\ 
	\]
\end{posledica}

\paragraph{Dokaz:}
Naj bo $H = \langle a \rangle \leq G$, po Langrangejevem izreku sledi $|H|$ deli $|G|$. Dokazati moramo le \v se $\red (a) = |\langle a \rangle |$. Elementi
$H$ so
\[
	\langle a \rangle = \{e, a^{\pm 1}, a^{\pm 2}, a^{\pm 3} \ldots\}.
\]
Ker je $G$ kon\v cna, se bodo za\v celi ponavljati. Naj bo $k\in \mathbb{N}$ najmanj\v se \v stevilo, da za nek $m \in \mathbb{Z}$
velja $a^{m+k} = a^m$. Od tod z mno\v zenjem z $a^{-m}$ dobimo $a^k = e$. Ker je $k$ najmanj\v si mo\v zni, je to ravno
$\red(a)$. Potem $\forall m \in \mathbb{Z}$ velja
\begin{align*}
	m &= qk + r,\quad q \in \mathbb{Z}, \quad 0 \leq r < k, \\
	a^m &= a^{qk} \cdot a^r = (a^k)^q a^r = e^q a^r = a^r.
\end{align*}
\begin{itemize}
	\item[$\then$]{v cikli\v cni grupi so natanko elementi $e, a, \ldots, a^{k-1}$.}
	\item[$\then$]{$|\langle a \rangle| = k = \red (a)$.}
\end{itemize}
\qed

\begin{zgled}
\v Ce je $[G:H] = 2$, je $H \lhd G$.
\paragraph{Re\v sitev:}
\begin{itemize}
	\item{$G$ razpade na dva odseka (vemo).}
	\item{Radi bi pokazali: $\forall a \in G$ je $aHa^{-1} \subseteq H$.
		\begin{itemize}
			\item{\v ce je $a \in H \then a H a^{-1} \subseteq H$, ker je $H$ grupa.}
			\item{izberimo poljuben $a \notin H$ ($\then a^{-1} \notin H$) -- spet \v zelimo $aHa^{-1} \subseteq H$. Opazimo:
				$aH \neq H = eH$ \then sta disjunktna $\then G = eH \cup aH$.}
			\item{Podobno velja, da sta $H$ in $Ha$ dva razli\v cna desna odseka $\then aH = Ha \neq H$. Ta izraz lahko
				z desne mno\v zimo z $a^{-1}$ dobimo
				\[
					aHa^{-1} = H.
				\]}
		\end{itemize}}
\end{itemize}
\qed
\end{zgled}

\begin{zgled}
	$G, H$ grupi, $H \leq G$. Velja $H \lhd G \iff \forall$ levi odsek, je tudi desni odsek.
	\paragraph{Re\v sitev:}
	\begin{itemize}
		\item[$(\then)$:]{$aHa^{-1} = Ha\ \then\ aH = Ha,\ \forall a \in G$. $\square$}
		\item[$(\Leftarrow)$:]{$\forall a\ \exists b$, tako da: $aH = Hb$. \v zelimo $b = a$. Ali smemo?
		
		Za $e \in H$ na desni dobimo $eb = b \in aH; b = ah, h \in H$. $a^{-1}b \in H$.
		\begin{align*}
			aH &= Hb, \\
			aHb^{-1} &= H, \\
			aH (ah)^{-1} &= a\underbrace{Hh^{-1}}_{H}a^{-1} = aHa^{-1}.\ \square
		\end{align*}}
	\end{itemize}
\qed
\end{zgled}

\begin{trditev}
	$G$ grupa, $H \lhd G$. Potem je mno\v zica levih odsekov $H$ v $G$ grupa za operacijo
	\[
		aH \cdot bH \stackrel{\text{def}}{=} abH,
	\]
	kar nakazuje, da je to natanko tedaj, ko so levi odseki enaki desnim.
	\begin{itemize}
		\item{Oznaka za to grupo je $\quot{G}{H} = \{aH\ |\ a\in G\}$. Imenujemo jo \emph{faktorska}, ali
			\emph{kvocientna} grupa grupe $G$ po edinki $H$ (z operacijo \ldots).}
		\item{Enota za $\quot{G}{H}$ je $eH = eH$, inverz pa $(aH)^{-1} = a^{-1} H$.}
	\end{itemize}
\end{trditev}

\paragraph{Dokaz}
\begin{itemize}
	\item{Asociativnost sledi iz asociativnosti mno\v zenja v $G$.}
	\item{Notranja: $aH \cdot bH \in \quot{G}{H}$.
		\[
			aH \cdot bH = abb^{-1}HbH = ab HH = ab H \in \quot{G}{H}.
		\]}
\end{itemize}
\qed

\ni Zgornji izrek nam da tole zaporedje grup in homomorfizmov:

\begin{align}
	p : a &\longrightarrow aH, \notag \\
	\{e\} \to H \stackrel{i}{\hookrightarrow} G &\stackrel{p}{\longrightarrow}\ \quot{G}{H} \to \{e\}.
\end{align}

\ni Tu se moramo spomniti
\begin{itemize}
	\item{inkluzija $i$ je injektivna,}
	\item{inducirana preslikava $G \stackrel{p}{\to} \quot{G}{H}$ je surjektivna,}
	\item{$\ker p = H$, $\im(i) = H$.}
\end{itemize}

\ni Vse preslikave tu so homomorfizmi. To zaporedje je \emph{eksaktno}: pri vski grupi je jedro izhodnega
homomorfizma enako sliki vhodnega.
\begin{itemize}
	\item{pri $H$: slika je $\{e\}$; jedro je $\{e\}$, ker je inkluzija injektivna.}
	\item{pri $G$: slika je $H$; jedro je $\{a \in G\ |\ p(a) = aH = eH = H\}$. Pogoj $aH = H \iff a \in H$.}
	\item{pri $\quot{G}{H}$: slika od $p$ je $\quot{G}{H}$; jedro je $\quot{G}{H}$, ker se vse slika v $e$.}
\end{itemize}

\begin{trditev}
	$H \lhd G,\ f : G \to K$ homomorfizem, za katerega velja $H \subseteq \ker f$ ($f : \{H\} \to \{e\}$).
	
	\ni Potem $f$ dolo\v ca homomorfizem $\overline{f} : \quot{G}{H} \to K$ s predpisom
	\begin{alignat*}{2}
		G \hspace{2ex} &\stackrel{f}{\longrightarrow} &K \\
		\stackrel[p]{}{\searrow} &\raisebox{-2ex}{$\quot{G}{H}$} &\stackrel[\overline{f}]{}{\nearrow}
	\end{alignat*}
	se pravi
	\begin{align*}
		G &\stackrel{f}{\to} K \\
		\text{in}\ &\text{hkrati} \\
		G \stackrel[p]{}{\to}\ &\quot{G}{H} \stackrel[\overline{f}]{}{\to} K
	\end{align*}
	za katerega velja $f = \overline{f}\circ p$.
\end{trditev}

\paragraph{Dokaz:}
\begin{itemize}
	\item{Preveriti moramo, da je $\overline{f}$, da je $\overline{f}$ dobro definiran in homomorfizem. Te\v zava
	je v tem, da odsek nima samo enega predstavnika. ($aH = bH$, $a \neq b$), $\overline{f}$ pa je dolo\v cena 
	predstavnikom. $aH = bH \then$ \v zelimo vedeti $f(a) = \overline{f}(ab) ) = \overline{f}(bH) = f (b)$.
	\begin{align*}
		\then a^{-1}b \in &H \subseteq \ker f \\
		f(a^{-1}b) = f(&a)^{-1} f(b) = e. \quad \square
	\end{align*}}
	\item{Da je $\overline{f}$ homomorfizem takoj sledi: $\overline{f}(aH \cdot bH) = \overline{f}(abH) = f(ab) = f(a)
		\cdot f(b) = \overline{f}(aH) \cdot \overline{f}(bH)$.}
\end{itemize}
\qed

\ni Od tod takoj sledi kanoni\v cna dekompozicija homomorfizma.
\begin{trditev}
	\paragraph{Prvi izrek o izomorfizmu:} Naj bo $f:G \to H$ homomorfizem (kot \v ze vemo, je jedro edinka). Potem je
	\[
		\overline{f}: \quot{G}{\ker f} \to \im f
	\] izomorfizem in zaporedje
	\[
		\{e\} \to \ker f \hookrightarrow G \stackrel{P}{\longrightarrow} \quot{G}{\ker f} \stackrel{f}{\to} \im f \hookrightarrow H
	\] je \emph{kanoni\v cna dekompozicija homomorfizma grup} $f$, kar lahko zapi\v semo tudi
	\[
		f = i \circ \overline{f} \circ P
	\]
\end{trditev}

\paragraph{Dokaz:} Jedro homomorfizma, $\ker f$, je edinka v $G$, zato sledi, da obstaja inducirani homomorfizem $\overline{f}: \quot{G}{\ker f} \to H$.
\underline{Trdimo:} $\overline{f}$ je injektiven in zato bijektiven na $\im f$. \v Ce je $a\cdot \ker f$ v jedru $\overline{f}$, torej
$\overline{f}(a\cdot \ker f) = e$, je $f(a) = e$ in $a \in \ker f \then a\cdot \ker f = \ker f$, ki je identiteta v $\quot{G}{\ker f}$.
Edini odsek, ki se s $\overline{f}$ preslika v $e$ je identiteta .
\qed

\begin{zgled}
$(\mathbb{Z}, +)$; $n\mathbb{Z} = \{nk\ |\ k \in \mathbb{Z}\}$ je podgrupa v $\mathbb{Z}$. Potem je $n\mathbb{Z} \lhd \mathbb{Z}$ (edinka).\\[6pt]

\ni {\bf Bolj na splo\v sno:} \v Ce je $G$ abelova in $H \leq G$, je $H$ edinka, tj. $aHa^{-1} = \{aha^{-1}\ |\ h \in H\} = \{aa^{-1}h\ |\ h\in H\} = H$.\\[6pt]

\ni Kaj je kvocientna grupa po edinki $n\mathbb{Z}$? Grupa $\quot{\mathbb{Z}}{n\mathbb{Z}}$ je mno\v zica odsekov $n\mathbb{Z}$ v $\mathbb{Z}$. Hitro vidimo,
da je $\quot{\mathbb{Z}}{n\mathbb{Z}} = \{n\mathbb{Z}, 1 + n\mathbb{Z}, \ldots, n - 1 + n\mathbb{Z}\}$ je izomorfna grupi ostankov pri deljenju z $n$, tj.

\begin{align*}
	f:\mathbb{Z} &\to \mathbb{Z}_n\\
	k &\mapsto k\ (\text{mod}\ n)
\end{align*}

\paragraph{Doka\v zi:} $f$ je surjetkivni homomorfizem in $\ker f = n\mathbb{Z}$, zato je $\overline{f}: \quot{\mathbb{Z}}{n\mathbb{Z}} \stackrel{\cong}{\longrightarrow} \mathbb{Z}_n.$
\end{zgled}

\paragraph{Ponovitev:}
$G, H$ grupi, $H \leq G \iff \forall a, b \in H: ab^{-1} \in H$.
\begin{itemize}
	\item[(1)]{$H$ je \emph{edinka}, \v ce jo ohranjajo konjugacije, $aHa^{-1} \subseteq H\ \forall a \in G\ 
		(\iff aHa^{-1} = H)$.}
	\item[(2)]{\emph{Kvocientna grupa}: $\quot{G}{H} = \{aH\ |\ a \in G\}$ je grupa, \v ce je $H$ edinka (tedaj so levi odseki enaki desnim in velja $aH \cdot bH = a HH b = a H b = ab H$).}
	\item[(3)]{$f : G \to H$ homomorfizem grup $\then\ \ker f \lhd G$.}
	\item[(4)]{$\im f$ inducira izomorfizem $\overline{f} : \quot{G}{\ker f} \stackrel{\cong}{\longrightarrow} \im f$}
	\item[(5)]{To da dekompozicijo kot $f : G \stackrel{p}{\longrightarrow} \quot{G}{\ker f}\stackrel{\overline{f}}{\longrightarrow} \im f \hookrightarrow H$}
\end{itemize}

\ni To\v cki (4) in (5) sta \emph{prvi izrek o izomorfizmu}. Grupa $\quot{G}{\ker f}$ ima za elemente mno\v zico $\{a\cdot \ker f\ |\ a \in G\}$.

\begin{zgled}
	Poka\v zi, da je $SL_n (\F)$ edinka v $GL_n (\F)$ in "`izra\v cunaj"' pripadujo\v co kvocientno grupo (tj. poi\v s\v ci znano grupo, ki ji je kvocientna grupa izomorfna).

	\paragraph{Ideja:} Poi\v scemo homomorfizem $f : GL_n (\F) \to H$ v primeru grupe $H$ tako, da je $\ker f = SL_n (\F)$. Potem bo prvi izrek o izomorfizmu
	\[
		\f : \quot{GL_n(\F)}{SL_n(\F)} \stackrel{\cong}{\longrightarrow} \im f \subseteq H
	\]

	\paragraph{Re\v sitev:} Ta homomorfizem je o\v citno determinanta (vse matrike iz $SL_n(\F)$ so v jedru), za $\F$ pa vzamemo $\F^* \equiv \F\backslash\{0\}$.
	\begin{align*}
		\det : GL_n (\F) &\to \F^* \\
		A &\mapsto \det A
	\end{align*}
	Velja tudi, da je $SL_n (\F)$ edinka:
	\paragraph{Dokaz:}
	Naj bosta $A, A^{-1} \in GL_n (\F)$ in $S \in SL_n(\F)$, $SL_n(\F) = \{S\ |\ \det S = 1\}$.
	\begin{align*}
		\det (A^{-1}) &= (\det A)^{-1},\ \text{lastnost homomorfizma},\ \\
		\det (A S A^{-1}) &= \det(A) \det(S) \big(\det (A)\big)^{-1} = \det S = 1, \\
		&\then A S A^{-1} \in SL_n (\F),\ \forall A \in GL_n (\F), \\
		&\then A \big(SL_n (\F)\big) A^{-1} = SL_n (\F),
	\end{align*}
	Kar pa pomeni $SL_n (\F) \lhd GL_n (\F)$. $\blacksquare$

	\ni Kaj pa je $\im (\det)$? Determinanta je surjektivna, saj za poljuben $a \in \F^*$ velja
	\[
		\det\begin{bmatrix}
			1 &   &        &   &   &   &        &  \\
			  & 1 &        &   &   &   &        &  \\
			  &   & \ddots &   &   &   &        &  \\
			  &   &        & 1 &   &   &        &  \\
			  &   &        &   & a &   &        &  \\
			  &   &        &   &   & 1 &        &  \\
			  &   &        &   &   &   & \ddots &  \\
			  &   &        &   &   &   &        & 1
		\end{bmatrix} = a.
	\]
	Potem to pomeni
	\[
		\overline{\det}: \quot{GL_n(\F)}{SL_n(\F)} \stackrel{\cong}{\longrightarrow} \F^*,
	\]
	torej je kvocientna grupa $\quot{GL_n(\F)}{SL_n(\F)} \cong \F^*$, kar pa pomeni, da je kvocientna
	grupa kar $\F^*$. 
\end{zgled}

\begin{trditev}
	\paragraph{Drugi in tretji izrek o izomorfizmu:}
	Naj bo $G$ grupa in $H, K \leq G$. Potem velja:
	\begin{itemize}
		\item[{\bf (2)}]{\v Ce je $K \lhd G$, potem je $H \cap K \lhd H$ in $\quot{H}{K\cap H} \cong \quot{HK}{K}$}
		\item[{\bf (3)}]{\v Ce je $K \leq H$ in sta obe edinki v $G$, potem je
			\begin{align*}
				\quot{H}{K} &\lhd \quot{G}{K} \\
				&\text{in} \\
				\Quot{G}{K}{H}{K} &\cong \quot{G}{H}
			\end{align*}}
	\end{itemize}
\end{trditev}

\paragraph{Dokaz:}
\begin{itemize}
	\item[{\bf (3)}]{Naj bo $f : \quot{G}{K} \to \quot{G}{H}$ homomorfizem s predpisom $f (aK) = aH$. Relacija $f$ je res homomorfizem, saj je
		\[
			f (aK \cdot bK) = f (ab K) = ab H = aH \cdot bH = f (aK) f(bK).
		\]
		
		$f$ je o\v citno surjektiven, saj dobimo v sliki vse odseke, zato moramo izra\v cunati le $\ker f$, pa dobimo izomorfizem
		\[
			\f: \Qquot{G}{K}{\ker f} \to \quot{G}{H}.
		\]
			
		V $\ker (f)$ so vsi odseki $aK$, ki se s $f$ preslikajo v identi\v cni odsek $eH \equiv H$, v $\quot{G}{H}$,
		\[
			aK \stackrel{f}{\longmapsto} aH = H \iff a \in H.
		\]
		
		Torej so v jedru ravno tisti odseki, ki imajo predstavnike v $H$:
		\[
			\{aK\ |\ a\in H\} \equiv \quot{H}{K} \cong \ker f.
		\]
		\qed}
	\item[{\bf (2)}]{Naj bo $p : G \to \quot{G}{K}$ kvocientni homomorfizem in $q \equiv \quot{p}{H}$,
		\[
			q : H \to \quot{G}{K}.
		\]

		Trdimo, da je jedro $q$: $\ker q = H \cap K$; in slika $q$: $\im q = \quot{HK}{K}$. Velja
		\[
			q(a) = \text{identiteta} \iff p(a)= \text{identiteta} \iff a \in K.
		\] Vendar $a \in H$. Torej velja $a \in K\cap H \then K\cap H \lhd H$.

		Po prvem izreku o izomorfizmu sledi $q : \quot{H}{K\cap H} \stackrel{\cong}{\longrightarrow} \im q$. V sliki so vsi odseki oblike $aK$, $a \in H$. Kot mno\v zica
		elementov v $G$ to ustreza produktu mno\v zic $HK = \{ab\ |\ a\in H,\ b\in K\}$. Ta mno\v zica je podgrupa v $G$, ker je $K$ edinka.
		\[
			a_1b_1 \cdot a_2 b_2  = \underbrace{a_1 a_2}_{\in H} \underbrace{(\underbrace{a_2^{-1} b_1 a_2}_{\in K\ \text{(edinka)}}) b_2}_{\in K} \in HK.
		\] Torej so $\im q$ odseki $HK$ po $K$, to je $\quot{HK}{K}$.
		\qed}
\end{itemize}

\subsection{Komutatorska podgrupa}

Grupa je \emph{abelova}, \v ce produkt komutira: $ab = ba$, $\forall a,b \in G$.
\begin{align*}
	ab &= ba \quad / \cdot a^{-1} \\
	aba^{-1} &= b \quad / \cdot b^{-1} \\
	aba^{-1}b^{-1} &= e \ \stackrel{(!)}{\longleftarrow}
\end{align*}

\begin{defin}
	Za $a,b \in G$ je $[a,b] \equiv aba^{-1}b^{-1}$ \emph{komutator} elementov $a$ in $b$. \emph{Komutatorska podgrupa}, $[G,G]$, je najmanj\v sa podgrupa, ki vsebuje vse komutatorje
	v $G$.

	\paragraph{Opomba:} $[G,G]$ o\v citno vsebuje produkte komutatorjev in vsi elementi so take oblike: $e [a,a]$.
	\[
		[a,b]^{-1} = (aba^{-1}b^{-1})^{-1} = bab^{-1}a^{-1} = [b,a].
	\]
\end{defin}

\begin{trditev}
	Naj bo $G$ poljubna grupa. Velja:
	\begin{itemize}
		\item[(1)]{$[G,G] \lhd G$,}
		\item[(2)]{$\quot{G}{[G,G]}$ je abelova in jo imenujemo \emph{abelacija} ali \emph{abelianizacija} grupe $G$.}
		\item[(3)]{\v Ce je $H$ poljubna abelova grupa in $f : G \to H$ homomorfizem, potem $[G,G] \leq \ker f$}
	\end{itemize}
\end{trditev}

\paragraph{Dokaz:}
\begin{itemize}
	\item[{\bf (1)}]{Ker je vsak element iz $[G,G]$ produkt komutatorjev za poljuben $a \in G$, tudi $c[a,b]c^{-1}$ nek komutator, saj za
		produkt
		\[
			[a_1, b_1] \cdot [a_2, b_2] \cdot \ldots \cdot [a_k, b_k]	
		\]velja
		\[
			c[a_1, b_1]\cdot\ldots\cdot[a_k,b_k]c^{-1} = c[a_1, b_1]c^{-1}c[a_2,b_2]c^{-1}\ldots c[a_k,b_k]c^{-1} 
		\]

		Izra\v cunajmo konjugacijo:
		\begin{align*}
			c[a,b]c^{-1} &= c ab a^{-1} b^{-1} c^{-1} = cac^{-1}c bc^{-1}c a^{-1}c^{-1}cb^{-1}c^{-1}\\
				&= (cac^{-1}) (cbc^{-1}) (cac^{-1})^{-1} (cbc^{-1})^{-1} = [cac^{-1}, cbc^{-1}] \in [G,G].\ \blacksquare
		\end{align*}}
	\item[{\bf (2)}]{Pokazati moramo, da odseki v $\quot{G}{[G,G]}$ komutirajo (tj. ta kvocientna grupa je abelova):
		\begin{align*}
			a[G,G] \cdot b[G,G] \stackrel{?}{=} b[G,G]\cdot a[G,G]
		\end{align*}
		Pokazati moramo, da $ab$ in $ba$ predstavljati isti odsek, to pa je $\iff (ba)^{-1}ab \in [G,G]$:
		\[
			(ba)^{-1} ab = a^{-1}b^{-1}ab = [a^{-1}, b^{-1}] \in [G,G].\ \blacksquare
		\]}
	\item[{\bf (3)}]{\v Ce je $f : G \to H$ homomorfizem v abelovo grupo, \v celimo videti, da je $[G,G] \leq \ker f$. Dovolj je, \v ce
		\[
			f\big([a,b]\big) = e,
		\]\v ce poka\v zemo za enega, to je
		\[
			f\big(aba^{-1}b^{-1}\big) = \overbrace{f(a)f(b)f(a^{-1})f(b^{-1})}^{\text{elementi v $H$ komutirajo}} =
				f(a) f(a^{-1}) f (b) f (b^{-1}) \stackrel{\text{homo.}}{=} f(a) f(a)^{-1} f(b) f(b)^{-1} = e
		\]\qed}
\end{itemize}

\ni \paragraph{Vsebina:} Komutatorska podgrupa je najmanj\v sa edinka v $G$, po kateri je kvocient abelov.

