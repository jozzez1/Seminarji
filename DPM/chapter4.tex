\chapter{Osnovni pojmi}

\paragraph{Ideja:} Mnogoterost je metri\v cni prostor, ki lokalno (v okolici vsake to\v cke) izgleda kot nek evklidski prostor $\R^n$, $n$
imenujemo dimenzija mnogoterosti Mnogoterost je metri\v cni prostor, ki lokalno (v okolici vsake to\v cke) izgleda kot nek evklidski prostor $\R^n$, $n$
imenujemo dimenzija mnogoterosti..

\paragraph{Primeri:}
\begin{itemize}
	\item{Mnogoterost (mnt) 1-D:
		\begin{itemize}
			\item{$S^1$ (odprt interval, kro\v znica),}
			\item{poljubna enostavna sklenjena krivulja,}
			\item{poljubna odprta krivulja (kraji\v s\v ci se ne stikata).}
		\end{itemize}}
	\item{2-D mnt:
		\begin{itemize}
			\item{2-sfera ($S^2$): okolica to\v cke je odprt disk, ki je v bistvu $\R^2$,}
			\item{odprt, ali neskon\v cen valj,}
			\item{torus, tudi taki z ve\v c luknjami.}
		\end{itemize}}
\end{itemize}

\begin{defin}
	\begin{itemize}
		\item{$\R^n_+ = \big\{x = (x_1,\ldots,x_n) \in \R^n\ |\ x_n \geq 0\big\}$ je \emph{zgornji polprostor}.}
		\item{\emph{Ekvivalenca} metri\v cnih prostorov in njihovih podmno\v zic je zvezna bijekcija, ki ima zvezen
			inverz.}
	\end{itemize}
\end{defin}

\begin{defin}
	$X$, $Y$ metri\v cna prostora. $f : X \to Y$ je \emph{homeomorfizem}, \v ce je $f$ zvezna in bijektivna in je tudi njen
	inverz, $f^{-1}$ zvezna.
\end{defin}

\paragraph{Primer:}
$f : [0,2\pi) \to S^1$, $t \mapsto e^{it} = \cos t  i\sin t$ je zvezna in bijektivna, vendar pa $f^{-1}$ ni zvezna v to\v cki 1:
to\v cki $e^{it}, t > 0$ se slikajo blizu $0$, ki je $f^{-1}(1)$, to\v cke $e^{it}, t < 0$ pa blizu $2\pi$, ki NI $f^{-1}(1)$.
\pagebreak
\begin{defin}
	$n$-mnt ($n$-mnogoterost) je metri\v cni prostor $M$, v katerem\ $\forall x \in M$ obstaja odprta okolica $V$ za $x$ in
	homeomorfizem $\vfi : V \to V'$, kjer je $V'$ odprta v $\R^n_+$.
\end{defin}

\paragraph{Terminologija:}
\begin{itemize}
	\item{V metri\v cnem prostoru $X$ je mno\v zica $V$ odprta, \v ce je unija odprtih krogel.}
	\item{Metrika je $d : X \times X \to \R$ in je nenegativna.}
	\item{Odprta krogla v metriki $d$ okoli to\v cke $x$ s polmerom $r$ je $K_d(x,r) = \{y \in X\ |\ d(x,y) < r\}$. Posebej v
		evklidski metriki, je odprta krogla res odprta krogla (dejansko tako izgleda).}
	\item{Odprte krogle v $\R^n_+$ z evklidsko metriko so odprte krogle v $\R^n$, presekane z $\R^n_+$ (tj. so odprte vsepovsod,
		vendar imajo lahko zravnan spodnji rob, ker je $x_n = 0$ v $\R^n$, se pravi, \v ce so preblizu roba $\R^n_+$).}
	\item{Odprta okolica za $x$ je odprta mno\v zica (unija odprtih krogel), ki vsebuje $x$.}	
\end{itemize}

\begin{defin}
	$M$ mnogoterost, $x \in M$ to\v cka in $V$ odprta okolica to\v cke $x$.
	\begin{itemize}
		\item{Homeomorfizem $\vfi : V \to V'$, $V' \odp \R^n_+$ imenujemo \underline{\emph{karta}} na $M$ v $x$ (iz krivega prostora $M$
			slikamo v evklidski (raven) prostor $\R^n_+$).}
		\item{Ozna\v cimo s $\pr_i$ projekcijo na $i$-to koordinatno os. Karta porodi koordinatne funkcije na $M$ v okolici $x$:
			\[
				\pr_i \circ \vfi : V \to V' \subseteq \R^n \to \R.
			\]
			Ker je $\vfi$ homeomorfizem, je njegov inverz zvezen in tega imenujemo \underline{\emph{lokalna parametrizacija}}
			$M$ v okolici $x$.}
		\item{To\v cke, ki so slike to\v ck z roba $\R^n_+$, tj. $\R^{n-1} \times \{0\}$ imenujemo \underline{\emph{robne to\v cke}}
			mnogoterosti $M$, ostale pa \underline{\emph{notranje}}.}
	\end{itemize}
\end{defin}

\paragraph{Primer:}
Zaprta krivulja brez prese\v ci\v s\v c, ki je slika nekega kompaktnega interala je 1-mnt, katere rob sta kon\v cni to\v cki, ostale so
notranje.

Na\v s cilj so gladke mnogoterosti, razreda $C^k, k \in \mathbb{N}$. Naj bo $V \odp \R^n$ in $f : V \to \R^m$.
\[
	f = (f_1, \ldots, f_m); \quad f_i : \R^n \to \R,
\]
\begin{center}
se pravi
\end{center}
\[
	(x_1,\ldots,x_n) \mapsto (f_1, \ldots, f_m).
\]
\begin{itemize}
	\item{$f$ je razreda $C^k$, \v ce je $k$-krat zvezno odveljiva na $V$, tj. \v ce obstajajo vsi parcialni odvodi reda $\leq k$, vseh
		komponent $f_i$ in so zvezni.
		\[
			\frac{\partial^\ell f_i}{\partial x_{i_1}\partial x_{i_2} \ldots \partial x_{i_\ell}}
		\]
		je parcialni odvod reda $\ell$, po spremenljivkah $x_{i_1} \ldots x_{i_\ell}$.}
	\item{Pogoj zvezne parcialne odvedljivosti zagotavlja, da vrstni red odvajanja ni pomemben.}
	\item{Na parcialne odvode gledamo kot na operatorje
		\begin{align*}
			\frac{\partial}{\partial x_j} : C^k &\to C^{k-1} \\
			f &\mapsto \frac{\partial f}{\partial x_j},
		\end{align*}
		po zgornji opombi ti operatorji komutirajo.}
\end{itemize}

Pojem ekvivalence, kot ga glede na zveznost dolo\@ ca homeomorfizem, raz\v sirimo v gladko kategorijo. \v Se vedno zahtevamo bijektivno preslikavo,
ki je razreda $C^k$ in tudi inverz je tak.

\begin{defin}
	Naj bosta $U$, $V \odp \R^n$. Potem je $f : U \to V$ \emph{difeomorfizem} razreda $C^k$, \v ce je bijektiven in $f, f^{-1} \in C^k$
	($k \in \mathbb{N}$).
\end{defin}

\paragraph{Opomba:} Difeomorfizem razreda $C^k$ je torej homeomorfizem, ki je v obe smeri \v se $k$-krat odvedljiv.\\[6pt]

Pogoj za $C^k$-difeomorfizem je te\v zko preveriti, saj obi\v cajno ne znamo izra\v cunati $f^{-1}$. Zato te\v zko direktno preverjamo $f^{-1} \in C^k$.
Pri tem nam pomaga izrek o inverzni preslikavi.\\[6pt]

Naj bo $f : U \to V$ razreda $C^k$, $U \odp \R^n$, $V \odp \R^m$. Odvod preslikave $f$ je dan z Jacobijevo matriko prvih parcialnih odvodov,
\[
	Jf(x) = \begin{bmatrix}
		\prt[1]{f_1} & \prt[2]{f_1} & \ldots & \prt[n]{f_1} \\
		\prt[1]{f_2} & \prt[2]{f_2} & \ldots & \prt[n]{f_2} \\
		\vdots       & \vdots       &        & \vdots       \\
		\prt[1]{f_m} & \prt[2]{f_m} & \ldots & \prt[n]{f_m}
	\end{bmatrix}
\] (slikamo iz $\R^n \to \R^m$, dimenzija matrike je $m \times n$, odvodi so izra\v cunani v to\v cki $x$). Matrika $Jf(x)$ predstavlja linearno
preslikavo iz $\R^n \to \R^m$, ki jo imenujemo \underline{\emph{odvod}} preslikave $f$ in ozna\v cimo
\[
	Df(x) : \R^n \to \R^m
\]
(to je tenzorsko polje: za vsak $x$ imamo linearno preslikavo iz $\R^n \to \R^m$). Skupaj to predstavlja
\begin{align*}
	Df : U &\to L\big(\R^n, \R^m\big) \\
	x &\mapsto Df(x)
\end{align*}

\begin{trditev}
\paragraph{Izrek o inverzni preslikavi:}[{\em Brez dokaza.}] Naj bo $U \odp \R^n$ in $f : U \to \R^n$ razreda $C^k$, $a \in U$. \v Ce je $Df(a)$
{\em linearni izomorfizem} (tj. $|Jf(a)| \neq 0$), potem je $f$ lokalni difemorfizem iz $C^k$ v okolici to\v cke $a$).\\[6pt]

\ni Z drugimi besedami $\exists$ odprta okolica $V \subset U$ za $a$, da je
\[
	f\big|_V : V \to f(V)
\]
difeomorfizem razreda $C^k$.

Posebej je odvod inverzne preslikave enak
\[
	D\big(f^{-1}\big) = \Big(Df\big(f^{-1}(y)\big)\Big)^{-1},
\]
kjer
\begin{align*}
	f(a) &= y \in f(U) \\
	D\big(f^{-1})\big(f(a)\big) &= \Big(Df(a)\Big)^{-1}.
\end{align*}
\end{trditev}

\paragraph{Opomba:} Iz formule za odvod sledi, da je $f^{-1} \in C^k$: Jakobijeva matria za $f$ vsebuje prve parcialne odvode $f$, ki so v $C^{k-1}$,
zato je tudi $(Jf)^{-1}$ razreda $C^{k-1}$, \v ce je le definiran; gotovo pa je definiran v neki okolici to\v cke $a$ saj je zaradi zveznosti
determinante in $Jf$, $\det\big(Jf(x)\big)$ zvezna in zato $\neq 0$ blizu $a$ $\then$ blizu $a$ ima inverz.

\begin{posledica}
	\v Ce je $f : U \to V$ homeomorfizem razreda $C^k$, potem je $f$ $C^k$ difeomorfizem $\iff$ odvod $Df(x)$ je linearen izomorfizem
	$\forall x \in U$. (pogoj o linearnem izomorfizmu nam zagotovi obrnljivost matrike na celotnem definicijskem obmo\v cju).
\end{posledica}

\paragraph{Cilj:} Definirati pojem abstraktne gladke mnogoterosti (gld mnt) razreda $C^k$.

\begin{defin}
	$C^k$ \emph{podmnogoterost} (podmnt) dimenzije $n$ v $\R^n$ je podmno\v zica $M \subseteq \R^m$, za katero velja $\forall x \in M$
	$\exists$ odp. okolica $W$ za $x \in \R^m$ in $C^k$-difeomorfizem $h : W \to W'$, $W' \odp \R^m$, da velja
	\[
		h (W \cap M) \subseteq R^n \times \{0^{m-n}\} \subseteq \R^m.
	\]
	(slike ni).
\end{defin}
\paragraph{Opomba:}
\begin{itemize}
	\item{$C^k$-podmnt dimenzije $n$ je o\v citno $n$-mnt,}
	\item{je metri\v cni prostor z metriko v $\R^m$}
	\item{karte so zo\v zitve $h\big|_{W \cap M} : W \cap M \to W' \cap \big(\R^n \times \big\{0^{m - n}\big\}\big) \subseteq \R^n$.}
\end{itemize}

\paragraph{Primeri:}
$C^k$ podmnt dobima na dva standardna na\v cina:
\begin{enumerate}
	\item{\underline{Graf $C^k$ preslikave:}
	$V \odp \R^n$ in $f : V \to \R^\ell$ razreda $C^k$.
	\[
		\Gamma f = \big\{\big(x, f(x)\big)\ |\ x \in V\big\} \subseteq V \times \R^\ell \subseteq \R^{n + \ell},
	\]
	(graf preslikave $f$) je  $C^k$-podmnt v $\R^{n + \ell}$ dimenzije $n$. Preslikava $f \in C^k$, zato je tudi
	$h \in C^k$. Vzamemo $W = V \times \R^\ell = W'$
	\[
		\left.\begin{array}{rl}
		h : W &\to W' \\
		(x,y) &\mapsto (x, y - f(x))
		\end{array} \right\}\ \ \text{graf preslika na $V \times \{0^\ell\}$.}
	\]
	Napisati znamo tudi inverz:
	\begin{align*}
		h^{-1} : W' &\to W, \\
		(x,y) &\mapsto (x, y + f(x)),
	\end{align*}
	$h$ je torej $C^k$-difeo, ki $\Gamma f$ zravna na $W' \cap \R^n = V$ (ni\v cle pozabimo).
	Projekcija $(x, y) \mapsto x$ je globalna karta za $n$-mnt, njen inverz pa je globalna parametrizacija $x \mapsto (x, f(x))$.}
	\item{\underline{Nivojska mno\v zica gladke preslikave:} Naj bo $g : \R^m \to \R^\ell$ razreda $C^k$. Izberimo $a \in \R^\ell$; zanima nas,
	pod kak\v snimi pogoji je $g^{-1} (a) = \{x \in \R^m\ |\ g(x) = a\}$ $C^k$-podmnt v $\R^m$ in kolik\v sna je dimenzija te podmnt. Odgovor
	da izrek o implicitni preslikavi.}
\end{enumerate}

\begin{trditev}
	\paragraph{Izrek o implicitni preslikavi:} [\emph{Brez dokaza.}] Naj bo
	\[
		g : \underbrace{\R^{n + \ell}}_{\R^n \times \R^\ell} \to \R^\ell
	\]
	razreda $C^k$, $k \in \mathbb{N}$, naj bo $a \in \R^\ell$ in in $(b, c) \in \R^n \times \R^\ell$, $g (b, c) = a$, tj. $(b, c) \in g^{-1} (a)$.
	\v Ce ozna\v cimo koordinate v $\R^n \times \R^\ell$ kot $(x,y) = (x_1, \ldots, x_n, y_1, \ldots, y_\ell)$ in velja
	\[
		J_y g(b ,c) = \begin{bmatrix}
			\party{g_1}{y_1}(b,c)    & \party{g_1}{y_2}(b,c)    & \ldots & \party{g_1}{y_\ell}(b,c)    \\
			\party{g_2}{y_1}(b,c)    & \party{g_2}{y_2}(b,c)    & \ldots & \party{g_2}{y_\ell}(b,c)    \\
			\vdots                   & \vdots                   &        & \vdots                      \\
			\party{g_\ell}{y_1}(b,c) & \party{g_\ell}{y_2}(b,c) & \ldots & \party{g_\ell}{y_\ell}(b,c)
		\end{bmatrix}
	\]
	ni singularna ($\det J_y g(b,c) \neq 0$), potem obstajata odprti okolici $V$ za $b$ v $\R^n$ in $W$ za $c$ v $\R^\ell$ ter
	preslikava $h : V \to W$ razreda $C^k$, za katero velja:
	\begin{itemize}
		\item{$h(b) = c$,}
		\item{za $(x,y) \in V \times W$ velja $g(x,y) = a \iff y = h(x)$.}
	\end{itemize}
\end{trditev}

\paragraph{Opomba:} Zgornje pomeni, da je v okolici to\v cke $(b,c)$ nivojska mno\v zica $g^{-1}(a)$ enaka grafu preslikave $h$.

\begin{posledica}
	Naj bo $g : \R^m \to \R^\ell$ $C^k$ preslikava ($m = n + \ell > \ell$), $a \in \R^\ell$. \v Ce je v vsaki to\v cki $x \in g^{-1}(a)$ Jacobijeva
	matrika $Jg(x)$ ranga $\ell$, potem je nivojska mno\v zica $g^{-1}(a)$ $C^k$-podmnt v $\R^m$ dimenzije $m - \ell$ (ker je v vaski to\v cki gra
	neke $C^k$ preslikave; spremenimo koordinatni sistem).
\end{posledica}

\paragraph{Dokaz:} Izberimo to\v cko $d \in g^{-1} (a)$. Po predpostavki  je $Jg(a)$ matrika ranga $\ell$. Velikost matrike je
\[
	\dim Jg(a) = \dim
	\begin{bmatrix}
		\prt[1]{g_1}    & \ldots & \prt[m]{g_1}   \\
		\vdots          &        & \vdots         \\
		\prt[1]{g_\ell} & \ldots & \prt[m]{g_\ell}
	\end{bmatrix} = \ell \times m.
\]
Rang $\ell$ pomeni, da matrika vsebuje $\ell$ linerno neodvisnih stolpcev, tiste stolpce lahko zberemo na koncu (na zadnjih $\ell$ mestih),
tako da preimenujemo koordinate. Sedaj $\ell \times \ell$ podmatrika (iz zadnjih $\ell$ stoplcev) v $Jg(a)$ izpolnjuje pogoje izreka $\then$ v okolici
to\v cke $d$ je $g^{-1}(a)$ graf neke preslikave $\R^{m - \ell} \to \R^\ell$, torej $C^k$-podmnt dimenzije $m - \ell$ v $\R^m$. $\blacksquare$


\begin{defin}
	\begin{itemize}
		\item{$g : \R^m \to \R^\ell$ razreda $C^k$; to\v cka $x \in \R^m$ je \emph{regularna}, \v ce je $Dg(x)$
			surjektivna preslikava ($m > \ell$, preslikava je surjektivna $\iff$ $\rang$ maskimalen) tj. $\rang Jg(x) = \ell$.}
		\item{To\v cka $a \in \R^\ell$ je \emph{regularna vrednost} za $g$, \v ce je vsak $x \in g^{-1}(a)$ regularna
			to\v cka.}
	\end{itemize}
\end{defin}

\paragraph{Opomba:} praslika regularne vrednosti je $x$; $x \in g^{-1}(a)$. V tem jeziku prej\v snja posledica pravi, da je praslika (tj. vsi $x$)
regularne vrednosti (to je $a$) $C^k$ preslikave $g : \R^m \to \R^\ell$ $C^k$-podmnt v $\R^m$ dimenzije $m - \ell$ (\v ce je neprazna).\\[6pt]

Sedaj \v zelimo definirati abstraktno gladko mnogoterost, ki ni vlo\v zena v evklidski prostor. \v Ce je $M$ mnt dim. $n$, ima vsak $x \in M$ odprto
okolico, ki je homeomorfna odprti mno\v zici $\R^n$.

\paragraph{Ideja:}
Radi bi, da bi bila karta $C^k$ preslikava. Ampak na $M$ nimamo koordinat, niti ne znamo odvajati. TO torej ne gre, lahko pa primerjamo
razli\v cne karte: Imamo npr. to\v cko $x \in M$ z odprto okolico $V$ in to\v cko $y \in M$ ($x \neq y$) z okolico $W$, tako da $V \cap
W \neq \{\}$. Ti okolici opremimo s kartama
\begin{align*}
	\vfi : V &\to V',\\
	\psi : W &\to W',
\end{align*}
presek $V \cap W$ se vidi v $V'$ kot $\vfi(V \cap W)$ in v $W'$ kot $\psi(V \cap W)$. Med tema dvema imamo neko preslikavo. Pravimo, da karti $\vfi$ in
$\psi$ dolo\v cata \underline{\emph{prehodno preslikavo}}
\[
	\psi \circ \vfi^{-1}\big|_{\vfi(V \cap W)} : \underbrace{\vfi(V \cap W)}_{\subseteq \R^n} \to \underbrace{\psi(V \cap W)}_{\subseteq \R^n}.
\]
Pogoj odvedljivosti na mnt "`zakodiramo"' v odvedljiost prehodne preslikave.

\begin{defin}
	\paragraph{Gladka struktura na mnogoterosti:}
	\begin{itemize}
		\item{Naj bo $M$ neka $n$-mnt, $k \in \mathbb{N}$, $\Lambda$ je mno\v zica indeskov. \emph{Atlas} razreda $C^k$ na $M$
			je dru\v zina kart na $M$, $\{(v_\lambda, \vfi_\lambda)\ |\ \lambda \in \Lambda\}$, kjer so $V_\lambda \odp M$,
			\[
				\bigcup_{\lambda \in \Lambda} V_\lambda = M,
			\]
			in $\vfi_\lambda : \stackrel{\text{homeo}}{\longrightarrow} V'_\lambda \subseteq \R^n$ je karta na $M$, poleg tega pa za
			poljubna $\lambda, \mu \in \Lambda$ velja: \v ce je $V_\lambda \cap V_\mu \neq \{\}$, potem je prehodna preslikava
			\[
				\vfi_\mu \circ \vfi^{-1}_\lambda : \vfi_\lambda(V_\lambda \cap V_\mu) \to \vfi_\mu (V_\lambda \cap V_\mu)
			\]
			$C^k$-difeo. To ozna\v cimo tudi kot
			\[
				\left.
				\begin{array}{rl}
					V_\lambda \cap V_\mu &= V_{\lambda\mu} \\
					\vfi_\mu \circ \vfi^{-1}_\mu &=  \vfi_{\mu\lambda}
				\end{array}
				\right\} \vfi_{\mu\lambda} : \vfi_\lambda(V_{\lambda\mu}) \to \vfi_\mu (V_{\lambda\mu}).
			\]}
		\item{Dva $C^k$-atlasa sta (imate razli\v cne karte na mnogoterosti) \underline{\emph{kompatibilna}} ali \underline{\emph{ekvivalentna}},
			\v ce je tudi njuna unija $C^k$-atlas.}
		\item{\underline{\emph{$C^k$-strukura na $M$}} je ekvivalen\v cni razred $C^k$-atlasov na $M$ (je maksimalen $C^k$ atlas, torej
			unija vseh kompatibilnih $C^k$ atlasov na $M$).}
	\end{itemize}
\end{defin}

\paragraph{Opomba:} $C^k$-struktura (ali gladka struktura) je dolo\v cena s poljubnim atlasom iz ekvivalen\v cnega razreda in obi\v cajno navedemo
atlas s \v cim manj kartami. Tega pa lahko vedno dopolnimi do maksimalnega.

\begin{zgled}
	$M = \R^n$ (je $n$-mnt). Najmanj\v si mo\v zen atlas ima eno karto, ki je $h : M \to \R^n$, $h$ homeomorfizem.
	\begin{itemize}
		\item{Mno\v zica $\{h : \R^n \to \R^n\}$ je $C^k$-atlas:
			\begin{itemize}
				\item{$\R^n$ pokriva $M = \R^n$,}
				\item{kompatibilni pogo je izpolnjen, saj ni drugih kart.}
			\end{itemize}}
		\item{{\em standardna} $C^k$-struktura na $\R^n$ je dana s karto $\{\id : \R^n \to \R^n\}$.}
		\item{Vzemimo $n = 1$: na \R obstajajo nekompatibilni (neekvivalentni) $C^k$-atlasi zgornje oblike:
			\begin{itemize}
				\item{standardni: $\{\id : \R \to \R\}$}
				\item{$h_0 \R \to \R$,
					\[
						h_0 (x) =
						\left\{\begin{array}{rl}
							x,  & x < 0 \\
							2x, & x \geq 0
						\end{array}\right.
					\]
					ni kompatibilen z identiteto, ker $h_0$ ni odvedljiv v to\v cki $0$. Prehodna preslikava je
					\[
						h_0 \circ \id^{-1} = h_0 \notin C^k
					\]}
				\item{$h_p : \R \to \R$, $h(x) = x^{2p + 1}$, $p \in \mathbb{N}$. V tem primeru je $h_p$ odvedljiv, inverz pa ne:
					$h_{p_1} \circ h_{p_2}$ ni razreda $C^k$. Tudi ti atlasi dajo same paroma nekompatibilne $C^k$ strukture (ker
					je $h_p'(0) = 0\ \then\ h^{-1}_p$ ni odvedljiv) $\then$ ni kompatibilen s standardno strukturo. Podobno sklepamo
					za ostale.}
			\end{itemize}}
	\end{itemize}
\end{zgled}
\begin{zgled}
	Vsaka odprta podmno\v zica gldadke mnogoterosti je gladka mnogoterost (gladka mnogoterost v tem primeru pomeni, da zanjo obstaja atlas).
	Karte lahko presekamo/zo\v zimo s to podmno\v zico, kar ne vpliva na gladkost prehodnih preslikav.\\[6pt]
	
	Npr. $GL_n(\R)$ -- obrnjljive matrike
	dimenzije $n \times n$ so gladka mnogoterost dimenzije $n^2$. $GL_n(\R)$ je odprta podmno\v zica v $\R^{n \times n} = \R^{n^2}$, $GL_n(\R)$ je
	tudi grupa in grupni operaciji (mno\v zenje in invertiranje) sta gladki preslikavi: to je primer \emph{Liejeve grupe}.
\end{zgled}
\pagebreak
\begin{zgled}
	Sfere $S^n = \{x \in \R^{n+1}\ |\ \|x\| = 1\}$ so gladke mnt, $\dim S^n = n$.
	\begin{itemize}
		\item{$S^n \subset \R^{n+1}$ je gladka podmnt, $f : \R^{n+1} \to \R$, $f(x) = \|x\|^2$, tedaj $S^n = f^{-1}(1)$. Preveriti moramo,
			da je $Jf$ ranga 1 v vsaki to\v cki $Jf(x) = \nabla f(x) = 2x$ ima rang 1, ko $x \neq 0$. Amapak na sferi je to vedno res.
			Dimenzija je $n + 1 - 1 = n$.}
		\item{En atlas na $S^n$ dobimo tako, da projekcije zo\v zimo na hemisfere. Domene kart bodo odprte hemisfere $S^n_{i,\pm} =
			\big\{x = (x_1,\ldots,x_{n+1}) \in S^n\ \big|\ x_i \gtrless 0\big\}$. Karta
			\[
				\vfi_{i,\pm} : S^n_{i,\pm} \to \B{}^n = \Big\{x = (x_1, \ldots, x_n),\ \sum_{j = 1}^n x^2_j < 1\Big\},
			\]
			slika v $\B{}^n$, ki je odprta krogla okrog izhodi\v s\v ca v $\R^n$. 
			\[
				\vfi_{i,\pm} (x_1, \ldots, x_{n+1}) = (x_1, \ldots, x_{i-1}, x_i, x_{i+1},\ldots,x_{n+1}).
			\]
			Ker je $\sum_j x_j^2 = 1$, je $\sum_{j \neq i} x_j^2 < 1$ $\then$ to je v $\B{}^n$. Da preverimo
			gladko kompatibilnost tega atlasa, moramo izra\v cunati prehodne preslikave.
			\begin{itemize}
				\item{inverzi kart:
					\begin{align*}
						\vfi^{-1}_{i, \pm} : \B{}^n & \to S^n_{i,\pm} \\
						(x_1,\ldots,x_n) &\mapsto \Big(x_1,\ldots,x_{i-1}, \pm\sqrt{1 - \textstyle{\sum_{j=1}^n x_j^2}}, x_i,
							\ldots, x_n\Big)
					\end{align*}}
				\item{prehodne preslikave: $[\epsilon, \delta = \pm]$, $j < i$. Karta
					\[
						\vfi_{j,\delta} \circ \vfi^{-1}_{i,\epsilon} (x_1,\ldots,x_n) = \Big(x_1, \ldots, x_{i-1},
							\epsilon\sqrt{1 - \textstyle{\sum_{k = 1}^n x_k^2}}, x_i, \ldots, x_{j-1}, x_{j+1}, \ldots, x_n\Big)
					\] je gladka v $C^\infty$ (\v se ve\v c, realno analit\v cna je -- enaka Taylorjevi vrsti).}
				\item{Za $\R$ funkcije (tj. funkcije realne spremenljivke) $C^\infty$ ne implicira analaiti\v cnosti. Kot primer vzemimo
					\[
						f(x) = \left\{
						\begin{array}{rl}
							0, & x \leq 0 \\
							e^{-1/x}, & x > 0
						\end{array}\right. {}\ \then\ \text{vsi odvodi v 0 so 0},
					\]
					torej je Taylorjeva vrsta v 0 ni\v celna (ker $f^{(n)} (0) = \forall n$), kljub temu, da $f$ ni povsod $0$.}
			\end{itemize}}
		\item{Zgornji atlas za $S^n$ ima $2(n+1)$ kart. \v Se lep\v se je, \v ce najdemo manj\v si atlas. \v Samo dve karti potrebujemo,
			\v ce uporabimo stereografsko projekcijo. Za karto z domeno $S^n\backslash\{N\}$, $N = (0,0,\ldots,0,1)$ (severni
			pol) je preslikava dana s pravilom, ki to\v cki $p \in S^n\backslash\{N\}$ priredi prese\v ci\v s\v ce premice skozi $p$ in
			$N$ z ekvatorialno ravnino $\R^n \times \{0\} \subseteq \R^{n+1}$. To\v cko $x \in \R^{n+1}$ lahko pi\v semo $x = (y,t)$,
			$y \in \R^n$, $t \in \R$. I\v s\v cemo formalno formulo za $\vfi(y,t) \in \R^n$. Smerni vektor premice je $p$-$N$, njena
			ena\v cba je
			\[
				N + s(p - N),\ s \in \R,
			\]\[
				(0,1) + s \big((y,t) - (0,1)\big) = \big(sy, 1 + s(t+1)\big).
			\]
			Prese\v ci\v s\v ce z "`ravnino"' $\R^n \times \{0\}$ ustreza ni\v celni zadnji koordinati,
			\[
				1 + s(t-1) = 0, \qquad s = \frac{1}{1 - t},\qquad \then \vfi(y,t) = \frac{1}{1 - t} \cdot y.
			\]
			Podobno za ju\v zni pol, le da je tam `$1 + t$'.}
		\item{Za prehodno preslikavo potrebujemo inverz ($\lambda z$ zaradi podobnih trikotnikov obdr\v zi smer v dimenziji $\R^n$):
			\begin{align*}
				\vfi^{-1} : \R^n &\to S^n \backslash \{\mathbb{N}\} \\
				\vfi^{-1} (z) &= (\lambda z, t), \qquad \lambda \in \R
			\end{align*}
			\begin{itemize}
				\item{Mora biti na sferi: $t^2 + \lambda^2\|z\|^2 = 1$.}
				\item{Mora biti na premici:
					\begin{align*}
						(\lambda z, t) &= (0,1) + s\big((z,0) - (0,1)\big) \\
						&= \underbrace{(sz, 1 - s) = (\lambda z, t)}_{\lambda = s,\ t = 1 - \lambda}
					\end{align*}
					\begin{align*}
						1 - 2\lambda + \lambda^2 + \lambda^2\|z\|^2 &= 1\qquad \slash :\lambda \\
						\lambda = \frac{2}{1 + \|z\|^2}; &\quad t = \frac{\|z\|^2 - 1}{\|z\|^2 + 1},
					\end{align*}
					kar pa nam na koncu vrne inverzno preslikavo
					\[
						\vfi^{-1}(z) = \bigg(\frac{2z}{1 + \|z\|^2}, \frac{\|z\|^2 - 1}{\|z\|^2 + 1}\bigg)
					\]}
			\end{itemize}}
		\item{Dobljene karte in njihovi inverzi imajo za komponente realno-analiti\v cne funkcije, zato enako velja tudi za njihove
			kompozitume $\then$ prehodne preslikave so realno-analiti\v cne in zato $\in C^\infty$. \v Se ve\v c, atlasa dolo\v cena s
			projekcijami na hemisferah in s stereografskima projekcijama sta tudi realno-analiti\v cna, kompatibilna, torej dolo\v cata
			isto gladko strukturo na sferi $S^n$, to je standardna gladka struktura na $S^n$.}
	\end{itemize}
\end{zgled}

\begin{defin}
	Naj bo $M$ $n$-mnt razreda $C^k$ in $f : M \to \R$. Re\v cemo, da je $f$ $C^k$-\emph{funkcija}, \v ce je za vsako karto iz nekega
	atlasa $M$ preslikava $f \circ \vfi^{-1} : V' \to \R$ razreda $C^k$ (tj. ker smo na abstraktni mnogoterosti moramo za\v ceti s karto).
\end{defin}

\begin{defin}
	Naj bosta $M$ in $N$ $C^k$-mnt dimenzije $m$ oz. $n$. Preslikava $f : M \to N$ je \emph{razreda} $C^k$, \v ce je $\forall$ karto
	$(\psi : W \to W')$ iz maksimalnega atlasa za $N$ in $\forall$ $(\vfi : V \to V')$ iz maks. atlasa za $M$ za katero velja $f(V)
	\subseteq W$ preslikava
	\[
		\psi \circ f \circ \vfi^{-1} : V' \to W'
	\]
	razreda $C^k$ med evklidskima prostoroma $\R^n$ in $\R^m$ (to mora veljati $\forall V \subseteq M$ in $\forall W \subseteq N$).
\end{defin}

\begin{defin}
	Homeomorfizem $f : M \to N$ je $C^k$-\emph{difeomorfizem}, \v ce sta $f$ in $f^{-1}$ razreda $C^k$.
\end{defin}

\paragraph{Opomba:} Pogoj za gladkost funkcije smo napisali glede na nek atlas (ne nujno maksimalni), saj za poljubni $C^k$ kompatibilni karti
$\vfi : V \to V'$ in $\psi : W \to W'$ na $M$ velja: 
\begin{align*}
	f \circ \psi^{-1} &= (f \circ \psi^{-1}) \circ (\vfi \circ \psi^{-1}) \\
	f \circ \vfi^{-1} &= (f \circ \psi^{-1}) \circ (\psi \circ \vfi^{-1}) \\
	\text{je $\in \C^k$} &\iff \text{je $\in C^k$}
\end{align*}

\paragraph{Opomba:}
$C^k$-odvedljivost je dobro definirana na odprtih podmno\v zicah v $\R^n$. \v Ce gledamo orp. podmno\v zice v polprostoru $\R^n_+$ ali bolj
splo\v sno mno\v zice, ki vsebujejo robne to\v cke, pa odedljivost v mejnih to\v ckah definiramo tako, da zahtevamo, da je preslikavo
mogo\v ce raz\v siriti do $C^k$-preslikave na okolici tak\v sne to\v cke.

\begin{zgled}
	Poka\v zi, da so vse gladke strukture (tj. vsi razli\v cni $C^k$-kompatibilni atlasi) v $\R^n$ dolo\v cene z eno karto
	$\{h : \R^n \to \R^n\}$, $h$ homeomorfizem, med sabo difeomorfne (med njimi obstajajo $C^k$-difeomorfizmi) $\iff$ $\exists$ difeomorfizem
	med standardno strukturo na $\R^n$ in zgornjo.

	\paragraph{Re\v sitev:}
	Slike ni, zato bom opisal: imamo dve mnogoterosti: $M$, $N = \R^n$. $M$ ima karto $h$ na $\R^n$, $N$ pa $\id$. Katera preslikava slika
	Med $M$ in $N$? Ker sta oba $\R^n$ lahko uporabimo spet kar $h$. Prehodna preslikava med slikami kart pa je potem $\id \circ h \circ
	h^{-1} = \id$. $\id : \R^n \to \R^n$ je homeomorfizem (njegov inverz prav tako), to je tisto.
	\qed

	\paragraph{Opomba:}
	\begin{itemize}
		\item{Na $\R^n$ obstajata do difeomorfizma natan\v cno ena gladka struktura za $n \neq 4$ (lahko z ve\v c kartami
			zgoraj z 1 karto)}
		\item{Na $\R^4$ obstajata ne\v stevno nedifeomorfnih gladkih struktur. Dokaz se uporablja v Yang-Millsovi teoriji polja
			(kvantna teorija polja -- masivna vektorska polja, kot npr QCD).}
		\item{Na $S^n$ je najve\v c kon\v cno mnogo gladkih struktur, ki niso difeomorfne standardni. Na $S^4$ ni znano koliko jih
			je (domneva se, da je 1). Prva sfera z znanimi ve\v cimi nedifeomorfnimi gladkimi strukturami je $S^7$, ki jih ima
			28.}
		\item{Za $n \geq 4$ na mnt dimenzije $n$ ne obstaja nujno gladka struktura!}
	\end{itemize}
\end{zgled}
