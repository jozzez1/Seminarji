\documentclass[a4paper, 12pt]{article}
\usepackage[slovene]{babel}
\usepackage[utf8]{inputenc}
\usepackage[T1]{fontenc}
\usepackage[small, width=0.8\textwidth]{caption}
\usepackage{amssymb, amsmath, fullpage, float, graphicx, pdflscape, hyperref}

\renewcommand{\d}{
	\ensuremath{\mathrm{d}}
}

\newenvironment{teorem}
{
	\vspace{6pt}
	\noindent [\emph{Trditev}]
}
{
	
}

\newenvironment{dokaz}
{
	\vspace{6pt}
	\noindent [\emph{Dokaz}]
}
{
	\begin{flushright}
		$\blacksquare$
	\end{flushright}
}

\newcommand{\e}{
	\ensuremath{\varepsilon}
}

\begin{document}

\begin{center}
\textsc{Dinami\v cna analiza}\\
\textsc{2013/14}\\[0.5cm]
\textbf{Zaklju\v cna naloga}
\end{center}
\begin{flushright}
\textbf{Jo\v ze Zobec}
\end{flushright}

\section{Naloga}

Razi\v s\v ci dvo-frekven\v cni brcani rotator s \v casovno odvisno (brezdimenzijsko) hamiltonko
\begin{equation}
	H(t) = \frac{p^2}{2} + \sum_{m = -\infty}^\infty \big\{k\delta(t - m) + k'\delta(t - m\tau - \theta)\big\}\cos\phi.
	\label{hamiltonian}
\end{equation}
\begin{itemize}
	\item{Zanimiva je predvsem dinamika za inkomenzurabilno razmerje brcanih frekvenc (t.j. $\tau$ iracionalen,
		npr. $\tau = (\sqrt{5} - 1)/2)$).}
	\item{Oglej si pojemanje korelacij, difuzijsko konstanto $\langle(p_t - p_0)^2\rangle/(2t)$, $t \to \infty$, ter
		Ljapunov eksponent v odvisnosti od konstant $k$, $k'$ in nekaj tipi\v cnih vrednosti razmerja frekvenc
		$\tau$.}
	\item{Ali ima fazni zamik $\theta$ kak vpliv?}
	\item{Ali obstaja kaka zanimiva predstavitev dinamike v faznem prostoru (pozor: koncept obi\v cajne stroboskopske
		preslikave tu odpove)?}
\end{itemize}

\section{Razmislek}

\subsection{Tipi\v cna izbira parametrov}

Hamiltonian~\eqref{hamiltonian} ima \v casovno periodi\v cen potencial. To s sabo prinese nekatere lepe lastnosti pri
dolo\v canju parametrov. Velja, da lahko brez zgube splo\v snosti delamo zgolj s $\tau \geq 0$ in $\theta \in [0, \tau)$.
Prav tako lahko za\v cetni \v cas (\v cas merjenja) postavimo na $t_0 = 0$ pri \v cemer lahko hamiltonian~\eqref{hamiltonian}
prepi\v semo v

\begin{equation}
	H(x,p,t) = \frac{p^2}{2} + \cos(x)\sum_{m = 0}^{\infty}\Big[k\delta(t - m) + k'\delta(t - m\tau - \theta)\Big].
\end{equation}
V primeru $\tau = 1$ in $\theta = 0$ imamo eno-frekve\v cni brcan rotator z jakostjo $K = k + k'$. Z njim lahko preverimo,
\v ce se rezultati ujemajo s pri\v cakovanimi. Zanimivo bi bilo tudi preveriti kaj se zgodi, ko $\tau = 2$. Mogo\v ce
imamo efektivni $K_\text{ef} = (k + k')/2$ (tj. mogo\v ce to ustreza eno-frekve\v cnemu brcanemu rotatorju s
$K = K_\text{ef}$). Na koncu pa tudi preveriti kako je za $\tau = \sqrt{2}-1$ in pa $\tau = (\sqrt{5} - 1)/2$.

Ob tem je smiselno preveriti vse rezultate za razli\v cne $\theta$, npr. za \v stiri razli\v cne ekvidistan\v cne izbire
$\theta$ znotraj $[0,\tau)$ (ob tem pa preveriti predvsem, ali to vpliva na korelacije).

\subsection{Izra\v cun dinamike}

O\v citno bomo morali delati z diskretnimi \v casi. Radi bi nekaj takega, kot je bila prej\v snja stroboskopska preslikava.
Opremimo z indeksi $j$ zaporedne sunke, tj. tik ob nekem sunku $j$ imamo vektor $(x_j, p_j)^T$. \v Cas bomo disktretizirali
tako, da bomo vedno gledali dogodke tik pred sunki in $\Delta_j \equiv t_{j+1} - t_j$. S $K_j$ bomo ozna\v cili jakost
$j$-tega sunka, tj. $K_j \in \{k, k', k + k'\}$, odvisno od situacije. Potem se je gibalna koli\v cina do tik pred
sunkom $j+1$ (ob \v casu $t_{j+1}$) spremenila za
\begin{align*}
	\int_{t_j}^{t_{j+1}} \dot{p}(t)\ \d t  &= \lim_{\e \searrow 0} \int_{t_j}^{t_{j+1}} \d t\ K_j \sin x_j
		\delta (t - (t_j + \e)) \\
		&= K_j \sin x_j \underbrace{\int_{t_j}^{t_{j+1}} \d t\ \lim_{\e \searrow 0} \delta(t - (t_j+\e))}_{ = 1},
\end{align*}
saj nas nosi sunek, ki se je zgodil tik za \v casom $t_j$ (zaradi tega smo v Diracovo $\delta$-funkcijo dodali \v se majhen
popravek $\e > 0$ k \v casu sunka). Nova gibalna koli\v cina je potem
\begin{equation*}
	p_{j+1} = p_j + K_j \sin x_j,
\end{equation*}
Do kraja tik pred novim sunkom pa smo pripotovali \v ze z "`novo"' gibalno koli\v cino, tj. gibalno
koli\v cino po trku, ki se je zgodil tik za \v casom $t_j$, to je $p_{j+1}$:
\begin{equation*}
	x_{j+1} = x_j + \Delta_j p_{j+1}\quad (\mathrm{mod}\ 2\pi).
\end{equation*}
Na\v sa nova "`stroboskopska preslikava"' je potem
\begin{align}
	p_{j+1} &= p_j + K_j \sin x_j, \notag \\
	x_{j+1} &= x_j + \Delta_j p_{j+1} \quad (\mathrm{mod}\ 2\pi).
	\label{stroboskopska}
\end{align}
ki tudi mo\v cno spominja na stroboskopsko preslikavo iz sktripte. Ta je dobra ne le za dvo-frekven\v cni
brcan rotator, ampak za poljubno \v stevilo frekvenc. Vse kar nam preostane je to, da poi\v s\v cemo $\Delta_j$ (\v casovne
intervale med zaporednimi sunki) in njim pripadajo\v ce $K_j$, torej s katero "`nogo"' brcnemo na\v s rotator, oboje pa lahko
izra\v cunamo vnaprej. Ena\v cba~\eqref{stroboskopska} bo na\v sa osnovna ena\v cba za modeliranje.

\section{Rezultati}

\subsection{Fazni portreti}

Dobro je, da si najprej ogledamo obna\v sanje v faznih portretov in mogo\v ce \v ze iz slik lahko razberemo kak\v sen
empiri\v cni zakon. To bomo naredili s postopkom:
\begin{itemize}
	\item{Dolo\v cimo $\tau$, $\theta$ in $t_\text{max}$.}
	\item{Med $0$ in $t_\text{max}$ poi\v s\v cemo vse $\Delta_j$ in $K_j$.}
	\item{Po ena\v cbi~\eqref{stroboskopska} izra\v cunamo vse $x_j$ in $p_j$.}
\end{itemize}
Vse grafe sem ra\v cunal s $t_\text{max} = 1000$ in z izbiro $G = 1000$ razli\v cnih za\v cetnih pogojih.
Za $\tau = 1$ in $k = 0.3$ ter $k' = 0.6$ dobimo res eno-frekven\v cni oscilator z jakostjo branja $K = k + k'$, kot vidimo
na sliki~\eqref{dokaz}.
\begin{figure}[H]\centering
	\includegraphics[width=10cm, keepaspectratio=1]{t100-q000-k030-K060-res}
	\caption{Res, ta fazni portret ustreza eno-frekven\v cnemu brcanemu rotatorju z jakostjo branja $K = 0.9$. Izbira
		parametrov je v naslovu slike.}
	\label{dokaz}
\end{figure}
Zanimivo bi bilo izmeriti vpliv $\theta$ na fazni portret: pustimo vse parametre nespremenjene, spremenimo pa $\theta = 0.25$,
kot ka\v ze slika~\ref{theta025}. Vidimo, da graf postane superpozicija dveh enakih fazno zamaknjenih grafov. Vsak izmed
teh grafov ustreza eno-frekven\v cnemu brcanemu rotatorju, vendar bi bil ustrezni $K < k + k'$. Podobne reuzultate dobimo
za $\tau = 2$ in $\theta = 0$, le da dobimo $K > k + k'$ (slika~\ref{graf2}), saj je stopnja nereda posameznega grafa v tem
primeru vi\v sja (vsaj sode\v c glede na sliko~\ref{dokaz}).
\begin{landscape}
\begin{figure}[H]\centering
	\includegraphics[width=22cm, keepaspectratio=1]{t100-q025-k030-K060-res}
	\caption{Kombiniran graf, je na skrajni levi, desno od njega pa sta grafa lo\v cena -- zeleni ustreza
		sunkom $K_i = k$, magenta pa sunkom $K_i \in \{k', k + k'\}$. Kombiniran graf je
		nekoliko kaoti\v cen, vendar pa imata lo\v cena grafa manj\v so stopnjo nereda.}
	\label{theta025}
\end{figure}
\begin{figure}[H]\centering
	\includegraphics[width=22cm, keepaspectratio=1]{t200-q000-k030-K060-res}
	\caption{Spet nam je uspelo graf razcepiti na dva enaka, a fazno zamaknjega.}
	\label{graf2}
\end{figure}
\end{landscape}
Ta razcep nam pove, da zgolj posplo\v sitev stroboskopske preslikave ne zado\v s\v ca. Te fazne portrete sem razlo\v cil
od predpostavki, da graf vedno razpade na dva grafa: eden ustreza sunku prve noge z jakostjo $k$, drugi pa sunku druge
noge z jakostjo $k'$ (ali pa $k + k'$). Za razmerja vrekvenc, kjer obstaja tak $n \in \mathbb{Z}$, da
$n \tau \in \mathbb{N}$ dobimo torej skupni graf, ki je superpozicija  ve\v c grafov. Naivno bi pri\v cakovali, da za
$\tau = 1.5$ dobimo superpozicijo treh grafov, tisti, ki ustrezajo $k$, $k'$ in $k + k'$. Vendar temu ni tako. Dobimo
superpozicijo \v stirih grafov (slika~\ref{graf3}). Intuitivno je to zato, ker je treba opazovati "`periodo"' brc, ki
je na tabeli~\ref{tab1}.
\begin{table}[H]\centering
	\caption{Primer tabele periode brc za primer $\tau = 1.5$, $\theta = 0$.}
	\begin{tabular}{c | c}
		$t_i$ & $K_i$ \\ \hline\hline
		$0$ & $k + k'$ \\
		$1$ & $k$ \\
		$1.5$ & $k'$ \\
		$2$ & $k$ \\ \hline
		$3$ & $k + k'$\\
		$\vdots$ & $\vdots$
	\end{tabular}
	\label{tab1}
\end{table}
Znotraj ene periode (tj. od $t = 0$ do tik pred $t = 3$) brcnemo \v stirikrat, zaradi \v cesar vidimo na sliki~\ref{graf3},
da nam razcep na $K_i = k$ in $K_i \in \{k', k + k'\}$ ni zadosten, saj je vsak izmed razcepljenih grafov sestavljen iz dveh,
tj. je razlika tudi po tem, kdaj udarimo in ne le s kak\v snim sunkom.

Za grafe z inkomenzurabilnim razmerjem frekvenc (tj. tak $n \in \mathbb{Z} \backepsilon: n\tau \in \mathbb{N}$ \emph{ne}
obstaja) nimamo cele periode, pa\v c pa imamo nekak\v sno kvazi-periodi\v cno obna\v sanje. Naredil sem primer
$\tau = \sqrt{2} - 1$ za razli\v cne $k$, $k'$ in pri\v sel do sklepa, da je kljub temu, da je obna\v sanje v ozadju videti
urejeno, \v se vedno preve\v c kaoti\v cno za vse mo\v zne izbire $k$, $k'$ (primer je na sliki~\ref{graf4}, kjer sem
izbral zelo pohlevne parametre, vendar zelo hitro zapolnimo cel prostor z izjemo orbit na sredini grafa).
\begin{landscape}
\begin{figure}[H]\centering
	\includegraphics[width=22cm, keepaspectratio=1]{t150-q000-k030-K060-res}
	\caption{Zeleni graf vsebuje vse sunke tipa $k$, magenta pa spet $K_i \in \{k', k' + k\}$. Po tabeli~\ref{tab1}
		vidimo, zakaj ta razcep ni zadosten -- niso vsi sunki z istim $k$ enakovredni -- pomembno je predvsem
		\emph{kdaj} se zgodijo.}
	\label{graf3}
\end{figure}

\begin{figure}[H]\centering
	\includegraphics[width=22cm, keepaspectratio=1]{t041-q000-k015-K010-res}
	\caption{Tukaj je $\tau = \sqrt{2}-1$. Vidimo, da so \v ze brce majnih jakosti premo\v cne. \v Ceprav vidimo nek
		red v ozadju, ga \v zal ne moremo izlu\v s\v citi.}
	\label{graf4}
\end{figure}

Za zelo majhne $k'$ bi moral zeleni graf biti manj za\v sumljen. To predpostavko potrdijo grafi slike~\ref{graf5}.
Grafa lahko separiramo, odfiltriramo \v sum in dobimo natanko sliko~\ref{dokaz}.
\begin{figure}[H]\centering
	\includegraphics[width=22cm, keepaspectratio=1]{new-t041-q000-k090-K000-res}
	\caption{$\tau = \sqrt{2}-1$, $k' = 10^{-6}$, $k = 0.9$. Vidimo, da v tem primeru magenta graf predstavlja nekak\v sen
		\v sum. Skupni graf je \v se vedno mo\v cno za\v sumljen. Zeleni graf je enak grafu slike~\ref{dokaz}.}
	\label{graf5}
\end{figure}
\begin{figure}[H]\centering
	\includegraphics[width=22cm, keepaspectratio=1]{new-t041-q000-k000-K090-res}
	\caption{$\tau = \sqrt{2}-1$, $k = 10^{-6}$, $k' = 0.9$, tj. neke vrste dualni graf sliki~\ref{graf5}. Zamenjamo
		$k$ in $k'$, in red se prestavi iz zelenega v magenta graf. Vendar pa grafa, ki predstavljata red nista
		enaka, zaradi tega, ker bi morali ustrezno popraviti tudi $\tau$.}
	\label{graf5a}
\end{figure}
\begin{figure}[H]\centering
	\includegraphics[width=22cm, keepaspectratio=1]{new-t041-q000-k050-K030-res}
	\caption{$\tau = \sqrt{2}-1$. Vidimo, da se vpliv $k'$ ka\v ze kot \v sum na zeleni graf, ki ga v tem primeru spet
		nismo mogli odfiltrirati (magenta graf).}
	\label{graf5b}
\end{figure}
\end{landscape}
Grafi z iracionalnim $\tau$ imajo v sredini oto\v cek stabilnosti -- tako je vsaj videti iz slik~\ref{graf4} in~\ref{graf5}.
Poglejmo si jih od blizu na slikah~\ref{graf6} in~\ref{graf7}.
\begin{figure}[H]\centering
	\includegraphics[width=10cm, keepaspectratio=1]{zoom-t041-q000-k010-K015-res.png}
	\caption{Center slike~\ref{graf4}, za $\tau = \sqrt{2} - 1$ in $G = 20$, tj. 20 razli\v cnih za\v cetnih pogojev.}
	\label{graf6}
\end{figure}
\begin{figure}[H]\centering
	\includegraphics[width=10cm, keepaspectratio=1]{zoom-t041-q000-k090-K000-res.png}
	\caption{Center slike~\ref{graf5}. Graf je podoben tistemu na sliki~\ref{graf6}.}
	\label{graf7}
\end{figure}
\begin{figure}[H]\centering
	\includegraphics[width=10cm, keepaspectratio=1]{zoom-t041-q000-k050-K030-res}
	\caption{Center slike~\ref{graf5b}. $G = 10$. Graf je na mo\v c podoben tistemu, ki je na sliki~\ref{graf6}.}
	\label{graf8}
\end{figure}
Videti je, kot da se orbite "`vrtijo"' v neki dodatni dimenziji in da vzbujanje s $k'$ in frekvenco $\tau$ poskrbi za to,
da je relaksacijski \v cas izjemno dolg. Pobli\v zan center slike~\ref{graf5a} manjka, ker je podoben tistemu iz
slike~\ref{graf7}.

\subsection{Korelacije}

Pod korelacije sem razumel predvsem difuzijsko konstanto, tj.
\begin{equation}
	D \equiv \lim_{t \to \infty} D_t, \quad D_t \equiv \frac{1}{2t}\bigg[ \frac{1}{G}\sum_{k = 1}^G \big(p_0^{(k)} -
		p_t^{(k)}\big)^2\bigg],
\end{equation}
in pa korelacije
\begin{equation}
	C_t^{(p)} \equiv \langle p_0 p_t \rangle - \langle p_0 \rangle\langle p_t\rangle,
	\qquad C_t^{(x)} \equiv \langle x_0 x_t \rangle - \langle x_0 \rangle\langle x_t \rangle,
\end{equation}
torej bomo `$\langle \bullet \rangle$' razumeli kot povpre\v cenje po razli\v cnih tirnicah, ozna\v cenih s $k$, ki se
lo\v cijo po za\v cetnih pogojih. Korelacije je treba verjetno \v se normirati na za\v cetno vrednost, tj $C_0 = 1$.
Rezultate vidimo na grafih spodaj. Da bi odpravil vplive statisti\v cnega \v suma, sem vzel zelo velik $G$. Ugotovil sem,
da se rezultati ne spremenijo \v ce je $G = 0.5 \cdot 10^6$ ali pa $G = 10^6$, zato sem delal kar z $G = 5 \cdot 10^6$. To
sem preveril za vse izmed spodnjih grafov.

\begin{figure}[H]\centering
	\input{corr.tex}
	\caption{Tu primerjamo korelacijske funkcije na grafih s slik~\ref{dokaz} in ~\ref{theta025}. Vidimo, da sprememba
		$\theta$ na $C^{(x)}_t$ prakti\v cno ne vpliva, spremeni pa se $C^{(p)}_t$.}
	\label{korelacije}
\end{figure}

\begin{figure}[H]\centering
	\input{corr2.tex}
	\caption{Za primer si poglejmo \v se bolj kaoti\v cen re\v zim in vpliv $\theta$ v tem primeru na korelacije.
		Glede sliko~\ref{korelacije} korelacije $C^{(x)}_t$ manj opletajo okoli ni\v cle. Vidimo, da zaradi
		kaoti\v cnosti sistema korelacije $C^{(p)}_t$ hitro padajo, $\theta \neq 0$ to upadanje pospe\v si.}
	\label{korelacije2}
\end{figure}

\begin{figure}[H]\centering
	\input{corr3.tex}
	\caption{Ta graf tudi ob velikem $G$ ostaja skrivnost. Morda je $G \sim 10^6$ \v se vedno premalo, saj
		takega nara\v s\v cujo\v cega trenda ne znam pojasniti.}
	\label{korelacije3}
\end{figure}

\begin{figure}[H]\centering
	\input{corr4.tex}
	\caption{V tem primeru je $\tau = \sqrt{2}-1$ vendar graf ni bistveno druga\v cen od tistega s slike~\ref{korelacije}.
		Vidimo le, da je vpliv $\theta$ v grafu $C^{(p)}$ v tem primeru zadu\v sen, vendar vseeno prinese fazno
		razliko, ki pa je prakti\v cno ne vidimo. Verjetno $\theta$ nima vpliva kadar je $\tau$ iracionalen.}
	\label{korelacije4}
\end{figure}

\begin{figure}[H]\centering
	\input{new-corr-1.tex}
	\caption{Na sliki so korelacije ki se nana\v sajo na fazni portret~\ref{graf4}. Zaradi majhnih $k$ in $k'$ se gibalna
		koli\v cina ne izniha tako hitro. Vpliv $\theta \neq 0$ je zanemarljiv. Zanimiv se mi zdi minimum
		v pri $t \sim 5$ za $C^{(x)}_t$, ki je sicer prisoten v vseh grafih, vendar tu bolj \v cist.}
	\label{n-kor-1}
\end{figure}

\begin{figure}[H]\centering
	\input{new-corr-2.tex}
	\caption{Na sliki so korelacije ki se nana\v sajo na fazni portret~\ref{graf5b}.
		V tem primeru sta $k$ in $k'$ ve\v cja, zato se tudi $C^{(p)}_t$ prej izniha.}
	\label{n-kor-2}
\end{figure}

\begin{figure}[H]\centering
	\input{new-corr-3.tex}
	\caption{Na sliki so korelacije ki se nana\v sajo na fazni portret~\ref{graf5}. Ta graf se izniha \v se prej.}
	\label{n-kor-3}
\end{figure}

\begin{figure}[H]\centering
	\input{new-corr-4.tex}
	\caption{Na sliki so korelacije ki se nana\v sajo na fazni portret~\ref{graf5a}.
		Ta graf je zelo lep za $C^{(p)}_t$, vendar $C^{(x)}_t$ vzbuja skrb zaradi nara\v scanja amplitude z
		ve\v canjem $t$.}
	\label{n-kor-4}
\end{figure}

Grafi na slikah~\ref{n-kor-1}, \ref{n-kor-2}, \ref{n-kor-3} in \ref{n-kor-4} potrjujejo domnevo, ki smo jo dobili
na sliki~\ref{korelacije4}, da $\theta$ nima vpliva na korelacije, v primeru ko je $\tau$ iracionalno \v stevilo.

\subsubsection{Difuzijska konstanta}

Rezultati numeri\v cnega poskusa so na sliki~\ref{difuzija}. Grafi so vsi prikazani s $k = 0.3$ in $k' = 0.6$, razen \v ce
pi\v se v legendi druga\v ce. Vidimo, da je $D = 0$ za $k = 0.3$, $k' = 0.6$ dokler ne
prese\v zemo neke kriti\v cne meje.

S posku\v sanjem sem ugotovil, da se to zgodi ko $\tau = \tau_\text{crit.} \sim 6$. Tudi \v ce pove\v cujemo $\tau \to \infty$
se rezultati (tj. $D$) ne spremenijo bistveno. Edini na\v cin, da $D$ mo\v cno pove\v camo, je prek $k$ in $k'$, kot se tudi
vidi na sliki~\ref{difuzija}.

Prav tako sem s posku\v sanjem ugotovil, da so grafi najbolj zamaknjeni, ko $\theta \sim 0.25$, zaradi \v cesar
sta grafa na sliki~\ref{difuzija} prikazana pri $\theta = 0$ in $\theta = 0.25$.

\begin{figure}[H]\centering
	\input{diff.tex}
	\caption{Vidimo, da fazni zamik $\theta$ malo vpliva tudi na difuzijsko konstanto. Ta je prakti\v cno ves \v cas
	$D = 0$, razen v primeru ko imamo kaoti\v cen sistem (tj. primer $k' = 5$). Vidimo, da je dovolj tudi, \v ce je
	$\tau > \tau_\text{crit.}$. V tem primeru se to nana\v sa na $\tau = 8$, tu je $D \neq 0$.}
	\label{difuzija}
\end{figure}

\subsection{Ljapunov eksponent}

Ljapunov eksponent dolo\v cimo prek
\begin{equation}
	\lambda = \lim_{t \to \infty} \frac{1}{t} \sum_{i = 1}^\infty t_i \lambda_i,
\end{equation}
kjer je $t_i$ \v cas ob katerem reskaliramo razdaljo med tirnicama, ki sta bili tik pred reskalacijo oddaljeni za
$\delta_0 \e^{\lambda_i}$. Vendar pa nimamo na voljo nesko\v cno dosti \v casa, zaradi tega bomo $\lambda$ izra\v cunali
s pomo\v cjo prilagajanja premice na to\v cke na\v sega numeri\v cnega poskusa s pomo\v cjo linearne regresije:
\begin{equation}
	t\lambda + C = \sum_{i = 1}^{t_i \leq t} t_i \lambda_i = \sum_{i = 1}^n y_i
\end{equation}
od koder je o\v citno, da je $\lambda$ smerni koeficient premice. Kot pravi literatura, je izra\v cun $\lambda$ mo\v cno
odvisen od \v casa pri katerem reskaliramo. Lahko reskaliramo vsaki\v c, ko $\delta_0 \mathrm{e}^{\lambda_i} = \delta_{t_i}
> \pi$, ali pa ob enakih \v casovnih intervalih. Spet, kot pravi literatura, to je mra\v cna umetnost.

Linearna regresija nam da slede\v ce rezultate:
\begin{align}
	\lambda &= \frac{nb - ga}{nh - g^2}, \\
	C &= \frac{ha - gb}{nh - g^2},
\end{align}
kjer so $g$, $h$, $a$ in $b$ definirani kot
\begin{equation}
	g = \sum_{i = 1}^n t_i,\quad h = \sum_{i = 1}^n t_i^2,\quad a = \sum_{i = 1}^n y_i,
	\quad b = \sum_{i = 1}^n t_i y_i.
\end{equation}
Ugotovil sem, da je orbito dobro reskalirati vsaki\v c, ko prete\v ce $t_\mathrm{r}$, ki je nekje med $3$ in $6$, odvisno
od $k$ in $k'$. Ker je rezultat zelo ob\v cutljiv tudi na izbiro za\v cetnih pogojev, sem delal z aritmeti\v cno sredino
ve\v cih tirnic ($10^4$ je dovolj).

Delovanje preverimo tako, da uporabimo znan rezultat za eno-frekven\v cen brcan rotator -- ta je marginalno kaoti\v cen za
$K \approx 0.971635$, (tj. Ljapunov eksponent mora biti v tem primeru $\lambda = 1$). Slika~\ref{deluje} potrdi na\v so
izbiro postopka za izra\v cun Ljapunovega eksponenta.
\begin{figure}[H]\centering
	\input{ljp-determ.tex}
	\caption{Marginalno kaoti\v cen eno-frekven\v cni brcan rotator ($\tau = 1$, $\theta = 0$, $k = 0.3$,
		$k' = 0.671635$). Napaka Ljapunovega eksponenta v tem primeru je znotraj $5\%$, kar zado\v s\v ca za
		smiseln izra\v cun na \v sir\v sem obmo\v cju.}
	\label{deluje}
\end{figure}
Iz tabele~\ref{tab2} razberemo, da so Ljapunovi eksponenti mo\v cno odvisni od $t_\mathrm{r}$, vendar imamo to sre\v co,
da je za inkomenzurabilne vrednosti $\tau < 1$ vedno dober $t_\mathrm{r} = 1$, in da je vpliv $\theta$ v tem primeru
ni\v cen (je zadu\v sen znotraj $5\%$ napake).
\begin{table}[H]\centering
	\caption{Ljapunovi eksponenti za nekatere fazne portrete v poro\v cilu. Navajal bom samo $\tau$ in $\theta$, ostali
		parametri ($k$ in $k'$) razvidni iz komentarja pod posameznimi faznimi portreti (slike~\ref{graf2},
		\ref{graf3} in~\ref{graf4}). Vidimo, da so grafi, pre\v sli kriti\v cno mejo. Separabilni grafi so kljub
		temu mo\v cno v kaoti\v cnem re\v zimu, sploh pa graf na sliki~\ref{graf4} za $\tau = \sqrt{2}-1$, kljub
		temu, da je bil v ozadju videti dosti pohleven. Ostale vrednosti so bile izra\v cunane (tj. vrstice pod
	$\tau = \sqrt{2} - 1$) so bile izra\v cunane za $k = 0.3$ in $k' = 0.6$.}
	\begin{tabular}{r | c | c | l}
		$\tau$ & $\theta$ & $t_\mathrm{r}$ & $\lambda$ \\
		\hline \hline
		2 & 0 & 4 & 1.26 \\
		1.5 & 0 & 3 & 1.44 \\
		\hline
		$\sqrt{2}-1$ & 0 & 1 & 1.77 \\
		$\sqrt{2}-1$ & 0.25 & 1 & 1.79 \\
		\hline
		1 & 0.25 & 1 & 3.67 \\
		1 & 0.25 & 1.25 & 0.64 \\
		\hline
		6 & 0 & 3 & 0.61 \\
		6 & 0 & 6 & 1.14
	\end{tabular}
	\label{tab2}
\end{table}
Funkcija $\lambda (k, k'; \tau = \sqrt{2}-1, t_\mathrm{r} = 1)$ je na sliki~\ref{ljapun}. Vidimo, da je sistem v bistvu cel
\v cas mo\v cno kaoti\v cen.
\begin{figure}[H]\centering
	\input{ljp-scan.tex}
	\caption{Vrednost ljapunovega eksponenta je ves \v cas $\lambda > 1.5$, kar pomeni, da z izbiro parametrov ne
		dose\v zemo niti marginalne stabilnosti orbit. Ker je ra\v cunalnik to ra\v cunal dlje \v casa, sem
		$G$ zmanj\v sal iz $10^4$ na $2000$.}
	\label{ljapun}
\end{figure}

\section{Zaklju\v cek:}

Ena\v cba~\eqref{stroboskopska} in slike~\ref{dokaz},~\ref{theta025},~\ref{graf2},~\ref{graf3},~\ref{graf4} prikazujejo
fazne portrete, tj. se da dinamiko posplo\v siti in predstaviti. Slika~\ref{theta025} dodatno prikazuje, da $\theta \neq 0$
lahko privede do novih, (ne)pri\v cakovanih situacij -- $\theta$ ima vpliv, kar potrujejo korelacijski grafi na
slikah~\ref{korelacije}, \ref{korelacije2}, \ref{korelacije3} in opazovanje $D_t$ na
sliki~\ref{difuzija}, vendar pa ne vpliva v primeru, ko je so razmerja frekvenc $\tau$ inkomenzurabilna, tj. $\tau$
iracionalen, kar se vidi na grafih~\ref{korelacije4}, \ref{n-kor-1}, \ref{n-kor-2}, \ref{n-kor-3} in~\ref{n-kor-4}.
\v Ce je $D_t \propto t^{-\alpha}$, potem s $\tau$ spreminjamo $\alpha$. Na ljapunov eksponent
$\theta$ nima opazljivega vpliva. Za inkomenzurabilne vrednosti $\tau$ je sistem vedno kaoti\v cen, kar lahko vidimo
na sliki~\ref{ljapun}.

\end{document}
