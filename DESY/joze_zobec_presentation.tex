\documentclass{beamer}
\usepackage[utf8]{inputenc}
\usepackage[english]{babel}

\usepackage{float, amsmath, amssymb, graphicx, stackrel, slashed, caption, subcaption}

\newcommand{\ttbar}{
	\ensuremath{t\bar{t}}
}

%\usepackage{utopia}

\usetheme{Warsaw}
\logo{
	\includegraphics[keepaspectratio=1, width=0.1\textwidth]{Pics/CMSLogo.png}~
	\includegraphics[keepaspectratio=1, width=0.1\textwidth]{Pics/DESYLogo.png}~
%	\includegraphics[keepaspectratio=1, width=0.1\textwidth]{Pics/CERNLogo.png}
}

%title info
\title{$\ttbar$ production cross-sections}
\subtitle{DESY Summer School, 2012}
\author{Jože Zobec}
\date[\today]{Hamburg, 30th of August, 2012}
\institute[DESY, CMS]{Supervisor: Andreas Meyer}

\begin{document}

\begin{frame}
	\titlepage
\end{frame}

\begin{frame}
	\frametitle{Overview}
	\tableofcontents
\end{frame}


\section{Introduction to top quarks at the LHC}
%1
\begin{frame}[t]{Differential cross sections at the CMS}
	\begin{itemize}
		\item{DESY is part of the CMS collaboration, one of the groups is the Differential
			cross-sections of the top quark, where I work,}
		\item{my work was split into two parts:}
		\begin{itemize}
			\item{Searching for new physics by applying cuts to the $m_{\ttbar}$ system}
			\item{decreasing the analysis cpu time from $\sim$ 3 h to $\sim$ 15 min using
				the PROOF-Lite package}
		\end{itemize}
		\item{Analysis was done on the 2011 data, $\ttbar$ decays in the dileptonic channel.}
	\end{itemize}
\end{frame}

\section{Physics of the $\ttbar$}
\subsection{$\ttbar$ production}
%2
\begin{frame}[t]{$\ttbar$ pair production}
	\begin{itemize}
		\item{Top quarks are heaviest known particles ($m_t \sim 175$ GeV),}
		\item{Extremely short-lived -- they don't hadronize!}
		\item{Interesting properties, that imply new physics,}
		\item{Might decay into new exotic particles \ldots}
	\end{itemize}
	\only <1,2,3,4,5>{
		\begin{itemize}
			\only<1->{\item{At the LHC $\ttbar$ pairs are produced via proton beam collisions}}
		\only<2-,3>{
			\only<2->{\item{Almost 90\% $\ttbar$ pairs are produced via the gluon fusion,}}
			\only<3>{
			\begin{figure}[H]
			\includegraphics[keepaspectratio=1, width=0.4\textheight]{Pics/ttbar-production.png}
			\end{figure}
			}
		}
		\only<4-,5>{
			\only<4->{\item{Remaining goes through quark-(anti)quark interactions,}}
			\only<5>{
			\begin{figure}[H]
			\includegraphics[keepaspectratio=1, width=0.5\textwidth]{Pics/ttbar-q-production.png}
			\end{figure}
			}
		}
		\end{itemize}
	}
\end{frame}

\subsection{$\ttbar$ decays}
%3
\begin{frame}[t]{$\ttbar$ decay modes}
%	\only <1, 2, 3, 4, 5, 6, 7, 8>{
%		\begin{itemize}
%			\only<1->{\item{Three decay modes:}}
%			\only<2-,3>{
%			\only<2->{\item{All-hadronic decay}}
%			\only<3>{
%			\begin{figure}[H]
%			\includegraphics[keepaspectratio=1, width=0.5\textwidth]{Pics/allhadronic.png}
%			\end{figure}}
%			}
%			\only<4-,5>{
%			\only<4->{\item{Semileptonic decay}}
%			\only<5>{
%			\begin{figure}[H]
%			\includegraphics[keepaspectratio=1, width=0.5\textwidth]{Pics/semileptonic.png}
%			\end{figure}}
%			}
%			\only<6-, 7>{
%			\only<6->{\item{Dileptonic decay}}
%			\only<7>{
%			\begin{figure}[H]
%			\includegraphics[keepaspectratio=1, width=0.5\textwidth]{Pics/dileptonic.png}
%			\end{figure}}
%			}
%		\end{itemize}
%	}
%	\only<8>{
		\begin{figure}[H]
		\includegraphics[keepaspectratio=1, width=0.5\textwidth]{Pics/decay-ch.pdf}
		\end{figure}
%	}
\end{frame}

%4
\begin{frame}[t]{Event Selection}
	\begin{itemize}
		\item{\bf At least two isolated leptons}
		\item[$\circ$]{opposite sign}
		\item[$\circ$]{{\bf muon} (combined + tracker): $p_T > 20$ GeV, $|\eta| < 2.4$}
		\item[$\circ$]{{\bf electron} $E_T > 20$ GeV, $|\eta| < 2.4$}
		\item[$\circ$]{{QCD veto}: $m_{\ell^+\ell^-} > 12$ GeV}
		\item{\bf At least two jets} 
		\item[$\circ$]{$p_T > 30$ GeV}
		\item[$\circ$]{$|\eta| < 2.4$}
		\item{{\bf system of} $e^+e^-/\mu^+\mu^-$}
		\item[$\circ$]{Z veto: 76 GeV$ < m_{\ell^+\ell^+} < 106$ GeV}
		\item[$\circ$]{$MET > 30$ GeV}
	\end{itemize}
\end{frame}

%5
\section{Physics analysis}
\begin{frame}[t]{Analysis work}
	\only<1,2,3,4,5,6,7>
	\begin{itemize}
		\only<1->{\item{Applying cuts at the $M_{\ttbar}$ -- interesting for two reasons:}}
			\only<2-, 3, 4, 5-,6>{
			\begin{itemize}
				\only<2->{\item[$\circ$]{decaying into high energetic exotic particles,}}
				\only<3>{
					\begin{figure}[H]
					\includegraphics[keepaspectratio=1, width=0.5\textwidth]{Pics/ttbarH.png}
					\end{figure}
					}
				\only<4>{
					\begin{figure}[H]
					\includegraphics[keepaspectratio=1, width=0.5\textwidth]{Pics/ttbarprime.eps}
					\end{figure}
				}
				\only<5->{\item[$\circ$]{the $\ttbar$ pair might be in fact produced from via an
					exotic particle in one of the physical processes from beyond the
					standard model}}
				\only<6>{
					\begin{figure}[H]
						\includegraphics[keepaspectratio=1, width=0.5\textwidth]{Pics/zprime.eps}
					\end{figure}
				}
			\end{itemize}
			}
		\only<7->{\item{We apply the cuts and then check how the differential cross-section changes
			in other systems, hoping for events that show evidence for new physics \ldots}}
	\end{itemize}
\end{frame}

%6
\subsection{Plots}
\begin{frame}[t]{Analysis results before cuts}
	\begin{figure}[H]
		\centering
		\begin{subfigure}[b]{0.30\textwidth}
			\centering
			\includegraphics[width=\textwidth, keepaspectratio=1]{Pics/Plots/UnCut/HypTTBarMass}
		\end{subfigure}~
		\begin{subfigure}[b]{0.30\textwidth}
			\centering
			\includegraphics[width=\textwidth, keepaspectratio=1]{Pics/Plots/UnCut/HypTopMass}
		\end{subfigure}~
		\begin{subfigure}[b]{0.30\textwidth}
			\centering
			\includegraphics[width=\textwidth, keepaspectratio=1]{Pics/Plots/UnCut/DiffXS_HypTTBarpT}
		\end{subfigure}
		\begin{subfigure}[b]{0.30\textwidth}
			\centering
			\includegraphics[width=\textwidth, keepaspectratio=1]{Pics/Plots/UnCut/DiffXS_HypTTBarMass}
		\end{subfigure}~
		\begin{subfigure}[b]{0.30\textwidth}
			\centering
			\includegraphics[width=\textwidth, keepaspectratio=1]{Pics/Plots/UnCut/DiffXS_HypToppT}
		\end{subfigure}~
		\begin{subfigure}[b]{0.30\textwidth}
			\centering
			\includegraphics[width=\textwidth, keepaspectratio=1]{Pics/Plots/UnCut/DiffXS_HypTopRapidity}
		\end{subfigure}
	\end{figure}
\end{frame}

%7
\begin{frame}[t]{Analysis results after the cuts were applied}
	\begin{figure}[H]
		\centering
		\begin{subfigure}[b]{0.30\textwidth}
			\centering
			\includegraphics[width=\textwidth, keepaspectratio=1]{Pics/Plots/700Cut/HypTTBarMass}
		\end{subfigure}~
		\begin{subfigure}[b]{0.30\textwidth}
			\centering
			\includegraphics[width=\textwidth, keepaspectratio=1]{Pics/Plots/700Cut/HypTopMass}
		\end{subfigure}~
		\begin{subfigure}[b]{0.30\textwidth}
			\centering
			\includegraphics[width=\textwidth, keepaspectratio=1]{Pics/Plots/700Cut/DiffXS_HypTTBarpT}
		\end{subfigure}
		\begin{subfigure}[b]{0.30\textwidth}
			\centering
			\includegraphics[width=\textwidth, keepaspectratio=1]{Pics/Plots/700Cut/DiffXS_HypTTBarMass}
		\end{subfigure}~
		\begin{subfigure}[b]{0.30\textwidth}
			\centering
			\includegraphics[width=\textwidth, keepaspectratio=1]{Pics/Plots/700Cut/DiffXS_HypToppT}
		\end{subfigure}~
		\begin{subfigure}[b]{0.30\textwidth}
			\centering
			\includegraphics[width=\textwidth, keepaspectratio=1]{Pics/Plots/700Cut/DiffXS_HypTopRapidity}
		\end{subfigure}
	\end{figure}
\end{frame}

%8

\section{PROOF}
\begin{frame}[t]{Introduction to PROOF}
	\begin{itemize}
		\item{PROOF -- {\bf P}arallel {\bf ROO}t {\bf F}acility}
		\item{Comes in several flavours:}
			\begin{itemize}
				\item{PROOF-Lite -- for a single multi-core machine}
				\item{normal PROOF -- for PROOF enabled facilities (if we have
					computer clusters)}
			\end{itemize}
		\item{PROOF is NOT a batch system!!!}
		\item{We use a {\tt TSelector} class for analysis, instead of {\tt TTree} class we use
			a {\tt TChain} class, other {\tt TObject} inherited structures are passed as
			arguments to the {\tt TList}s}
	\end{itemize}
\end{frame}

\begin{frame}[t]{Functions involved in PROOF}
	\begin{itemize}
		\item{{\tt Begin} -- initialize the PROOF system,}
		\item{{\tt SlaveBegin} -- initialize the ``workers'',}
		\item{{\tt Init} -- initialize the {\tt TTree},}
		\item{{\tt Process} -- do the processing on the workers,}
		\item{{\tt SlaveTerminate} -- close the work on the workers,}
		\item{somewhere here merge the results of the histos,}
		\item{{\tt Terminate} -- close the work on the master.}
	\end{itemize}

	Using PROOF we can linearly decrease the computational time by using more cpu-s. We can put more
	than one machine to work using the PROOF computer clusters.

	My analysis for example:
	With PROOF on 8 cpu-s: 15 min CPU time vs. 120 min user time
	without PROOF: 180 min CPU time vs. 210 min user time 
\end{frame}

\section{Conclusion}
%9
\begin{frame}[t]{Summary}
	\begin{itemize}
		\item{No new physics found,}
		\item{Analysis successfully changed to {\bf parallel processing!}}
%		\item{Plans for the future:}
%		\begin{itemize}
%			\item{Modify bash scripts to further decrease computing time,}
%			\item{Upgrade the script for designing the booklets}
%			\item{Have a centralized control file containing all the cuts}
%			\item{Upgrade the parallel Analysis.C script -- for versioning and running several
%				instances of Analysis.C without any problems\ldots}
%		\end{itemize}
	\end{itemize}
\end{frame}

\begin{frame}
	Thank you for your attention!
\end{frame}

\end{document}

