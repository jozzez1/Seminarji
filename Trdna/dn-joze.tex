\documentclass[a4paper, 12pt]{article}
\usepackage[slovene]{babel}
\usepackage[T1]{fontenc}
\usepackage[utf8]{inputenc}
\usepackage{fullpage, amsmath, amssymb, graphicx, caption, float}

\newcommand{\vq}{
	\ensuremath{\vec{q}}
}

\newcommand{\w}{
	\ensuremath{\omega}
}

\newcommand{\e}{
	\ensuremath{\epsilon}
}

\renewcommand{\u}{
	\ensuremath{\uparrow}
}

\renewcommand{\d}{
	\ensuremath{\downarrow}
}

\newcommand{\hn}{
	\ensuremath{\hat{n}}
}

\newcommand{\oh}{
	\ensuremath{\hat{h}}
}

\newcommand{\vk}{
	\ensuremath{\vec{k}}
}

\newcommand{\koren}{
	\ensuremath{\frac{1}{\sqrt{V}}}
}

\newcommand{\vR}{
	\ensuremath{\vec{R}_{ij}}
}

\newcommand{\s}{
	\ensuremath{\sigma}
}

\newcommand{\ex}{
	\ensuremath{\vec{e}_x}
}

\newcommand{\ey}{
	\ensuremath{\vec{e}_y}
}

\newcommand{\fV}{
	\ensuremath{\frac{1}{V}}
}

\renewcommand{\ni}{
	\noindent
}

\newcommand{\pred}{
	\ensuremath{\frac{e^2}{\e_0 q^2 V}}
}

\newcommand{\dd}{
	\ensuremath{\text{d}}
}

\newcommand{\I}{
	\ensuremath{\mathcal{I}}
}

\begin{document}

\begin{center}
\textsc{Teorija trdne snovi}\\
\textsc{2012/13}\\[0.5cm]
\textbf{Doma\v ca naloga}\\
\end{center}
\begin{flushright}
\textbf{Jože Zobec}\\
\end{flushright}

\section{Navodilo}

Atomi z enim valen\v cnim elektronom na povr\v sini tvorijo kvadratno mre\v zo. Elektronski pas
popi\v semo v pribli\v zku tesne vezi. Zapi\v si dielektri\v cno odzivno funkcijo za tako kovino.
Razi\v s\v ci (frekven\v cno, momentno, temperaturno) obna\v sanje dielektri\v cne konstante v
bli\v zini valovnih vektorjev (nesting) gnezdenja.

\section{Naloga}

V skripti, poglavje 8, imamo na voljo ti. \emph{Lindhardovo ena\v cbo},

\begin{equation}
	\e (\vq, \w) = 1 - \frac{e^2}{\e_0 q^2 V} \sum_{\vec{k}} \frac{f_{\vec{k}}^0 -
		f^0_{\vec{k}+\vq}} {\e_{\vec{k}} - \e_{\vec{k}+\vq} + \hbar\w + i\hbar\alpha}
	\label{lindhard}
\end{equation}

V izpeljavi le-te smo delali s splo\v sno valovno funkcijo, tj. velja tudi za nas. Ugotoviti moramo
le energijsko disperzijo, $\e_{\vq}$. Za to moramo re\v siti Hamiltonian. Delali bomo po zgledu
Hubbardovega modela v skripti, vendar brez interakcije. Uporabimo operatorje polja

\begin{align}
	\psi_{i,j,\sigma}(\vec{r}) &= \sum_n c_{n,\sigma} \phi_n (\vec{R}_{ij} + \vec{r}),\notag  \\
	\psi^\dagger_{i,j,\sigma}(\vec{r}) &= \sum_n c^\dagger_{n,\sigma} \phi_n (\vec{R}_{ij} +
		\vec{r}),
	\label{polje}
\end{align}

\ni ki ustvarijo (uni\v cijo) delec s spinom $\sigma$ na mre\v zi -- $\vec{R}_{ij}$ je vozli\v sce
kristalne mre\v ze (tj. lokacija atoma), $\vec{r}$ pa je odmik elektrona v okolici. Delamo v
adiabatnem pribli\v zku, tj. mre\v za je za na\v se potrebe stacionarna. Funkcije $\phi_n (\vec{r})$
so znamenite Wannierove funkcije. Vendar, ker imamo samo en elektron, lahko zapi\v semo tako:

\begin{align}
	\psi_{ij,\sigma,m}(\vec{r}) &= \sum_n c_{n,\sigma} \phi_n (\vec{R}_{ij} + \vec{r})
		\delta_{mn} \notag \\
	&= c_{m,\sigma} \phi_m (\vec{R}_{ij} + \vec{r}).
\end{align}

\ni Ker imamo kreacijsko anihilacijske operatorje le za en delec, jih bomo opremili druga\v ce -- indeks
ne bo ve\v c pomenil stanja, ampak lokacijo na mre\v zi:

\begin{align}
	\psi_{ij,\sigma,n} (\vec{r}) &= c_{ij,\sigma} \phi_n (\vec{R}_{ij} + \vec{r}), \notag \\
	\psi^\dagger_{ij,\sigma,n} (\vec{r}) &= c^\dagger_{ij,\sigma} \phi_n (\vec{R}_{ij} + \vec{r}).
	\label{polje2}
\end{align}

Pribli\v zek tesne vezi pomeni, da imamo neni\v celne prispevke le za najbli\v zje sosede in sicer
dobimo `$-t$' na par sosedov. V na\v sem primeru so za energijo vozli\v sca $\oh_{ij}$ to

\begin{equation}
	\oh_{ij} = -t \sum_{\sigma} [c^\dagger_{i,j,\sigma} (c_{i,j+1,\sigma} + c_{i+1,j,\sigma} +
		c_{i,j-1,\sigma} + c_{i-1,j,\sigma}) + \text{h.c.}].
	\label{oh}
\end{equation}

\ni Celotno kineti\v cno energijo lahko zapi\v semo kot vsoto po vseh vozli\v s\v cih, vendar moramo
biti pazljivi: na tak na\v cin bomo vsako "`vez"' \v steli dvakrat\footnote{To v splo\v snem ni res,
zaradi efektov na robu. Vendar mi delamo s periodi\v cnimi robnimi pogoji, zato je vse v redu.}, zato
moramo vsoto obte\v ziti z ustrezno multiplikativno konstanto, tj. $1/2$:

\begin{equation}
	\hat{H} = \frac{1}{2}\sum_{ij} \oh_{ij}.
\end{equation}

\ni Operator $\hn$ je operator \v stetja. Zapisal sem ga s stre\v sico, da se ga lo\v ci od indeksa
energijskega stanja.

Imamo dva na\v cina, kako nadaljujemo z re\v sevanjem problema: $h_{ij}$ lahko izra\v cunamo v
matri\v cni obliki in dobimo energijski spekter, ali pa Hamiltonian pretvorimo v impulzno bazo.

Za pretvorbo v impulzno bazo poglejmo najprej operatorje polja, iz en.~\eqref{polje}:
\begin{equation}
	\psi_{ij,\sigma} (\vec{r}) = \frac{1}{\sqrt{V}}\sum_{\vec{k}}
		c_{\vec{k},\sigma}\varphi_{\vec{k}}(\vec{R}_{ij} + \vec{r}) = 
		\frac{1}{\sqrt{V}}\sum_{\vec{k}} c_{\vec{k},\sigma} e^{i\vec{k}\cdot\vec{R}_{ij}}
		\varphi_{\vec{k}}(\vec{r})
\end{equation}

\ni Po primerjavi s staro izjavo hitro uganemo, da je
\begin{align}
	c_{ij,\sigma} &= \koren \sum_{\vk} c_{\vk,\sigma} e^{i\vk\cdot\vec{R}_{ij}}, \\
	c^\dagger_{ij,\sigma} &= \koren \sum_{\vk} c^\dagger_{\vk, \sigma} e^{-i\vk\cdot\vec{R}_{ij}}.
\end{align}

\ni Sedaj transformirajmo \v clene ki nastopajo v en.~\eqref{oh}. Recimo $\vec{R}_{i,j\pm1} = \vec{R}_{ij}
\pm a\vec{e}_y$, $a > 0$, $a$ je razdalja med ioni na mre\v zi. Ker je mre\v za kvadratna, lahko
isto naredimo v drugi smeri, tj. $\vec{R}_{i\pm1, j} = \vec{R}_{ij} \pm a\vec{e}_x$.

\begin{align}
	\sum_\s\sum_{i,j} c^\dagger_{i,j,\sigma}c_{i,j \pm 1,\sigma} 
	&=\fV \sum_\s\sum_{i,j}\sum_{\vk} c^\dagger_{\vk,\s} e^{-i\vk \cdot \vR}
		\sum_{\vq}c_{\vq,\s} e^{i\vq\cdot\vec{R}_{i,j \pm 1}}, \notag \\
	&=\fV \sum_\s \sum_{\vk,\vq} \sum_{i,j}c^\dagger_{\vk,\s} c_{\vq,\s} e^{-i(\vk - \vq)\cdot \vR}
		e^{\pm i\vq \cdot a \ey}, \notag \\
	&=\fV \sum_\s \sum_{\vk, \vq} c^\dagger_{\vk, \s} c_{\vq, \s} e^{\pm i q_y a}\ 
		V \delta_{\vk,\vq}\ ,\notag \\
	&= \sum_{\vk} \sum_\s c^\dagger_{\vk, \s} c_{\vk, \s} e^{\pm ik_y a}, \notag \\
	&= \sum_{\vk} (\hat{n}_{\vk,\u} + \hat{n}_{\vk,\d})e^{\pm ik_y a}, \notag \\
	&= \sum_{\vk} \hat{n}_{\vk}\ e^{\pm ik_y a}.
\end{align}

\ni Opazimo, da je analogno za primer, ko imamo $c^\dagger_{i,j,\s} c_{i\pm 1,j\s}$ in sicer
\begin{equation}
	\sum_\s \sum_{i,j} c^\dagger_{i,j,\s} c_{i\pm 1,j\s} = \sum_{\vk}
		\hat{n}_{\vk}\ e^{\pm ik_x a}.
\end{equation}

\ni Torej lahko hamiltonian $\hat{H}$ zapi\v semo kot

\begin{align}
	\hat{H} &= -\frac{t}{2} \sum_{\vk} \hat{n}_{\vk}\ 2(e^{ik_xa} + e^{-ik_xa} +
		e^{ik_ya} + e^{-ik_ya}), \notag \\
	&= \sum_{\vk} \underbrace{(-2t) \hat{n}_{\vk} \big[\cos k_xa + \cos k_ya \big]}_{\oh_{\vk}}.
\end{align}

\ni Dodatni faktor `$2$' se prikrade notri od hermitsko konjugiranih \v clenov. Se pravi, na\v sa
re\v sitev je

\begin{equation}
	\e_{\vk} = -2t (\cos k_x a + \cos k_y a).
\end{equation}

\ni Za $f_{\vk}^0$ upo\v stevamo Fermi-Diracovo porazdelitev, tj.

\begin{equation}
	f_{\vk}^0 = \frac{1}{e^{\beta(\e_{\vk} - \mu)} + 1}.
	\label{favela}
\end{equation}

Nimamo \v se vsega, kar potrebujemo -- $\mu$, ki nastopa v porazdelitvi, je neznanka. Dobili ga bomo iz
identitete

\begin{equation}
	\bar{n} = F (\mu (\beta)) = \int \dd \e \mathcal{D}(\e) f(\e),
\end{equation}

\ni kjer $\bar{n}$ predstavlja polzapolnjen pas, torej ima vrednost 1/2. $f(\e) = f_{\vk}^0$ in 
$\mathcal{D}(\e)$ je gostota stanj, definirana kot

\begin{equation}
	\mathcal{D}(\e) = \sum_{\vk} \delta (\e - \e_{\vk}).
\end{equation}

\ni Naredimo torej

\begin{align}
	\frac{1}{2} &= \int \dd \e \mathcal{D} (\e) f (\e) = \int \dd \e f(\e) \sum_{\vk}
		\delta (\e - \e_{\vk}) \notag \\
	&= \int \dd \e f(\e) \int\limits_\text{1. BC} \dd^2 \vk \bigg(\frac{a}{2\pi}\bigg)^2
		\delta (\e - \e_{\vk}) \notag \\
	&= \bigg(\frac{a}{2\pi}\bigg)^2 \int\limits_\text{1. BC} \dd^2 \vk \int \dd \e\ f(\e)
		\delta (\e - \e_{\vk}) \notag \\
	&= \bigg(\frac{a}{2\pi}\bigg)^2 \int\limits_\text{1. BC} \dd^2 \vk\ f(\e_{\vk}).
\end{align}

Za ta integral uvedemo brezdimenzijske spremenljivke $x = k_x a$ in $y = k_y a$ in dobimo
rezultat

\begin{align}
	2\pi^2 &= \int_{-\pi}^\pi \dd x \int_{-\pi}^\pi \dd y\ \overbrace{\frac{1}
		{e^{-2\beta t(\cos x + \cos y) -\beta \mu (\beta)} + 1}}^{f(x,y)}\notag \\
	\frac{\pi^2}{2} &= \int_0^\pi \dd x \int_0^\pi \dd y \frac{1}{e^{-2\beta t(\cos x + \cos y)
		- \beta \mu (\beta)} + 1}
\end{align}

Z numeri\v cno obravnavo tega integrala dobimo rezultat $\mu \equiv 0$, kar sledi tudi iz
simetrije.

\ni Izraz in \eqref{favela} vstavimo v en.~\eqref{lindhard}. Na\v so vsoto pretvorimo
v integral

\begin{align}
	\e(\vq, \w) &= 1 - \pred \Big(\frac{a}{2\pi}\Big)^2 \underbrace{ \iint\limits_\text{1. BC}\dd^2 \vk \frac{f^0_{\vk} -
		f^0_{\vk+\vq}}{\e_{\vk} - \e_{\vk+\vq} + \hbar\w + i\hbar\alpha}}_{\I}.
\end{align}

\ni Integral $\I$ bomo raz\v clenili v $\I = \I_1 - \I_2$, kjer sta

\begin{align}
	\I_1 &= \iint\limits_\text{1. BC}\dd^2 \vk \frac{1}{(e^{\beta(\e_{\vk} - \e_F)} + 1)
		(\e_{\vk} - \e_{\vk + \vq} + \hbar\w + i\hbar\alpha)}, \notag \\
	\I_2 &= \iint\limits_\text{1. BC}\dd^2 \vk \frac{1}{(e^{\beta(\e_{\vk+\vq} - \e_F)} + 1)
		(\e_{\vk} - \e_{\vk + \vq} + \hbar\w + i\hbar\alpha)}.
	\label{integrali}
\end{align}

\ni Da bo manj pisanja bomo kar vnaprej uvedli oznake

\begin{align}
	c_{x,y}^k \equiv \cos k_{x,y} a, &\quad	c_{x,y}^q \equiv \cos q_{x,y} a, \notag \\
	s_{x,y}^k \equiv \sin k_{x,y} a, &\quad s_{x,y}^q \equiv \sin q_{x,y} a.
\end{align}

\ni Torej

\begin{equation}
	\cos [(k_{x,y} + q_{x,y}) a] = c_{x,y}^k c_{x,y}^q - s_{x,y}^k s_{x,y}^q.
\end{equation}

\ni To uporabimo v en.~\eqref{integrali} in dobimo, 

\begin{align}
	\I_1 &= -\frac{1}{2t}\iint \dd^2 \vk \frac{1}
		{\big[e^{-2t\beta(c_x^k + c_y^k)} + 1\big]
		\big[c_x^k + c_y^k + s_x^k s_x^q + s_y^k s_y^q - c_x^k c_x^q - c_y^k c_y^q -
		\hbar(\w + i\alpha)/2t\big]}, \notag \\
	\I_2 &= -\frac{1}{2t}\iint \dd^2 \vk \frac{1}
		{\big[e^{-2t\beta (c_x^k c_x^q + c_y^k c_y^q - s_x^k s_x^q - s_y^k s_y^q)} + 1\big]
	[ \ldots ]}.
\end{align}

Sedaj si moramo ogledati obna\v sanje v bli\v zini vektorjev gnezdenja, tj. $\vq = \vq_0 =
\pi/a (1,1)$. To na\v se integrale zelo poenostavi. V primeru $\vq = \vq_0$ velja

\begin{equation}
	c_{x,y}^q = -1, \quad s_{x,y}^q = 0,
\end{equation}

\ni kar poenostavi na\v sa integrala v

\begin{align}
	\I_1^0 &= -\frac{1}{2t}\int \dd k_x\int \dd k_y \frac{1}
		{\big[e^{-2t\beta(c_x^k + c_y^k)} + 1\big]\big[2(c_x^k + c_y^k) -
		\hbar(\w + i\alpha)/2t\big]}, \notag \\
	\I_2^0 &= -\frac{1}{2t}\int \dd k_x\int \dd k_y \frac{1}
	{\big[e^{+2t\beta(c_x^k + c_y^k)} + 1\big]\big[2(c_x^k + c_y^k) - \hbar(\w + i\alpha)/2t\big]}.
\end{align}

Dielektri\v cna funkcija, $\e$, nas zanima v brezdimenzijski obliki (ozna\v cena z $\varepsilon$, namesto $\e$), ki jo bomo
definirali kot

\begin{align}
	\varepsilon (\vq_0, \beta, \w) -1 = (\I^0_1 - \I^0_2),
\end{align}

kjer bomo vzeli $\alpha = 0$, $t = 1$, $\mathcal{U} = 0$, $\hbar = 1$, in $x = 1$, tj. $k_x$, $\w$ itd. so vsi brezdimenzijski.

\begin{align}
	\varepsilon -1 \equiv + \frac{1}{4\pi^2}\iint\limits_\text{1. BC} \dd^2 \vk \frac{1}{2(c_x^k + c_y^k) - \w/2 -
	i\alpha/2}
		\bigg[\frac{1}{e^{-2\beta(c_x^k + c_y^k)} + 1} - \frac{1}{e^{+2\beta(c_x^k + c_y^k)} +
1}\bigg].
\end{align}

\section{Rezultati}

Dielektri\v cno funkcijo $\varepsilon - 1$ pri vektorju gnezdenja $\vq = (\pi, \pi)$ prikazuje
sl.~\ref{sl2}.

\begin{figure}[H]
	\centering
	\input{graf2.tex}
	\caption{Dielektri\v cna funkcija $\varepsilon$ v odvisnosti $\beta$ in $\omega$.
		Integriral sem numeri\v cno, z delitvijo $M = 250$, in du\v silnim
		regularizacijskim parametrom $\alpha = 0.05$. Za $\w = 0, \beta \to \infty$
		funkcija $\varepsilon -1$ divergira korensko.}
	\label{sl2}
\end{figure}

\ni Kot vidimo, $\beta$ skrbi za to, da po\v casi naraste, vendar po neki vrednosti saturira,
in se z $\beta$ ne spreminja ve\v c. Tu $\beta$ v bistvu predstavlja multiplikativni
predfaktor in opazujemo le dogajanje v odvisnosti od $\w$, kar lahko vidimo na sl.~\ref{sl1}.

\begin{figure}[H]
	\centering
	\input{graf1.tex}
	\caption{Na tej sliki vidimo, kako se spreminja dielektri\v cna funkcija v odvisnosti
		od frekvenc, pri konstantnem parametru $\beta$. Izbira integracijskih parametrov je ista, kot
		na prej\v snji sliki (sl.~\ref{sl2}).}
	\label{sl1}
\end{figure}

\begin{figure}[H]
	\centering
	\input{graf3.tex}
	\caption{Na tej sliki vidimo, kako se spreminja dielektri\v cna funkcija v odvisnosti
		od temperature pri frekvenci $\w = 0$. Parametri so isti, kot
		v sliki (sl.~\ref{sl2}).}
	\label{sl3}
\end{figure}

\subsection{Diskusija}

Integral ima dve divergenci, to je \v crta $\w = 8$ ($\beta$ je tu karkoli), in pa to\v cka $(\beta, \w) = (\infty, 0)$. 
Nekoliko nas preseneti, da imamo negativne vrednosti $\varepsilon -1$. Te seveda ne pridejo od razlike eksponentov, ampak
od prvega \v clena,

\begin{equation}
	\frac{1}{2(c_x^k + c_y^k) - \w/2 - i\alpha/2} = \frac{1}{f(\w) - i\alpha/2} = \frac{f(\w) + i\alpha/2}{f^2(\w)
		+ \alpha^2/4},
\end{equation}

\ni s katerim je $\varepsilon - 1$ sorazmeren. Ker $\alpha$ izberemo tako, da je majhen, lahko imaginarni del zanemarimo in
delamo le z realnim. \v Stevec je v tem primeru $f(\w)$, ki pa je lahko tudi negativen:

\begin{align}
	f(\w) &< 0, \notag \\
	2(c_x^k + c_y^k) - \w/2 &< 0, \notag \\
	\w/2 &> 2(c_x^k + c_y^k), \notag \\
	\w &> 4 (c_x^k + c_y^k).
\end{align}

\ni Tu za oceno vzamemo zgornjo mejo cosinusa in dobimo $\w > 4 \cdot 2 = 8$, kar nam ravno vrne frekvenco, pri kateri izraz
divergira -- seveda:

\begin{equation}
	\text{Re}\bigg\{\frac{f(\w) + i\alpha/2}{f^2(\w) + \alpha^2/4}\bigg\} = \frac{f(\w)}{f^2(\w) + \alpha^2/4}
		\approx \frac{f(\w)}{f^2(\w)} = \frac{1}{f(\w)}.
\end{equation}

\ni Ko numeri\v cno integriramo delamo v bistvu

\begin{equation}
	\varepsilon - 1 \propto \sum_i f_i (\w),
\end{equation}

\ni kjer pa so za nekatere indekse $i$ \v cleni $f_i$ divergenti -- to je $f$, ki je vzet pri $k_x$ in $k_y$ na robu
prve Brillouinove cone, saj sta tam $c_x^k = 1$ in $c_y^k = 1$, zaradi \v cesar ima tak $1/f(\w)$ tam pol. Ko ga pri\v stejemo
h kon\v cnim \v clenom ta vsota divergira. Tu je ta divergenca seveda zadu\v sena zaradi parametra $\alpha$, ki stabilizira
integracijo.

\end{document}

