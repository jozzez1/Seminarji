\documentclass[a4paper, 12pt]{article}
\usepackage[slovene]{babel}
\usepackage[T1]{fontenc}
\usepackage[utf8]{inputenc}
\usepackage{fullpage, amsmath, amssymb}

\newcommand{\vq}{
	\ensuremath{\vec{q}}
}

\newcommand{\w}{
	\ensuremath{\omega}
}

\newcommand{\e}{
	\ensuremath{\epsilon}
}

\renewcommand{\u}{
	\ensuremath{\uparrow}
}

\renewcommand{\d}{
	\ensuremath{\downarrow}
}

\newcommand{\hn}{
	\ensuremath{\hat{n}}
}

\newcommand{\oh}{
	\ensuremath{\hat{h}}
}

\newcommand{\vk}{
	\ensuremath{\vec{k}}
}

\newcommand{\koren}{
	\ensuremath{\frac{1}{\sqrt{V}}}
}

\newcommand{\vR}{
	\ensuremath{\vec{R}_{ij}}
}

\newcommand{\s}{
	\ensuremath{\sigma}
}

\newcommand{\ex}{
	\ensuremath{\vec{e}_x}
}

\newcommand{\ey}{
	\ensuremath{\vec{e}_y}
}

\newcommand{\fV}{
	\ensuremath{\frac{1}{V}}
}

\renewcommand{\ni}{
	\noindent
}

\newcommand{\pred}{
	\ensuremath{\frac{e^2}{\e_0 q^2 V}}
}

\newcommand{\dd}{
	\ensuremath{\text{d}}
}

\newcommand{\I}{
	\ensuremath{\mathcal{I}}
}

\begin{document}

\section{Navodilo}

Atomi z enim valen\v cnim elektronom na povr\v sini tvorijo kvadratno mre\v zo. Elektronski pas
popi\v semo v pribli\v zku tesne vezi. Zapi\v si dielektri\v cno odzivno funkcijo za tako kovino.
Razi\v s\v ci (frekven\v cno, momentno, temperaturno) obna\v sanje dielektri\v cne konstante v
bli\v zini valovnih vektorjev (nesting) gnezdenja.

\section{Naloga}

V skripti, poglavje 8, imamo na voljo ti. \emph{Lindhardovo ena\v cbo},

\begin{equation}
	\e (\vq, \w) = 1 - \frac{e^2}{\e_0 q^2 V} \sum_{\vec{k}} \frac{f_{\vec{k}}^0 -
		f^0_{\vec{k}+\vq}} {\e_{\vec{k}} - \e_{\vec{k}+\vq} + \hbar\w + i\hbar\alpha}
	\label{lindhard}
\end{equation}

V izpeljavi le-te smo delali s splo\v sno valovno funkcijo, tj. velja tudi za nas. Ugotoviti moramo
le energijsko disperzijo, $\e_{\vq}$. Za to moramo re\v siti Hamiltonian. Za re\v sevanje bomo
uporabili ti. \emph{Hubbardov model}, ki upo\v steva tesno vez.

Uporabimo operatorje polja

\begin{align}
	\psi_{i,j,\sigma}(\vec{r}) &= \sum_n c_{n,\sigma} \phi_n (\vec{R}_{ij} + \vec{r}),\notag  \\
	\psi^\dagger_{i,j,\sigma}(\vec{r}) &= \sum_n c^\dagger_{n,\sigma} \phi_n (\vec{R}_{ij} +
		\vec{r}),
	\label{polje}
\end{align}

\ni ki ustvarijo (uni\v cijo) delec s spinom $\sigma$ na mre\v zi -- $\vec{R}_{ij}$ je vozli\v sce
kristalne mre\v ze (tj. lokacija atoma), $\vec{r}$ pa je odmik elektrona v okolici. Delamo v
adiabatnem pribli\v zku, tj. mre\v za je za na\v se potrebe stacionarna. Funkcije $\phi_n (\vec{r})$
so znamenite Wannierove funkcije. Vendar, ker imamo samo en elektron, lahko zapi\v semo tako:

\begin{align}
	\psi_{ij,\sigma,m}(\vec{r}) &= \sum_n c_{n,\sigma} \phi_n (\vec{R}_{ij} + \vec{r})
		\delta_{mn} \notag \\
	&= c_{m,\sigma} \phi_m (\vec{R}_{ij} + \vec{r}).
\end{align}

\ni Ker imamo kreacijsko anihilacijske operatorje le za en delec, jih bomo opremili druga\v ce -- indeks
ne bo ve\v c pomenil stanje, ampak lokacijo na mre\v zi:

\begin{align}
	\psi_{ij,\sigma,n} (\vec{r}) &= c_{ij,\sigma} \phi_n (\vec{R}_{ij} + \vec{r}), \notag \\
	\psi^\dagger_{ij,\sigma,n} (\vec{r}) &= c^\dagger_{ij,\sigma} \phi_n (\vec{R}_{ij} + \vec{r}).
	\label{polje2}
\end{align}

Pribli\v zek tesne vezi pomeni, da imamo neni\v celne prispevke le za najbli\v zje sosede in sicer
dobimo `$-t$' na par sosedov. V na\v sem
primeru so za energijo vozli\v sca $\oh_{ij}$ to

\begin{equation}
	\oh_{ij} = -t \sum_{\sigma} [c^\dagger_{i,j,\sigma} (c_{i,j+1,\sigma} + c_{i+1,j,\sigma} +
		c_{i,j-1,\sigma} + c_{i-1,j,\sigma}) + \text{h.c.}]
	\label{oh}
\end{equation}

\ni Celotno kineti\v cno energijo lahko zapi\v semo kot vsoto po vseh vozli\v s\v cih, vendar moramo biti
pazljivi: na tak na\v cin bomo vsako "`vez"' \v steli dvakrat\footnote{To v splo\v snem ni res, zaradi
efektov na robu. Vendar mi delamo s periodi\v cnimi robnimi pogoji, zato je vse v redu.}, zato moramo
vsoto obte\v ziti z ustrezno multiplikativno konstanto, tj. $1/2$:

\begin{equation}
	\hat{H} = \frac{1}{2}\sum_{ij} \oh_{ij} + \mathcal{U}\sum_{ij} \hn_{ij,\u} \hn_{ij,\d}.
\end{equation}

\ni Operator $\hn$ je operator \v stetja. Zapisal sem ga s stre\v sico, da se ga lo\v ci od indeksa
energijskega stanja.

Imamo dva na\v cina, kako nadaljujemo z re\v sevanjem problema: $h_{ij}$ lahko izra\v cunamo v
matri\v cni obliki in dobimo energijski spekter, ali pa Hamiltonian pretvorimo v impulzno bazo.

Za pretvorbo v impulzno bazo poglejmo najprej operatorje polja, iz en.~\eqref{polje}:
\begin{equation}
	\psi_{ij,\sigma} (\vec{r}) = \frac{1}{\sqrt{V}}\sum_{\vec{k}}
		c_{\vec{k},\sigma}\varphi_{\vec{k}}(\vec{R}_{ij} + \vec{r}) = 
		\frac{1}{\sqrt{V}}\sum_{\vec{k}} c_{\vec{k},\sigma} e^{i\vec{k}\cdot\vec{R}_{ij}}
		\varphi_{\vec{k}}(\vec{r})
\end{equation}

\ni Po primerjavi s staro izjavo hitro uganemo, da je
\begin{align}
	c_{ij,\sigma} &= \koren \sum_{\vk} c_{\vk,\sigma} e^{i\vk\cdot\vec{R}_{ij}}, \\
	c^\dagger_{ij,\sigma} &= \koren \sum_{\vk} c^\dagger_{\vk, \sigma} e^{-i\vk\cdot\vec{R}_{ij}}.
\end{align}

\ni Sedaj transformirajmo \v clene ki nastopajo v en.~\eqref{oh}. Recimo $\vec{R}_{i,j\pm1} = \vec{R}_{ij}
\pm a\vec{e}_y$, $a > 0$, $a$ je razdalja med ioni na mre\v zi. Ker je mre\v za kvadratna, lahko
isto naredimo v drugi smeri, tj. $\vec{R}_{i\pm1, j} = \vec{R}_{ij} \pm a\vec{e}_x$.

\begin{align}
	\sum_\s\sum_{i,j} c^\dagger_{i,j,\sigma}c_{i,j \pm 1,\sigma} 
	&=\fV \sum_\s\sum_{i,j}\sum_{\vk} c^\dagger_{\vk,\s} e^{-i\vk \cdot \vR}
		\sum_{\vq}c_{\vq,\s} e^{i\vq\cdot\vec{R}_{i,j \pm 1}}, \notag \\
	&=\fV \sum_\s \sum_{\vk,\vq} \sum_{i,j}c^\dagger_{\vk,\s} c_{\vq,\s} e^{-i(\vk - \vq)\cdot \vR}
		e^{\pm i\vq \cdot a \ey}, \notag \\
	&=\fV \sum_\s \sum_{\vk, \vq} c^\dagger_{\vk, \s} c_{\vq, \s} e^{\pm i q_y a}\ 
		V \delta_{\vk,\vq}\ ,\notag \\
	&= \sum_{\vk} \sum_\s c^\dagger_{\vk, \s} c_{\vk, \s} e^{\pm ik_y a}, \notag \\
	&= \sum_{\vk} (\hat{n}_{\vk,\u} + \hat{n}_{\vk,\d})e^{\pm ik_y a}, \notag \\
	&= \sum_{\vk} \hat{n}_{\vk}\ e^{\pm ik_y a}.
\end{align}

\ni Opazimo, da je analogno za primer, ko imamo $c^\dagger_{i,j,\s} c_{i\pm 1,j\s}$ in sicer
\begin{equation}
	\sum_\s \sum_{i,j} c^\dagger_{i,j,\s} c_{i\pm 1,j\s} = \sum_{\vk}
		\hat{n}_{\vk}\ e^{\pm ik_x a}.
\end{equation}

\ni Torej lahko kineti\v cni del $\hat{H}_\text{kin}$ zapi\v semo kot

\begin{align}
	\hat{H}_\text{kin} &= -\frac{t}{2} \sum_{\vk} \hat{n}_{\vk}\ 2(e^{ik_xa} + e^{-ik_xa} +
		e^{ik_ya} + e^{-ik_ya}), \notag \\
	&= \sum_{\vk} \underbrace{(-2t) \hat{n}_{\vk} \big[\cos k_xa + \cos k_ya \big]}_{\oh_{\vk}}.
\end{align}

\ni Dodatni faktor `$2$' se prikrade notri od hermitsko konjugiranih \v clenov.

Torej operator kineti\v cne energije pri momentu $\vk$,

\begin{equation}
	\oh_{\vk} = -2t\ \hat{n}_{\vk} \big(\cos k_x a + \cos k_y a),
\end{equation}

\ni oz. \v ce gledamo $\e_{\vk}^\text{kin}$,

\begin{equation}
	\e_{\vk}^\text{kin} = -2t (\cos k_x a + \cos k_y a).
\end{equation}

Transformirati moramo \v se interakcijski \v clen, da se dokon\v cno prepri\v camo o njegovi naravi:

\begin{align}
	\hat{H}_\text{int} &= \mathcal{U}\sum_{i,j}\hat{n}_{i,j,\u}\hat{n}_{i,j,\d}
		= \mathcal{U}\sum_{i,j} c^\dagger_{i,j,\u}c_{i,j,\u} c^\dagger_{i,j,\d} c_{i,j,\d}
		\notag \\
	&= \frac{\mathcal{U}}{V^2} \sum_{\vk,\vq}\sum_{\vk^\prime,\vq\ ^\prime}
		c^\dagger_{\vk,\u}c_{\vq,\u}c^\dagger_{\vk',\d}c_{\vq\ ',\d} \sum_{i,j}
		e^{-i \vR\cdot (\vk + \vk' - \vq - \vq\ ')} \notag \\
	&= \frac{\mathcal{U}}{V^2} \sum_{\vk,\vq}\sum_{\vk^\prime,\vq\ ^\prime}
		c^\dagger_{\vk,\u}c_{\vq,\u}c^\dagger_{\vk',\d}c_{\vq\ ',\d}
		V \delta_{\vq\ ',\ \vk + \vk' - \vq} \notag  \\
	&= \frac{\mathcal{U}}{V} \sum_{\vk, \vk', \vq} c^\dagger_{\vk,\u} c_{\vq, \u}
		c^\dagger_{\vk', \d} c_{\vk + \vk' - \vq, \d}
\end{align}

\ni Na tem mestu upo\v stevamo identiteto iz kurza vi\v sje kvantne mehanike,

\begin{equation}
	\hat{n}_{\vq,\s} = \frac{1}{V}\sum_{\vk} c^\dagger_{\vk, \s} c_{\vk + \vq, \s},
\end{equation}

\ni ki nam vrne (po ustreznih fermionskih zamenjavah)

\begin{equation}
	\hat{H}_\text{int} = \mathcal{U} \sum_{\vk,\vq} c^\dagger_{\vk,\u}
		\hat{n}_{\vk - \vq,\d} c_{\vq, \u}.
\end{equation}

\ni To \v zal ni enako ni\v c. Vendar lahko ta \v clen prepi\v semo
nazaj v kreacijsko-anihilacijske operatorje in dobimo

\begin{equation}
	\hat{H}_\text{int} = \mathcal{U}\sum_{\vk, \vq} c^\dagger_{\vk, \u} c^\dagger_{\vk
		-\vq, \d} c_{\vk - \vq, \d} c_{\vq, \u},
\end{equation}

\ni koder lahko sedaj uporabimo Wickov izrek, ki nam vrne

\begin{equation}
	\hat{H}_\text{int} = \mathcal{U}\sum_{\vk} \hat{n}_{\vk,\u} \hat{n}_{\vec{0},\d}.
\end{equation}

\ni Vendar gremo lahko isto z druge smeri, na za\v cetku se pa\v c ne iznebimo $\vq\ '$ ampak kak\v snega
drugega vektorja in dobimo

\begin{equation}
	\hat{H}_\text{int} = \mathcal{U}\sum_{\vk} \hat{n}_{\vec{0},\u} \hat{n}_{\vk, \d},
\end{equation}

\ni kar nam vrne identiteto

\begin{equation}
	\hat{n}_{\vk, \u} \hat{n}_{\vec{0}, \d} = \hat{n}_{\vec{0},\u} \hat{n}_{\vk, \d},
\end{equation}

\ni ki jo bomo uporabili v naslednjih en.~\eqref{zamenjava}. Stanja so zapolnjena v Fermijevi krogli, ki je
popolnoma polna z ostrim robom v osnovnem stanju, v vi\v sjih vzbujenih stanjih pa bolj napihnjena in preluknjana.
Vseeno recimo, da $\hat{n}_{\vec{0}, \u} = 1$ in $\hat{n}_{\vec{0}, \d} = 1$, kar je res za nizke temperature.
Potem sledi

\begin{equation}
	\hat{H}_\text{int} = \mathcal{U}\frac{1}{2}\sum_{\vk} (\hat{n}_{\vk, \u} \hat{n}_{\vec{0}, \d}
		+ \hat{n}_{\vk, \u} \hat{n}_{\vec{0}, \d}) = \frac{\mathcal{U}}{2}\sum_{\vk} (\hat{n}_{\vk, \u}
		\overbrace{\hat{n}_{\vec{0}, \d}}^1 + \overbrace{\hat{n}_{\vec{0},\u}}^1 \hat{n}_{\vk, \d}) =
		\frac{\mathcal{U}}{2}\sum_{\vk} \hat{n}_{\vk}
	\label{zamenjava}
\end{equation}

Sedaj imamo vse, kar potrebujemo za $\e_{\vk}$:

\begin{equation}
	\e_{\vk} = -2t (\cos k_x a + \cos k_y a) + \mathcal{U}/2.
	\label{eps}
\end{equation}

\ni Za $f_{\vk}^0$ upo\v stevamo Fermi-Diracovo porazdelitev, tj.

\begin{equation}
	f_{\vk}^0 = \frac{1}{e^{\beta(\e_{\vk} - \e_F)} + 1}.
	\label{favela}
\end{equation}

\ni Izraza \eqref{eps} in \eqref{favela} vstavimo v en.~\eqref{lindhard}. Na\v so vsoto pretvorimo
v integral

\begin{align}
	\e(\vq, \w) &= 1 - \pred \Big(\frac{a}{2\pi}\Big)^2 \underbrace{ \iint\limits_\text{1. BC}\dd^2 \vk \frac{f^0_{\vk} -
		f^0_{\vk+\vq}}{\e_{\vk} - \e_{\vk+\vq} + \hbar\w + i\hbar\alpha}}_{\I}.
\end{align}

\ni Integral $\I$ bomo raz\v clenili v $\I = \I_1 - \I_2$, kjer sta

\begin{align}
	\I_1 &= \iint\limits_\text{1. BC}\dd^2 \vk \frac{1}{(e^{\beta(\e_{\vk} - \e_F)} + 1)
		(\e_{\vk} - \e_{\vk + \vq} + \hbar\w + i\hbar\alpha)}, \notag \\
	\I_2 &= \iint\limits_\text{1. BC}\dd^2 \vk \frac{1}{(e^{\beta(\e_{\vk+\vq} - \e_F)} + 1)
		(\e_{\vk} - \e_{\vk + \vq} + \hbar\w + i\hbar\alpha)}.
	\label{integrali}
\end{align}

\ni Da bo manj pisanja bomo kar vnaprej uvedli oznake

\begin{align}
	c_{x,y}^k \equiv \cos k_{x,y} a, &\quad	c_{x,y}^q \equiv \cos q_{x,y} a, \notag \\
	s_{x,y}^k \equiv \sin k_{x,y} a, &\quad s_{x,y}^q \equiv \sin q_{x,y} a.
\end{align}

\ni Torej

\begin{equation}
	\cos [(k_{x,y} + q_{x,y}) a] = c_{x,y}^k c_{x,y}^q - s_{x,y}^k s_{x,y}^q.
\end{equation}

\ni To uporabimo v en.~\eqref{integrali} in dobimo, 

\begin{align}
	\I_1 &= -\frac{1}{2t}\iint \dd^2 \vk \frac{1}
		{\big[e^{-2t\beta(c_x^k + c_y^k) + \beta\mathcal{U}/2 - \beta\e_F} + 1\big]
		\big[c_x^k + c_y^k + s_x^k s_x^q + s_y^k s_y^q - c_x^k c_x^q - c_y^k c_y^q -
		\hbar(\w + i\alpha)/2t\big]}, \notag \\
	\I_2 &= -\frac{1}{2t}\iint \dd^2 \vk \frac{1}
		{\big[e^{-2t\beta (c_x^k c_x^q + c_y^k c_y^q - s_x^k s_x^q - s_y^k s_y^q) + 
		\beta\mathcal{U}/2 - \beta\e_F} + 1\big][ \ldots ]}.
\end{align}

Sedaj si moramo ogledati obna\v sanje v bli\v zini vektorjev gnezdenja, tj. $\vq = \vq_0 =
\pi/a (1,1)$. To na\v se integrale zelo poenostavi. V primeru $\vq = \vq_0$ velja

\begin{equation}
	c_{x,y}^q = -1, \quad s_{x,y}^q = 0,
\end{equation}

\ni kar poenostavi na\v sa integrala v

\begin{align}
	\I_1^0 &= -\frac{1}{2t}\int \dd k_x\int \dd k_y \frac{1}
		{\big[e^{-2t\beta(c_x^k + c_y^k) + \beta\mathcal{U}/2 - \beta\e_F} + 1\big]\big[
		2(c_x^k + c_y^k) - \hbar(\w + i\alpha)/2t\big]}, \notag \\
	\I_2^0 &= -\frac{1}{2t}\int \dd k_x\int \dd k_y \frac{1}
	{\big[e^{+2t\beta(c_x^k + c_y^k) + \beta\mathcal{U}/2 - \beta\e_F} + 1\big]\big[
		2(c_x^k + c_y^k) - \hbar(\w + i\alpha)/2t\big]}.
\end{align}

\ni Izra\v cunati moramo \v se energijo vakuuma, saj je to pribli\v zno kemijski potencial in nastopa v na\v si
porazdelitveni funkciji. V prvi Brillouinovi coni so izoenergijske \v crte za energijo ni\v c tiste, ki te\v cejo od
to\v cke $A (0, -\pi/a)$ do $B (\pi/a, 0)$, od $B$ do $C (0, \pi/a)$, od $C$ do $D (-\pi/a, 0)$ in potem \v se $D$ od $A$.

To nam predstavlja kvadrat, ki je v\v crtan robu prve Brillouinove cone in ima polovi\v cno povr\v sino le-te.
Namesto, da integriramo po celem podro\v cju, lahko zaradi simetrije integracijskih mej in simetrije integranda na
zamenjavo $k_x$ in $k_y$ integriramo le po enem izmed kvadrantov, nato pa pomno\v zimo s 4. Torej pri\v cnemo z

\begin{align}
	\e_F &= \bigg(\frac{a}{2\pi}\bigg)^2 \iint\limits_{ABCD} \dd k_x \dd k_y \big[-2t (\cos k_x a + \cos k_y a)
		\big] \notag \\
	&= -2t \cdot 4 \bigg(\frac{a}{2\pi}\bigg)^2 \int_0^{\pi/a} \dd k_x \int_0^{\pi/a - k_x} \dd k_y (\cos k_x a +
		\cos k_y a) \notag \\
	&= -2t \frac{a^2}{\pi^2} \int_0^{\pi/a} \dd k_x \bigg[\Big(\frac{\pi}{a} - k_x\Big)\cos k_x a +
		\frac{1}{a}\sin(\pi - k_x a)\bigg] \notag \\
	&= -2t \frac{a^2}{\pi^2} \int_0^{\pi/a} \dd k_x \bigg[\Big(\frac{\pi}{a} - k_x\Big)\cos k_x a +
		\frac{1}{a}\sin k_x a \bigg] \notag \\
	&= -2t \frac{a^2}{\pi^2} \bigg\{\Big[\frac{\pi}{a^2}\sin k_x a - \frac{1}{a^2}\cos k_x a\Big]_0^{\pi/a} -
		\int_0^{\pi/a} \dd k_x k_x \cos k_x a\bigg\} \notag \\
	&= -2t \frac{a^2}{\pi^2} \cdot \frac{1}{a^2} \bigg\{\pi \underbrace{(\sin\pi - \sin 0)}_{=\ 0} - \cos\pi + \cos 0 - a^2
		\int_0^{\pi/a} \dd k_x k_x \cos k_x a\bigg\} \notag \\
	&= -\frac{2t}{\pi^2} \bigg[2 - a^2 \int_0^{\pi/a} \dd k_x k_x \cos k_x a\bigg] = -\frac{2t}{\pi^2}
		\bigg[2 - \int_0^\pi \dd x x \cos x\bigg] \notag \\
	&= -\frac{2t}{\pi^2}\big[2 - (\cos x + x\sin x)\big]_0^\pi = -\frac{2t}{\pi^2} \big[2 - (-2)\big] =
		-2t \frac{4}{\pi^2}.
\end{align}

Dielektri\v cna funkcija, $\e$, nas zanima v brezdimenzijski obliki (ozna\v cena z $\varepsilon$, namesto $\e$), ki jo bomo
definirali kot

\begin{align}
	\varepsilon (\vq_0, \beta, \w) = 1 - x (\I^0_1 - \I^0_2),
\end{align}

kjer bomo vzeli $\alpha = 0$, $t = 1$, $\mathcal{U} = 0$, $\hbar = 1$, in $x = 1$, tj. $k_x$, $\w$ itd. so vsi brezdimenzijski.

\begin{align}
	\varepsilon = 1 + \frac{1}{4\pi^2}\iint\limits_\text{1. BC} \dd^2 \vk \frac{1}{2(c_x^k + c_y^k) - \w/2}
		\bigg[\frac{1}{e^{-2\beta(c_x^k + c_y^k - 4/\pi^2)} + 1} - \frac{1}{e^{+2\beta(c_x^k + c_y^k + 4/\pi^2)} + 1}\bigg]
\end{align}

\end{document}
