\documentclass[a4paper, 12pt]{article}
\usepackage[slovene]{babel}
\usepackage[T1]{fontenc}
\usepackage[utf8]{inputenc}
\usepackage{fullpage, amsmath, amssymb}

\newcommand{\vq}{
	\ensuremath{\vec{q}}
}

\newcommand{\w}{
	\ensuremath{\omega}
}

\newcommand{\e}{
	\ensuremath{\epsilon}
}

\renewcommand{\u}{
	\ensuremath{\uparrow}
}

\renewcommand{\d}{
	\ensuremath{\downarrow}
}

\newcommand{\hn}{
	\ensuremath{\hat{n}}
}

\newcommand{\oh}{
	\ensuremath{\hat{h}}
}

\newcommand{\vk}{
	\ensuremath{\vec{k}}
}

\newcommand{\koren}{
	\ensuremath{\frac{1}{\sqrt{V}}}
}

\begin{document}

\section{Navodilo}

Atomi z enim valen\v cnim elektronom na povr\v sini tvorijo kvadratno mre\v zo. Elektronski pas
popi\v semo v pribli\v zku tesne vezi. Zapi\v si dielektri\v cno odzivno funkcijo za tako kovino.
Razi\v s\v ci (frekven\v cno, momentno, temperaturno) obna\v sanje dielektri\v cne konstante v
bli\v zini valovnih vektorjev (nesting) gnezdenja.

\section{Naloga}

V skripti, poglavje 8, imamo na voljo ti. \emph{Lindhardovo ena\v cbo},

\begin{equation}
	\e (\vq, \w) = 1 - \frac{e^2}{\e_0 q^2 V} \sum_{\vec{k}} \frac{f_{\vec{k}}^0 - f^0_{\vec{k}+\vq}}
		{\e_{\vec{q}} - \e_{\vec{k}+\vq} + \hbar\w + i\hbar\alpha}
\end{equation}

V izpeljavi le-te smo delali s splo\v sno valovno funkcijo, tj. velja tudi za nas. Ugotoviti moramo
le energijsko disperzijo, $\e_{\vq}$. Za to moramo re\v siti Hamiltonian. Za re\v sevanje bomo
uporabili ti. \emph{Hubbardov model}, ki upo\v steva tesno vez.

Uporabimo operatorje polja

\begin{align}
	\psi_{i,j,\sigma}(\vec{r}) &= \sum_n c_{n,\sigma} \phi_n (\vec{R}_{ij} + \vec{r}),\notag  \\
	\psi^\dagger_{i,j,\sigma}(\vec{r}) &= \sum_n c^\dagger_{n,\sigma} \phi_n (\vec{R}_{ij} +
		\vec{r}),
	\label{polje}
\end{align}

ki ustvarijo (uni\v cijo) delec s spinom $\sigma$ na mre\v zi -- $\vec{R}_{ij}$ je vozli\v sce
kristalne mre\v ze (tj. lokacija atoma), $\vec{r}$ pa je odmik elektrona v okolici. Delamo v
adiabatnem pribli\v zku, tj. mre\v za je za na\v se potrebe stacionarna. Funkcije $\phi_n (\vec{r})$
so znamenite Wannierove funkcije. Vendar, ker imamo samo en elektron, lahko zapi\v semo tako:

\begin{align}
	\psi_{ij,\sigma,m}(\vec{r}) &= \sum_n c_{n,\sigma} \phi_n (\vec{R}_{ij} + \vec{r})
		\delta_{mn} \notag \\
	&= c_{m,\sigma} \phi_m (\vec{R}_{ij} + \vec{r}).
\end{align}

Ker imamo kreacijsko anihilacijske operatorje le za en delec, jih bomo opremili druga\v ce -- indeks
ne bo ve\v c pomenil stanje, ampak lokacijo na mre\v zi:

\begin{align}
	\psi_{ij,\sigma,n} (\vec{r}) &= c_{ij,\sigma} \phi_n (\vec{R}_{ij} + \vec{r}), \notag \\
	\psi^\dagger_{ij,\sigma,n} (\vec{r}) &= c^\dagger_{ij,\sigma} \phi_n (\vec{R}_{ij} + \vec{r}).
	\label{polje2}
\end{align}

Pribli\v zek tesne vezi pomeni, da imamo neni\v celne prispevke le za najbli\v zje sosede in sicer
dobimo `$-t$' na soseda. V na\v sem
primeru so za energijo vozli\v sca $h_{ij}$ to

\begin{equation}
	\oh_{ij} = -t \sum_{\sigma} [c^\dagger_{i,j,\sigma} (c_{i,j+1,\sigma} + c_{i+1,j,\sigma} +
		c_{i,j-1,\sigma} + c_{i-1,j,\sigma}) + \text{h.c.}]
\end{equation}

Celotno kineti\v cno energijo lahko zapi\v semo kot vsoto po vseh vozli\v s\v cih, vendar moramo biti
pazljivi: na tak na\v cin bomo vsako "`vez"' \v steli dvakrat\footnote{To v splo\v snem ni res, zaradi
efektov na robu. Vendar mi delamo s periodi\v cnimi robnimi pogoji, zato je vse v redu.}, zato moramo
vsoto obte\v ziti z ustrezno multiplikativno konstanto, tj. $1/2$:

\begin{equation}
	\hat{H} = \frac{1}{2}\sum_{ij} \oh_{ij} + \mathcal{U}\sum_{ij} \hn_{ij,\u} \hn_{ij,\d}.
\end{equation}

Operator $\hn$ je operator \v stetja. Zapisal sem ga s stre\v sico, da se ga lo\v ci od indeksa
energijskega stanja. Na vsak ion imamo le en elektron, tj. bo interakcijski \v clen neutola\v zljivo
enak ni\v c.

Imamo dva na\v cina, kako nadaljujemo z re\v sevanjem problema: $h_{ij}$ lahko izra\v cunamo v
matri\v cni obliki in dobimo energijski spekter, ali pa Hamiltonian pretvorimo v impulzno bazo.

Za pretvorbo v impulzno bazo poglejmo najprej operatorje polja, iz en.~\eqref{polje}
\begin{equation}
	\psi_{ij,\sigma} (\vec{r}) = \frac{1}{\sqrt{V}}\sum_{\vec{k}}
		c_{\vec{k},\sigma}\varphi_{\vec{k}}(\vec{R}_{ij} + \vec{r}) = 
		\frac{1}{\sqrt{V}}\sum_{\vec{k}} c_{\vec{k},\sigma} e^{i\vec{k}\cdot\vec{R}_{ij}}
		\varphi_{\vec{k}}(\vec{r})
\end{equation}

Po primerjavi s staro izjavo hitro ugotovimo, da je
\begin{align}
	c_{ij,\sigma} &= \koren \sum_{\vk} c_{\vk,\sigma} e^{i\vk\cdot\vec{R}_{ij}}, \\
	c^\dagger_{ij,\sigma} &= \koren \sum_{\vk} c^\dagger_{\vk, \sigma} e^{-i\vk\cdot\vec{R}_{ij}}.
\end{align}

\end{document}
