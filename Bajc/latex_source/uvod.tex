\chapter{Uvod}

Standardni model je ta hip najbolj precizna teorija v fiziki. Delce, za katere ta hip empirično
verjamemo, da so nedeljivi, smiselno razvrsti in opiše njihove interakcije ter tako pojasni
makroskopske pojave s pomočjo najosnovnejših pojavov, ki se dogajajo na zelo majhni skali.

Kljub dobrim in zelo lepin lastnostim standardnega modela, obstajajo eksperimentalne indikacije, da
standardni model ni popoln. S teoretičnega vidika bi zelo težko pojasnili izmerjeno asimetrijo $CP$
in pa nesorazmernost v vsebnosti snovi proti vsebnosti anti-snovi v našem vesolju. To bi lahko pojasnili
z razpadom protona, ki ni dovoljen v standardnem modelu. Prav tako zahteva zahteva, da so nevtrini
brezmasni, ni več podprta s strani eksperimentov, a standardni model jih ne more umestiti drugače.
Še ena hiba standardnega modela je ta, da ne more pojasniti kvantizacije električnega naboja in ne more
predvideti števila generacij fermionov.

Standardni model med fermionskimi polji opisuje tri fundamentalne interakcije in njihove nosilce, ki so
bozoni. Te tri interakcije so močna, šibka in elektromagnetna. Ker so osnovni delci izjemno lahki,
gravitacijo zanemarimo, ta bi sicer bila četrta fundamentalna interakcija. Kako naj torej pojasnimo
npr. razpad protona? Lahko ugibamo o obstoju nove, še neodkrite interakcije, katere vloga postane
bolj očitna pri višjih energijah. Poleg bozonov, ki so nosilci interakcij obstajajo tudi ti. Higgsovi
bozoni, ki dajo maso prek Higgsovega mehanizma, na račun tega, da interakcijo zadušijo na nizkih
energijskih skalah.  Tako bi lahko dali nevtrinom mase in omogočili protonski razpad.

Tak način razmišljanja je odprl novo raziskovalno področje v teoriji osnovnih delcev, ki se imenuje
teorija poenotenja\footnote{\emph{ang.} Grand Unified Theory -- GUT.}, ki skuša iti korak dlje -- namesto
tega, da bi dodali le eno novo interakcijo, trdimo da v osnovi obstaja le ena sama, ki se zaradi
prisotnosti Higgsovih polj prikazuje v treh oblikah. Tak pristop nam omogoča poenostavitev, saj poleg
poenotenja znanih interakcij, poenotimo tudi fermionska polja. S seboj pa prinese tudi komplikacije,
saj ima tipično več nosilcev interakcij -- te uporabimo za teoretično pojasnilo eksperimentov, ki se
jih ni dalo znotraj standardnega modela. S pravo teorijo poenotenja bi lahko pojasnili vse to, kar
nam manjka.

Na žalost, pa je tako teorijo zelo težko preveriti. Poenotenje se predvidoma zgodi na energijski skali
$\Lambda_\text{GUT} \approx 2 \cdot 10^{16}$ GeV~\cite{mohapatra}. Zaenkrat smo z eksperimenti šele
pričeli z raziskovanjem področja od $100$ GeV do $1000$ GeV kar je zaenkrat inženirski podvig znanih
prijemov. Obstaja realna možnost, da ne bomo nikdar mogli potrditi oz. ovreči teorije poenotenja.

Vendar, če bomo opazili protonski razpad, bomo zmožni prav tega. In ne le to -- prek študije
protonskega razpada bomo mogli tudi povedati katera interakcija je tista v visokih energijah, tj. s
katero umeritveno grupo jo lahko opišemo. To je možno narediti z eksperimenti pri nizkih energijah,
ki se odvijajo na Japonskem, v Super-Kamiokande. Odsotnost protonskega razpada pa žal teorije poenotenja
ne more ovreči, saj so obstajajo tako teorije, ki ga vsebujejo, kot tudi, ki ga nimajo.

V teoriji poenotenja so glavni kandidati za umeritveno grupo poenotene interakcije predvsem $SU(5)$,
$SO(10)$ in $E_6$. Prva teorija poenotenja je bila narejena ravno prek $SU(5)$, katera pa ima še vedno
sterilne desnoročne nevtrine, ki bi jih potrebovali za maso. Pričakujemo, da tudi slednji sodelujejo
v interakcijah. To lahko storimo z umeritveno grupo $SO(10)$, ki pa nam da pravo bogastvo. Poleg tega,
da kvantizira električni naboj, je to tudi teorija "`levo-desno"' simetričnega vesolja. Pri nizkih
energijah imajo namreč ti. levoročni delci poseben status. Še vedno pa to ni teorija, ki bi napovedala
število generacij -- za to bi potrebovali vsaj $SU(9)$, ki ima za podgrupo
$SU(3)_L\times SU(3)_C\times U(1)_X$, kot so to storili v \cite{SU9}.

Tudi mi bomo tekom tega dela uporabili model poenotenja prek $SO(10)$. Osredotočili se bomo predvsem
na problem nevtrinskih mas in mešalne matrike v leptonskem sektorju, to je matrike
Pontecorvo-Maki-Nakagawa-Sakata (PMNS), ki je analogija mešalne matrike v kvarkovskem sektorju,
Cabbibo-Kobayashi-Maskawa (CKM).

To delo je v grobem razdeljeno na tri dele. V prvem delu bomo na kratko povzeli standardni model,
teorijo poenotenja $SO(10)$ in mehanizem, ki da nevtrinom mase. V drugem delu bomo predstavili
teoretičen model, ki ga bomo preučili tekom tega dela. Tretji del bo namenjen diskusiji rezultatov.
Delali bomo v enotah $\hbar = c = \varepsilon_0 = 1$.
