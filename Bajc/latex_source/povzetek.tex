\newpage

\thispagestyle{empty}

\centerline{}

\vfill

\centerline{\bf Izjava o avtorstvu in objavi elektronske oblike zaključnega dela:}

\bigskip
\bigskip

\noindent

Podpisani Jože Zobec, rojen 30.~10.~1989 v Ljubljani, izjavljam:
\begin{itemize}
\item{da je magistrsko delo z naslovom \emph{Mase fermionov standardnega modela znotraj minimalne teorije
	poenotenja 	$SO(10)$ z dodatnimi težkimi generacijami fermionov} rezultat mojega samostojnega dela
	pod mentorstvom prof. dr. Boruta Bajca,}
\item{da je tiskani izvod dela identičen z elektronskim izvodom in}
\item{da Fakulteti za matematiko in fiziko Univerze v Ljubljani dovoljujem objavo \\ elektronske oblike
	svojega dela na spletnih straneh Repozitorjia Univerze v Ljubljani.}
\end{itemize}
\bigskip
\bigskip
\noindent
Ljubljana, \today \hfill Jože Zobec
\vfill
\newpage
\thispagestyle{empty}
\null
\newpage


\thispagestyle{empty}
\vspace{5cm}
\section*{Povzetek}

V tej magistrski nalogi predstavimo minimalen model znotraj teorije poenotenja $SO(10)$, ki da mase lahkim
fermionom. To naredimo tako, da poleg treh generacij, ki so v 16-dimenzionalni upodobitvi, dodamo še dve novi
težki generaciji fermionov, od katerih je ena v 16-dimenzionalni upodobitvi ena pa v dualni 16-dimenzionalni
upodobitvi. Tako lahko sestavimo minimalen model, kjer zadošča da se lahki fermioni sklapljajo le s Higgsovim
poljem v dualni 126-dimenzionalni upodobitviji. Neničelne vakuumske pričakovane vrednosti tega polja lahko
izberemo tako, da damo mase vsem fermionom standardnega modela. Glavna motivacija je predvsem napoved mas
nevtrinov, ki dobijo mase z gugalničnim mehanizmom tipa I in tipa II. Sprva bomo testirali model v približku
dveh generacij, nato pa posplošili na tri.

\vspace{10pt}
\paragraph{Ključne besede:} velika teorija poenotenja, $SO(10)$, gugalnični mehanizem, mase nevtrinov, mešalni
koti nevtrinov, mešalni koti kvarkov.

\vfill

\newpage
\thispagestyle{empty}
\null
\newpage
\vfill

\thispagestyle{empty}
\vspace{5cm}
\section*{Abstract}

In this thesis we consider a minimal $SO(10)$ grand unified model, by which we give masses to the standard
model fermions. The main premise of the model is that by adding to the standard model fermions in the
three 16-dimentsional representations two new generations in 16-dimensional and dual 16-dimensional
representation, we can now couple the light fermions with just one Higgs field in dual 126-representation.
The main motivation of this work is predicting the neutrino masses, which are given through see-saw mechanism
of type I and type II. At first we will test this model in two-generation approximation and then we will try to
predict using all three generations.

\vspace{10pt}
\paragraph{Keywords:} grand unification theory, $SO(10)$, see-saw mechanism,  neutrino masses, neutrino mixing
angles, quark mixing angles.

\vspace{10pt}
\paragraph{PACS:} 12.10.Kt, 12.10.Dm, 12.15Ff, 14.60.Pq, 14.60.St
\vfill

\newpage
\thispagestyle{empty}
\null