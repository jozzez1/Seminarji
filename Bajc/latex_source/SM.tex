\chapter{Standardni model}

Standardni model (SM) z zelo dobro preciznostjo opiše naše razumevanje sveta osnovnih delcev od nekaj
$10$ eV (pri nizkih energijah postane barvna interakcija neperturbativna), pa do nekaj $100$ GeV, do
koder smo zaenkrat eksperimentalno raziskali energijsko področje. To je model, ki uspe celotno znano
snov opisati prek interakcij fermionskih in bozonskih polj.

\section{Fermionska polja}
Fermionska polja, tj. polja s pol-celim spinom, delimo na kvarke in leptone, prikazana pa so v
tabeli~\ref{SMdelci}.
\begin{table}[H]\centering
	\caption{Fermionska polja standardnega modela. Vidimo, kako so razvrščena v družine (generacije),
		zaradi česar so stolpci oštevilčeni, `$Q$' pa predstavlja električni naboj, ki ga to polje nosi.}
	\begin{subtable}[H]{0.45\linewidth}\centering
		\begin{tabular}{c c c | l}
			1. & 2. & 3. & $Q$ \\
			\hline
			$u$ & $c$ & $t$ & $+2/3$ \\
			$d$ & $s$ & $b$ & $-1/3$
		\end{tabular}
		\caption{kvarki}
	\end{subtable}
	\begin{subtable}[H]{0.45\linewidth}\centering
		\begin{tabular}{c c c | l}
			1. & 2. & 3. & $Q$ \\
			\hline
			$\nu_e$ & $\nu_\mu$ & $\nu_\tau$ & $0$ \\
			$e$ & $\mu$ & $\tau$ & $-1$
		\end{tabular}
		\caption{leptoni}
	\end{subtable}
	\label{SMdelci}
\end{table}
Leptone lahko dodatno delimo na električno nabite leptone (kot je na primer elektron) in na
električno nevtralne leptone, katere imenujemo \emph{nevtrini}, ki imajo v SM poseben status, saj
so znotraj SM edino fermionsko polje, ki nima mase.

Kot posledica Lorentzovih transformacij imamo poleg teh polj tudi njim pripadajoče anti-delce, oz.
polja s konjugiranim nabojem. Npr. fermionsko polje $\psi$ bi s konjugacijo naboja postalo $\psi^c$,
\begin{equation}
	\psi^c = \psi^T i\gamma_0\gamma_2 = \psi^T C^{-1},
\end{equation}
kjer je $C$ operator konjugacije naboja. To pomeni, da imamo poleg fermionov tudi anti-fermione,
ki se od njih razlikujejo le po predznaku električnega naboja. Prikazani so v tabeli~\ref{SManti}.
\begin{table}[H]\centering
	\caption{Anti-fermionska polja SM, tj. fermioni s konjugiranim nabojem.}
	\begin{subtable}[H]{0.45\linewidth}\centering
		\begin{tabular}{c c c | l}
			1. & 2. & 3. & $Q$ \\
			\hline
			$u^c$ & $c^c$ & $t^c$ & $-2/3$ \\
			$d^c$ & $s^c$ & $b^c$ & $+1/3$
		\end{tabular}
		\caption{anti-kvarki}
	\end{subtable}
	\begin{subtable}[H]{0.45\linewidth}\centering
		\begin{tabular}{c c c | l}
			1. & 2. & 3. & $Q$ \\
			\hline
			$\nu_e^c$ & $\nu_\mu^c$ & $\nu_\tau^c$ & $0$ \\
			$e^c$ & $\mu^c$ & $\tau^c$ & $+1$
		\end{tabular}
		\caption{anti-leptoni}
	\end{subtable}
	\label{SManti}
\end{table}
Nevtrini imajo tu spet poseben status, saj so električno nevtralni -- lahko so sami sebi anti-delci,
tj. $\nu = \nu^c$. To so ti. nevtrini Majorane. Kadar velja $\psi^c \neq \psi$ pa govorimo o Diracovih
fermionih. Lahko zgradimo teorijo, kjer so fizikalni nevtrini sestavljeni iz obeh prispevkov, več
o tem v nadaljevanju.

Kvarkovska (in anti-kvarkovska) polja nosijo poleg električnega naboje tudi ti. \emph{barvni naboj},
ker interagirajo prek barvne interakcije. Nosijo lahko katerokoli izmed treh barv, rdečo -- $r$,
zeleno -- $g$, ali pa modro -- $b$. Anti-kvarki nosijo anti-barve, $\bar{r}$, $\bar{g}$ in $\bar{b}$.
To pomeni da lahko kvarkovska polja dobimo v treh različnih variantah. Take barve so izbrane, ker so
vsi fizikalni sestavljeni delci navzven brezbarvni, tj. \emph{bele} barve.

Še ena stvar nas v SM spravlja v zagato. Delce lahko namreč razločimo po "`ročnsti"'. Dočim so
nabiti fermioni lahko levo- in desnoročni, eksperimenti niso odkrili desnoročnega nevtrina, zaradi
česar so nevtrini v SM izključno levoročni. Za anti-delce velja
\begin{equation}
	\psi_L \stackrel{C}{\longrightarrow} \psi^c_R, \quad \psi_R \stackrel{C}{\longrightarrow} \psi^c_L,
\end{equation}
torej levoročen fermion po konjugaciji naboja postane desnoročen anti-fermion in obratno. Taka kiralna stanja
dobimo z operatorjem ročnosti,:
\begin{equation}
	P_L = \frac{1}{2}(1 - \gamma_5), \quad P_R = \frac{1}{2}(1 + \gamma_5).
\end{equation}
To sta projekcijska operatorja z lastnostmi $P_{L,R}^2 = 1$ in $P_L P_R = 0$, saj $\gamma_5^2 = 1$ in
$\gamma_5^\dagger = \gamma_5$. Matrika $\gamma_5$ je ena izmed matrik $\gamma$ iz Lorentzove grupe. Od
tod dobimo identiteto
\begin{equation}
	\bar{\psi} = \psi^\dagger \gamma_0, \Rightarrow \bar{\psi}_L = (\psi_L)^\dagger \gamma_0 =
		\psi^\dagger P_L \gamma_0 = \bar{\psi} P_R,
\end{equation}
torej je $\bar{\psi}_L$ v resnici desnoročen. To pomeni, da
\begin{equation}
	\bar{\psi}_L\psi_L = \bar{\psi}_R\psi_R = 0,
\end{equation}
zaradi česar ne moremo eksplicitno zapisati masnih členov takih polj, ampak so lahko le brezmasna~\cite{quang}.
Taka polja imenujemo Weylova. Fermioni v SM so tipično sestavljeni iz obeh komponent, tj.
\begin{equation}
	\psi = \psi_L + \psi_R,
\end{equation}
Tu je $\psi$ Diracovo polje, zapisano z Weylovimi. Lagrangian za prosto Diracovo polje z maso $m$ lahko v tem
primeru z Weylovimi polji zapišemo kot
\begin{equation}
	\mathcal{L} = i\bar{\psi}_R\slashed{\partial}\psi_R + i\bar{\psi}_L\slashed{\partial}\psi_L -
		m(\bar{\psi}_R\psi_L + \bar{\psi}_L\psi_R).
	\label{weylovaMasa}
\end{equation}
Nevtrini so v SM le levoročni $\nu = \nu_L$ in nimajo desnoročnega prispevka -- zaradi tega morajo biti
brezmasni. Ker imajo dobro definirano ročnost so to Weylova polja.

Ob upoštevanju barve imamo v SM skupaj 24 različnih fermionskih polj, ki jih moramo nekako urediti
in klasificirati, pri čemer lahko $\nu_R$ v princupu obstaja, vendar ne interagira z ostalimi polji
(kar pojasni zakaj ni bil eksperimentalno potrjen).

\section{Bozoni in umeritvene interakcije}

Bozonska polja nosijo celoštevilski spin. SM ima dva tipa bozonov -- umertivene in Higgsove. Umeritveni
bozoni so nosilci osnovnih interakcij. V nizkih energijah je dovolj če vzamemo le elektromagnetno in
pa barvno, za celosten opis pa moramo namesto elektromagnetne vzeti \emph{elektro-šibko}. Interakcije
lahko opišemo s pomočjo ti. umeritvene grupe. To je Lieva grupa, ki je veljavna globalno. Primer
invariance na tako grupo je v sliki~\ref{Umeritvena}.
\begin{figure}[H]\centering
	\begin{subfigure}{0.45\textwidth}\centering
	\[
		\psi \to \exp (\alpha_i T_i) \psi = \psi
	\]
	\caption{lokalna simetrija}
	\end{subfigure}
	\begin{subfigure}{0.45\textwidth}\centering
	\[
		\psi \to \exp \big(\alpha_i(x) T_i\big) \psi = \psi
	\]
	\caption{globalna simetrija}
	\end{subfigure}
	\caption{Razlika med globalno in lokalno simetrijo. Koeficienti $\alpha_i$ so v globalni
		simetriji lahko zvezne funkcije koordinat prostora Minkowskega (tj. prostor-časa).}
	\label{Umeritvena}
\end{figure}
\noindent Umeritvena grupa standardnega modela je produktna grupa posameznih interakcij:
\begin{itemize}
	\item{barvna interakcija -- imamo tri osnovne barve, vzamemo $SU(3)_C$,}
	\item{elektromagnetna -- en nosilec naboja, vzamemo $U(1)_Q$,}
	\item{levoročna šibka interakcija -- $SU(2)_L$.}
\end{itemize}
Kot je bilo že omenjeno, $U(1)_Q$ ni zadostna pri visokih energijah, ampak moramo vzeti
elektrošibko, to je $SU(2)_L\times U(1)_Y$, kjer $Y$ imenujemo \emph{hipernaboj}. Šibka interakcija ima
dodaten indeks `$L$', s čimer smo želeli povdariti, da se sklaplja le z levoročnimi delci! Zaradi tega
je masne člene napisati zelo težko na nivoju SM. Umeritvena
grupa SM, označena z $\mathcal{G}_\text{SM}$ je tako
\begin{equation}
	\mathcal{G}_\text{SM} = SU(2)_L \times SU(3)_C \times U(1)_Y.
\end{equation}

Nosilci takih umeritvenih interakcij so umeritveni bozoni. Njihovo število je enako dimenziji grupe.
V $SU(3)_C$ jih imamo 8 in jih imenujemo \emph{gluoni}. To so $G_{\mu,j}$, $j \in \{1,2,\dots,8\}$.
V $SU(2)_L\times U(1)_Y$ vzamemo njihove linearne kombinacije, da dobimo foton iz $U(1)_Q$, $A_\mu$
in pa ti. šibke bozone $W_\mu$, $W^\dagger_\mu$ in $Z_\mu$. Ker imajo ta polja po en Lorentzov indeks,
$\mu$, jih pravimo vektorska polja in nosijo spin 1. Z izjemo šibkih bozonov so umeritveni bozoni
brezmasni.

Množico fermionskih polj lahko razvrstimo v multiplete $\mathcal{G}_\text{SM}$. Ti so multipleti so
kvarkovski multiplet, $Q$, ki se obnaša kot dublet nad $SU(2)_L$, triplet nad $SU(3)_C$ in ima
hipernaboj
$Y = 1/3$, kar označimo
\begin{equation}
	Q = \irep{2}{3}{1/3} = \begin{pmatrix}
		u_r & u_g & u_b \\
		d_r & d_g & d_b
	\end{pmatrix},
	\label{multipletiZ}
\end{equation}
leptonski multiplet, $L$,
\begin{equation}
	L = \irep{2}{1}{-1} = \begin{pmatrix}
		\nu_e \\ e
	\end{pmatrix},
\end{equation}
anti-kvarkovska tripleta $u^c$ in $d^c$,
\begin{equation}
	u^c = \irepb{1}{3}{-4/3} = \begin{pmatrix}
		u^c_{\bar{r}} & u^c_{\bar{g}} & u^c_{\bar{b}}
	\end{pmatrix}, \quad
	d^c = \irepb{1}{3}{2/3} = \begin{pmatrix}
		d^c_{\bar{r}} & d^c_{\bar{g}} & d^c_{\bar{b}}
	\end{pmatrix},
\end{equation}
pozitron
\begin{equation}
	e^c = \irep{1}{1}{2}
\end{equation}
in pa anti-nevtrino, ki je singlet nad $\mathcal{G}_\text{SM}$,
\begin{equation}
	\nu^c_e = \irep{1}{1}{0}
\end{equation}
in ga ponavadi ne pišemo, saj to da je singlet nad $\mathcal{G}_\text{SM}$ pomeni, da ne interagira
prek umeritvenih interakcij. Ker je $\nu = \nu_L$, pomeni $\nu^c = \nu_R$ po prispevku Majorane.
Taka razvrstitev fermionov je torej ustrezna, saj pravilno uvrstimo tudi morebitni desnoročni nevtrino.

Tri generacije fermionskih polj pomeni v vsaki generaciji 5 (z desnoročnim nevtrinom 6)
različnih multipletov. 

Tako žal ne moremo eksplicitno zapisati masnih členov naših nabitih fermionskih polj, saj
$m\bar{\psi}_L\psi_R = 0$, tj. $m = 0$. Prav tako masivni šibki bozoni napravijo $SU(2)_L$
nerenormalizabilno. Iz zagate nas rešijo Higgsovi bozoni. To je kompleksen dublet nad $SU(2)_L$,
\begin{equation}
	\phi = \irep{2}{1}{1} = \begin{pmatrix}
		\phi^+ \\
		\phi^0
	\end{pmatrix},
	\label{multipletiK}
\end{equation}
ki zaradi neničelne vakuumske pričakovane vrednosti\footnote{Ang. \emph{vacuum expectation value} -- od
tod okrajšava \emph{vev}.} (vev) spontano zlomi elektrošibko umeritveno simetrijo in dá maso vsem
prisotnim poljim prek interakcij Yukawe, ki so členi v Lagrangianu, ki predstavljajo interakcijo
med fermioni in skalarnim poljem (tj. skalarno polje ima spin 0),
\begin{equation}
	\mathcal{L}_Y \sim y\bar{\psi}\psi\phi,
\end{equation}
kjer $y$ imenujemo \emph{sklopitvena konstanta Yukawe}.

\paragraph{Pozvetek polj SM.}
Umeritvene bozone SM lahko potem na kratko povzamemo z umeritveno grupo $SU(3)_C\times SU(2)_L\times
U(1)_Y$, preostale delce pa z multipleti $Q_g$, $L_g$, $u^c_g$, $d^c_g$, $e^c_g$ in $\nu^c_g$ in $\phi$,
kjer $g$ pomeni generacijo.























