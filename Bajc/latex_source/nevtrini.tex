\chapter{Nevtrinske mase in gugalnični mehanizem}

Sedaj, ko smo malo bolj seznanjeni s formalizmom teorije grup in raznih form, se bomo lotili mas nevtrinov.
Kot smo že omenili, so električno nabiti fermioni rešitve Diracove
enačbe~\cite{ryder},~\cite{palash},~\cite{quang},
\begin{equation}
	i\slashed{\partial} \psi - m\psi = 0.
\end{equation}
kar lahko zapišemo z Lagrangianom~\cite{palash},~\cite{quang},~\cite{ryder}
\begin{equation}
	\mathcal{L}_F = \bar{\psi}(i\slashed{\partial} - m)\psi
\end{equation}
njihov masni člen v Lagrangianu
je oblike $m\psi^\dagger\gamma_0\psi = m\bar{\psi}\psi$. Vendar pa imamo lahko še en tip delcev. Matrike
$\gamma_\mu$ lahko nastavimo tako, da še vedno zadostijo Cliffordovi algebri in ostalim lastnostim
Lorentzove grupe, vendar pa nam dajo vedno le kompleksne rešitve $\psi$. Tako dobimo enačbo Majorane
\begin{equation}
	i\slashed{\partial}\psi - m\psi^c = 0, \quad \psi^c \equiv i\gamma_0\gamma_2\psi^* = C\psi^*.
\end{equation}
Pogosto dodamo zahtevo $\psi^c = \psi$. Takemu delcu pravimo polje Majorane in je sam sebi anti-delec.
Masni člen takega spinorja je oblike
\begin{equation}
	m\psi^\dagger \psi = m\psi^T C\psi,
\end{equation}
torej na levi in na desni nastopa isto polje. V primeru, da nevtrini niso Weylova polja, ampak Majoranova,
dobimo poleg levoročnega nevtrina tudi desnoročni nevtrino, saj
\begin{equation}
	\nu = \nu_L, \quad \nu^c = (\nu_L)^c = \nu_R,
\end{equation}
prvi pripada leptonskemu dubletu, $\irep{1}{2}{-1}$, drugi pa je singlet nad $\mathcal{G}_\text{SM}$.
Da taki objekti res niso več Weylova polja, je očitno, saj je tak nevtrino kombinacija $\nu_L$ in $\nu_R$.
Ob tej predpostavki so nevtrini tako Diracova kot Majoranova polja hkrati in imajo lahko oba masna prispevka.
Kadar smo Diracovim poljim s pomočjo Weylovih dali maso, nismo imeli ločenega zapisa za mase levoročnih
in desnoročnih polj, pač pa je bila masa posledica mešanja le-teh, kot v enačbi~\eqref{weylovaMasa}.
Vendar pa za polja Majorane ni tako -- levoročni delci imajo lahko različno maso od desnoročnih.
Vse masne člene nevtrinov bi lahko (shematsko) zapisali kot~\cite{strumia},~\cite{miha},~\cite{nugut}
\begin{equation}
	\mathcal{L}_N \sim m_{\nu_D} \nu^c \nu + m_{\nu_R} \nu^c\nu^c + m_{\nu_L} \nu\nu,
\end{equation}
torej bi levoročni in desnoročni nevtrini lahko imeli različne mase. Tu je $m_{\nu_D}$ Diracov masni
člen, $m_{\nu_R}$ pripada desnoročnemu nevtrinu Majorane, $m_{\nu_L}$ pa se sklaplja z levoročnim
nevtrinom Majorane. Mase dobimo prek skalarnih bozonov, tako kot prej.

Z dodatkom desnoročnega nevtrina rešimo poleg nevtrinskih mas še problem kvantizacije električnega naboja,
ki se ga teoretično ne da pojasniti drugače~\cite{mohapatra} (v principu bi to lahko storili le še s
z dodajanjem magnetnih monopolov, kot je to storil Dirac).

\section{Gugalnični mehanizem tipa I}

Sam SM ima nevtrine le v leptonskem dubletu, $L$, $\nu^c$ pa nima, saj je singlet, ki ne interagira
prek $\mathcal{G}_\text{SM}$. Najpreprosteje je, da ga dodamo in zapišemo člene Yukawe za $\nu^c$,
s čimer bomo po spontanem zlomu simetrije dali mase. Ti se glasijo~\cite{strumia},~\cite{miha},~\cite{nugut}
\begin{equation}
	\mathcal{L}_N = (Y_{\nu_D})_{ij} \nu^c_i L_j \Phi_1 +
		\frac{1}{2}(Y_{\nu_R})_{ij}\nu^c_i\nu^c_j\Phi_2 + \text{h.k.}
	\label{tipI}
\end{equation}
Polji $\Phi_1$ in $\Phi_2$ dobita ob zlomu $SU(2)_L\times U(1)_Y$ neničelno vakuumsko pričakovano
vrednost, ki nam da masne matrike. Ugotoviti moramo, katerim upodobitvam $\mathcal{G}_\text{SM}$
pripadata, da bodo interakcijski členi neničelni in invariantni na $\mathcal{G}_\text{SM}$,
\begin{align}
	\Phi_1 &\in \nu^c \otimes L = \irep{1}{1}{0} \otimes \irep{1}{2}{-1} = \irep{1}{2}{1}, \notag \\
	\Phi_2 &\in \nu^c \otimes \nu^c = \irep{1}{1}{0} \otimes \irep{1}{1}{0} = \irep{1}{1}{0}.
\end{align}
Vidimo, da je $\Phi_1 = \phi$, tj. Higgsov dublet, ki da mase vsem ostalim fermionom in dobi
vev $\langle \phi \rangle = v$, polje $\Phi_2$ pa je singlet, torej lahko eksplicitno zapišemo masno
matriko $Y_{\nu_R} = M_{\nu_R}$ (dati singletu vev je ekvivalentno eksplicitno zapisati
masni člen). Enačbo~\eqref{tipI} lahko potem prepišemo v
\begin{equation}
	\mathcal{L}_N = \underbrace{v(Y_{\nu_D})_{ij}}_{(M_{\nu_D})_{ij}} \nu^c_i L_j + (M_{\nu_R})_{ij}
		\nu^c_i\nu^c_j + \text{h.k} = \begin{pmatrix}
			\nu & \nu^c
		\end{pmatrix}_i \underbrace{\begin{pmatrix}
			0 & (M_{\nu_D}^T)_{ij} \\
			(M_{\nu_D})_{ij} & (M_{\nu_R})_{ij}
		\end{pmatrix}}_{M_N} \begin{pmatrix}
			\nu \\ \nu^c
		\end{pmatrix}_j + \text{h.k.}
\end{equation}
Interakcijske člene $e_j$ in $\nu^c_i$ smo izpustili. Kombinirana nevtrinska masna matrika bi se torej
zapisala kot~\cite{strumia}
\begin{equation}
	M_N = \begin{pmatrix}
		0_{3\times 3} & M^T_{\nu_D} \\ M_{\nu_D} & M_{\nu_R},
	\end{pmatrix}
\end{equation}
ima pa dimenzijo $6\times 6$, saj prispevek Majorane doseže to, da so mase desnoročnih nevtrinov
različne od mas levoročnih. Celotna nevtrinska masna matrika mora za tri generacije imeti 6 različnih
nevtrinskih mas. Matrika, ki se lahko bločno zapiše takole ima lastnosti ti. \emph{gugalnične matrike},
kot gugalnice za dva. Za nas sta zanimiva sta predvsem dva režima~\cite{strumia}:
\begin{itemize}
	\item{$M_{\nu_R} \gg M_{\nu_D}$ -- prevlada člen Majorane,}
	\item{$M_{\nu_R} \ll M_{\nu_D}$ -- prevlada Diracovih mas.}
\end{itemize}
Tu matriki primerjamo po neskončni normi, tj. največji lastni vrednosti.

\paragraph{Režim $M_{\nu_R} \gg M_{\nu_D}$.}

Mase desnoročnih nevtrinov so ogromne $m_{\nu^c}\ \approx\ M_{\nu_R}$, mase levoročnih nevtrinov pa so
izjemno majhne. Diracovi členi predstavljajo mešanje, ki je prav tako zelo majhno. Mase levoročnih
nevtrinov v tem primeru dobimo tako, da si predstavljamo, da je $M_N$ le matrika $2\times 2$ nato pa bi
z diagonalizacijo dobili lastne vrednosti~\cite{miha}
\begin{equation}
	M_N \sim M_{\nu_R}\bigg[1 \pm \sqrt{1 - M_{\nu_D}^TM_{\nu_R}^{-2}M_{\nu_D}}\bigg] = M_{\nu^c} +
		M_\nu,
\end{equation}
kjer so $M_{\nu^c}$ mase desnoročnih nevtrinov in $M_\nu$ mase levoročnih nevtrinov. Ob upoštevanju
limite lahko zgornji izraz razvijemo in ugotovimo~\cite{strumia}
\begin{align}
	M_{\nu^c} &\approx M_{\nu_R}, \notag \\
	M_\nu &\approx -M_{\nu_D}^T M_{\nu_R}^{-1}M_{\nu_D}.
\end{align}
Če je $M_{\nu^c}\ \propto\ M_{\nu_R}$, potem je $M_\nu\ \propto\ M_{\nu_R}^{-1}$, zaradi česar tak
mehanizem imenujemo gugalnični, saj težke mase desnoročnih nevtrinov prevagajo levoročne, tako da
"`obvsisijo v zraku"'. Tak režim v teorijah poenotenja ni težko doseči~\cite{strumia}.

\paragraph{Režim $M_{\nu_D} \gg M_{\nu_R}$.} Levoročni nevtrini imajo tu predvsem Diracovo maso, tj.
$M_\nu \approx M_{\nu_D}$, za katero vemo eksperimentalno, da je izjemno majhna. Masa desnoročnih
nevtrinov je kar $M_{\nu^c}\ \approx\ M_{\nu_R}$, ki pa mora biti še manjša. To lahko dosežemo le z
zahtevo, da je leptonsko število ohranjeno. Taka zahteva se ne porodi sama po sebi iz modela poenotenja,
prav tako pa tudi eksperimentalno vemo, da je mešanje v leptonskem sektorju vse prej kot
dušeno~\cite{pdg:neutrinos} (ohranitev leptonskega števila ni dobra simetrija) zaradi česar se zdi ta
režim v neskladju z naravo~\cite{strumia}.

\section{Gugalnični mehanizem s tipom I in tipom II}

Gugalnični mehanizem tipa I je minimalen, vendar pa v bistvu ni razloga, zakaj tudi levoročni
nevtrino ne bi smel imeti prispevka Majoranove mase. Prispevek mase levoročnega nevtrina se imenuje
gugalnični mehanizem tipa II. Oba tipa lahko med seboj kombiniramo in v enačbo~\eqref{tipI} s skalarnim
singletom $\Phi_2$, ki da desnoročnim nevtrinom maso, dodamo novo skalarno polje $\Delta$, ki se sklaplja
z leptonskim dubletom,~\cite{nugut},~\cite{strumia}
\begin{equation}
	\mathcal{L}_\nu = (Y_{\nu_D})_{ij} \nu^c_i L_j \phi +
		\frac{1}{2}(Y_{\nu_R})_{ij}\nu^c_i\nu^c_j\Phi_2 + \frac{1}{2}(Y_{\nu_L})_{ij}L_iL_j\Delta +
		\text{h.k.}
	\label{gugalnice}
\end{equation}
Skalarno polje $\Delta$ mora biti singlet nad $SU(3)$, saj je $L_i$ singlet nad $SU(3)$, nad
 $SU(2)_L$ pa
je lahko bodisi singlet, bodisi triplet, saj sta $L_{i,j}$ dubleta nad $SU(2)_L$ in tako
\begin{equation}
	\rep{2} \otimes \rep{2} = \rep{1}\oplus\rep{3}.
\end{equation}
Singlet ne prinese nobene novosti, pač pa se splača vzeti triplet $\irep{1}{3}{2}$, katerega vev
dá neničelno maso $M_{\nu_L}$. Nevtrinska masna matrika, ki bi jo dobili po zlomu
$SU(2)_L\times U(1)_Y \to U(1)_Q$, bi se tako zapisala kot~\cite{strumia},~\cite{miha}
\begin{equation}
	M_N = \begin{pmatrix}
		M_{\nu_L} & M_{\nu_D}^T \\
		M_{\nu_D} & M_{\nu_R}
	\end{pmatrix}
\end{equation}
Taka matrika ima prispevke gugalničnega mehanizama tipa I in tipa II. Po vzoru tipa I imamo na
voljo različne režime, vendar vemo, da bo fizikalno najbolj sprejemljiv tak, ki bo
$M_{\nu_L} \ll M_{\nu_D} \ll M_{\nu_R}$, saj bo imel izjemno lahke levoročne nevtrine,
izjemno težke desnoročne nevtrine, med katerimi pa bo zelo malo mešanja. Masni matriki
levoročnih in desnoročnih nevtrinov, sta v tem režimu v prvem redu razvoja~\cite{miha}
\begin{align}
	M_{\nu^c} &\approx M_{\nu_R}, \notag \\
	M_\nu &\approx M_{\nu_L} - M_{\nu_D}^T M_{\nu_R}^{-1} M_{\nu_D}.
\end{align}
Mase desnoročnih nevtrinov nas ne zanimajo, saj so singleti nad $\mathcal{G}_\text{SM}$ in mase dobijo
na skali reda $\Lambda_\text{GUT} \sim 10^{16}$ GeV, mnogo pred zlomom $\mathcal{G}_\text{SM}$. Mase
levoročnih nevtrinov pa so mase nevtrinov, ki jih poznamo in jih lahko tudi izmerimo. Tak model je možen
znotraj teorij poenotenja, med drugim tudi $SO(10)$.

Naj omenimo, da poleg teh dveh tipov obstaja še gugalnični mehanizem tipa III~\cite{nugut},~\cite{strumia},
pri čemer namesto skalarnega tripleta $\irep{1}{3}{2}$ uporabimo fermionski triplet $\irep{1}{3}{0}$. Taki
modeli so redki~\cite{strumia} in niso relevantni za naš problem.













































