\chapter{Umeritvena grupa $SO(10)$}

Grupa $SO(10)$ je posebna ortogonalna Lieva grupa, ki spada v isto kategorijo kot grupe
$SO(2N)$~\cite{mohapatra},~\cite{compact},~\cite{oezer}. Ker imajo take grupe veliko skupnih lastnosti
si poglejmo najprej splošneje kako je z Lievimi grupami in $SO(2N)$.

\section{Grupe $SO(2N)$}

Elementi posebne ortogonalne grupe so matrike z dvema osnovnima lastnostima~\cite{compact},
\begin{equation}
	O^TO = \one, \quad \det O = 1, \quad \forall O \in SO(2N).
\end{equation}
Te lastnosti imajo tudi operatorji vrtenja, zato jim pravimo tudi rotacijske matrike~\cite{compact}.
Število v oklepaju, $2N$, je celo in nam pove da matrike delujejo na $2N$ dimenzionalnem vektorskem
prostoru. Včasih zapišemo še bolj specifično $SO(2N, \mathbb{R})$ ali pa $SO(2N, \mathbb{C})$, da povdarimo
nad katerim obsegom je vektorski prostor, ki ga imamo v mislih.

Lastnost Lievih grup je ta, da vsak element dobimo z eksponenciacijo elementov, ki ležijo na
zvezni mnogoterosti~\cite{compact}. Kot posledica sledi, da imajo Lieve grupe neštevno neskončno
elementov. Mnogoterost imenujemo algebra grupe. Na mnogoterosti lahko izberemo glavne smeri in tako
določimo njeno dimenzijo. To dimenzijo imenujemo tudi dimenzija grupe, ki je za splošne grupe
$SO(N)$ enaka~\cite{mohapatra},~\cite{compact},
\begin{equation}
	\dim SO(N) = \frac{N(N - 1)}{2}.
\end{equation}
To pomeni, da lahko izberemo $N(N-1)/2$ grupnih elementov za matrično bazo algebre in z njimi
popišemo tako celo mnogoterost, kot tudi celotno grupo. Bazne matrike imenujemo \emph{generatorji}.
Diagonalne operatorje\footnote{Stroga definicija pravi, da niso nujno diagonalni, zadošča da med
seboj komutirajo, kar se lahko zgodi v nespretno izbranih bazah.} tvorijo ti.
\emph{Cartanovo podalgebro}, ki ima to lepo lastnost, da je njena dimenzija enaka dobrim
"`kvantnim"' številom, tj. količinam, ki so konstantne glede na delovanje grupe. Dimenzijo
Cartanove podalgebre imenujemo \emph{rang} grupe. V primeru $SO(2N)$ velja~\cite{compact}
\begin{equation}
	\rank SO(2N) = N.
\end{equation}

\subsection{Splošne lastnosti Lievih grup}

Tukaj so navedena dejstva, ki veljajo za vse Lieve grupe, kljub temu, da je za bolj specifične
primere bila vzeta grupa, ki je relevantna za to nalogo, tj. $SO(2N)$.

Eksponenciacija iz algebre v grupo je izomorfizem, torej je vsaka grupa enolično določena z
mnogoterostjo, ki ji pripada. Matematična konvencija je, da algebro grupe označimo z malimi
gotskimi črkami, tj. algebro grupe $SO(2N)$ označimo z $\mathfrak{so}_{2N}$. Da lahko razločimo med
različnimi mnogoterostmi, izračunamo \emph{strukturne konstante} grupe, ki jih dobimo s komutatorji
med generatorji grupe~\cite{compact},
\begin{equation}
	[T_i, T_j] = -if_{ijk} T_k,
	\label{structureConst}
\end{equation}
kjer so $T_i$, $T_j$ in $T_k$ generatorji grupe, $f_{ijk}$ strukturne konstante, $[\bullet, \bullet]$
pa je komutator\footnote{Komutator je le, kadar imamo znotraj oklepaja elemente algebre. Med
elementi grupe je to Liev oklepaj, $[A,B] = ABA^{-1}B^{-1},\ A,B \in SO(2N)$.},
\begin{equation}
	[A, B] = AB - BA, \quad A,B \in \mathfrak{so}_{2N}.
\end{equation}
Strukturne konstante, definirane v enačbi~\eqref{structureConst} nam poda zahtevo, da so generatorji
vedno hermitski, tj. $T_j = T_j^\dagger$. Če bi pozabili $-i$ bi imali pogoj, da so generatorji
anti-hermitski. Strukturne konstante različnih grup se razlikujejo.

Kot smo povedali prej, lahko vse grupne elemente dobimo z eksponenciacijo elementov iz algebre.
Če imamo $A \in SO(2N, \mathbb{F})$, kjer je $\mathbb{F}$ obseg števil, se potem lahko zapiše kot
\begin{equation}
	A = \exp i\alpha_k T_k, \quad \alpha_k \in \mathbb{F}\ \forall\ k.
\end{equation}
Ker smo imeli `$-i$' v enačbi~\eqref{structureConst} imamo tukaj v eksponentu `$i$', sicer ga ne bi
potrebovali.

Če najdemo $N(N-1)/2$ matrik, ki nam dajo iste strukturne konstante, kot jih zahteva grupa
$SO(N)$, potem tvorijo algebro $\mathfrak{so}_N$. Vendar pa ni nujno da so te matrike dimenzije
$N\times N$ -- lahko so manjše ali večje. Še vedno tvorijo isto grupo, vendar pa pravimo,
da pripadajo drugi \emph{upodobitvi}. Dimenzija upodobitve je enaka rangu matrik.

Lahko se zgodi, da vse matrike vsebujejo blok na istem mestu, ki se jih lahko obravnava kot
neodvisne matrike in skupaj spet tvorijo isto algebro. Potem pravimo, da je originalna upodobitev
\emph{razcepna} (\emph{reducibilna}). Bloke prepišemo ven kot matrike in spet skušamo upodobitev
razcepiti na manjše \emph{podupodobitve}. Postopek lahko ponavljamo, dokler se ne da več poiskati
blokov, ki bi tudi tvorili algebro, ampak moramo za to vzeti cele matrike. Takim upodobitvam pravimo
\emph{nerazcepne} (\emph{ireducibilne}).

Vse grupe imajo ti. \emph{trivialno} upodobitev dimenzije 1, ki je nerazcepna. Prva netrivialna
nerazcepna upodobitev je fundamentalna, ki ima isto dimenzijo, kot številka v oklepaju, tj.
$SU(N)$ in $SO(N)$ imata fundamentalni upodobitvi dimenzije $N$, vendar pa imata različno število
generatorjev in različne strukturne konstante, čeprav nekateri generatorji lahko nastopajo tako v
prvi, kot v drugi grupi.

Ponavadi imamo v vsaki dimenziji kvečjemu eno nerazcepno upodobitev, zato se v fiziki polja
upodobitev označi kar z dimenzijo~\cite{slansky}. Da ne bomo mešali z navadnimi števili jih bomo tekom te
naloge označevali v mastnem tisku, tj.
\begin{itemize}
	\item{trivialna upodobitev -- $\rep{1}$,}
	\item{fundamentalna upodobitev -- $\rep{N}$,}
	\item{neka druga upodobitev -- $\rep{V}$,}
	\item{upodobitev, ki ima prav tako dimenzijo $V$, vendar ni $\rep{V}$, se označi s črtico: $\rep{V}'$.}
\end{itemize}
Pogosto zamenjujemo med elementi vektorskega prostora upodobitve in med vektorskim prostorom, ki mu
upodobitev pripada. Tako je lahko $\rep{V}$ vektorski prostor upodobitve, ali pa njegov element.
Vseeno je razvidno iz konteksta o čem govorimo in ne pride do zmešnjav.

Vektorski prostor $\rep{V}$ ima lahko dualni vektorski prostor $\repb{V}$, katerega elementi so
iz \emph{dualne} upodobitve. Nekatere upodobitve so lahko sebi-dualne, tako da $\rep{V} = \repb{V}$.
En tak primer je trivialna upodobitev.

Vsako razcepno upodobitev lahko dobimo iz nerazcepnih. Na voljo imamo dva načina~\cite{compact}:
\begin{itemize}
	\item{Direktna vsota vektorskih prostorv:} $\rep{V} \oplus \rep{W}$,
	\item{Direkten produkt vektorskih prostorov:} $\rep{V} \otimes \rep{W}$.
\end{itemize}
Vendar pa je zapis s prouktom odveč, saj kot posledica Schurove leme sledi, da ob naboru
neraczepnih upodobitev $\rep{R}_i$ lahko zapišemo vsako razcepno upodobitev kot~\cite{compact}
\begin{equation}
	\rep{V} \otimes \rep{W} = \bigoplus_{i = 1}^\infty n_i \rep{R}_i,
\end{equation}
kjer $n_i$ pove pogostost upodobitve $\rep{R}_i$, tj. kolikokrat nastopa v direktni vsoti. Ker
imamo opravka z Lievimi grupami, imamo na voljo števno neskončno nerazcepnih
upodobitev\footnote{Število nerazcepnih upodobitev je sorazmerno s številom elementov grupe.}. Vendar
pa za matrike končne dimenzije potrebujemo le končno mnogo členov v direktni vsoti, saj mora veljati
da se dimenziji izraza na desni in na levi ujemata.

Imejmo matriko $\rep{V}$, ki je v upodobitvi neke Lieve grupe. Potem ji lahko priredimo
(anti-)fundamentalni ikdeks, s katerim označimo njene elemente. V primeru fundamentalne
in dualne fundamentalne updobitve
\begin{equation}
	\rep{N}_i,\ \repb{N}^j, \quad i,j \in \{1,2,\ldots, N\},
\end{equation}
kot vidimo, smo v anti-fundamentalnem primeru uporabili zgornji indeks. Indeksov je lahko tudi več,
npr.
\begin{equation}
	\rep{V}_{i_1,i_2,\ldots,i_k}^{j_1,j_2,\ldots,j_\ell}, \quad i_1,i_2,\ldots,i_k,j_1,j_2,
		\ldots,j_\ell
	\in \{1,2,\ldots,N\},
\end{equation}
tako, da ima neka upodobitev hkrati fundamentalne in anti-fundamentalne indekse. Seveda so med
določenimi elementi dodatna pravila, npr. matrika mora biti v dani upodobitve popolnoma
anti-simetrična, (anti-)sebi-dualna itd. tako da dimenzija upodobitev ni zgolj $N^{k+\ell}$.

V tem smislu so tudi multipleti SM v enačbah (\ref{multipletiZ}-\ref{multipletiK}) zapisani v taisti
notaciji, tj. h katerim nerazcepnim upodobitvam grupe $\mathcal{G}_\text{SM}$ pripadajo.
Tam je le ta razlika, da gre za produktno grupo, zato je prva številka upodobitev nad $SU(2)_L$,
druga upodobitev nad $SU(3)_C$, tretja pa nerazcepna upodobitev $U(1)_Y$. Lieva grupa $U(1)$
ima to lastnost, da ima le en generator, elementi ležijo na enotski krožnici in vsak element
je v svoji upodobitvi. Vse nerazcepne upodobitve so eno-dimenzionalne. Elementi se ločijo le po
"`naboju"', ki ga v produktnih upodobitvah pišemo v podpisanem oklepaju brez mastnega tiska.

\subsection{Upodobitve grup $SO(2N)$}

Grupe $SO(2N)$ imajo dva tipa upodobitev:
\begin{itemize}
	\item{spinorske,}
	\item{tenzorske.}
\end{itemize}
Tenzorske opremljamo samo s fundamentalnimi indeksi\footnote{V $SO(N)$ ni razlike med spodnjimi in
zgornjimi indeksi. Notacija izvira iz teorije funktorjev, ne iz dualnosti upodobitev.}, spinorske
pa so čista analogija nam znane Lorentzove grupe $SO(3,1) \sim SU(2)\times SU(2)$ iz posebne
relativnosti. To pomeni, da lahko poiščemo $SO(2N)$ analogijo matrik $\gamma_\mu$, ki zadoščajo
Cliffordovi algebri, definiramo oprator ročnosti, analogijo operatorja konjugacije naboja itd.
Spinorske upodobitve $SO(2N)$ so zapisane v bazi $SU(N)$~\cite{mohapatra}.

Da bi sestavili spinorsko upodobitev $SO(2N)$ potrebujemo množicio $N$ operatorjev, $\chi_i$
in njihove hermitsko konjugirane različice, $\chi_i^\dagger$, ki zadostijo fermionski
algebri~\cite{mohapatra}
\begin{align}
	\{\chi_i,\chi_j^\dagger\} &= \delta_{ij}, \notag \\
	\{\chi_i,\chi_j\} &= 0.
\end{align}
Tu smo z $\{\bullet,\bullet\}$ označili anti-komutator. Z oklepajem $[\bullet,\bullet]$ bomo
označili komutator. S temi $2N$ operatorji lahko sestavimo generatorje grupe $U(N)$
\begin{equation}
	T_j^i = \chi_i^\dagger \chi_j.
\end{equation}
Ki zadostijo komutacijskim lastnostim algebre:
\begin{equation}
	[T_j^i, T^k_\ell] = \delta^k_j T_\ell^i - \delta_\ell^i T_j^k.
\end{equation}
Z operatorji $\chi_j$ in $\chi_j^\dagger$ lahko zgradimo tudi analogijo matrik $\gamma_\mu$
nad $SO(2N)$, če nam le uspe najti matrike ki zadostijo Cliffordovi algebri~\cite{palash}:
\begin{align}
	\{\gamma_\mu,\gamma_\nu\} &= 2g_{\mu\nu}, \notag \\
	\gamma_0 \gamma_\mu \gamma_0 &= \gamma^\dagger
\end{align}
Nad $SO(2N)$ so to matrike~\cite{mohapatra}
\begin{equation}
	\Gamma_{2j} = \chi_j + \chi_j^\dagger, \quad \Gamma_{2j-1} = -i(\chi_j - \chi_j^\dagger), \quad
		j \in \{1,2,\ldots,N\},
\end{equation}
relacija ki je pogoj za Cliffordovo algebro pa je~\cite{mohapatra}
\begin{equation}
	\{\Gamma_i,\Gamma_j\} = 2\delta_{ij}.
\end{equation}
S pomočjo matrik $\Gamma_j$ lahko dobimo generatorje $SO(2N)$, kjer je ista definicija, kot v spinorski
upodobitvi Lorentzove grupe~\cite{mohapatra},~\cite{palash},~\cite{quang},
\begin{equation}
	\Sigma_{\mu\nu} = \frac{1}{2i}[\gamma_\mu,\gamma_\nu] \longrightarrow
		\Sigma_{jk} = \frac{1}{2i}[\Gamma_j,\Gamma_k].
	\label{genratorji}
\end{equation}
Spinorska upodobitev grupe $SO(2N)$ je $2^N$-dimenzionalna. Iz matrik $\Gamma_i$ bi radi po vzoru
Lorentzove grupe dobili še Kazimirjev operator, $\gamma_5$,
\begin{equation}
	\gamma_5 = i\gamma_0\gamma_1\gamma_2\gamma_3.
\end{equation}
V $SO(2N)$ je ta defiran kot~\cite{mohapatra}
\begin{equation}
	\Gf = i^N \Gamma_1 \Gamma_2 \ldots \Gamma_{2N}.
	\label{g5def}
\end{equation}
Če je to res dobra definicija (tj. operator je res Kazimirjev) moramo preveriti, če komutira z
vsemi generatorji grupe. Potem moramo izraz~\eqref{g5def} prepiati s pomočjo izvornih operatorjev
$\chi_i$ kot
\begin{equation}
	\Gf = [\chi_1,\chi^\dagger_1][\chi_2,\chi_2^\dagger]\ldots[\chi_N,\chi_N^\dagger],
\end{equation}
od koder lahko upoštevamo, da $\chi_j$ in $\chi_j^\dagger$, tvorijo fermionsko algebro in definiramo
operator štetja
\begin{equation}
	n_j = \chi_j^\dagger \chi_j, \quad [\chi^\dagger_j,\chi_j] = 1 - 2n_j,
\end{equation}
kar nam vrne
\begin{equation}
	\Gf = \prod_{j - 1}^N(1 - 2n_j).
\end{equation}
Fermionski operator štetja ima lastnost
\[
	1 - 2n_j = (-1)^{n_j},
\]
s čimer si lahko pomagamo do končne identitete~\cite{mohapatra}
\begin{equation}
	\Gf = (-1)^n, \quad n = \sum_{j = 1}^N n_j.
\end{equation}
Sedaj je postalo očitno, da $\Gf$ komutira z generatorji, $\Sigma_{ij}$, torej je ta analogija
ustrezna. S pomočjo Kazimirjevega operatorja $\Gf$ lahko definiramo operator abstraktne
ročnosti nad $SO(2N)$
\begin{equation}
	P_{\pm} = \frac{1}{2}(1 \pm \gamma_5)\quad\Longrightarrow\quad P_{\pm} = \frac{1}{2}(1 \pm \Gf).
\end{equation}

\section{Upodobitve $SO(10)$}

Grupa $SO(10)$ ima dimenzijo $45$ in je ranga 5. Spinorksa upodobitev ima dimenzijo $2^5$, tj.
$\rep{32}$. Slednja je razcepna, kar lahko pokažemo s pomočjo operatorja "`ročnosti"',
\begin{equation}
	\frac{1}{2}(1 + \Gf)\rep{32} = \rep{16}, \quad \frac{1}{2}(1 - \Gf)\rep{32} = \repb{16}.
\end{equation}
Izkaže se, da sta upodobitvi $\rep{16}$ in $\repb{16}$ nerazcepni, tj.
\begin{equation}
	\rep{32} = \rep{16}\oplus\repb{16}.
\end{equation}
Ker ju dobimo kot projekciji z operatorjem "`ročnosti"', ju imenujemo kiralni upodobitvi. Če je $\rep{32}$
analogija Diracovih spinorjev, potem sta $\rep{16}$ in $\repb{16}$ analogiji Weylovih polj.

Prvih nekaj nerazcepnih upodobitev $SO(10)$ je navedenih v tabeli~\ref{irreps}. Prikazan je tudi
razcep posamezne upodobitve nad Pati-Salamovo podgrupo, $\mathcal{G}_\text{PS}$, katerega pomen
bo bolj jasen v nadaljevanju.
\begin{table}[H]\centering
	\caption{Prvih nekaj nerazcepnih upodobitev $SO(10)$ in njihove dekompozicije znotraj
	Pati-Salamove (PS) grupe~\cite{slansky}. Kot v multipletih iz SM prva številka pomeni upodobitev glede
	na prvo grupo, druga gelede na druo itd.}
	\begin{tabular}{r|l}
		upodobitev & razcep nad $SU(2)_L\times SU(2)_R \times SU(4)_C$ \\
		\hline
		$\rep{1}$ & $\irrep{1}{1}{1}$ \\
		$\rep{10}$ & $\irrep{2}{2}{1} \oplus \irrep{1}{1}{6}$ \\
		$\rep{16}$ & $\irrep{2}{1}{4} \oplus \irrepb{1}{2}{4}$ \\
		$\rep{45}$ & $\irrep{3}{1}{1} \oplus \irrep{1}{3}{1} \oplus \irrep{2}{2}{6}
				\oplus \irrep{1}{1}{15}$ \\
		$\rep{54}$ & $\irrep{1}{1}{1} \oplus \irrep{3}{3}{1} \oplus \irrep{2}{2}{6} \oplus
				\irrep{1}{1}{20'}$ \\
		$\rep{120}$ & $\irrep{2}{2}{1} \oplus \irrep{3}{1}{6} \oplus \irrep{1}{3}{6} 
			\oplus \irrep{1}{1}{10} \oplus \irrepb{1}{1}{10} \oplus \irrep{2}{2}{15}$ \\
		$\rep{126}$ & $\irrep{1}{1}{6} \oplus \irrepb{3}{1}{10} \oplus \irrep{1}{3}{10}
				\oplus \irrep{2}{2}{15}$ \\
		$\rep{144}$ & $\irrep{2}{1}{4} \oplus \irrepb{1}{2}{4} \oplus \irrep{2}{3}{4}
				\oplus \irrepb{3}{2}{4} \oplus \irrep{2}{1}{20} \oplus \irrepb{1}{2}{20}$
	\end{tabular}
	\label{irreps}
\end{table}

Kiralna upodobitev $\rep{16}$ je zelo pomembna v poenotenju prek $SO(10)$, saj je edina možna
upodobitev, ki lahko s svojimi kvantnimi števili vključuje fermione SM~\cite{mohapatra},~\cite{oezer}.
Hkrati je tudi dimenzija ravno pravšnja, tako da lahko 5 upodobitev $\mathcal{G}_\text{SM}$ na generacijo
nadomestimo s $\rep{16}$ iz $SO(10)$, s čimer smo poenotili fermionska polja\footnote{Ker nastopajo v isti
upodobitvi grupe.} znotraj generacije,
\begin{equation}
	\rep{16} = (Q, u^c, d^c, L, \nu^c, e^c).
	\label{opis16}
\end{equation}
Kot vidimo ima ta upodobitev tudi $\nu^c$, ki je singlet nad $\mathcal{G}_\text{SM}$. To pomeni,
da nam ni treba desnoročnega nevtrina dodajati posebej, ampak je že sam po sebi vključen teorijo
$SO(10)$, zaradi česar je $SO(10)$ lahko teorija masivnih nevtrinov~\cite{mohapatra}~\cite{gutproton}.

\section{Invariante umeritvene grupe $SO(10)$}

Ko Lagrangianu priredimo umeritveno simetrijo $SO(10)$, pomeni, da je le-ta invarianten na
rotacije iz $SO(10)$, poleg seveda tega, da je invarianten na Lorentzove transformacije.
Fermionske mase v Lagrangianu bodo izvirali iz Yukawovih členov tipa~\cite{mohapatra}
\begin{equation}
	\mathcal{L}_Y \sim y \psi^c \psi \phi =  y\psi^T C^{-1} \psi \phi,
	\label{primer1}
\end{equation}
kjer je $\psi$ v kiralni upodobitvi (tj. $\rep{16}$ ali pa $\repb{16}$), $\phi$ pa je v eni izmed
tenzorskih upodobitev in ga lahko opremimo s fundamentalnimi indeksi. Operator $C^{-1} = C$ je
operator konjugacije naboja, v Yukawovem členu je zato, da ohranimo Lorentzovo invarianco.

Invarianto iz take tenzorske upodobitve dobimo zelo enostavno -- vse tenzorje med seboj zmnožimo 
in seštejemo po vseh indeksih~\cite{gutproton},~\cite{miha},~\cite{mohapatra}, kot npr.
\begin{equation}
	\phi_{i_1i_2\dots i_k} M_{i_1i_2\dots i_kj_1j_2\dots j_\ell} N_{j_1j_2\dots j_\ell}.
	\label{tenzorji}
\end{equation}
Iz zapisa je očitno, da niso nujno enakih dimenzij, je pa pomembno, da na koncu ne ostane več prostih
indeksov.

Poglejmo še kako je z invariantami spinorske upodobitve. Matrika $C$ je skalar nad grupo $SO(10)$,
zato jo bomo na kratko pozabili iz en.~\eqref{primer1}. Ker je $SO(10)$ grupa ortogonalnih
transformacij, bomo ugibali, da je invarianta nad $SO(10)$ kar $\psi^T \psi$. To bomo preverili s
pomočjo transformacij grupe $\exp(i\alpha_{jk}\Sigma_{jk}) \in SO(10)$:
\begin{align}
	\psi &\to \psi' = \e^{i\alpha_{jk}\Sigma_{jk}} \psi \approx \psi
		+ i\alpha_{jk}\Sigma_{jk}\psi, \notag \\
	\psi^\dagger &\to (\psi^\dagger)' \approx \psi - i\alpha_{jk}(\Sigma_{jk}\psi)^\dagger
		= \psi - i\alpha_{jk}\ \psi^\dagger\Sigma_{jk},
		\notag \\
	\psi^T &\to (\psi^T)' \approx \psi + i\alpha_{jk}\ \psi^T\Sigma_{jk}^T,
\end{align}
kar v prvem redu pomeni
\begin{equation}
	\psi^T\psi \to \psi^T\psi + \underbrace{i\alpha_{jk}\psi^T\Sigma_{jk}^T\psi +
		i\alpha_{jk}\psi^T\Sigma_{jk}\psi}_{\neq 0} + \mathcal{O}(\alpha_{jk}^2) \neq \psi^T\psi,
	\label{zagata}
\end{equation}
se pravi $\psi^T\psi$ ni invarianta nad $SO(10)$, saj $\Sigma^T_{jk} \neq -\Sigma_{jk}$.

\subsection{Operator konjugacije naboja nad $SO(10)$}

Da se rešimo iz zagate v en.~\eqref{zagata} bomo v člen $\psi^T\psi$ dodali matriko $B$, tako bomo
imeli invarianto nad $SO(10)$
\begin{equation}
	\psi^T B\psi \to \psi^TB\psi.
\end{equation}
Transformacija $SO(10)$ se potem v prvem redu glasi
\begin{equation}
	\psi^TB\psi \to (\psi^TB)'\psi' = \psi^TB\psi + \underbrace{i\alpha_{jk}\psi^TB\Sigma_{jk}\psi +
		\delta(\psi^T B)\psi}_{= 0} + \mathcal{O}(\alpha_{jk}^2) = \psi^TB\psi,
\end{equation}
od koder dobimo pogoj
\begin{equation}
	\delta(\psi^TB) = -i\alpha_{jk}\psi^T B\Sigma_{jk},
	\label{pogoj1}
\end{equation}
hkrati pa vemo, da se $\psi^TB$ transformira kot
\begin{equation}
	\psi^TB \to (\psi^TB)' = (\psi^T)'B \approx \psi^TB +
		\underbrace{i\alpha_{jk}\psi^T\Sigma_{jk}^T B}_{\delta(\psi^TB)}.
	\label{pogoj2}
\end{equation}
Če združimo enačbi~\eqref{pogoj1} in~\eqref{pogoj2} dobimo
\begin{equation}
	i\alpha_{jk}\psi^T\Sigma_{jk}^TB = -i\alpha_{jk}\psi^TB\Sigma_{jk},
\end{equation}
kar se poenostavi v
\begin{equation}
	B^{-1}\Sigma_{jk}^TB = -\Sigma_{jk}.
	\label{konjugacija}
\end{equation}
Isti pogoj pa mora veljati za operator konugacije naboja iz Lorentzove grupe~\cite{palash},
\begin{equation}
	C^{-1}\Sigma_{\mu\nu}^TC = -\Sigma_{\mu\nu},
\end{equation}
ki je definiran kot
\begin{equation}
	C = i\gamma_0\gamma_2,\quad C^{-1} = C^\dagger = C, \quad C^T = -C
\end{equation}
Opazimo, da je $B$ analogija operatorja konjugacije naboja Lorentzove grupe~\cite{mohapatra}.
Od tod kar uganemo rešitev enačbe~\eqref{konjugacija}~\cite{mohapatra},
\begin{equation}
	B = i\Gamma_1\Gamma_3\ldots\Gamma_9 = i\prod_{j\ \text{lih}}\Gamma_j.
\end{equation}

\subsection{Bilinearne kombinacije v $SO(10)$}

S pomočjo matrike $B$ lahko napravimo, po zgledu Lorentzove grupe za Diracove spinorje, bilinearne forme
umeritvene grupe $SO(10)$ za spinorje $\psi$. Te so prikazane v tabeli~\ref{bilinearne}.
\begin{table}[H]\centering
	\caption{Tu so prikazane možne bilinearne forme, ki jih dobimo s spinorske upodobitve $SO(10)$.
		Tu smo izpustili $\Gf$, ki bi služil zgolj kot projektor, s katerim bi $\rep{32}$ razbili
		na $\rep{16}$ in $\repb{16}$, vendar bomo tako ali tako uporabljali zgolj ti dve upodobitvi.}
	\begin{tabular}{c c}
		transformacijsko pravilo & forma nad $SO(10)$ \\
		\hline
		skalar	& $\psi^T BC^{-1}\psi$ \\
		vektor & $\psi^TBC^{-1}\Gamma_j\psi$ \\
		tenzor & $\psi^TBC^{-1}\Gamma_{j_1}\Gamma_{j_2}\dots\Gamma_{j_k}\psi$
	\end{tabular}
	\label{bilinearne}
\end{table}
Če hočemo s polja $\phi_{j_1j_2\dots j_k}$ napraviti invarianto ga moramo pomnožiti s tenzorjem, ki
ima prav toliko indeksov. Seveda bomo izbrali tenzorsko upodobitev, ki jo dobimo iz bilinearnih
form. Potem je dobra invarianta nad $SO(10)$ npr. Yukawow člen, ki bi se zapisal kot
\begin{equation}
	\mathcal{L}_Y \sim y\ \psi^TBC^{-1}\Gamma_{j_1}\Gamma_{j_2}\dots\Gamma_{j_k}\psi\ 
		\phi_{j_1j_2\dots j_k},
\end{equation}
saj je produkt dveh tenzorjev, pri čemer se vsi indeksi seštejejo. Iz pogoja, da je $\psi$
v spinorski upodobitvi ugotovimo, da je $\psi = \rep{16}$ ali pa $\psi = \repb{16}$. Potem
so tile tenzorji v sledečih upodobitvah:
\begin{align}
	\rep{16}^TBC^{-1}\Gamma_{j_1}\Gamma_{j_2}\dots\Gamma_{j_k}\rep{16} &\in \rep{16}\otimes\rep{16},
		\notag \\
	\rep{16}^TBC^{-1}\Gamma_{j_1}\Gamma_{j_2}\dots\Gamma_{j_k}\repb{16} &\in \rep{16}\otimes\repb{16},
		\notag \\
	\repb{16}^TBC^{-1}\Gamma_{j_1}\Gamma_{j_2}\dots\Gamma_{j_k}\repb{16} &\in \repb{16}\otimes\repb{16},
		\notag \\
\end{align}
Od tod lahko ugotovimo, katere tenzorje lahko sestavimo, saj lahko te produktne upodobitve razcepimo
in dobimo
\begin{align}
	\rep{16} \otimes \rep{16} &= \rep{10} \oplus \rep{120} \oplus \repb{126}, \notag \\
	\rep{16} \otimes \repb{16} &= \rep{1} \oplus \rep{45} \oplus \rep{210}, \notag \\
	\repb{16} \otimes \repb{16} &= \rep{10} \oplus \rep{120} \oplus \rep{126}.
	\label{produktne:spinorske}
\end{align}
Fundamentalna upodobitev $\rep{10}$ ima en indkes, $\rep{120}$ ima tri in $\rep{126}$ oz. $\repb{126}$
imata 5. Singlet $\rep{1}$ je brez indeksov, $\rep{45}$ ima dva in $\rep{210}$ jih ima štiri. Vse
te upodobitve so popolnoma anti-simetrične. Iz spinorjev lahko sestavimo tenzorje, ki pripadajo
sledečim nerazcepnim upodobitvam,
\begin{align}
	\rep{16}^TBC^{-1}\Gamma_i\rep{16} &\in \rep{10}, \notag \\
	\repb{16}^TBC^{-1}\Gamma_i\Gamma_j\rep{16} &\in \rep{45}, \notag \\
	\rep{16}^TBC^{-1}\Gamma_i\Gamma_j\Gamma_k\rep{16} &\in \rep{120}, \notag \\
	\repb{16}^TBC^{-1}\Gamma_i\Gamma_j\Gamma_k\Gamma_\ell\rep{16} &\in \rep{210}, \notag \\
	\rep{16}^TBC^{-1}\Gamma_i\Gamma_j\Gamma_k\Gamma_\ell\Gamma_m\rep{16} &\in \repb{126}.
	\label{tenzor:spinor}
\end{align}
kjer se $\repb{16}^TBC^{-1}\rep{16}$ lahko dodatno sklopi s singletom $SO(10)$, kar lahko preberemo
iz~\eqref{produktne:spinorske}.
Skalarno polje $\phi$, s katerim bomo sklopili fermionska polja morajo imeti isto število
indeksov\footnote{V splošnem morajo biti v upodobitvi, katere podupodobitev je vsaj ena izmed
navedenih v direktni vsoti enačb~\eqref{produktne:spinorske}.}, tj. morajo biti v isti
upodobitvi -- dober člen v Lagrangianu\footnote{Dober, ker je invarianten na Lorentzovo grupo in
umeritveno grupo $SO(10)$.} bi se torej glasil npr.
\begin{equation}
	\mathcal{L}_Y \sim \rep{16}^TBC^{-1}(Y_{120}\Gamma_i\Gamma_j\Gamma_k\rep{16}\ \rep{120}_{ijk}
		+ Y_{\overline{126}}\Gamma_i\Gamma_j\Gamma_k\Gamma_\ell\Gamma_m\rep{16}\ 
		\repb{126}_{ijk\ell m}),
	\label{komplikaicija}
\end{equation}
kar pa je mučno brati.

\section{Okrajšan zapis v teorijah poenotenja}

Ker vemo koliko indeksov morajo imeti posamezne tenzorske upodobitve in se prav tako zavademo, da
je člen $BC^{-1}$ vedno prisoten, se v teoriji poenotenja vse to ponavadi kar spušča zavoljo čistejše
notacije. Kar preostane, so le polja in njihove upodobitve, tj.
\begin{equation}
	y \psi^TBC^{-1}\Gamma_{j_1}\Gamma_{j_2}\dots\Gamma_{j_k}\psi\ \phi_{j_1j_2\dots j_k}
		\to y\psi^c\phi\psi.
\end{equation}
Tak zapis bi izraz~\eqref{komplikaicija} poenostavil v
\begin{equation}
	\mathcal{L}_Y \sim \rep{16}^c (Y_{120}\rep{120} + Y_{\overline{126}}\repb{126})\rep{16},
\end{equation}
to pa je veliko preglednejše. 

Ob uporabi seveda ne smemo pozabiti na vse matrike $\Gamma_i$ in indekse v ozadju, saj nam ravno ti
povedo, katere kombinacije polj so možne, niso pa bistvene za nadaljno obravnavo mas v Yukawovih
členih. Zaradi preglednosti se bomo v nadaljevanju držali takšnega zapisa.






































