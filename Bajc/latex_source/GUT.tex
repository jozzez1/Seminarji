\section{Teorija poenotenja -- v delu}

Vse tri umeritvene interakcije imajo svojo sklopitveno konstanto, ki meri jakost interakcije. Te tri
sklopitvene konstante so
\begin{itemize}
	\item{elektromagnetna: $\alpha_S \sim 1$,}
	\item{šibka: $\alpha_Q \sim 10^{-2}$,}
	\item{močna: $\alpha_W \sim 10^{-6}$.}
\end{itemize}
Kot vidimo, med njimi velja hierarhično razmerje. Gravitacija, kot četrta fundamentalna interakcija ima
po teoretičnih napovedih sklopitveno konstanto $\alpha_G \sim 10^{-39}$, zaradi česar jo v teoriji
osnovnih delcev zanemarjamo.

Vendar se vrednosti sklopitvenih konstant spreminjajo z energijo. Z odkritjem renormalizacijskega
postopka za odpravo teoretičnih divergenc je bilo odkrito tudi dejstvo, da so izmerjene vrednosti
odvisne od energijske skale na katerih jih merimo. To velja za mase in tudi za sklopitvene konstante.

