%My header and style options
\documentclass[a4paper, 12 pt, titlepage]{article}
\usepackage[slovene]{babel}
\usepackage[utf8]{inputenc}
\usepackage[T1]{fontenc}

%\usepackage{vicent}
%\usepackage[0T1]{fontenc}

%custom colour package
\usepackage[usenames, dvipsnames]{xcolor}

%graphics, captions etc.
\usepackage[pdftex]{graphicx}
\usepackage{amssymb, float, amsmath, fullpage}

%to get the colourful hyperlinks ... not just square boxes around them ...
\usepackage{hyperref}
\hypersetup{
	colorlinks=true,
	linkcolor=black!60!red,
	citecolor=black!60!green,
	urlcolor=black!60!cyan,
	filecolor=black!60!magenta
}

%set up custom captions
\usepackage{caption}
\captionsetup{
	font=small,
	margin=10pt,
	labelfont=it,
%	labelsep=endash,
	format=hang,
	width=0.7\textwidth
}

%bibliography
%\usepackage[round]{natbib}

%LOOKS WAY BETTER WITHOUT THESE ... :P
%costum matter fonts and section fonts
%\usepackage{mathpazo}
%\usepackage{sectsty}
%\allsectionsfont{\LARGE\sffamily\bfseries}

\newcommand{\parc}[2]{
	\ensuremath{\frac{\partial#1}{\partial#2}}
}

\newcommand{\vac}[1][\phi]{
	\ensuremath{\langle#1\rangle}
}

%\renewcommand{\to}{
%	\ensuremath{\longrightarrow}
%}

% New definition of square root:
% it renames \sqrt as \oldsqrt
\let\oldsqrt\sqrt
% it defines the new \sqrt in terms of the old one
\def\sqrt{\mathpalette\DHLhksqrt}
\def\DHLhksqrt#1#2{%
\setbox0=\hbox{$#1\oldsqrt{#2\,}$}\dimen0=\ht0
\advance\dimen0-0.2\ht0
\setbox2=\hbox{\vrule height\ht0 depth -\dimen0}%
{\box0\lower0.4pt\box2}}

\newenvironment{myfig}[2][10cm]
{
	\vspace{-20pt}
	\begin{figure}[H]
		\begin{center}
			\includegraphics[keepaspectratio=1, width=#1]{#2}
		\end{center}
		\vspace{-24pt}
}
{
	\end{figure}
	\vspace{-6pt}
}

\renewenvironment{abstract}[1][1.0]
{
	\begin{center}
		{\bf Povzetek}\\[12pt]
		\begin{minipage}{#1\textwidth}
}
{
		\end{minipage}
	\end{center}
}

\newcommand{\rot}{
	\ensuremath{\vec{\nabla}\times}
}

\renewcommand{\div}{
	\ensuremath{\vec{\nabla}\cdot}
}

\newcommand{\ve}{
	\ensuremath{\vec{E}}
}

\newcommand{\vb}{
	\ensuremath{\vec{B}}
}

\begin{document}

%titlepage
\begin{titlepage}
	\begin{figure}[H]
		\centering
		\includegraphics[width = 7cm, keepaspectratio=1]{pics/logo.pdf}\\[12pt]
		{\sc Oddelek za fiziko}\\[4cm]
	\end{figure}
	\begin{center}
		\large{Seminar -- 2. letnik 2. bolonjske stopnje}\\[0.5cm]
		\LARGE\textbf{Masivna elektrodinamika}\\[1.0cm]

		\vspace{0.0cm}

		\begin{minipage}{0.4\textwidth}\small
			\begin{flushleft}
				\textsc{Avtor:}\\[0.2cm]
				Jože Zobec
			\end{flushleft}
		\end{minipage}
		\begin{minipage}{0.4\textwidth}\small
			\begin{flushright}
				\textsc{Mentor:}\\[0.2cm]
				Prof. Dr. Borut Bajc
			\end{flushright}
		\end{minipage}
	\end{center}

	\vspace{4.5cm}

	\begin{abstract}
		Kot je verjetno potrebno venomer, so ob zori teorije o elektromagnetnem polju znanstveniki novosti
		sprejemali z zdravo mero dvoma. Tako so dlakocepili ob podrobnostih, ki so poznejše fizike privedle
		do možnosti neničelne mase fotona, katere eksperimentalno zaenkrat še ne moremo zanemariti. V tem
		seminarju bom na kratko opisal različne mehanizme, po katerih lahko foton dobi maso in naredil
		zgodovinski pregled meritev le-te.
	\end{abstract}
	
	\vfill

	\centering{\footnotesize Ljubljana, \today}
\end{titlepage}

%table of contents, obviously ...
\tableofcontents

\pagebreak

\section{Uvod}

Newtonova teorija gravitacije je s svojo učinkovitostjo ter preprostostjo narave presunila mnoge tedanje mislece in
služila za navdih novincem, ki so se tedaj ravno pričeli ukvarjati z elektrodinamiko. Predpostavka je bila, da
električna sila med naboji prav tako pada z $r^{-2}$, tako kot gravitacijska. Nekateri so temu oporekali in so
poskušali z $r^{-2 + \alpha}$, malenkost, ki je povezana prav z maso fotona.

Fiziki francoske šole so se o tem nekako največ spraševali. Med njimi prednjači Proca, ki je prvi zapisal ekvivalent
Maxwellovih enačb za masivne fotone. Za njim so se s tem ukvarjali Stueckelberg, de Broglie, in med drugimi tudi
Schroedinger.

S pojavom kvantne elektrodinamike so poskušali slednjo uporabiti za masivne fotone, pri čemer so naleteli na
različne nevšečnosti, pri čemer so pričeli z razvojem novih teoretičnih mehanizmov, s katermi bi foton dobil maso
in hkrati ohranil lepe lastnosti brezmasne elektrodinamike.

Masivna elektrodinamika je tudi dandanes zanimiva, ker s seboj prinaša velike posledice tudi v primeru, ko je masa
fotona zelo majhna.

\section{Začetki elektrodinamike}

Maxwellove enačbe so nas dosegle konec devetnajstega stoletja, nekaj desetletij za tem, jih je Proca prepisal v obliko,
ki zado\v s\v ca neni\v celni fotonski masi. Ena\v cbe se potem glasijo

\begin{align}
	\div\ve = \rho/\varepsilon_0 - m_\gamma^2\varphi, &\qquad \div\vb = 0, \notag \\
	\rot\ve = \parc{\vb}{t}, &\qquad \rot\vb = \mu_0\vec{\jmath} + \frac{1}{c_0^2}\parc{\ve}{t} - m_\gamma^2\vec{A},
\end{align}

kjer je $c_0 = 1/\sqrt{\varepsilon_0\mu_0}$ \v se vedno dobro definirana in nespremenjena. Kot vidimo, so ena\v cbe Proca
v Lorentzovo kovariantni obliki nespremenjene za dualni napetostni tenzor elektromagnetnega polja, medtem ko dobi
napetostni tenzor sam popravke, $\partial_\nu F^{\mu\nu} + m^2_\gamma A^\mu = \jmath^\mu$.

\end{document}


