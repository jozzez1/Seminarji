%My header and style options
\documentclass[a4paper, twocolumn, titlepage]{article}
\usepackage[slovene]{babel}
\usepackage[utf8]{inputenc}
\usepackage[T1]{fontenc}

%\usepackage{vicent}
%\usepackage[0T1]{fontenc}

%custom colour package
\usepackage[usenames, dvipsnames]{xcolor}

%graphics, captions etc.
\usepackage[pdftex]{graphicx}
\usepackage{amssymb, float, amsmath, fullpage}

%to get the colourful hyperlinks ... not just square boxes around them ...
\usepackage{hyperref}
\hypersetup{
	colorlinks=true,
	linkcolor=black!60!red,
	citecolor=black!60!green,
	urlcolor=black!60!cyan,
	filecolor=black!60!magenta
}

%set up custom captions
\usepackage{caption}
\captionsetup{
	font=small,
	margin=10pt,
	labelfont=it,
%	labelsep=endash,
	format=hang,
	width=0.7\textwidth
}

%bibliography
%\usepackage[round]{natbib}

%LOOKS WAY BETTER WITHOUT THESE ... :P
%costum matter fonts and section fonts
%\usepackage{mathpazo}
%\usepackage{sectsty}
%\allsectionsfont{\LARGE\sffamily\bfseries}

\newcommand{\parc}[2]{
	\ensuremath{\frac{\partial#1}{\partial#2}}
}

\newcommand{\vac}[1][\phi]{
	\ensuremath{\langle#1\rangle}
}

%\renewcommand{\to}{
%	\ensuremath{\longrightarrow}
%}

% New definition of square root:
% it renames \sqrt as \oldsqrt
\let\oldsqrt\sqrt
% it defines the new \sqrt in terms of the old one
\def\sqrt{\mathpalette\DHLhksqrt}
\def\DHLhksqrt#1#2{%
\setbox0=\hbox{$#1\oldsqrt{#2\,}$}\dimen0=\ht0
\advance\dimen0-0.2\ht0
\setbox2=\hbox{\vrule height\ht0 depth -\dimen0}%
{\box0\lower0.4pt\box2}}

\newenvironment{myfig}[2][10cm]
{
	\vspace{-20pt}
	\begin{figure}[H]
		\begin{center}
			\includegraphics[keepaspectratio=1, width=#1]{#2}
		\end{center}
		\vspace{-24pt}
}
{
	\end{figure}
	\vspace{-6pt}
}

\renewenvironment{abstract}[1][1.0]
{
	\begin{center}
		{\bf Povzetek}\\[12pt]
		\begin{minipage}{#1\textwidth}
}
{
		\end{minipage}
	\end{center}
}

\newcommand{\rot}{
	\ensuremath{\vec{\nabla}\times}
}

\renewcommand{\div}{
	\ensuremath{\vec{\nabla}\cdot}
}

\newcommand{\ve}{
	\ensuremath{\vec{E}}
}

\newcommand{\vb}{
	\ensuremath{\vec{B}}
}

\begin{document}

%titlepage
\begin{titlepage}
	\begin{figure}[H]
		\centering
		\includegraphics[width = 7cm, keepaspectratio=1]{pics/logo.pdf}\\[12pt]
		{\sc Oddelek za fiziko}\\[4cm]
	\end{figure}
	\begin{center}
		\large{Seminar -- 2. letnik 2. bolonjske stopnje}\\[0.5cm]
		\LARGE\textbf{Masivna elektrodinamika}\\[1.0cm]

		\vspace{0.0cm}

		\begin{minipage}{0.4\textwidth}\small
			\begin{flushleft}
				\textsc{Avtor:}\\[0.2cm]
				Jože Zobec
			\end{flushleft}
		\end{minipage}
		\begin{minipage}{0.4\textwidth}\small
			\begin{flushright}
				\textsc{Mentor:}\\[0.2cm]
				Prof. Dr. Borut Bajc
			\end{flushright}
		\end{minipage}
	\end{center}

	\vspace{4.5cm}

	\begin{abstract}
		Kot je verjetno potrebno venomer, so ob zori teorije o elektromagnetnem polju znanstveniki novosti
		sprejemali z zdravo mero dvoma. Tako so dlakocepili ob podrobnostih, ki so poznejše fizike privedle
		do možnosti neničelne mase fotona, katere eksperimentalno zaenkrat še ne moremo zanemariti. V tem
		seminarju bom na kratko opisal različne mehanizme, po katerih lahko foton dobi maso in naredil
		zgodovinski pregled meritev le-te.
	\end{abstract}
	
	\vfill

	\centering{\footnotesize Ljubljana, \today}
\end{titlepage}

%table of contents, obviously ...
%\tableofcontents

\pagebreak

\section{Uvod}

Newtonova teorija gravitacije je s svojo učinkovitostjo ter preprostostjo narave presunila mnoge tedanje mislece in
služila za navdih novincem, ki so se tedaj ravno pričeli ukvarjati z elektrodinamiko. Predpostavka je bila, da
električna sila med naboji prav tako pada z $r^{-2}$, tako kot gravitacijska. Nekateri so temu oporekali in so
poskušali z $r^{-2 + \alpha}$, malenkost, ki je povezana prav z maso fotona.~\cite{nieto2}

Fiziki francoske šole so se o tem nekako največ spraševali. Med njimi prednjači Proca, ki je prvi zapisal ekvivalent
Maxwellovih enačb za masivne fotone~\cite{nieto1,over}. Za njim so se s tem ukvarjali Stueckelberg~\cite{over,nieto1},
de Broglie~\cite{nieto1,over}, in med drugimi tudi Schroedinger~\cite{nieto1}.

S pojavom kvantne elektrodinamike so poskušali slednjo uporabiti za masivne fotone, pri čemer so naleteli na
različne nevšečnosti, pri čemer so pričeli z razvojem novih teoretičnih mehanizmov, s katermi bi foton dobil maso
in hkrati ohranil lepe lastnosti brezmasne elektrodinamike.

Masivna elektrodinamika je tudi dandanes zanimiva, ker s seboj prinaša vdaljnose\v zne posledice tudi v primeru, ko je masa
fotona zelo majhna~\cite{over}

\section{Začetki elektrodinamike}

Maxwellove enačbe so nas dosegle konec devetnajstega stoletja, nekaj desetletij za tem, jih je Proca prepisal v obliko,
ki zado\v s\v ca neni\v celni fotonski masi. Ena\v cbe se potem glasijo

\begin{align}
	&\div\ve = \rho/\varepsilon_0 - m^2\varphi, \qquad \div\vb = 0, \notag \\
	&\rot\ve = \parc{\vb}{t}, \qquad \rot\vb = \mu_0\vec{\jmath} + \frac{1}{c_0^2}\parc{\ve}{t} - m^2\vec{A},
	\label{eqn:max-proca}
\end{align}

kjer je $c_0 = 1/\sqrt{\varepsilon_0\mu_0}$ \v se vedno dobro definirana in nespremenjena. Kot vidimo, so ena\v cbe Proca
v Lorentzovo kovariantni obliki nespremenjene za dualni napetostni tenzor elektromagnetnega polja, medtem ko dobi samo
napetostni tenzor masne popravke, 

\begin{equation}
	\partial_\nu F^{\mu\nu} + m^2 A^\mu = \jmath^\mu.
	\label{eqn:proca}
\end{equation}

\v Ce ena\v cbo \eqref{eqn:proca} diferenciramo, lahko od odondod dobimo pogoj za Lorentzovo umeritev, $\partial_\mu
A^\mu$, ki nam reducira eno od \v stirih prostostnih stopenj $A^\mu$, od koder sledi, da gre za foton, s spinom $1$ in
maso $m$. Gostota Lagrangejeve funkcije\footnote{v nadaljevanju zaradi preprostosti raje kar Lagrange-ian}, $\mathcal{L}$,
se v tem primeru glasi

\begin{equation}
	\mathcal{L} = -\frac{1}{4}F_{\mu\nu}F^{\mu\nu} + \frac{1}{2}m^2A_\mu A^\mu.
	\label{eqn:proca-lagrangian}
\end{equation}

To polje lahko nato kvantiziramo, poka\v zemo, da zado\v s\v ca bozonski algebri.

Vendar kot vidimo \v ze v en. \eqref{eqn:max-proca} in \eqref{eqn:proca}, postanejo potenciali, torej $A_\mu =
(\varphi, \vec{A})$ fizikalne opazljivke v fiksni umeritveni skali (ki je v tem primeru Lorentzova). Taka teorija nima
umeritvene invariance.

\subsection{Posledice}

Kot sem namignil, ena\v cbe Proca s seboj prina\v sajo dolo\v cene nev\v se\v cnosti/inovacije:

\begin{itemize}
	\item{kr\v senje umeritvene invariance~\cite{nieto1,nieto2},}
	\item{svetlobna disperzija~\cite{nieto1,nieto2}}
	\item{kr\v senje (lokalne) ohranitve naboja~\cite{nieto2},}
	\item{dodatna longitudinalna komponenta elektromagnetnega sevanja~\cite{nieto1,nieto2},}
	\item{elektromagnetna interakcija ima kon\v cni doseg~\cite{nieto1,nieto2}}
\end{itemize}

kar nas lahko nekoliko preseneti -- da bo $\varphi$ dobil Yukawovo obliko gre pri\v cakovati, vendar kr\v senje ohranitve
naboja ni tako od muh. Pod kr\v sitvijo umeritvene invariance mislimo umeritveno invarianco prve vrste.

Svetlobna disperzija je najve\v cja razlika, ki jo je Proca povdarjal. Ni jim bilo o\v citno, da bi bila elektrodinamika
tako zaradi Yukawovega potenciala kon\v cna in da bi to nato prevedlo v lokalno kr\v senje ogranitve elektri\v cnega
naboja.



\begin{thebibliography}{9}
	\bibitem{nieto2}
	A. S. Goldhaber and M. M. Nieto,
	"`Photon and Graviton Mass Limits"',
	arXiv:0809.1003v5 [hep-ph],
	(2010)

	\bibitem{nieto1}
	A. S. Goldhaber and M. M. Nieto,
	"`Terrestrial and Extraterrestrial Limits on The Photon Mass"',
	Rev. Mod. Phys. {\bf 43}, 277-296,
	(1971)

	\bibitem{stueckelberg}
	H. Ruegg and M. Ruiz-Altaba,
	"`The Stueckelberg Field"'
	arXiv:hep-th/0304245v2,
	(2003)

	\bibitem{higgs}
	E. Adelberger, G. Dvali and A. Gruzinov,
	"`Photon Mass Bound Destroyed by Vortices"',
	arXiv:hep-ph/0306245v2,
	(2003)

	\bibitem{over}
	G. Spavieri, J. Quintero, G.T. Gillies and M. Rodriguez,
	"`A survey of existing and proposed classical and quantum approaches to the photon mass"',
	Eur. Phys. J. D {\bf 61} 531-550,
	(2011)
\end{thebibliography}

\end{document}


