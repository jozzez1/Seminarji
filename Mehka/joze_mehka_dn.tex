\documentclass[12pt, a4 paper]{article}
\usepackage[slovene]{babel}
\usepackage[T1]{fontenc}
\usepackage[utf8]{inputenc}
\usepackage[small,width=0.8\textwidth]{caption}
\usepackage[usenames,dvipsnames]{xcolor}
\usepackage{graphicx}
\usepackage{amsmath, amssymb, float, fullpage, hyperref, mathrsfs}

\hypersetup{
	colorlinks=true,
	linkcolor=black!60!red,
	citecolor=black!60!green,
	urlcolor=black!60!cyan,
	filecolor=black!60!magenta
}

\newcommand{\rr}{
	\ensuremath{\langle r^2 \rangle}
}

\newcommand{\cor}{
	\ensuremath{\langle \vec{u}_i\cdot\vec{u}_{i+1}\rangle}
}

\renewcommand{\u}{
	\ensuremath{\hat{u}}
}

\renewcommand{\t}{
	\ensuremath{\hat{t}}
}

\renewcommand{\d}{
	\ensuremath{\mathrm{d}}
}

\newcommand{\od}[2]{
	\ensuremath{\frac{\d #1}{\d #2}}
}

\newcommand{\e}{
	\ensuremath{\mathrm{e}}
}

\newcommand{\tp}{
	\ensuremath{\vec{t}_\perp}
}

\newcommand{\tv}{
	\ensuremath{t_\parallel}
}

\newcommand{\III}{
	\ensuremath{\frac{Na}{4(1 - r/Na)}}
}

\newcommand{\F}{
	\ensuremath{\mathcal{F}}
}

\newcommand{\sh}{
	\operatorname{sh}
}

\newcommand{\ch}{
	\operatorname{ch}
}

\newcommand{\arctg}{
	\operatorname{arc\,tg}
}

\begin{document}

\begin{center}
\textsc{Fizika mehke snovi}\\
\textsc{2012/13}\\[0.5cm]
\textbf{Doma\v ca naloga -- model \v crvje verige}
\end{center}
\begin{flushright}
\textbf{Jo\v ze Zobec}
\end{flushright}

\section{Uvod}

To nalogo bi rad predstavil na malo druga\v cen na\v cin. Najprej bi rad pokazal kon\v cne rezultate in
\v sele potem pot, kako pridemo do njih.

Tekom naloge bom ulomke v eni vrstici pisal kot

\[
   \frac{a_1 a_2 a_3 \ldots a_N}{b_1 b_2 b_3 \ldots b_M} \equiv a_1 a_2 a_3 \ldots a_N/b_1 b_2 b_3 \ldots b_M,
\]

to pa zgolj zato, da se izognemo nepotrebnih oklepajev, tj. zavoljo la\v zjega branja.

V tej doma\v ci nalogi moram primerjati dva modela verig monomerov -- tj. polimerov in sicer: model
popolnoma gibke verige (PGV) in pa model \v crvje verige (WLC -- \emph{worm-like chain}). Klju\v cne razlike
so podane v tabeli~\ref{tab1}.

\begin{table}[H]
	\centering
	\caption{tu so predstavljene razlike med modeloma PGV in WLC. Model WLC je v limiti enak PGV,
		torej je WLC naravna raz\v siritev PGV in smo res na pravi poti.}
	\vspace{6pt}
	\begin{tabular}{c|c|c}
			& PGV					& WLC \\ \hline
	$F$		& $\frac{3}{2}k_B T\frac{r^2}{Na^2}$	& $\frac{k_B T}{\ell_p}[\frac{r^2}{2Na}
		- \frac{r}{4} + \III]$\\
	$\rr$		& $a^2N$				& $\ell_p^2 (2L/\ell_p + \e^{-2L/\ell_p})$ \\
	$\ell_p$	& $a/2$					& $\kappa/k_B T = a\kappa_0/k_B T$\\
	\end{tabular}
	\label{tab1}
\end{table}

Shajamo iz Hamiltoniana

\begin{equation}
	H = H_0 + H_\text{upogib} = H_0 + \frac{\kappa_0}{2}\sum_{i = 0}^{N-1}|\u_i - \u_{i+1}|^2 =
		H_0 - \kappa_0 \sum_{i = 0}^{N-1}\u_i \cdot \u_{i+1} + C,
	\label{osnova}
\end{equation}

kjer je $C$ konstanta, ki pove zgolj dol\v zine verige in jo lahko mirne du\v se od\v stejemo. Vektorji $\u_i$ so
normirani, tj. $\u \equiv \vec{u}/|\vec{u}|$. V zvezni limiti lahko taisti Hamiltonian zapi\v semo kot

\begin{equation}
	H \to \frac{\kappa}{2}\int_0^L \bigg|\od{\hat{t}}{s}\bigg|^2 \d s = \frac{\kappa}{2}
		\int_0^L c(s)^2 \d s, \qquad \hat{t} = \vec{t}/t, \quad \vec{t} = \od{\vec{r}}{s},
	\label{hamilton}
\end{equation}

kjer smo uvedli nove skalirane parametre: $L = aN$, kjer je $a$ dol\v zina monomera, $\kappa = a\kappa_0$, $s$ pa
je naravni parameter na na\v si krivulji (polimeru). Zaradi elasti\v cnega \v clena v Hamiltonianu ne bomo imeli
ostrih zavojev -- polimer bo gladka krivulja, $c(s)$ bo definiran po vsem obmo\v cju.

\section{Upogibni modul in persisten\v cna dol\v zina}

Upogibni modul $\kappa$ je povezan s persisten\v cno dol\v zino prek slede\v ce identitete. 

\begin{equation}
	\kappa = k_B T \ell_p,
\end{equation}

To lahko izra\v cunamo prek korelacijskih \v clenov v Hamiltoninanu~\eqref{osnova}

\begin{equation}
	\langle\u_i\cdot\u_{j}\rangle = \langle\cos\theta\rangle =
		\frac{1}{\mathcal{N}}\int\d\Omega\cos\theta\exp(\beta\kappa_0\cos\theta),
	\label{integral}
\end{equation}

kjer je $\mathcal{N}$ ustrezna normalizacija, ki je kar isti integral, vendar brez $\cos\theta$.
Integral~\eqref{integral} lahko zapi\v semo (po uvedbi nove spremenljivke $\cos\theta = x$) kot

\begin{equation}
	\langle\u_i\cdot\u_{j}\rangle = \frac{\int_{-1}^1 x\e^{x\beta\kappa_0}\d x}{\int_{-1}^1
		\e^{x\beta\kappa_0}\d x}.
\end{equation}

Spodnji integral, tj. $\mathcal{N}$, lahko kar takoj izra\v cunamo

\begin{equation}
	\mathcal{N} = \int_{-1}^1\d x\e^{\beta\kappa_0 x} = \frac{1}{\beta\kappa_0}\e^{\beta\kappa_0}
		\bigg|_{-1}^1 = \frac{\e^{\beta\kappa_0} - \e^{-\beta\kappa_0}}{\beta\kappa_0} =
		\frac{2\sh\beta\kappa_0}{\beta\kappa_0}.
\end{equation}

Z integralom v \v stevcu je nekoliko druga\v ce. Integrirati ga je treba "`per partes"', vendar je tudi
enostaven integral

\begin{align}
	\int_{-1}^1x\e^{\beta\kappa_0x}\d x &= \frac{1}{\beta\kappa_0}x\e^{\beta\kappa_0x}\Big|_{-1}^1
		 - \frac{1}{\beta\kappa_0} \int_{-1}^1\e^{\beta\kappa_0x}\d x = \notag \\
	&= \frac{1}{\beta\kappa_0}\bigg[x\e^{\beta\kappa_0x} - \frac{1}{\beta\kappa_0}
		\e^{\beta\kappa_0x}\bigg]_{x=-1}^{x=1} = \notag \\
	&= \frac{2}{\beta\kappa_0}\Big(\ch\beta\kappa_0 - \frac{1}{\beta\kappa_0}\sh\beta\kappa_0\Big).
\end{align}

Izraz~\eqref{integral} je torej enak

\begin{equation}
	\langle\u_i\cdot\u_j\rangle = \langle\cos\theta\rangle = \frac{\ch\beta\kappa_0}{\sh\beta\kappa_0} -
		\frac{1}{\beta\kappa_0}.
	\label{persist}
\end{equation}

Z razvojem za majhne $\beta$ dobimo obna\v sanje okrog $T \to \infty$, ki je linearno, vendar takrat verjetno
nimamo ve\v c polimerne faze. Izraz~\eqref{persist} je treba razviti za nizke temperature, oz. velike $\beta$
in dobimo rezultat

\begin{equation}
	\langle\cos\theta\rangle \stackrel{\beta \to \infty}{\longrightarrow} 1 - \frac{1}{\beta\kappa_0}.
	\label{betavelik}
\end{equation}

Vendar pa lahko ena\v cbo~\eqref{integral} izra\v cunamo tudi druga\v ce. Ker vemo da gre za korelacije in
vemo, da morajo slednje padati eksponentno glede na persisten\v cno dol\v zino, mora hkrati veljati tudi to

\begin{equation}
	\langle\u_i\cdot\u_j\rangle = \exp\bigg(-\frac{|i - j|a}{\ell_p}\bigg).
\end{equation}

Hamiltonian nam narekuje, da moramo gledati korelacije sosedov, zatorej vzamemo $j = i + 1$,

\begin{equation}
	\langle\u_i\cdot\u_j\rangle = \exp\bigg(-\frac{a}{\ell_p}\bigg) \approx 1 - \frac{a}{\ell_p}.
	\label{ellp}
\end{equation}

\v Ce sedaj primerjamo ena\v cbi~\eqref{ellp} in~\eqref{betavelik}, dobimo

\begin{equation}
	\ell_p = a\kappa_0\beta = \kappa\beta = \frac{\kappa}{k_B T}
\end{equation}


\section{Entropijska elasti\v cnost}

To smo v PGV ponazorili z $F$, tj. prosto energijo. V potu profesorjevega obraza smo se prebili do izraza

\begin{equation}
	F = \frac{3}{2} k_B T \frac{r^2}{Na^2}.
\end{equation}

Tu se bomo vsega seveda lotili prek Hamiltoniana~\eqref{hamilton}. Na\v s polimer bomo sedaj raztegnili s
silo $\F$ v smeri $\hat{r}$ in tako dobili

\begin{equation}
	H' = H - \F\hat{r} \cdot \int_0^L \t(s)\d s. \label{prham}
\end{equation}

Dodatek v hamiltonianu predstavlja prejeto delo. Recimo, da je sila mo\v cna, tako da je $\t$ skoraj
vzporeden na $\hat{r}$. Potem se spla\v ca $\t$ parametrizirati glede na smer $\hat{r}$ tako, da
$\t(s) = \tv\hat{e}_x + \tp (s)$. Vektorja $\t(s)$ lahko v vsaki to\v cki opi\v semo le z dvema komponentama,
\v ce ustrezno izberemo koordinatni sistem, vendar se bo ta zaradi zvijanja polimerne klobke moral vrteti,
zaradi \v cesar $\tp (s)$ ostaja dvokomponentni vektor.

Zaradi velike obremenitve lahko izraze v en.~\eqref{prham} poenostavimo tako

\begin{align}
	\hat{r}\cdot\t \approx \tv &= \sqrt{1 - |\tp|^2} \approx 1 - \frac{1}{2}|\tp|^2 \label{tivu} \\
	\bigg|\od{\tv}{s}\bigg|^2 &= \bigg(\od{\tv}{s}\bigg)^2 + \bigg|\od{\tp}{s}\bigg|^2 \approx
		\bigg|\od{\tp}{s}\bigg|^2.
\end{align}

Na\v s Hamiltonian~\eqref{prham} tako postane

\begin{equation}
	H \approx \int_0^L\bigg(\frac{\kappa}{2}\bigg|\od{\tp}{s}\bigg|^2 + \frac{\F}{2}|\tp|^2\bigg)\d s - \F L.
	\label{harm}
\end{equation}

Radi bi se znebili diferencialov in operirali le s $\tp$-ji, zato bomo naredili Fourierovo transformacijo
in dobili \v cisto integralsko ena\v cbo. Pre\v ziveli bodo le kosinusni \v cleni,

\begin{align}
	\tp &= 2\sum_q \vec{a}_q\cos qs, \\
	\vec{a}_q &\equiv \frac{1}{L}\int_0^L\d s\ \tp \cos qs,
\end{align}

s \v cimer dodatno poenostavimo en.~\eqref{harm}

\begin{equation}
	H = L\sum_{q = 1}^{\infty} (\kappa q^2 + \F)|\vec{a}_q|^2 - \F L.
	\label{popravek}
\end{equation}

Porazdelitev spremenljivk takega hamiltoniana je Gaussova in iz ekviparticijskega izreka vemo, da na vsako
prostostno stopnjo pride $k_B T/2$ energije. Ker ima na\v s vektor $\vec{a}_q$ dimenzijo 2, pomeni da bomo
imeli ravno $k_B T$, tj. \v ce nadaljujemo od~\eqref{popravek}

\begin{equation}
	H = L \sum_{q = 0}^\infty \overbrace{(\kappa q^2 + \F)\langle|\vec{a}_q|^2\rangle}^{k_B T}.
\end{equation}

To vedo\v c lahko izra\v cunamo raztezek $r$ prek en.~\eqref{tivu}

\begin{align}
	\frac{r}{L} &= \frac{1}{L}\int_0^L \langle\hat{r}\cdot\t\rangle \approx
		1 - \frac{1}{2L}\int_0^L \langle|\tp(s)|^2\rangle \d s = \notag \\
	&= 1 - 2\sum_q \langle|\vec{a}_q|^2\rangle \approx 1 -\frac{1}{\pi}\int_0^\infty \d q\ \langle|\vec{a}_q|^2
		\rangle = 1 - \frac{k_B T}{\pi}\int_0^\infty \frac{\d q}{\kappa q^2 + \F} = \notag \\
	&= 1 - \frac{k_B T}{\pi\kappa}\int_0^\infty\frac{\d q}{q^2 + \F/\kappa} =
		1 - \frac{k_B T}{\pi\kappa}\sqrt{\frac{\kappa}{\F}}\arctg\bigg(q\sqrt{\frac{\kappa}{\F}}\bigg)
		\bigg|_{q = 0}^{q = \infty} = \notag \\
	&= 1 - \frac{k_B T}{\pi\sqrt{\kappa\F}}\frac{\pi}{2} = 1 - \frac{k_B T}{\sqrt{4\kappa\F}},
\end{align}

sedaj pa \v se uporabimo $\kappa = \ell_p k_B T$ in dobimo

\begin{equation}
	\frac{r}{L} \approx 1 - \sqrt{\frac{k_B T}{4\ell_p \F}}, \quad \text{oz.} \quad
	\F \approx \frac{k_B T}{4\ell_p}\frac{1}{(1 - r/L)^2}.
\end{equation}

Ta ena\v cba je veljavna v re\v zimu velikih obremenitev. Pravi rezultat izgleda takole

\begin{equation}
	\F(r) = \frac{k_B T}{\ell_p}\bigg[\frac{r}{L} + \frac{1}{4}\bigg(\frac{1}{(1 - r/L)^2} - 1\bigg)\bigg].
	\label{sila}
\end{equation}

Sedaj se \v ze kar sama od sebe ponuja substitucija v brezdimenzijsko obliko
\[
	\mathscr{F} = \frac{\F\ell_p}{k_B T}, \quad x = r/L.
\]

Funkcijo $\mathscr{F}(x)$ vidimo na grafu~\ref{fig1} v log-log skali.

\begin{figure}[H]\centering
	\input{sila.tex}
	\caption{Kot vidimo, imamo na za\v cetku res v zelo dobrem pribli\v zku linearen odziv,
		ki pa blizu pola hitro naraste \v cez vse meje in pri `$x = 1$' dobimo poten\v cno
		singularnost.}
	\label{fig1}
\end{figure}

Izraz~\eqref{sila} lahko integriramo, in tako dobimo prosto energijo $F$, ki tukaj predstavlja pro\v znostno
energijo. Naredimo nedolo\v cen integral po $r$,

\begin{align}
	F &= \int\F(r)\d r = \frac{k_B T}{\ell_p}\int\d r\bigg[\frac{r}{L} - \frac{1}{4} +
		\frac{1}{4(1 - r/L)^2}\bigg] = \notag \\
	&= \frac{k_B T}{\ell_p}\bigg[\frac{r^2}{2L} - \frac{r}{4} + \frac{L}{4}\frac{1}{1 - r/L}\bigg]
		+ \text{konst.}
	\label{prosta}
\end{align}

\v Ce izraz~\eqref{prosta} razvijemo za $r \ll L$ in upo\v stevamo, da v tem primeru velja $\ell_p = a/2$
in $L = Na$, potem dobimo

\begin{align}
	F &= \frac{2k_B T}{a}\bigg[\frac{r^2}{2L} - \frac{r}{4} + \frac{L}{4}(1 + r/L + r^2/L^2)\bigg]
		+ \text{konst.} = \notag \\
	&= \frac{2k_BT}{a} \frac{3r^2}{4L} + \text{konst.} = \frac{3}{2}k_B T \frac{r^2}{Na^2} + \text{konst.}
	\label{entropijskaF}
\end{align}

kjer sem izraz razvil do $\mathcal{O}(r^3)$ in do aditivne konstante natan\v cno. Izraz~\eqref{entropijskaF} je
natanko enak tistemu, ki ga dobimo v re\v zimu PGV.

\section{Povpre\v cen kvadrat oddaljenosti v $r$}

Pa izra\v cunajmo ta $\rr$. Tega se bomo lotili tako

\begin{align}
	\rr &= \langle \vec{r} \cdot \vec{r} \rangle = \notag \\
	&= \bigg\langle \int_0^L \vec{t}(s) \d s \cdot \int_0^L \vec{t}(s')\d s'\bigg\rangle = \notag \\
	&= \int_0^L\d s\int_0^L\d s' \langle \vec{t}(s) \cdot \vec{t}(s') \rangle = \notag \\
	&= \int_0^L \d s\int_0^L \d s'\ \e^{|s - s'|/\ell_p}, \label{rr}
\end{align}

kjer smo uporabili dejstvo, da morajo korelacije padati eksponentno, saj je sistem zadosti ergodi\v cen. Zaradi te
absolutne vrednosti moramo paziti, kako bomo naprej integrirali. Integral bomo razbili na dva primera:

\begin{align}
	s < s' &\longrightarrow \e^{-\frac{s - s'}{\ell_p}}, \notag \\
	s > s' &\longrightarrow \e^{-\frac{s' - s}{\ell_p}}. \notag
\end{align}

Sedaj lahko nadaljujemo od~\eqref{rr} in uporabimo aditivnost integralov

\begin{align}
	\rr &= \int_0^L\d s\bigg[\int_0^s \d s'\ \e^{-(s - s')/\ell_p} + \int_s^L \d s'\ \e^{-(s' -
		s)/\ell_p}\bigg] = \notag \\
	&= \ell_p \int_0^L \d s\ \Big[1 - \e^{-s/\ell_p} + 1 - \e^{-s/\ell_p}\e^{-L/\ell_p}\Big] = \notag \\
	&= \ell_p^2 \Big(2L/\ell_p + \e^{-2L/\ell_p}\Big).
\end{align}

Naredimo limito za dolge verige -- $L \gg \ell_p$,

\begin{equation}
	r^2 \sim 2\ell_p L = 2\ell_p N a \approx Na^2 \quad \Longrightarrow \quad r \sim a\sqrt{N},
	\label{rkvadrat}
\end{equation}

kjer smo upo\v stevali, da je v tej limiti $\ell_p = a/2$. To je \v cisto isti rezultat kot v PGV. Kadar pa je
veriga prekratka se ne bo obna\v sala kot PGV, ampak bo za $L \ll \ell_p$ veljalo

\begin{equation}
	r \propto \ell_p,
	\label{konec}
\end{equation}

kar je seveda smiselno: takrat bo veriga namre\v c bila tako kratka, da bo edina karakteristi\v cna dol\v zina
ki jo veriga pozna, kar persisten\v cna dol\v zina $\ell_p$.

Spet lahko na podlagi ena\v cbe~\eqref{rkvadrat} vpeljemo brezdimenzijski $\rho^2(x)$, kjer smo tokrat vzeli
$x = 2L/\ell_p$. Graf lahko vidimo na sliki~\ref{fig2}.

\begin{figure}[H]\centering
	\input{sila1.tex}
	\caption{Pri\v cnemo s konstantnim \v clenom, kot napoveduje ena\v cba~\eqref{konec}. Ko
		polimerna klobka postaja dalj\v sa, smo \v cedalje bolj v linearnem re\v zimu.}
	\label{fig2}
\end{figure}


\end{document}
