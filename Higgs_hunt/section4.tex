\section{Latest Higgs boson search results}

There are two experiments at the LHC which joined the Higgs hunt: ATLAS and CMS.

Cross section per each channel has been calculated by the LHC Higgs Cross Section Working Group~\cite{CERN}. Graphs per channel are 
depicted below (fig.~\ref{fig:channels}).

\vspace{12pt}

\begin{myfig}[8cm]{pics/XSBR-7TeV-SM-3.eps}
	\caption{Cross sections for each channel at $\sqrt{s} = 7$ TeV. This plots combine the cross sections with branching ratio,
		thus giving the total individual channel cross section.}
	\label{fig:channels}
\end{myfig}

The plot below shows combined exclusions by Tevatron D\O and CDF compared to latest LHC exclusion zones.

\begin{myfig}[9cm]{pics/tev28febsmbayeslimits.eps}
	\caption{Latest exclusion plots given by~\cite{tev:review}. They are from Tevatron report from March 16, 2012.}
	\label{fig:exclusions}
\end{myfig}

The CMS collaboration reported it's largest excess (local significance of 3.1 $\sigma$) is reported at $M_H = 124 GeV$~\cite{cms:prelim}.
The ATLAS collaboration reported largest excess at 126 GeV (local significance 3.6 $\sigma$). When
uncertainties of ATLAS are taken into account this significance drops to 2.5 $\sigma$~\cite{atlas:prelim} and for CMS to 1.5 $\sigma$.

\begin{figure}[H]
	\centering
	\begin{subfigure}[b]{0.45\textwidth}
		\centering
		\includegraphics[width=\textwidth, keepaspectratio=1]{pics/atlassigma}
		\caption{ATLAS results~\cite{atlas:prelim}.}
		\label{fig:atlas:prelim}
	\end{subfigure}~
	\begin{subfigure}[b]{0.45\textwidth}
		\centering
		\includegraphics[width=\textwidth, keepaspectratio=1]{pics/cmssigma}
		\caption{CMS results~\cite{cms:prelim}.}
		\label{fig:cms:prelim}
	\end{subfigure}
	\caption{CMS and ATLAS preliminary results in 7th of February 2012. Plots depict local p-values and their respective standard deviations.}
	\label{cern:prelim}
\end{figure}

High mass Higgs bosons weren't excluded yet. For CMS they have been excluded in the 127-600 GeV mass range at 95\%CL~\cite{cms:prelim} and
for ATLAS in the 131-238 GeV and then in the range of 251-466 GeV mass range at the same CL~\cite{atlas:prelim}.

In any case, low mass Higgs boson is preferred (and for now more probable). Looking at the fig.~\ref{fig:exclusions} it also seems that we are gaining
on the Higgs boson. The question of the Higgs boson existence will most likely be resolved within this or the next year.
