\section{Search strategies at the LHC}

Even if we don't know all the numbers, we can still bound the mass of the Higgs boson, so that the theory will still have
sense (tab.~\ref{tab:vacuum:scenarios}). Upper bound is based upon the scale where the SM breaks down\footnote{Perturbation theory
fails to give us credible results.}, lower bound is based on the $\vac$ stability and unitary constraints from the
electroweak interaction.

\begin{table}[H]
	\centering
	\caption{The theoretically preferred scenarios. Metastable vacuum expectation state has decay time
		longer than the age of the universe~\cite[pg.~5]{higgs:review}.}
	\begin{tabular}{l c r}
		\hline
		Stable $\vac$ & \hspace{2cm} & {\color{BrickRed}$130$ GeV} $\lesssim M_H \lesssim$ $180$ GeV \\
		Metastable $\vac$ & \hspace{2cm} & {\color{Green} $115$ GeV} $\lesssim M_H \lesssim$ $180$ GeV \\
		\hline
	\end{tabular}
	\label{tab:vacuum:scenarios}
\end{table}

Needless to say, the only way, physicists are considering to produce $H$ is using high energy beam collisions in colliders.
Hadron colliders are preferred, since they are the exploratory machines\footnote{While lepton colliders are precision machines.}.
During the course of time there were three that were at least partially dedicated to the $H$ hunt and got some results:
electron--positron (LEP), proton--anti-proton (Tevatron) and proton--proton (LHC).

Why do we need such a high energy? When we are colliding protons, the problem is this that each proton behaves as a group of
free particles (QCD effect). The constituents are then responsible for interactions. Each of them carries only a part of the
total momentum. And it turns out that each valence quark carries approximately 1/6 of it. So in order to probe mass regions up
to 1 TeV, we would need at least $\sim 6$ TeV energy. They decided to make it 7 TeV, just to be on the safe side.

The reason why this project was accepted and others were refused\footnote{Namely the UNK in the USSR and SSC in the USA.}, was
probably because LHC is using the old LEP tunnel. But in order to make more energetic collisions, much stronger magnets were needed.
In circular colliders, energy yield -- $E$ [TeV], accelerator circumference\footnote{If there was just the radius then there would be a different equation.}
$\rho$ [km] and magnetic field strength $B$ [T], can be estimated using this formula

\[
	B\rho \sim E \cdot 3.36 \frac{\rm T\cdot km}{\rm TeV} ,
\]

from which we obtain incredible $B \sim 8.71$ T (using 7 TeV and 27 km). The temperatures needed to cool such strong magnets and keep
them in the
superconducting phase\footnote{We want this, so the current passes with no resistance and so we keep the energy losses low.} are so
low, that the liquid helium coolant must be in the superfluid phase. This resulted in greater complications, since superfluid helium
passes through most of the known materials as if through a strainer. Magnets are now expected to operate at $B = 8.36$ T.

In order to predict cross sections and branching ratios, theoretical models were used. The biggest problem was this: of all the
possible reactions, that can take place, the cross section for the Higgs boson is lower by several orders of magnitude
(fig.~\ref{hadron:cross}). For this reason, sufficiently high energy and detector luminosity have to be reached.

\vspace{12pt}
\begin{myfig}[9cm]{pics/cross1.jpeg}
	\caption{Some cross sections at modern hadron colliders~\cite{paussdittmar}. We can see that Higgs boson production is very rare.
		We are producing a lot of background (known events) and very little signal.}
	\label{hadron:cross}
\end{myfig}

Higgs boson cross sections are low due to the fact that its couplings are proportional to mass. Ordinary matter is very light (electrons,
up and down quarks) and so production rates are lower. Higgs
couplings are by several orders of magnitude higher if we used boson collisions\footnote{As we saw in one of the previous seminars,
photon colliders look very appealing for this very fact.}. 

So now that we have Higgs boson production within our reach (if it exists), let us consider the possible production channels (see
fig~\ref{lhc:prod}).

\begin{myfig}[11.5cm]{pics/lhc.pdf}
	\caption{Theoretical Higgs boson production cross sections for the proton-proton collisions at the energy of collisions
		$\sqrt{s} = 14$ TeV. Computed by the TeV4LHC Higgs working group~\cite{tev4lhc}.}
	\label{lhc:prod}
\end{myfig}

\subsection{Gluon fusion}

The coupling constants for ordinary matter are actually so low, that at the LHC, Higgs boson will most likely be produced via the
gluon fusion (fig.~\ref{diagrams}) -- gluons form an intermediate top quark loop, to which Higgs boson can
couple really well because of their 175 GeV mass.
Higgs boson production is then more likely to occur at the loop order, rather than the tree order, which is
rather surprising. This channel, however, has a lot of background and cannot easily be distinguished.

Even so, with our well defined BRs this is not enough. We need two signals in order to know that we produced the Higgs boson:
we need some sort of ``tagging'' to determine the production mode as well. And so production modes with low background and some
simple tagging mechanism are preferred. That's why the gluon fusion, though it is the dominant production mode, isn't preferred,
since it isn't distinctive enough (large backgrounds). This doesn't mean that the channel is useless, it just means that Higgs boson
identification through this channel is harder than the rest.

\subsection{Weak gauge boson fusion}

Thankfully enough, the 2nd dominant mode, the weak gauge boson fusion (fig.~\ref{diagrams}), is quite distinctive: $W$ and $Z$ fusion
processes. Such reactions
can then be tagged quite easily and they also have low background~\cite[pg.~157]{elena:green}. Other possible way are annihilation
processes -- quarks
annihilate through electroweak interaction. These are the so called `Drell-Yan' processes, where off-shell $Z$ or $W$ radiates a Higgs
boson~\cite[pg.~158]{elena:green}. These processes are preferred since through them we can directly measure the Higgs boson
coupling to the weak gauge bosons.

We can then use the different decay modes in order to clearly distinguish between different channels and get a really clean signal.

\subsection{Other modes}

There are still other production modes, Higgs boson pair production, top quark fusion and the so called ``\emph{associated production}'', because in these
reactions the Higgs boson is produced in association with other particles. These modes don't play a crucial role in the Higgs discovery, since
they are of a very low probability, but they will play more important roles later if the Higgs is discovered, since using the associated
products we can measure the spin, charge conservation and other properties of the Higgs boson.

\vspace{20pt}
\begin{myfig}[12cm]{pics/production.jpg}
	\vspace{12pt}
	\caption{The most common production modes on the Feynman diagrams for better visualization.}
	\label{diagrams}
\end{myfig}

\subsection{Higgs decay modes}

Higgs boson decay width in that mass region is $10^{-21} \lesssim \Gamma \lesssim 10^{-23}$ s, so the we
can only identify it indirectly from reconstruction, the transverse energy $E_T$, the transverse linear momentum $\vec{p}_T$ and through its
decay products which are shown in the lower graph (see fig.~\ref{fig:SMHbrsmall}).

Since Higgs couplings are proportional to masses of the particles, our surest bets are heavy particles. Cleanest channels are produced
via weak gauge boson decays, which subsequently decay into leptons, or into $q\bar{q}$ pairs, which produce the so called `jets' of
particles\footnote{Since we cannot have bare quarks they immediately ``dress'' themselves with additional quarks which then do the same until
energy equilibrium has been reached. The remaining quarks are all bound hadron states. In this event we say that we get the jet of particles.}.

Of course the Higgs hunt takes place in all the channels, but the cleanest are, as said before, in the $H \to WW^{(*)}$ or $H \to ZZ^{(*)}$ decays.
We must identify Higgs boson through subsequent decays.

The cleanest channel is $ZZ^{(*)} \to \ell^+\ell^-\ell^+\ell^-$, since we cannot get this situation in many ways. It is very unlikely to happen,
but if it happens, it's very likely to have come from the Higgs boson.

Then there are three others we really like to consider: $WW^{(*)} \to \ell^+\nu\ell^-\bar{\nu}$ and $WW^{(*)} \to \ell\nu q\bar{q}$ (two-jet event with
missing $E_T$) and $WW^{(*)} \to q\bar{q}q\bar{q}$ (four-jet event).

\vspace{12pt}
\begin{myfig}{pics/YRHXS2_BR_Fig1.eps}
	\caption{SM Higgs boson branching ratios (ratio between this channel decay rate to total decay rate) including uncertainties.
		Plotted by LHC Higgs Cross Section Working Group~\cite{CERN}.}
	\label{fig:SMHbrsmall}
\end{myfig}
