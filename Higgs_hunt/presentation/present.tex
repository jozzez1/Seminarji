\documentclass{beamer}
\usepackage[utf8]{inputenc}
\usepackage[english]{babel}

\usepackage{amsmath, amssymb, graphicx, float, stackrel, slashed}

%\usepackage{utopia}

\usetheme{Warsaw}

%title info
\title{The hunt for the Higgs boson}
\subtitle{Seminar -- 1st year of the Physics 2nd cycle}
\author{Jože Zobec, BSc}
\date[23. 5. 2012]{Ljubljana, 23rd of March, 2012}
\institute[UL, FMF]{Supervisor: Marko Mikuž, PhD}

\newcommand{\vac}{
	\ensuremath{\langle\phi\rangle}
}

\begin{document}

\begin{frame}
	\titlepage
\end{frame}

\begin{frame}
	\frametitle{Overview}
	\tableofcontents
\end{frame}

\section{Introduction}
\begin{frame}
	\frametitle{Beginnings}
	\begin{itemize}
		\item{Electroweak theory has problems -- cannot be renormalizable,}
		\item{use of contact interactions,}
		\item{use of massive propagators.}
		\item{Theory {\bf still} not renormalizable.}
	\end{itemize}
\end{frame}

\begin{frame}{The Higgs field}
	  \begin{itemize}
	   \item{1962-1964 -- Several independent descriptions of a new mechanism.}
	   \item{Introduction of a new complex scalar field -- $\phi$.
		\begin{figure}[H]
			\includegraphics[keepaspectratio=1, width=0.5\textwidth]{../pics/potential.jpeg}
		\end{figure}
		}
	   \item{Due to symmetry reasons, only even powers are included.}
	   \item{Higher powers again render the theory non-renormalizable.}
	  \end{itemize}
\end{frame}

\begin{frame}{Symmetry breaking}
	Minimum is not at the $V(\phi) = 0 = |\phi|$, but at $\phi_{\pm} = \pm \sqrt{\frac{-2\mu}{\lambda}}$.
	We use $\phi_+ \equiv \vac$. The moment we do a Taylor expansion around this minimum, we break the
	symmetry.
\[
	\phi = \vac + \varphi,
\]

Where $\varphi$ is now the Higgs field and

\[
	\varphi \stackrel{\vac}{\slashed{\longrightarrow}} \varphi + C
\]

\end{frame}

\begin{frame}{Couplings and masses}
	After solving the interactions for this field, we see that $\vac$ induces masses for the Weak gauge
	bosons and fermions:

\[
	M_W = g_W\vac\sqrt{2} \qquad M_Z = \frac{g_W\vac\sqrt{2}}{\cos\theta_W},
\]
\[
	m_f = 2 g_f\vac = M_W \frac{g_f}{g_W}\sqrt{2},
\]

	where $g_f$ and $g_W$ are couplings of the interactions.

\begin{itemize}
	\item{Couplings are proportional to mass \ldots}
	\item{More mass -- stronger coupling -- more interactions.}
	\item{Top quark -- heaviest known elementary particle}
\end{itemize}

\end{frame}

\begin{frame}[t]{Self couplings and Higgs bosons}
	How do we get Higgs bosons from this?
	\only<1, 2, 3, 4, 5>{
	\begin{itemize}
		\only<1->{\item{Higgs also couples to itself -- like gluons.}}
		\only<2->{\item{It gives the mass to itself.}}
		\only<3->{\item{$\sqrt{-\mu^2}$ from the $V(\phi)$ can be interpreted as mass.}}
		\only<4->{\item{We so quantize the Higgs field $\varphi$ -- we get the Higgs bosons.}}
	\end{itemize}}

\only<5->{
\[
	M_H = \sqrt{-\mu^2} = \vac\sqrt{\lambda/2} = v\sqrt{\lambda} = 246\ \mathrm{GeV} \cdot \sqrt{\lambda} .
\]}
\end{frame}

\section{LHC search strategies}
\subsection{LHC construction strategy}
\begin{frame}{$M_H$ boundaries}
	\only<1, 2, 3, 4, 5, 6, 7, 8 ,9>{
	Theoretical boundaries:
	\begin{enumerate}
		\only<1-, 2->{\item{\only<1->{Stable $\vac$}:
			\only<2->{130 $\lesssim M_H \lesssim$ 180 GeV,}}}
		\only<3-, 4->{\item{\only<3->{metastable $\vac$}:
			\only<4->{115 $\lesssim M_H \lesssim$ 180 GeV.}}}
	\end{enumerate}
	\only<5->{
	Experimentally it could be anywhere in the $M_H \lesssim 1$ TeV area, since the upper boundary is
	based on  perturbativity.}
	\only<6-, 7-, 8-, 9->{
	\begin{itemize}
		\only<6->{\item{Only hadron colliders can reach such energies.}}
		\only<7->{\item{Each constituent carries its own part of the momentum.}}
		\only<8->{\item{To produce $M_H \sim 1$ TeV each valence quark needs $\sim$ 1 TeV}}
		\only<9->{\item{Hence $E \sim 6$ TeV.}}
	\end{itemize}}
	}
\end{frame}

\begin{frame}[t]{Magnets}
	\only<1, 2, 3, 4, 5, 6, 7, 8>{
	\begin{itemize}
		\only<1->{\item{LHC uses 2 proton beam collisions.}}
		\only<2->{\item{Each proton in the beam has 7 TeV.}}
		\only<3->{\item{We can calculate the necessary magnet strength:}}
	\end{itemize}
\only<4->{
\[
	B\rho \sim E \cdot 3 \mathrm{\frac{T \cdot km}{TeV}}
\]}
\only<5-, 6-, 7-, 8->{
	\begin{itemize}
		\only<5->{\item{$B \sim 7.7$ T.}}
		\only<6->{\item{With corrections -- $B \approx 8.44$ T.}}
		\only<7-, 8->{\item{\only<7->{To keep magnets superconducting with He coolant}
			\only<8->{we need \alert{superfluid} phase.}}}
	\end{itemize}
}}
\end{frame}

\begin{frame}[t]{Higgs boson production rates and cross sections}
	\only<1,2,3,4,5,6>{

		\only<1,2>{
		\only<1->{Cross sections for hadron colliders:}
		\only<2->{
		\begin{figure}[H]
			\includegraphics[keepaspectratio=1, height=0.8\textheight]{../pics/cross1.jpeg}
		\end{figure}}}

		\only<3,4,5,6>{
		\begin{itemize}
			\only<3->{\item{For the region $100 < M_H < 200$ GeV we produce only
				$\sim 10^{-4}$ events per second.}}
			\only<4->{\item{Each year has $\sim 10^7$ seconds.}}
			\only<5->{\item{We only produce $\sim 1000$ events per year at the energy
				 \hbox{$\sqrt{s} \sim 14$ TeV}.}}
			\only<6->{\item{We will have to run for many years before getting enough statistics.}}
		\end{itemize}}
	}
\end{frame}

\subsection{Detection strategies}
\begin{frame}[t]{Higgs boson production modes at the LHC}
	\only<1,2,3,4,5>{

	\only<1>{
		Theoretical production cross sections at $\sqrt{s} = 14$ TeV:
		\begin{figure}[H]
			\includegraphics[keepaspectratio=1, width=0.8\textwidth]{../pics/lhc.pdf}
		\end{figure}}

	\only<2>{
		Let us now take a look at the Feynman diagrams for the four dominant production modes:
		\begin{figure}[H]
			\includegraphics[keepaspectratio=1, width=0.8\textwidth]{../pics/production.jpg}
		\end{figure}
	}

	\only<3,4,5>{
		\begin{itemize}
			\only<3->{
			\item{We can see that the {\bf gluon fusion} occurs only at the loop order, but is by
				far the most common production mode. It has a lot of background and the
				signal is unclean.}}
			\only<4->{
			\item{The {\bf weak gauge boson fusion} on the other hand is a form of a Drell-Yan
				process and it produces a cleaner signal.}}
			\only<5->{
			\item{The remaining two processes are called the {\bf associated production} and are
				less common. Important role later, if the Higgs is confirmed.}}
		\end{itemize}
	}}
\end{frame}

\begin{frame}[t]{Higgs boson decay modes}
	\only<1,2>{
		\only<1->{
	How the channel is clean is determined by the Higgs boson decay modes that follow. We so need the
	Ratios between the decay width of the particular channel with the total decay width -- 
		{\bf branching ratio (BR)}:}
		
		\only<2->{
		\begin{figure}[H]
			\includegraphics[keepaspectratio=1, width=0.6\textwidth]{../pics/YRHXS2_BR_Fig2.eps}
		\end{figure}}
	}
\end{frame}

\begin{frame}[t]{Secondary decays}
	\only<1, 2>{
		\only<1->{
	We determine the total channel cross sections by combining both the branching ratios of the
	Higgs boson and the secondary products.}
		\only<2->{In turn we obtain
	\begin{figure}[H]
		\includegraphics[keepaspectratio=1, width=0.5\textwidth]{../pics/XSBR-7TeV-SM-3.eps}
	\end{figure}
		}
	}
\end{frame}

\section{Past search attempts}
\begin{frame}[t]{LEP}
	\only<1,2,3,4,5,6,7,8,9,10>{
	\begin{itemize}
		\only<1->{\item{CERN, 1989 -- 2000.}}
		\only<2->{\item{$e^+e^-$ collider.}}
		\only<3->{\item{Prequel to the LHC (same tunnel).}}
		\only<4->{\item{Energies up to $\sqrt{s} \sim 200$ GeV.}}
		\only<5->{\item{Four experiments: ALEPH, OPAL, DELPHI and L3.}}
		\only<6->{\item{Dominant mode is associated production: $e^+e^- \to HZ$.}}
		\only<7->{\item{ALEPH reported excess of $\sim 3 \sigma$ at the 115 GeV \ldots}}
		\only<8->{\item{\ldots others did not \ldots}}
		\only<9->{\item{\ldots but they couldn't exclude it!}}
		\only<10->{\item{In turn, Higgs boson with $M_H < 114.4$ GeV was excluded
			at 95\% confidence level (CL).}}
	\end{itemize}
	}
\end{frame}

\begin{frame}[t]{Tevatron}
	\only<1,2,3,4,5,6,7>{
	\begin{itemize}
		\only<1->{\item{Fermilab, 1983 -- 2011.}}
		\only<2->{\item{$p\bar{p}$ collider.}}
		\only<3->{\item{Second largest proton collider after LHC.}}
		\only<4->{\item{Energies up to $\sqrt{s} = 1.96$ TeV in 2011.}}
		\only<5->{\item{Two collaborations: CDF and D\O{}.}}
		\only<6->{\item{Mostly used weak gauge boson fusion for production.}}
		\only<7->{\item{Closed down in September 2011. Data is still under analysis.}}
	\end{itemize}}
\end{frame}

\section{Latest results}
\subsection{Exclusion plots}
\begin{frame}{Exclusion graphs -- how they work:}
	\only<1,2>{
	\only<1>{
	We don't know where the Higgs boson is. So scientists are searching everywhere, excluding where there
	is nothing and thus narrowing down the mass range where the Higgs boson could experimentally exist
	at a certain CL. Usually 95\% CL.
	}
	
	\only<2>{
	\begin{figure}
		\includegraphics[keepaspectratio=1, width=0.6\textwidth]{../pics/sexclusion.jpg}
	\end{figure}
	}
	}
\end{frame}

\subsection{Tevatron}
\begin{frame}[t]{Tevatron preliminary results}
	\only<1,2>{
	\only<1->{
	Latest reports from 16th of March, 2012.}

	\only<2->{
	\begin{figure}[H]
		\includegraphics[keepaspectratio=1, width=0.6\textwidth]{../pics/justtev.pdf}
	\end{figure}}}
\end{frame}

\subsection{LHC preliminary results}
\begin{frame}[t]{CMS preliminary results}
	\only<1,2,3>{
	\only<1->{
	Latest reports from 7th of February, 2012. There is $\sim 5$ fb$^{-1}$ of data.}

	\only<2->{
	\begin{figure}[H]
		\includegraphics[keepaspectratio=1, width=0.4\textwidth]{../pics/cmssigma.jpeg}
	\end{figure}}

	\only<3->{
		To avoid biased results, we take convert local significance into global. After this we get the
		that the $M_H \approx 124$ GeV at 1.5 $\sigma$.}
	}
\end{frame}

\begin{frame}[t]{ATLAS preliminary results}
	\only<1,2,3>{
	\only<1->{
	Latest reports from 7th of February, 2012. There is $\sim 5$ fb$^{-1}$ of data.}

	\only<2->{
	\begin{figure}[H]
		\includegraphics[keepaspectratio=1, width=0.4\textwidth]{../pics/atlassigma.jpeg}
	\end{figure}}

	\only<3->{
		Global significance of this sample is again smaller: $M_H \approx 126$ GeV at 2.5 $\sigma$.}
	}
\end{frame}

\subsection{Total preliminary results}
\begin{frame}[t]{Combined exclusion}
	\only<1,2>{
	\only<1->{
	Latest reports from 16th of March, 2012, by Tevatron.}

	\only<2->{
	\begin{figure}[H]
		\includegraphics[keepaspectratio=1, width=0.6\textwidth]{../pics/tev28febsmbayeslimits.eps}
	\end{figure}}}
\end{frame}

\section{Conclusion}
\begin{frame}[t]{Conclusion}
\begin{itemize}
	\only<1,2,3,4,5,6,7,8>{
	\only<1->{\item{How will the discovery affect me?}}
	\only<2->{\item{It won't.}}
	\only<3->{\item{Why do we do it then?}}
	\only<4->{\item{To wrap up the electroweak theory.}}
	\only<5->{\item{Why do we want to do that?}}
	\only<6->{\item{To see if we were on the right track for the past 50 years \ldots}}
	\only<7->{\item{\ldots and avoid physics turning into metaphysics \ldots}}
	\only<8->{\item{\ldots *khm* String theory *khm*}}
	}
\end{itemize}
\end{frame}

\begin{frame}
	\begin{center}
	\only<1,2,3>{
		\only<1>{\LARGE Thank you for your attention!}
		\only<2,3>{\LARGE \only<2>{Stop!} \only<3>{It's question time!}}
	}
	\end{center}
\end{frame}

\end{document}
