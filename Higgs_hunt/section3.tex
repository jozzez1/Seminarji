\section{Overview of past attempts}

The attempt to finally complete the SM ``periodic table'' is by no means a new endeavour. Before LHC there have been several attempts
to find the Higgs boson, and the ones who actually managed to obtain some results\footnote{There are other results as well, but they didn't
have the same energy reach and they only served to exclude very low-mass Higgs boson, because they didn't see anything.} were at the LEP
collider (CERN) and Tevatron (Fermilab).

Higgs bosons were sought both directly (did we get the Higgs boson?) and indirectly (if the Higgs boson exists, then certain parameters should
have prescribed values). The main reason why they didn't manage to get the Higgs boson was mostly due to the low energy reach of
the colliders and consequently very low cross sections for production. Some detectors reported slight excess, while others managed to experimentally
bound $M_H$. Still, there were enough indications, which later served as a motivation for construction of the LHC~\cite{elena:green}.

\subsection{LEP}

Although the hunt for the Higgs particle has been popularized recently with the completion of the LHC, CERN made an attempt to find it
before with their previous collider, LEP\footnote{Operated from 1989 to 2000.}.

LEP was an $e^-e^+$ collider, which meant cleaner signals and no QCD background, which modern hadron colliders now have to face. The
dominant production process was associated production via $e^+e^-\to HZ$~\cite{higgs:review}.
It was also possible to produce Higgs through other channels, but the cross sections were much lower. Those modes were secondary via
$WW$ and $ZZ$ fusion~\cite{higgs:review}. The $Z$ in the $HZ$ production can be either virtual (LEP1 phase) or on mass shell
(LEP2 phase)~\cite{higgs:review}. Combined results were from center of mass energy $\sqrt{s} =$ 65 GeV to 209 GeV.

There were four decay modes that were particularly interesting for their distinctions~\cite{higgs:review}:

\begin{itemize}
	\item[(a)] Final products are all quarks:
		\[ H \to b\bar{b}, \quad Z \to q\bar{q}, \] \vspace{-32pt}
	\item[(b)] Final products are quarks and tau leptons:
		\[ \left\{\begin{matrix}
		H \to \tau^+\tau^-, \quad Z \to q\bar{q}, \\
		H \to b\bar{b}, \quad Z \to \tau^+\tau^-,
		\end{matrix}\right.\] \vspace{-22pt}
	\item[(c)] Final products are quark and neutrino pairs:
		\[ H \to b\bar{b}, \quad Z \to \nu\bar{\nu}, \] \vspace{-32pt}
	\item[(d)] Final products are light leptons and quarks:
		\[ H \to b\bar{b}, \quad Z \to \ell^+\ell^-, \] \vspace{-32pt} \label{higgs:decay}
\end{itemize}

where $q\bar{q}$ denotes a quark--anti-quark pair, $\nu$ is a neutrino regardless of generation, and $\ell^\pm \in \{e^\pm,\mu^\pm\}$.
LEP1 only used modes (c) and (d), while the LEP2 phase included all four of them.

LEP at the time had four collaborations working on trying to find Higgs: ALEPH, DELPHI, OPAL and L3. ALEPH found an excess of
$\sim 3 \sigma$, suggesting Higgs boson with mass $M_H \sim 115$ GeV~\cite{higgs:review}. Other experiments, however, couldn't confirm it, but then
again, they couldn't reject it at the 95\% confidence level (CL). Using this data they excluded existence of
SM Higgs boson below 114.4 GeV at 95\% CL, comparing data with theoretically calculated BRs (fig.~\ref{fig:SMHbrsmall}).

\vspace{12pt}

SM Higgs boson was also sought indirectly from fits to electroweak observables~\cite{higgs:review} (masses of $M_Z$ and $M_W$ depend on $M_H$
through loop-order corrections~\cite{higgs:review}). Thus they obtained the upper limit
for the Higgs boson mass and at 95\% CL constrained it to be in the $114.4$ GeV $< M_H < 186$ GeV range\footnote{Accumulated data of this
indirect measurements over the last 20 years at LEP, SLC, Tevatron and elsewhere gave $M_H = 87^{+32}_{-26}$ GeV and $M_H < 157$ GeV at 95\% CL.},
which coincides with theoretical predictions for metastable electroweak vacuum (tab.~\ref{tab:vacuum:scenarios}).

\subsection{Tevatron}

Just like there is CERN in Europe, there is Fermilab in the USA. The LHC counterpart is called Tevatron, a $p\bar{p}$ collider with
much smaller circumference and for a change above the ground. Unfortunately the USA government stopped it's funding in the September of 2011, which
forced Fermilab to close down the collider. The data, however, is currently still pending analysis and so far we've been only fed the preliminary
report.

Two collaborations were working on the experiment: CDF and D\O. Higgs boson production modes for Tevatron collisions are depicted
in the figure below (fig.~\ref{tev:prod}).

\begin{myfig}[11cm]{pics/tev.pdf}
	\vspace{-12pt}
	\caption{Tevatron SM Higgs boson production cross sections. Computed by the TeV4LHC working group~\cite{tev4lhc}, for $p\bar{p}$
		collisions at $\sqrt{s} = 1.96$ TeV}
	\label{tev:prod}
\end{myfig}

Fig.~\ref{tev:prod} shows that the most promising channels are gluon fusion and production with vector bosons $Z,W^\pm$. QCD effects
give greater background and uncertainties are greater than at LEP. They tried to get rid of background using auxiliary measurements
and then Monte Carlo simulations for each specific channel in order to have as high background estimation as possible. For clarification
and easier visualization see the Higgs boson branching ratios (fig.~\ref{fig:SMHbrsmall}).

The ``bread and butter'' production channel was the aforementioned weak gauge boson fusion, through which they obtained most results.
They used almost the same decay channels as the LHC experiments, so no use mentioning them again.

The collisions  were finally made at the beam energy of $\sqrt{s} = 1.96$ TeV. With 95\% CL they excluded two mass regions: 100 < $M_H$ < 106 GeV
and 147 < $M_H$ < 179 GeV. They also expect to exclude regions 100 < $M_H$ < 119 GeV and 141 < $M_H$ < 184 GeV. The greatest reported
discrepancy between SM and experiment was a 2.2 $\sigma$ deviation at the mass $M_H = 120$ GeV~\cite{tev:review}.
