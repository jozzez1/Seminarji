\section{Standard Model Higgs boson}

Contrary to common belief, Higgs bosons \emph{are already} part of the Standard Model (SM).
In the sixties, new questions arose in the world of particle physics: the theories weren't renormalizable\footnote{Renormalizable theories have
singularities which can be removed using redefinition (renormalization) of non-physical parameters. Lagrangian is in that case a part of the
renormalization group -- giving us the freedom to gauge it accordingly.}, and the question
of the hierarchy. While such a mechanism cannot give the direct answer to the latter\footnote{Though it is responsible for
the electroweak symmetry breaking.}, it does provide the masses for the gauge bosons and fermions.

The hypothesis is that our observable space is immersed in a field, which spontaneously breaks the symmetry of the Lagrangian density
$\mathcal{L}$. The original Lagrangian retains the symmetry (all gauge bosons and particles are massless), but the field is the one
makes particles behave, as though they had mass.

\subsection{The Higgs field}

Let us make a quick overview of some very basic theory behind the Higgs boson just so we will get familiar with some of the
physical observables.

We could start all with Quantum Field Theory (QFT). However, since the majority of students is unfamiliar with it, we'll
commence with a different and a more qualitative approach.

The Higgs mechanism is intimately connected with the electroweak theory. With the discovery of the weak interaction, physicists were
puzzled as to how to parametrize it. Fermi suggested to do the same thing as before for the electromagnetic, just change the
constant. Later they fixed it with additional `$\gamma_5$' terms to include left-handedness. However there was still something
wrong with the theory. It wasn't renormalizable. While such theories can occur at anytime, and are allowed in physics, they are
more complex and so they tried to ``fix'' electroweak theory.

Of course, when Fermi first wrote down his low-energy parametrization it was just a ``contact'' interaction. In order to fix it,
theorists tried substituting the coupling with a propagator, containing mass of the weak bosons. Even so, with introducing
masses of the weak bosons ``by hand'' into the Lagrangian, they didn't manage to make the theory renormalizable.

Then in the early sixties, a group of physicists independently realized that the theory can be made renormalizable by using the
electroweak symmetry, which would then be spontaneously broken through another, complex scalar field. This mechanism was later dubbed
the Higgs mechanism, after one of the theorists who postulated it.

Just like we can remember from the classical Hamiltonian and Lagrangian mechanics, we can substitute the coordinates with
fields~\cite{peskin}: one way to introduce the field, $\phi$, is with a new potential in the Lagrangian\footnote{Alternatively
$V(\phi) = \mu^2|\phi|^2 + \lambda|\phi|^4$, like in~\cite[pg. 18]{elena:green}, which only changes this: $\vac\sqrt{2}
\to \vac/\sqrt{2}$ and vice-versa. I used this fixing in equations \eqref{mass:weak}, \eqref{mass:fermions} and
\eqref{mass:higgs}, which in turn look slightly different from what the book suggests. This parametrization also changes the
definition of the Higgs boson mass, $M_H$, like in~\cite[pg. 267]{perkins}. In the one I used, the constants are set in such a way,
that the $M_H = \sqrt{-\mu^2}$}~\cite{perkins,peskin,elena:green}

\begin{equation}
	V(\phi) = \mu^2|\phi|^2 + \frac{\lambda}{4}|\phi|^4, \label{potential}
\end{equation}

like it's shown in the fig.~\ref{higgsf}.

\begin{figure}[H]
	\centering
	\input{pics/higgsf/higgsf.tex}
	\caption{The Higgs potential for the complex field $\phi$. Parameter $\lambda$ is always greater than zero.}
	\label{higgsf}
\end{figure}

Lagrangian retains the symmetries of the potential. But to solve the problems via QFT, we need to make a series expansion around
the minimum. However, for negative $\mu^2$ we have two of them. And as soon as we start to make expansion around either of
those, we break the symmetry\footnote{We can imagine, that such solutions aren't symmetric around the $\phi = 0$ axis, though
the original Lagrangian suggests that they are.}. The new approximation from the expansion

\[
	\varphi \equiv \phi - \phi_\pm,
\]

is then called the Higgs field~\cite{wikipedia}, which, upon 2nd quantization, yields the Higgs bosons -- $H$. We usually
name the $\phi_+ \equiv \vac$.

\subsection{Properties of the Higgs field}

The Higgs field boson is a scalar -- it's spin is theoretically zero, and it obeys the Klein-Gordon differential equation. It
doesn't carry any charge, colour or weak hypercharge.

We are dealing with mass here and so our field doesn't have the usual translational symmetry (the ground state isn't arbitrary)

\[
	\varphi\ \stackrel{\vac}{\slashed{\longrightarrow}}\ \varphi + C.
\]

We can see this, because vacuum states are the ones, that give weak gauge bosons mass\footnote{And because $\mathcal{L}$
isn't invariant on such transformations -- we cannot move both minimums and retain the symmetry over the $\phi = 0$ axis.}.
Positive minimum of the potential  \eqref{potential} is referred to as the `vacuum expectation value'~\cite{elena:green},
$\vac$, which is obviously a nonzero quantity

\begin{equation}
	\vac \equiv \phi_+ = \sqrt{-\frac{2\mu^2}{\lambda}}, \label{vacuum:field}
\end{equation}

from which we can obtain the vacuum energy of the potential (which is quite irrelevant, but anyway),

\begin{equation}
	V(\vac) = V_0 = -\mu^4/\lambda. \label{vacuum:energy}
\end{equation}

The value of $\vac$ is all that we need in order to give masses to the gauge bosons and fermions. For instance,
masses of the $W^\pm$ and $Z$ bosons can be written as \hbox{(see~\cite[pg. 19]{elena:green})}

\begin{equation}
	M_W = g_W \vac\sqrt{2}, \qquad M_Z = \frac{M_W}{\cos\theta_W} = \frac{g_W \vac\sqrt{2}}
		{\cos\theta_W}. \label{mass:weak}
\end{equation}

Higgs bosons couple to themselves\footnote{They give masses to each other -- self-coupling.}, to other bosons and to fermions. We
have different possible couplings for each case. Fermions get their masses from the so called `Yukawa interactions', and Higgs bosons
couple to them through Yukawa couplings, `$g_f$'. For each fermion there might be different couplings, as the coupling
strength is proportional to the fermion mass. Mass for a fermion `$f$' can then be written as \hbox{(see~\cite[pg. 19]{elena:green})}

\begin{equation}
	m_f = 2g_f\vac = M_W\frac{g_f}{g_W}\sqrt{2}. \label{mass:fermions}
\end{equation}

The vacuum expectation value has been theoretically set to $\vac = v/\sqrt{2}$, with $v \approx 246$
GeV~\cite[pg. 1]{higgs:review},~\cite[pg. 21]{elena:green} for the theory to be renormalizable and experimentally confirmed to be
$\vac \sim 174$ GeV~\cite[pg. 19]{elena:green} ($v$ is also a vacuum expectation state~\cite[pg. 1]{higgs:review}, but
from the alternative definition of the Higgs potential $V(\phi)$~\cite[pg. 267]{perkins}; people at CERN mostly use $v$ instead of
$\vac$.).

Parameter $\sqrt{-\mu^2}$ from such $V(\phi)$, defined in eq.~\eqref{potential} can be identified as the Higgs mass, which we can
parametrize with this expression \hbox{(see~\cite[pg. 21]{elena:green})}

\begin{equation}
	M_H \equiv \sqrt{-\mu^2} = \underbrace{\sqrt{-\frac{2\mu^2}{\lambda}}}_{\vac} \cdot \sqrt{\frac{\lambda}{2}}
		= \vac\sqrt{\lambda/2} = v\sqrt{\lambda} = 246\text{ GeV} \cdot \sqrt{\lambda}, \label{mass:higgs}
\end{equation}

so we see that $\lambda$ is the remaining free parameter in the theory to compute the Higgs boson mass\footnote{There are
still other free parameters, namely the coupling constants.}.

\vspace{12pt}
