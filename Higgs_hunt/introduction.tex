\section{Introduction}
Higgs boson, the holy grail of particle physics\ldots It has come a long way from just a hypothesis, a
mathematical trick to make the theory more feasible. There are now indications, that it is no joke.

We live perhaps in a unique time, where the discovery of ``Higgs'' might become a reality.
Journalists went so far as to dub it `the God particle'. While such nicknames are nothing but exaggerations, the
discovery of the Higgs boson, will mean that for the last 50 years we've been mostly on the right track, it
will be the so called `experimentum crucis', that will round up and complete the Standard model experimentally
as well.

The theory of particle physics has advanced to such an extent, that many are wondering if this still makes any
sense at all and the 
problem is, that some theories have begun to border on metaphysics, rather than physics\footnote{String theory is such an example.}.
New theories and models have been proposed and we don't know which to follow.

Despite what many theorists would like to think, physics is an empirical science, we study nature. And so theory of nature must be
in agreement with nature itself. If it doesn't describe it, it's not physics anymore.

So in truth it doesn't really matter if we confirm the Higgs boson or reject it (although it would be really
appreciated if we did), because either way it will give us hints and clues on how to continue our description
of nature.
