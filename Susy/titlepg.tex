% titlepage
\begin{titlepage}
	\begin{figure}[H]
		\centering
		\includegraphics[width = 7cm, keepaspectratio=1]{pics/logo.pdf}\\[12pt]
		{\sc Oddelek za fiziko}\\[4cm]
	\end{figure}
	\begin{center}
		\large{Seminar -- I$_b$, II bolonjska stopnja}\\[0.5cm]
		\LARGE\textbf{Supersimetrija v nerelativisti\v cni kvantni mehaniki}\\[1.0cm]

		\vspace{0.0cm}

		\begin{minipage}{0.4\textwidth}\small
			\begin{flushleft}
			\textsc{Avtor:}\\[0.2cm]
			Jo\v ze Zobec, dipl. fiz. (UN)
			\end{flushleft}
		\end{minipage}
		\begin{minipage}{0.4\textwidth}\small
			\begin{flushright}
				\textsc{Mentor:}\\[0.2cm]
				Prof. Dr. Svjetlana Fajfer,\\[0.1cm]
				Prof. Dr. Toma\v z Prosen
			\end{flushright}
		\end{minipage}
	\end{center}

	\vspace{3.5cm}

	\begin{abstract}
		Supersimetrija je \v sir\v se podro\v cje, ki ni omejeno zgolj na fiziko osnovnih
		delcev, pa\v c pa lahko najde rabo tudi drugod v fiziki. V tem seminarju bom pokazal
		ra\v cunske prijeme v obravnavi \v solskih kvantno-mehanskih problemov s pomo\v cjo
		supersimetri\v cnega ogrodja in novosti v teoriji izospektralnih Hamiltonianov ter
		teoriji perturbacij.
	\end{abstract}
	
	\vfill

	\centering{\footnotesize Ljubljana, \today}
\end{titlepage}

