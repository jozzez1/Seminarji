% titlepage
\begin{titlepage}
	\begin{figure}[H]
		\centering
		\includegraphics[width = 7cm, keepaspectratio=1]{pics/logo.pdf}\\[12pt]
		{\sc Oddelek za fiziko}\\[4cm]
	\end{figure}
	\begin{center}
		\large{Seminar -- 1. letnik fizike, druge bolonjske stopnje}\\[0.5cm]
		\LARGE\textbf{Supersimetrija v klasi\v cni kvantni mehaniki}\\[1.0cm]

		\vspace{0.0cm}

		\begin{minipage}{0.4\textwidth}\small
			\begin{flushleft}
			\textsc{Avtor:}\\[0.2cm]
			Jo\v ze Zobec, dipl. fiz. (UN)
			\end{flushleft}
		\end{minipage}
		\begin{minipage}{0.4\textwidth}\small
			\begin{flushright}
				\textsc{Mentor:}\\[0.2cm]
				Prof. Dr. Svjetlana Fajfer,\\[0.1cm]
				Prof. Dr. Toma\v z Prosen
			\end{flushright}
		\end{minipage}
	\end{center}

	\vspace{5.0cm}

	\begin{abstract}
		Supersimetrija ni ve\v c tako vro\v ce podro\v cje, kot je bilo pred leti. To je predvsem
		zaradi meritev v fiziki delcev, ki zaenkrat nakazujejo, da je ni v na\v sem energijskem
		dosegu. Vendar pa je supersimetrija \v sir\v se podro\v cje, ki ni omejeno zgolj na fiziko
		delcev. V tem seminarju bom pokazal napredne ra\v cunske prijeme v obravnavi klasi\v cnih
		kvantno-mehanskih problemih.
	\end{abstract}
	
	\vfill

	\centering{\footnotesize Ljubljana, \today}
\end{titlepage}

