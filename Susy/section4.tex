\section{Izospektralni Hamiltoniani}

Izospektralni Hamiltoniani v klasi\v cni kvantni mehaniki so taki, ki imajo strogo enake energijske
spektre vezanih stanj in enake transmisijske/refleksijske koeficiente sipalnih stanj. Edino, kar se
med njima razlikuje, so valovne funkcije in posledi\v cno nekateri momenti ($\langle x \rangle$, $\langle
x^2 \rangle$ \ldots).

Izospektralne dru\v zine hamiltonianov so intimno povezane z multisolitonskimi re\v sitvami nelinarnih
evolucijskih ena\v cb, ki nam vra\v cajo solitonske re\v sitve.

\subsection{Enoparametri\v cne dru\v zine izospektralnih potencialov}

Ideja je ta: superpotencial $W(x)$ ni enoli\v cen, zato lahko poi\v scemo dru\v zino potencialov
$\{\tilde{V}_1(x;\lambda_1\}$, ki imajo vsi istega supersimetri\v cnega partnerja. Da se bomo ognili
nanavadnim koeficientom bomo delali v enotah $\hbar = 2m = 1$, zaradi \v cesar se bomo iznebili raznoraznih
koeficientov $(\sqrt{2})^{\pm 1}$ (pozor, cele potence $2$ ostanejo). Hamiltoniani, oblike

\begin{equation}
	H = \frac{\d^2}{\d x^2} + \tilde{V}_1 (x;\lambda_1),
\end{equation}

\ni so vsi izospektralni glede na parameter $\lambda_1$.

Recimo, da $W(x)$ ni enoli\v cen. Potem poleg $W(x)$ obstaja \v se $\tilde{W}(x)$, ki prav tako ustreza
en.~\eqref{v2pot}. Najpreprostej\v sa ideja bi bila potem

\begin{equation}
	W(x) \to \tilde{W}(x) = W(x) + \phi(x),
\end{equation}

\ni kjer zahtevamo, da $\tilde{W}(x)$ prav tako uboga en.~\eqref{riccati} za $V_2(x)$, ki se v teh enotah
glasi

\begin{equation}
	V_2 (x) = W^2(x) + \px W(x) = \tilde{W}^2(x) + \px\tilde{W}(x),
\end{equation}

\ni od koder sledi

\begin{align}
	W^2 + \odv W &= W^2 + \odv W + 2W\phi + \odv \phi + \phi^2, \notag \\
	2W(x)\phi(x) + \phi^2(x) &= -\odv \phi (x), \notag \\
	\frac{2W(x)}{\phi(x)} + 1 &= -\frac{1}{\phi^2(x)}\odv\phi(x),\qquad y(x) = 1/\phi(x), \notag \\
	2W(x)y(x) + 1 &= -\odv y(x).
\end{align}

\ni Ko to ena\v cbo re\v simo, dobimo

\begin{equation}
	\phi(x) = \odv \ln \bigg[\int_{-\infty}^x \psi_0^2(u)\d u + \lambda_1\bigg] =
		\odv \ln \big[\mathcal{I}_1(x) + \lambda_1\big],
\end{equation}

\ni kjer je $\psi_0(x)$ spet normirana funkcija izvornega potenciala $V_1 (x)$.

Dru\v zina potencialov $\tilde{V}_1 (x; \lambda_1)$, ki ima partnerski potencial $V_2(x)$ je torej

\begin{equation}
	\tilde{V}_1 (x;\lambda_1) = V_1(x) - 2\frac{\d^2}{\d x^2}\ln\big[\mathcal{I}_1(x) + \lambda_1\big].
\end{equation}

\ni Parameter $\lambda_1$ se je notri pri\v stulil kot integralska konstanta in ne more biti \v cisto poljuben, ampak
$\lambda_1 \notin [0,1]$, tj $\lambda_1 \in \mathbb{R}\text{\textbackslash}[-1,0]$ -- v tistem re\v zimu je osnovno stanje
$\psi_0 (x; \lambda_1)$ potenciala $\tilde{V}_1(x; \lambda_1)$ nenormalizabilno, ampak je sipalno stanje -- torej
nam kot pri supersimetri\v cnih partnerjih manjka ostnovno stanje, \v ceprav so vsa ostala stanja nespremenjena.
Prvotni potencial $V_1(x)$ dobimo kot limito $\lambda_1 \to \pm \infty$.

\subsection{Ve\v cparametri\v cne dru\v zine izospektralnih potencialov}
Isto lahko naredimo tudi na malo druga\v cen na\v cin. Partnerski potencial $V_2$ je v bistvu $V_1$ brez osnovnega stanja.
Dru\v zina potencialov $\tilde{V}_1$ niso ni\v c drugega, kot $V_1$ z modificiranim osnovnim stanjem.

Malo bom spremenil notacijo: $\psi_1$ je osnovno stanje $V_1$ in ima energijo $E_1$, $\psi_2$ je osnovno stanje $V_2$ in
ima energijo $E_2$, $\psi_n$ je osnovno stanje $V_n$ z energijo $E_n$.

Na enoparametri\v cnem primeru odre\v zemo\footnote{Od potenciala od\v stejemo funkcijo $g(x)$, zaradi \v cesar izgubi
osnovno stanje.} osnovno stanje kot $V_2 = V_1 - 2\px^2 \ln\psi_1$. Nato moramo $V_2$ dodati splo\v sno stanje, ki bi ustrezalo
energiji $E_1$ To stanje je linearna kombinacija $1/\psi_1$ in
$\mathcal{I}_1/\psi_1$, torej je novo osnovno stanje

\begin{equation}
	\Phi_1 (x;\lambda_1) = \frac{\alpha\mathcal{I}_1 + \beta}{\psi_1} = \frac{\mathcal{I}_1 + \lambda_1}{\psi_1}.
\end{equation}

\ni Tu smo upo\v stevali, da $\alpha$ in $\beta$ v resnici nista neodvisna parametra, saj imamo \v se pogoj, da je to novo stanje
normirano -- potem zado\v s\v ca le en parameter -- $\lambda_1$. Ko to stanje vstavimo nazaj v $V_2$ dobimo

\begin{align}
	\tilde{V}_1(x;\lambda_1) &= V_2 - 2\odv[2]\ln\Phi_1(x;\lambda_1) = \notag \\
		&= V_1 - 2\odv[2]\ln\psi_1 - 2\odv[2]\ln\Phi_1 = \notag \\
		&= V_1 - 2\odv[2]\ln (\psi_1 \Phi_1) = \notag \\
		&= V_1 - 2\odv[2]\ln (\mathcal{I}_1 + \lambda_1).
\end{align}

\ni Osnovno stanje takega potenciala je $\tilde{\psi_1}(x;\lambda_1) = 1/\Phi_1 (x;\lambda_1)$ in v limiti $\lambda_1 \to \pm
\infty$ vrne $\psi_1$. Kot prej sledi, da mora biti $\lambda_1 \in \mathbb{R}\text{\textbackslash [-1,0]}$. Ta postopek
lahko posplo\v simo na ve\v cparametri\v cne primere: ne samo, da od\v stejemo $\psi_1$ in ga vrnemo s parametrom $\lambda_1$,
od\v stejemo \v se osnovno stnanje $V_2$, $V_3$, \ldots in na koncu dobimo
$\tilde{V}_1 (x;\lambda_1,\lambda_2\ldots\lambda_n) = \tilde{V}_1 (x; \underline{\lambda})$, kjer imamo toliko $\lambda_i$, da
parametriziramo vsa vezana stanja.

