\section{Uvod}

Supersimetrija je podro\v cje, ki ima razne zanimive posledice in aplikacije ve\v c kot le v fiziki
delcev, kjer je verjetno ta hip najbolj aktualna.

V klasi\v cni\footnote{tj. nerelativisti\v cni} kvantni mehaniki je supersimetrija izjemno mo\v cno orodje v \v studiji
izospektralnih in kvazi-izospektralnih Hamiltonianov in je prav tako mo\v cno povezana z integrabilnostjo v kvantni
mehaniki.

Ideja je predvsem to, da bi naredili konsistentno teorijo, po kateri bi lahko v enem zamahu naenkrat izra\v cunali
cele dru\v zine hamiltonianov in iz obstoje\v cih re\v sljivih naredili nove, ki so prav tako re\v sljivi.

Supersimetri\v cni pristopi so relativno novi in s seboj prina\v sajo razne nadgradnje \v ze znanim prijemom
(npr. novi na\v cini perturbacijske teorije).
