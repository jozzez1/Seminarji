\section{Uvod}

Supersimetrija je podro\v cje, ki je nastalo v fiziki visokih energij, vendar pa ima zanimive aplikacije lahko
tudi drugod. Omejil se bom na enodel\v cne probleme v nerelativisti\v cni kvnatni mehaniki.

To je simetrija, ki velja med fermionskimi in bozonskimi operatorji. Npr. kot to\v ckovna
simetrija $C_2$ slika $-\psi$ v $\psi$, ne da bi se pri tem spremenila energija sistema, tako supersimetrija pomeni, da lahko bozonske
operatorje zamenjamo s fermionskimi, ne da bi pri tem vplivali na fizikalne opazljivke.

V nerelativisti\v cni kvantni mehaniki supersimetri\v cni opis problema predstavlja alternativno in marsikdaj
enostavnej\v so pot do re\v sitve. Omogo\v ca konstrukcijo konsistentne teorije, s katero lahko konstruiramo izospektralne
Hamiltoniane in nove integrabilne sisteme.

