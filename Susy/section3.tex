\section{Podobni Hamiltoniani}

Prava mo\v c supersimetrije se poka\v ze pri obravnavi re\v sljivih problemov. Njihov nabor je sicer neskon\v cen,
vendar pa obstaja kon\v cna podmno\v zica, iz katere lahko dobimo vse ostale. Tudi znotraj te podmno\v zice imamo
v resnici transformacije, ki "`generatorski"' potencial slika v drugega.

Imejmo potencial $V_1 (x; \pr_1)$ in potencial
$V_2 (x; \pr_2)$, kjer sta $\pr_1$ in $\pr_2$ urejena nabora parametrov v potencialih. \v Ce sta
$V_1$ in $V_2$ podobna, potem veljata slede\v ci identiteti:

\begin{align}
	V_2 (x; \pr_1) &= V_1 (x; \pr_2) + R (\pr_1), \\
	\pr_2 &= \underline{f} (\pr_1). \label{osnova}
\end{align}

\ni Prek en.~\eqref{osnova} dobimo dru\v zino potencialov, ki so podobni $V_1$. Nove parametre dobimo iz
kompozituma funkcije $f$

\begin{align}
	\pr_2 &= \underline{f} (\pr_1), \notag \\
	\pr_3 &= \underline{f} (\pr_2) = (\underline{f} \circ \underline{f}) (\pr_1), \notag \\
	&\ldots \notag \\
	\pr_n &= (\underbrace{\underline{f} \circ \underline{f} \circ \ldots \circ \underline{f}}_{n-1}) (\pr_1),
\end{align}

\ni potenciali pa so potem

\begin{align}
	V_2 (x; \pr_1) &= V_1 (x; \pr_2) + R (\pr_1), \notag \\
	V_3 (x; \pr_1) &= V_2 (x; \pr_2) + R (\pr_2) = V_1 (x; \pr_3) + R (\pr_1) + R (\pr_2), \notag \\
	&\vdots \notag \\
	V_n (x; \pr_1) &= V_1 (x; \pr_n) + \sum_{k = 1}^{n-1} R (\pr_k),
\end{align}

\ni ki je spet o\v citno podoben $V_1 (x)$. Sedaj dobimo dru\v zino podobnih Hamiltonianov,

\begin{equation}
	H_n = - \frac{1}{2}\px^2 + V_n (x;\pr_1) = - \frac{1}{2}\px^2+V_1(x;\pr_n)+\sum_{k = 1}^{n-1}R(\pr_k).
\end{equation}

\ni Dobljeni Hamiltoniani imajo enak spekter, z izjemo vakuumov in prvih nekaj stanj. Energije 
osnovnih stanj so o\v citno

\begin{equation}
	E^{(n)}_0 = \sum_{k = 1}^{n - 1} R(\pr_k).
\end{equation}

Ker so spektri od neke energije dalje enaki, sledi da energije vakuumov teh Hamiltonianov sovpadajo z energijami
vzbujenih stanj Hamiltonianov z indeksom $m < n$. Seveda gremo lahko do konca nazaj, pridemo do
$m = 1$ in $E^{(1)}_0 = 0$ in tako dobimo celoten vezani spekter Hamiltoniana $H_1$:

\begin{equation}
	E_n^{(1)} (\pr_1) = \sum_{k = 1}^n R (\pr_k).
	\label{v2-pogoj}
\end{equation}

\ni Seveda to pomeni, da $V_2$ (in posledi\v cno $\pr_2$) ne sme biti arbitraren, ampak tak, da zadosti pogoju
en.~\eqref{v2-pogoj}, sicer pademo lahko v poljubno vzbujeno stanje. Tak na\v cin iskanja spektra je dosti
enostavnej\v si, vendar moramo za to poznati funkcijo $\underline{f}$. 

\subsection{Operatorji vi\v sanja}

Razumeti je treba, da so $a$ in $a^\dagger$ posplo\v sitvi pravi opeatorjev vi\v sanja ali ni\v zanja iz harmonskega
oscialtorja. Pokazali smo, kako z njimi me\v samo bozone s fermioni, sedaj bom pokazal, kako jih
je treba uporabiti, da z njimi dejansko lahko dvignemo oz. spustimo stanje.

Spet imejmo Hamiltonian $H_1$ z energijo osnovnega stanja $E^{(1)}_0 = 0$ in valovno funkcijo
$\psi_0^{(1)} (x; \pr_1)$. Operatorje $a^\dagger$ bi morali dejansko ves \v cas pisati kot $a^\dagger (x; \pr)$,
saj

\[
	a^\dagger \equiv a^\dagger (x; \pr) \equiv W(x; \pr) - \frac{1}{\sqrt{2}}\px,
\]

Hamiltonianu $H_1$ pripada supersimetri\v cni partner $H_2$, ki ima enak spekter, hkrati pa nima lastnega
stanja pri vakuumski energiji $H_1$, zato lahko $H_2$ obravnavamo kot $H_1$ podoben Hamiltonian.

Vi\v sja vzbujena stanja moramo o\v citno dobiti kot

\begin{equation}
	\psi^{(1)}_n (x; \pr_1) \propto a^\dagger (x;\pr_1)\ a^\dagger (x;\pr_2)\ \ldots\ a^\dagger (x; \pr_n)\
		\psi^{(1)}_0 (x; \pr_{n+1}),
\end{equation}

\ni Odtod sledi pomembna posledica: supersimetri\v cna partnerja $V_1$ in $V_2$ sta si podobna potenciala, kar
pomeni, da je med njima lahko netrivialna podobnostna transformacija -- konkretno si lahko spet za zgled vzamemo
neskon\v cno potencialno jamo in en.~\eqref{pot-nes-jama}.

Ker je neprikladno ra\v cunati poljubno visok $\pr_k$ na zalogo, se po navadi raje uporabi kar

\begin{equation}
	\psi^{(1)}_n (x; \pr_1) = a^\dagger (x; \pr_1)\ \psi_{n-1}^{(1)} (x; \pr_2),
\end{equation}

\ni ki nam namiguje, da je za dviganje in spu\v s\v canje spet dejansko dovolj le poznavanje supersimetri\v cnih
partnerjev.

\subsection{Klasifikacija podobnih potencialov}

Posebej prikladno je, \v ce znamo podobne potenciale, ki pripadajo podobnostnim transformacijam iste ba\v ze,
razvrstiti v mno\v zice. Ta problem je \v se nere\v sen, saj pravzaprav splo\v sen eksaktno re\v sljivi potencial
\v se ni definiran. Kljub temu so fiziki na\v sli dva tipa podobnih potencialov, v katere sodi ve\v cina
potencialov iz raznih u\v cbenikov.

\begin{itemize}
	\item{$\pr^\prime = \pr + \underline{q}$ -- potenciali, podobni na translacije parametrov,}
	\item{$\pr^\prime = q\pr$ -- potenciali, podobni na skaliranje parametrov.}
\end{itemize}

Poleg teh imamo \v se primere, ko je $\underline{f}$ nelinearna transformacija, takrat se lahko
zgodi npr. $p^\prime_k = q \cdot p_k^2$, teh niso na\v sli veliko.

Potenciali, ki imajo lastnost podobnosti ne rabijo biti centralni.

\ni Tabela~\ref{tab1} prikazuje potenciale, ki pripadajo translatorni dru\v zini.

\begin{table}[H]\centering
	\caption{Nekaj preprostih potencialov, ki pripadajo podobnim na translacije. Rosen-Morse I je res podoben
		neskon\v cni potencialni jami -- Rosen-Morse I, pri $\alpha = \pi$, $A = \pi$ in $B = 0$. Coulombov
		potencial se da transformirati v 3D harmonski oscilator. Rosen-Morse I ima pogoj $0 \leq \alpha x \leq \pi$,
		Eckart pa $B > A^2$.}
	\label{tab1}
	\vspace{0.2cm}
	\begin{tabular}{l || c | c}
		Naziv & $W(x)$ & $V(x)$\\ \hline
		Rosen-Morse I & $-A \cot \alpha x - B/A$ & $\frac{A (A - \alpha)}{\sin^2 \alpha x} + 2B\cot\alpha x - A^2
			+ (B/A)^2$ \\
		Coulomb & $\frac{e^2}{2(\ell+1)} - \frac{(\ell+1)}{r}$ & $-\frac{e^2}{r} + \frac{\ell(\ell+1)}{r^2} -
			\frac{e^4}{4(\ell + 1)^2}$ \\
		1D harmonski oscilator & $\frac{1}{2}\w x - b$ & $\frac{1}{4}\w^2\big(x - \frac{2b}{\w}\big)^2 - \frac{\w}{2}$ \\
		3D harmonski oscilator & $\frac{1}{2}\w r - \frac{\ell + 1}{r}$ & $\frac{1}{4}\w^2r^2 +
			\frac{\ell(\ell + 1)}{r^2} - (\ell + 3/2)\w$ \\
		Eckart & $-A\coth\alpha r + B/A$ & $A^2 + (B/A)^2 - 2B\coth\alpha r + \frac{A(A + \alpha)}{\sinh^2 \alpha r}$ \\
	\end{tabular}
\end{table}

