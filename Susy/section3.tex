\section{Podobni Hamiltoniani}

Prava mo\v c supersimetrije se poka\v ze pri obravnavi re\v sljivih problemov. Izka\v ze se, da
imamo kon\v cen nabor potencialov, za katere poznamo to\v cne re\v sitve, med njimi pa lahko definiramo
preproste transformacije, ki en potencial preoblikuje v drugega.

Najprej poka\v zimo kaj so podobni potenciali. Imejmo potencial $V_1 (x; \pr_1)$ in potencial
$V_2 (x; \pr_2)$, kjer sta $\pr_1$ in $\pr_2$ urejena nabora parametrov v potencialih. \v Ce sta
$V_1$ in $V_2$ podobna, potem veljajo slede\v ci identiteti:

\begin{align}
	V_2 (x; \pr_1) &= V_1 (x; \pr_2) + R (\pr_1), \\
	\pr_2 &= \underline{f} (\pr_1). \label{osnova}
\end{align}

Prek en.~\eqref{osnova} lahko definiramo celo dru\v zino potencialov, ki so podobni $V_1$:

parametre dobimo prek kompozituma funkcije $f$,
\begin{align}
	\pr_2 &= \underline{f} (\pr_1), \notag \\
	\pr_3 &= \underline{f} (\pr_2) = (\underline{f} \circ \underline{f}) (\pr_1), \notag \\
	&\ldots \notag \\
	\pr_n &= (\underbrace{\underline{f} \circ \underline{f} \circ \ldots \circ \underline{f}}_{n-1}) (\pr_1),
\end{align}

potenciali pa so potem

\begin{align}
	V_2 (x; \pr_1) &= V_1 (x; \pr_2) + R (\pr_1), \notag \\
	V_3 (x; \pr_1) &= V_2 (x; \pr_2) + R (\pr_2) = V_1 (x; \pr_3) + R (\pr_1) + R (\pr_2), \notag \\
	&\ldots \notag \\
	V_n (x; \pr_1) &= V_1 (x; \pr_n) + \sum_{k = 1}^{n-1} R (\pr_k),
\end{align}

ki je spet o\v citno podoben $V_1 (x)$. Sedaj dobimo dru\v zino podobnih Hamiltonianov,

\begin{equation}
	H_n = - \frac{1}{2}\px^2 + V_n (x;\pr_1) = - \frac{1}{2}\px^2+V_1(x;\pr_n)+\sum_{k = 1}^{n-1}R(\pr_k).
\end{equation}

Ti Hamiltoniani imajo degeneriran spekter, z izjemo vakuumov, ki so o\v citno

\begin{equation}
	E^{(n)}_0 = \sum_{k = 1}^{n - 1} R(\pr_k).
\end{equation}

Ker so spektri degenerirani, sledi da so vakuumske energije teh hamiltonianov sovpadajo z energijami vzbujenih
stanj Hamiltonianov z indeksom $m < n$. Seveda gremo lahko do konca nazaj, pridemo do $m = 1$ in $E^{(1)}_0 = 0$
in tako dobimo celoten vezani spekter Hamiltoniana $H_1$:

\begin{equation}
	E_n^{(1)} (\pr_1) = \sum_{k = 1}^n R (\pr_k).
	\label{v2-pogoj}
\end{equation}

Seveda to pomeni, da $V_2$ (in posledi\v cno $\pr_2$) ne sme biti arbitraren, ampak tak, da zadosti pogoju
en.~\eqref{v2-pogoj}, sicer pademo lahko v poljubno vzbujeno stanje. Tak na\v cin iskanja spektra je dosti
enostavnej\v si, vendar moramo za to poznati funkcijo $\underline{f}$. 

\subsection{Operatorji dviganja}

V prej\v snji sekciji sem na koncu zgleda povedal, da z naivnim pristopom z operatorjem dviganja ali
spu\v s\v canje lahko le pretvarjamo med supersimetri\v cnima potencialoma. Sedaj bom pokazal, kako jih
je treba uporabiti, da z njimi dejansko lahko dvignemo oz. spustimo stanje.

Spet imejmo Hamiltonian $H_1$, za katerega velja $E^{(1)}_0 = 0$, ki ima valovno funkcijo
$\psi_0^{(1)} (x; \pr_1)$. Operatorje $a^\dagger$ bi morali dejansko ves \v cas pisati kot $a^\dagger (x; \pr)$,
saj

\[
	a^\dagger \equiv a^\dagger (x; \pr) \equiv W(x; \pr) - \frac{1}{\sqrt{2}}\px,
\]

Hamiltonian $H_1$ ima supersimetri\v cnega partnerja $H_2$, ki ima degeneriran spekter. $H_2$ zato lahko
obravnavamo ko podoben Hamiltonian, vemo, da se razlikuje ravno za vakuumsko stanje.

Vi\v sja vzbujena stanja moramo o\v citno dobiti kot

\begin{equation}
	\psi^{(1)}_n (x; \pr_1) \propto a^\dagger (x;\pr_1)\ a^\dagger (x;\pr_2)\ \ldots\ a^\dagger (x; \pr_n)\
		\psi^{(1)}_0 (x; \pr_{n+1}),
\end{equation}

Odtod sledi pomembna posledica: supersimetri\v cna partnerja $V_1$ in $V_2$ sta si podobna potenciala, kar
pomeni, da je med njima lahko netrivialna podobsnostna transformacija -- konkretno lahko spet za zgled vzamemo
neskon\v cno potencialno jamo in en.~\eqref{pot-nes-jama}.

Ker je neprikladno ra\v cunati poljubno visok $\pr_k$ na zalogo, se po navadi raje uporabi kar

\begin{equation}
	\psi^{(1)}_n (x; \pr_1) = a^\dagger (x; \pr_1)\ \psi_{n-1}^{(1)} (x; \pr_2),
\end{equation}

ki nam sugerira, da je za dviganje in spu\v s\v canje dejansko dovolj poznavanje supersimetri\v cnih
partnerjev.

Sipalnih stanj se ne bom dotikal.

\subsection{Klasifikacija podobnih potencialov}

Posebej prikladno je, \v ce znamo podobne potenciale, ki pripadjo podobnostnim transformacijam iste ba\v ze,
pogrupirati, saj lahko prek tega z zamahom roke naenkrat re\v simo vse probleme klasi\v cne kvantne mehanike,
kar se jih da analiti\v cno re\v siti.

Splo\v sen problem je \v se nere\v sen, saj pravzaprav splo\v sen re\v sljivi potencial \v se ni definiran.
Je pa fitzikom kljub temu uspelo zaenkrat pokazati, da obstajata dve pasmi podobnih potencialov:

\begin{itemize}
	\item{$\pr^\prime = \pr + \underline{q}$ -- potenciali, podobni na translacije parametrov,}
	\item{$\pr^\prime = q\pr$ -- potenciali, podobni na skaliranje parametrov.}
\end{itemize}

\subsubsection{Re\v sitve povezane s translacijami}

Iz $\pr_1$ lahko preidemo v $\pr_2$ prek transformacije $\underline{f}$, ki je v tem primeru enostavna
translacija:

\[
	\pr_2 = \underline{f}(\pr_1) = \pr_1 + \underline{q}.
\]

Kot sem napisal prej, je sta si supersimetri\v cna partnerja "`podobna"'. Torej poleg vseh do sedaj zapisanih
identitet, velja tudi pravilo

\begin{equation}
	V_2 (x; \pr_1) - V_1 (x; \pr_1 + \underline{q}) = R (\pr_1).
\end{equation}

Oba sta povezana prek istega superpotenciala v en.~\eqref{v2pot}, torej

\begin{multline}
	W^2 (x; \pr_1) + \frac{1}{\sqrt{2}}\px W(x; \pr_1) + \frac{1}{\sqrt{2}}\px W (x; \pr_1 + \underline{q}) - W^2
		(x; \pr_1 + \underline{q}) =\\= R (\pr_1) = L (\pr_1) - L (\pr_1 + \underline{q})
\end{multline}
