\section{Supersimetrija v teoriji perturbacij}

Supersimetrija omogo\v ca dve novi perturbacijski metodi. Variacijska metoda je bolj intuitivna in tudi bistveno bolj
natan\v cna, ti. $\delta$-razvoj pa spominja na razvoj po zankah iz kvantne teorije polja, saj uporablja podobne prijeme
ki se uporabljajo pri regularizaciji interakcijskih \v clenov.

\subsection{Variacijski pristop}

Potencial $V_1$ ima \v clene, ki jih je treba obravnavati perturbativno. Za metodo potrebujemo testno valovno funckijo, $\psi_v$,
ki aproksimira osnovno stanje. Pri\v cakovana energija tega stanja bo na\v sa aproksimacija prave osnovne energije.
Upo\v stevamo en.~\eqref{riccati}.

\begin{equation}
	V_1(x) - E_0 = W^2(x) - \koren\px W(x).
\end{equation}

s katero izra\v cunamo $W(x)$ do neke natan\v cnosti. Z njim lahko prek en.~\eqref{anihilator} in~\eqref{kreator} izra\v cunamo
operatorje dviganja, $a^\dagger$, in spu\v s\v canja, $a$, s katerima se lahko zavihtimo v vzbujena stanja.

Za\v cetno aproksimacijo $W(x)$ in $E_0$ dobimo prek variacijske metode, s testno funkcijo, ki aproksimira osnovno stanje. Po
navadi se jo aproksimira z Gaussovo krivuljo. Imejmo testno funkcijo $\psi_v (x;\eta)$, kjer parameter $\eta$ dolo\v cimo tako,
da minimizira energijo -- s tem dobimo aproksimacijo osnovnega stanja. Pri\v cakovana energija take testne valovne funkcije je

\begin{align}
	\langle E \rangle_\eta &= \langle \psi_v| H |\psi_v \rangle \notag \\
	&= \intinf\ \psi_v (x; \eta) H \psi_v (x; \eta) \notag \\
	&= \intinf\ \psi_v (x;\eta) \bigg[-\frac{1}{2}\px^2 + V_1(x)\bigg]\psi_v(x;\eta) \notag \\
	&= -\frac{1}{2}\intinf\ \psi_v(x;\eta)\ \px^2\ \psi_v (x;\eta) + \intinf\ \psi_v(x;\eta)V_1(x)\psi_v(x;\eta) \notag \\
	&= \intinf \bigg[\koren \px \psi_v(x;\eta)\bigg]^2 + \intinf\ \psi_v^2 (x; \eta) V_1 (x), \label{eeta}
\end{align}

kjer smo v en.~\eqref{eeta} predpostavili, da je na\v sa testna funkcija integrabilna, analiti\v cna in da dovolj hitro pada, ko
$x \to \pm \infty$ -- ker je testna funkcija arbitrarna, lahko zadostimo vsem tem pogojem. Prav tako smo zgolj zaradi preprostosti
pisave predpostavili, da potencial $V_1$ komutira z falovno funkcijo, vendar to ni pravi pogoj.

Parameter $\eta$ dolo\v cimo tako, da minimizira energijo $\langle E \rangle_\eta$, tj. zahtevamo

\begin{equation}
	\frac{\partial \langle E \rangle_\eta}{\partial\eta}\bigg|_{\eta = \xi} = 0, \qquad
	\frac{\partial^2 \langle E \rangle_\eta}{\partial\eta^2}\bigg|_{\eta = \xi} > 0.
\end{equation}

Energijo zato odvajamo po parametru $\eta$,

\begin{equation}
	\pe\langle E\rangle_\eta = \intinf\bigg[(\px \psi_v) \pe (\px \psi_v) +  2 \psi_v (\pe \psi_v) V_1\bigg].
\end{equation}

Sedaj, ko imamo pribli\v zek za $E_0$ in $\psi_0$ lahko izra\v cunamo $W(x)$ prek identitete~\eqref{superpotencial}, tj.

\begin{equation}
	W (x; \xi) = -\koren\px\ln\psi_v (x;\xi),
\end{equation}

s \v cimer dobimo par $V_1^\prime$ in $V_2^\prime$, ki sta pribli\v zka za $V_1$ in $V_2$. Potenciala $V_1^\prime$ in $V_2^\prime$
bi radi tak\v sna, da poznamo njuno to\v cno re\v sitev.

Ta metoda ima zelo veliko pomankljivost -- zelo dobro moramo uganiti $\psi_v$. Po navadi se uporablja ve\v c parametrov $\eta_i$, kjer
potem zahtevamo

\begin{equation}
	\partial_{\eta_i} \langle E \rangle_{\underline\eta}\Big|_{\underline{\eta} = \underline{\xi}} = 0, \ \forall \eta_i
\end{equation}

in pozitivnost determinante Hessejeve matrike glede na $n$-terec parametrov $\eta_i$ (pogoj za lokalni minimum).

\subsection{$\delta$-razvoj}

Razvoj po potencah $\delta$ nekoliko spominja na razvoj po zankah iz kvantne teorije polja. Metodo bom predstavil na konkretnem
primeru. Naj bo $V_1 (x) = 2gx^4$ anharmonski oscilator. Potenca $4$ je previsoka, re\v siti znamo samo za $x^2$, zato bomo
$V_1$ definirali kot analiti\v cno nadaljevanje harmonskega oscilatorja

\begin{equation}
	V_1 = 2gx^4 = M^{2 + \delta}x^{2 + 2\delta} - C(\delta) = W^2 - \koren\px W.
\end{equation}

Parameter $M$ se je notri pri\v stulil tako kot pri dimenzijski regularizaciji iz kvantne teorije polja. Sicer je res, da delamo
z brezdimenzijskimi koli\v cinami, vendar ta skalirni faktor ostaja. Parameter $\delta$ nam predstavlja "`anharmonskost"' oscilatorja.
Energijo osnovnega stanja, $C(\delta)$ od\v stejemo za faktorizacijo hamiltoniana z operatorji $a$ in $a^\dagger$.

Pri $\delta = 0$ imamo $M = m$, pri kvarti\v cnem anharmonskem oscilatorju, $\delta = 1$, pa je $M = (2g)^{1/3}$. \v Ce sta $V_1$ in
$W$ analiti\v cna, potem za oba obstaja Taylorjev razvoj po potencah $\delta$,

\begin{equation}
	V_1 = M^2x^2\sum_{k = 0}^\infty \delta^k \frac{[\ln(Mx^2)]^k}{k!} - 2 \sum_{k = 0}^\infty \delta^k E_k,
	\label{delta-razvoj}
\end{equation}

$E_k$ je energija osnovnega stanja potenciala $V_1$, ki je znan do reda $\delta^k$ nata\v cno. Analogno aproksimiramo tudi superpotencial
$W(x)$,

\begin{equation}
	W(x) = \sum_{k = 0}^\infty \delta^k W_{(k)}(x).
	\label{w-razvoj}
\end{equation}

Nadaljujemo tako, da na obeh straneh en.~\eqref{riccati} upo\v stevamo en.~\eqref{delta-razvoj} in~\eqref{w-razvoj}, ki mora veljati v
vseh redih razvoja -- dobimo sistem diferencialnih ena\v cb po redih $\delta^k$.

V prvem redu dobimo harmonski oscilator

\begin{equation}
	W_{(0)}^2 - \px W_{(0)} = M^2x^2 - 2E_0,
	\label{diffena}
\end{equation}

ki ima re\v sitve $W_{(0)} = Mx$ in $E_0 = M/2$. V drugem redu dobimo poleg en~\eqref{diffena} \v se ena\v cbo

\begin{equation}
	\px W_{(1)} - 2W_{(1)}W_{(0)} = - M^2x^2 \ln Mx^2 + 2E_1,
\end{equation}

z re\v sitvami

\begin{equation}
	W_{(1)} = -\text{e}^{Mx^2} \int_0^x \d y\ \text{e}^{-My^2}\big[M^2y^2 \ln My^2 - 2E_1\big].
\end{equation}

Ta metoda je dobra, \v ce nimamo nobene predstave o problemu in si \v zelimo relativno grobo oceno osnovnega stanja. Za vzbujena stanja
je variacijska metoda natan\v cnej\v sa, vendar za visoka vzbujena stanja tudi ta podle\v ze vsled \v cesar moramo pose\v ci po
standarnih metodah.

