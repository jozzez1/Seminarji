\section{Neskon\v cna potencialna jama}

Stvari je la\v zje razumeti na konkretnem zgledu, zato si bomo pogledali supersimetrijo na primeru neskon\v cne
potencialne jame. Gre za\v cetni\v ski potencial, (vstavi sliko), omejen od $0$ do $1$. Lastne fukcije so potem kar

\begin{equation}
	\psi_n^{(1)} (x) = \frac{1}{2}\sin n\pi x, \quad E_n^{(1)} = \frac{(n\pi)^2}{2}
		\quad n = 1,2,\ldots
\end{equation}

\ni stanja za $n = 0$ ni, ker to stanje ustreza situaciji brez delca\footnote{Verjetnost, da se delec nahaja v jami
je natanko 0.}. Osnovni lastni par je torej za $n = 1$, vendar ga bolj kljub temu ozna\v cil z indeksom 0.

\begin{equation}
	\psi_0^{(1)} (x) = \frac{1}{2}\sin\pi x, \quad \frac{2}{\pi^2}E_0^{(1)} = 1 \neq 0.
\end{equation}

\ni Supersimetrija je v tem primeru zlomljena, saj $E_0 \neq 0$, zato moramo za\v cetnemu Hamiltonianu
od\v steti energijo osnovnega stanja.

\begin{equation}
	(\underbrace{H - E_0^{(1)}}_{H_1})\psi_0^{(1)} = (E_0^{(1)} - E_0^{(1)})\psi_0^{(1)} =
		0\cdot\psi_0^{(1)} = 0. \label{translacija}
\end{equation}

\ni Od tod lahko poi\v s\v cemo superpotencial $W(x)$ iz en.~\eqref{superpotencial}

\begin{equation}
	W(x) = -\frac{1}{\sqrt{2}} \frac{\px \sin \pi x}{\sin \pi x} = -\frac{\pi}{\sqrt{2}}\cot \pi x.
\end{equation}

\ni S pomo\v cjo en.~\eqref{v2pot} lahko poi\v s\v cemo supersimetri\v cnega partnerja neskon\v cne potencialne jame

\begin{equation}
	V_2 (x) = \frac{\pi^2}{2}\bigg(\cot^2 \pi x + \frac{1}{\sin^2 \pi x}\bigg)
		= \frac{\pi^2}{2}\bigg(\frac{1}{\sin^2 \pi x} - 1\bigg),
	\label{pot-nes-jama}
\end{equation}

\ni kar o\v citno ni ve\v c neskon\v cna potencialna jama, za katero dobimo $V_1(x) = \pi^2/2$, ki pa se ravno
prikladno od\v steje s konstantno $E_0^{(1)}$ v ena\v cbi~\eqref{translacija}. Potencial $V_2$ je poseben primer
potenciala Rosen Morse I (tj. triginometri\v cna izvedenka).

Sedaj smo dobili oba Hamiltoniana,

\begin{align}
	H_1 &= -\frac{1}{2}\px^2, \notag \\
	H_2 &= -\frac{1}{2}\px^2 - V_2(x),
\end{align}

\ni od koder lahko \v ze sklepamo, kako bo izgledal supersimetri\v cni Hamiltonian $\H$.

Poglejmo \v se kako izgledajo valovne funkcije $\psi_n^{(2)}(x)$. Za to bomo seveda uporabili
operatorje vi\v sanja in ni\v zanja iz en.~\eqref{degenener}.
Vemo, da $H_2$ nima stanja pri $n = 1$, ampak da se \v stetje za\v cne pri $n = 2$, vendar bomo tudi
sedaj pisali indeks 0. Uporabili bomo identiteto

\begin{align}
	\psi_0^{(2)} (x) = a\psi_2^{(1)} &= -\frac{1}{2\sqrt{2}}(\pi \cot \pi x -
		\px)\sin2\pi x \\
	&= \frac{2\pi}{2\sqrt{2}}(\cos^2\pi x - \cos 2\pi x) = \frac{\pi}{\sqrt{2}}\sin^2\pi x.
\end{align}

Splo\v sen izraz za $\psi_n^{(2)}$ je dolg. Zapisal bom raje le \v se za
$\psi_3^{(2)}$, za $n=3$, tj. prvo vzbujeno stanje $H_2$:

\begin{equation}
	\psi_3^{(2)}(x) \propto \sin (\pi x) \sin (2\pi x),
\end{equation}

\ni Operatorja $a$ in $a^\dagger$ seveda po pretvorbi pokvarita normalizacijo, tako da rezultati niso nujno
normirani.
