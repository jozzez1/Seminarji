\section{Zgledi}

Take stvari je vedno la\v zje razumeti na konkretnem zgledu, zato poka\v zimo, bolj kot zanimivost,
kako lahko to teorijo uporabimo na starih znancih iz takih ali druga\v cnih kurzov kvantne mehanike.

\subsection{Neskon\v cna potencialna jama}

Za\v cetni\v ski potencial, (vstavi sliko), omejen od $0$ do $1$. Lastne fukcije so kar

\begin{equation}
	\psi_n^{(1)} (x) = \frac{1}{2}\sin n\pi x, \quad E_n^{(1)} = \frac{(n\pi)^2}{2}
		\quad n = 1,2,\ldots
\end{equation}

stanja za $n = 0$ ni, ker to stanje ustreza situaciji brez delca (verjetnost, da se delec nahaja v jami
je natanko 0). Osnovni lastni par je torej za $n = 1$, vendar ga bolj kljub temu ozna\v cil z indeksom 0.

\begin{equation}
	\psi_0^{(1)} (x) = \frac{1}{2}\sin\pi x, \quad \frac{2}{\pi^2}E_0^{(1)} = 1 \neq 0.
\end{equation}

Supersimetrija je v tem primeru zlomljena, saj $E_0 \neq 0$, zato moramo za\v cetni Hamiltonian
translirati v energiji:

\begin{equation}
	(\underbrace{H - E_0^{(1)}}_{H_1})\psi_0^{(1)} = (E_0^{(1)} - E_0^{(1)})\psi_0^{(1)} =
		0\cdot\psi_0^{(1)} = 0. \label{translacija}
\end{equation}

Od tod lahko poi\v s\v cemo superpotencial $W(x)$ in to kar po definiciji iz
ena\v cbe~\eqref{superpotencial}

\begin{equation}
	W(x) = -\frac{1}{\sqrt{2}} \frac{\px \sin \pi x}{\sin \pi x} = -\frac{\pi}{\sqrt{2}}\cot \pi x.
\end{equation}

Sedaj lahko poi\v cemo supersimetri\v cnega partnerja neskon\v cne potencialne jame s pomo\v cjo
ena\v cbe~\eqref{v2pot}

\begin{equation}
	V_2 (x) = \frac{\pi^2}{2}\bigg(\cot^2 \pi x + \frac{1}{\sin^2 \pi x}\bigg)
		= \pi^2\bigg(\frac{1}{\sin^2 \pi x} - \frac{1}{2}\bigg),
	\label{pot-nes-jama}
\end{equation}

kar o\v citno ni ve\v c neskon\v cna potencialna jama, za katero dobimo $V_1(x) = \pi^2/2$, ki pa se ravno
prikladno od\v steje s konstantno $E_0^{(1)}$ v ena\v cbi~\eqref{translacija}. Potencial $V_2$ je poseben primer
potenciala Rosen Morse I (tj. triginometri\v cna izvedenka).

Sedaj smo dobili oba Hamiltoniana,

\begin{align}
	H_1 &= -\frac{1}{2}\px^2, \notag \\
	H_2 &= -\frac{1}{2}\px^2 - V_2(x),
\end{align}

od koder lahko \v ze sklepamo, kako bo izgledal supersimetri\v cni Hamiltonian $\H$.

Poglejmo \v se kako izgledajo valovne funkcije $\psi_n^{(2)}(x)$. Za to bomo seveda uporabili
operatorje "`vi\v sanja"' in "`ni\v zanja"', vendar bi morali za "`brute force"' pristop
izra\v cunati ne preve\v c lepo diferencialno ena\v cbo za $H_2$.

Vendar lahko tudi tokrat uporabimo izraze iz prej\v snjega poglavja in sicer~\eqref{degenener}.
Vemo, da $H_2$ nima stanja pri $n = 1$, ampak da se \v stetje za\v cne pri $n = 2$.

Da bomo dobili $\psi_0^{(2)}$ bomo uporabili identiteto

\begin{align}
	\psi_0^{(2)} (x) = a\psi_2^{(1)} &= -\frac{1}{2\sqrt{2}}(\pi \cot \pi x -
		\px)\sin2\pi x \\
	&= \frac{2\pi}{2\sqrt{2}}(\cos^2\pi x - \cos 2\pi x) = \frac{\pi}{\sqrt{2}}\sin^2\pi x.
\end{align}

Operatorja $a$ in $a^\dagger$ ne slu\v zita ve\v c vi\v sanju in ni\v zanju energije, ampak
pretvorbi stanj med $H_1$ in $H_2$.

Splo\v sen izraz za $\psi_n^{(2)}$ je dolg in grd. Zapisal bom raje le \v se za
$\psi_3^{(2)}$, za $n=3$, se pravi prvo vzbujeno stanje $H_2$:

\begin{equation}
	\psi_3^{(2)}(x) \propto \sin (\pi x) \sin (2\pi x),
\end{equation}

se pravi do normalizacijske konstante natan\v cno. Operatorja $a$ in $a^\dagger$ seveda
po pretvorbi pokvarita normalizacijo, tako da so rezultati narobe normirani.
