\section{Osnove supersimetrije}

S simetrijami problema lahko v veliko primerih pojasnimo dobljene fizikalne re\v sitve: energijska stanja molekul,
sipanje na kristalih, ohranjene koli\v cine itd. Pravilno upo\v stevanje simetrij lahko poenostavi marsikateri problem,
v nekaterih vejah fizike pa je uporaba teorije grup obvezna.

V grobem simetrije delimo na zunanje in notranje.
\begin{itemize}
	\item{Zunanje simetrije so tiste, ki niso vezane na specifi\v cen problem: fizikalni zakoni so invariantni
		na translacije v prostoru (ohranitev gibalne koli\v cine), \v casu (ohranitev energije) in rotacije
		(ohranitev vrtilne koli\v cine). To so simetrije makroskopskega sistema.}
	\item{Notranje simetrije so povezane z mikroskopsko sliko. \v Ce pogledamo res od blizu in zamrznemo
		\v cas, lahko ugotovimo v katerem mikroskopskem stanju smo in preu\v cimo njegove simetrije. Lahko ugotovimo
		kako so delci urejeni, kaj se zgodi ob enostavni zamenjavi dveh ali ve\v c delcev, ali \v ce nad njimi
		izvedemo abstraktno rotacijo. Lahko Te simetrije nam vrnejo npr. ohranitev elektri\v cnega naboja, fermionskega
		\v stevila, itd.}
\end{itemize}

Obstaja izrek, ki pravi, da z Liejevo algebro ni mo\v c zdru\v ziti notranjih in zunanjih simetrij. Fiziki so iskali
razne alterntive, edina mo\v zna je supersimetrija, ki je Liejeva superalgebra.

\ni Ko je bila ustvarjena vizija ti. velike teorije poenotnenja~\footnote{Ang. \emph{Grand Unification Theory}, ali na kratko GUT.}
so fiziki iskali poleg poenotenja sil, tudi enoten zapis za bozone in fermione, hkrati pa tudi zdru\v zitev notranjih
z zunanjimi simetrijami. Supersimetrija je otrok tega prizadevanja. Uvede namre\v c enoten zapis za obravnavo tako bozonov,
kot fermionov.

Supersimetrija je od takrat naprej napredovala in prekora\v cila meje relativisti\v cne kvantne mehanike. Formalna obravnava
supersimetrije kot matemati\v cnega pripomo\v cka, je raz\v sirila njeno uporabo na nerelativisti\v cno kvantno mehaniko

Tekom celotnega seminarja bomo imeli v mislih re\v sevanje Schr\" odingerjeve ena\v cbe,

\begin{equation}
	H = -\frac{\hbar^2}{2m}\nabla^2 + V(\vec{x}).
	\label{s1}
\end{equation}

\ni Zaradi enostavnosti pisave bomo razdalje in energijo merili v primeru $\hbar = m = 1$, kar ustreza preprosti
brezdimenzijski transformaciji. Omejili se bomo na enodimenzionalne probleme, tj. Hamiltonian iz en.~\eqref{s1}
prepi\v semo v

\begin{equation}
	H = -\frac{1}{2}\px^2 + V(x).
	\label{sx}
\end{equation}

\ni Na\v s potencial naj bo neni\v celen in navzdol omejen\footnote{Obstaja tak $V_0$, da $V(x) \geq V_0,\ \forall x$.}. Potem
lahko izmed vezanih stanj poi\v s\v cemo lastne pare $E_n$, $\psi_n (x)$ Hamiltonovega operatorja $H$,

\begin{equation}
	H\psi_n(x) = E_n\psi_n(x) = i\pt\psi_n(x).
\end{equation}

\ni Za $E_0 = 0$ velja $H|\psi_0\rangle = 0$. Od tod dobimo pogoj za potencial

\begin{equation}
	V(x) = \frac{1}{2}\frac{\px^2\psi_0}{\psi_0},
	\label{vpogoj}
\end{equation}

Na\v s Hamiltonian bi radi razcepili na produkt posplo\v senega kreacijskega in anihilacijskega operatorja, tj.
$H = a^\dagger a$. Vidimo, da je~\eqref{sx} oblike
$H \sim (A + B)(A - B)$, tako bomo definirali superpotencial $W(x)$, da bo

\begin{align}
	a &= W(x) + \frac{1}{\sqrt{2}}\px,          \label{anihilator} \\
	a^\dagger &= W(x) - \frac{1}{\sqrt{2}}\px.  \label{kreator}
\end{align}

\ni Taka operatorja sta res drug drugemu hermitsko adjungirana, saj je operator $\px$ anti-hermitski (do totalnega
odvoda natan\v cno).

Tak $W(x)$ mora vrniti Schr\" odingerjevo ena\v cbo~\eqref{sx}, torej mora veljati

\begin{align}
	H\phi(x) = a^\dagger a\ \phi(x)&= \Big[W(x) - \frac{1}{\sqrt{2}}\px\Big]\Big[W(x) +
		\frac{1}{\sqrt{2}}\px\Big]\phi(x) \notag \\
	&= \Big[-\frac{1}{2}\px^2 + W^2(x)\Big]\phi(x) - \frac{1}{\sqrt{2}}\Big[
		\underbrace{\px \big(W(x)\phi(x)\big) - W(x)\px\phi(x)}_{(\px W(x))\phi(x)} \Big] \notag \\
	&= \bigg\{-\frac{1}{2}\px^2 +
		\underbrace{W^2(x) - \frac{1}{\sqrt{2}}\big[\px W(x)\big]}_{V(x)}\bigg\}\phi(x).
	\label{dokaz.aad}
\end{align}

\ni To je prvotna Schr\" odingerjeva ena\v cba~\eqref{sx} natanko tedaj, kadar $W (x)$ spo\v stuje slede\v ca izraza:

\begin{align}
	V(x) &=\frac{1}{2}\frac{\px^2\psi_0(x)}{\psi_0(x)} = W^2(x) - \frac{1}{\sqrt{2}}\px
		W(x) \label{riccati} \\
	W(x) &= -\koren\frac{\px\psi_0(x)}{\psi_0(x)} = -\koren \px \ln \psi_0 (x)\label{superpotencial},
\end{align}

\ni kjer smo izraz~\eqref{superpotencial} dobili z re\v sevanjem Riccatijeve ena\v cbe~\eqref{riccati}.

Tako smo dobili $H = a^\dagger a$, z lastnimi funkcijami $\psi_n(x)$ in $E_n$, ki jih poznamo
od prej. Poglejmo, kaj se zgodi, \v ce zamenjamo vrstni red operatorjev $a$ in $a^\dagger$ -- definirajmo $\tilde{H} = aa^\dagger$,
z lastnimi funkcijami $\tilde{\psi}_n$. Dobimo par Hamiltonianov $H$ in $\tilde{H}$,

\begin{align}
	H &= -\frac{1}{2}\px^2 + V(x) = a^\dagger a, \\
	\tilde{H} &= -\frac{1}{2}\px^2 + \tilde{V}(x) = aa^\dagger,
\end{align}

\ni kjer sta $V(x)$ in $\tilde{V}(x)$ povezana prek

\begin{align}
	V(x) &= W^2(x) - \frac{1}{\sqrt{2}}\px W(x), \notag \\
	\tilde{V}(x) &= W^2(x) + \frac{1}{\sqrt{2}}\px W(x). \label{v2pot}
\end{align}

\ni Ena\v cbo~\eqref{v2pot} lahko doka\v zemo z istim postopkom kot v en.~\eqref{dokaz.aad}.

Pravimo, da je $\tilde{V}(x)$ supersimetri\v cni partner potenciala $V(x)$. Lastne funkcije in energijski spekter $\tilde{H}$
lahko dobimo z re\v sevanjem, ali pa ga uganemo:

\begin{equation}
	\tilde{\psi}_n(x) = a\psi_n(x),
\end{equation}

\ni od koder vidimo da $\psi_0(x)$ ne obstaja, saj anihilacijski operator\footnote{Operatorja $a$ in $a^\dagger$ nista nujno
operatorja vi\v sanja/ni\v zanja, vendar se v tem konkretnem primeru zgodi ravno to, da $a\psi_0 = 0$.} iz vakuuma po definiciji
naredi ni\v clo. Energijski spekter $\tilde{H}$ je enak tistemu iz $H$, s tem da nima osnovnega stanja $E_0$, kar lahko
poka\v zemo kot

\begin{align}
	\tilde{H}\tilde{\psi}_n(x) &= aa^\dagger [a\psi_n(x)] = a(a^\dagger a)\psi_n(x) = \notag \\
		&= aE_n\psi_n = E_n(a\psi_n) = E_n\psi_n^{(2)},
	\label{degenener}
\end{align}

\ni zaradi \v cesar lahko sklepamo
\begin{itemize}
	\item{energijski nivoji $\tilde{H}$ sovpadajo z nivoji iz $H$, vendar pa $\tilde{H}$ nima vezanega stanja pri
		$E_0$, zato se zanj \v stetje energij pri\v cne pri $E_1$,}
	\item{$a$ in $a^\dagger$ ne moremo interpretirati, kot kreacijsko-anihilacijska operatorja, ampak kot operatorja
		"`me\v sanja"', saj brez spremembe energije delec iz $H$ preslikata v $\tilde{H}$.}
\end{itemize}

Energija vakuuma $E_0$ je v mnogih primerih lahko arbitrarna in se je z ustrezno konstanto lahko znebimo:

\begin{equation}
	H \to H^\prime = H - E_0.
\end{equation}

\ni Takrat sta $H$ in $\tilde{H}$ v supersimetriji, sicer pa pravimo, da je spontano zlomljena.

Tu je supersimetrija uporabljena zgolj kot matemati\v cni pripomo\v cek, vendar pa se kljub temu poznajo ostanki
formalizma iz fizike visokih energij, \v ce definiramo supersimetri\v cni hamiltonian $\H$:

\begin{equation}
	\H \equiv H \oplus \tilde{H} \equiv \begin{bmatrix}H & \\
		& \tilde{H} \end{bmatrix},
\end{equation}

\begin{equation}
	Q^\dagger = \sigma^+ a^\dagger = \begin{bmatrix} 0 & a^\dagger \\
		0 & 0 \end{bmatrix}, \quad
	Q = \sigma^- a = \begin{bmatrix} 0 & 0 \\
		a & 0 \end{bmatrix},
	\label{superop}
\end{equation}

\begin{equation}
	Q^\dagger Q + QQ^\dagger = \{Q, Q^\dagger\} = \H.
\end{equation}

Operatorja $Q$ in $Q^\dagger$ skupaj z generatorji grupe zunanjih simetrij tvorijo supersimetri\v cno algebro. Od tod
se, \v ceprav nismo ve\v c v relativisti\v cnem pribli\v zku, \v se vedno pozna supersimetrijsko izro\v cilo.

\subsection{Me\v sanje bozonskih in fermionskih stanj}

\v Ceprav v enodimenzionalnem nerelativisti\v cnem modelu enega delca nimamo dobrega opisa, ki bi lo\v cil med bozoni in
fermioni, lahko uporabimo supersimetrijsko izro\v cilo in poka\v zemo kje je ideja supersimetrije v relativisti\v cni
kvantni mehaniki.

V supersimetri\v cnih teorijah bozonske in fermionske operatorje opi\v semo zgolj z eno samo dru\v zino operatorjev -- to
so $Q$ in $Q^\dagger$, ki v relativisti\v cnem opisu ne lo\v cijo med bozoni in fermioni. Operatorja $a$ in $a^\dagger$
zado\v s\v cata kanoni\v cnim komutacijskim relacijam za bozone, zato sklepamo da sta to bozonska operatorja

\begin{equation}
	[a, a^\dagger] = (\px W), \qquad [a, a] = 0.
\end{equation}

\ni V splo\v snem je $\px W = 1$ le za harmonski oscilator, za vi\v sje \v clene pa je problem zaradi anharmonske
sklopitve bolj kompliciran, vendar obstaja dokaz za splo\v sen polinomski potencial, v katerem
opis v Fockovem prostoru ni ve\v c tako enostaven, so pa $a$, $a^\dagger$ \v se vedno bozonski.
Definirajmo operatorje $c$ in $c^\dagger$, tako da

\begin{equation}
	c = \sigma^+, \qquad c^\dagger = \sigma^-,
\end{equation}

\ni Poka\v zemo lahko, da operatorji $c$ in $c^\dagger$ zadostijo enostavnim fermionskim kanoni\v cnim anti-komutacijskim
relacijam

\begin{equation}
	\{c, c^\dagger\} = 1, \quad \{c, c\} = 0, \quad n_F = \frac{1 - [c,c^\dagger]}{2},
\end{equation}

\ni kjer je $n_F$ fermionski operator \v stetja. Sevada je to le trik in v nerelativisti\v cnem opisu ne gre za prave fermione,
ampak to mo\v cno spominja na Jordan-Wignerjevo transformacijo -- spinski prostor ima namre\v c kon\v cno dimenzijo glede na vrednost
celotnega spina, Fockov prostor za prave fermione pa je neskon\v cno dimenzionalen. Se pravi, spinski operatorji zado\v scajo
kanoni\v cnim anti-komutacijskim relacijem za fermione, vendar kljub temu niso fermioni. Kljub temu, lahko z njimi \v se
vedno poka\v zemo globlji pomen operatorjev $Q$ in $Q^\dagger$, ki ju lahko zapi\v semo kot

\begin{equation}
	Q = c^\dagger a, \qquad Q^\dagger = a^\dagger c.
\end{equation}

\ni Sedaj vidimo, kako je pravzaprav treba interpretirati operatorje $Q$ in $Q^\dagger$. Ker namre\v c
velja

\begin{equation}
	[Q, \mathbf{H}] = [Q^\dagger, \mathbf{H}] = 0,
\end{equation}

\ni vidimo, da $Q$ in $Q^\dagger$ menjata bozonska in fermionska stanja, ne da bi pri tem spremenila energijo sistema.
\v Ceprav imata $Q$ in $Q^\dagger$ velik pomen v kvantni teoriji polja, sta na tem nivoju odve\v c -- za celosten opis
za na\v se potrebe zado\v s\v cata namre\v c \v ze $a$ in $a^\dagger$.

