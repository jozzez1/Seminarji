\section{Osnove supersimetrije}

V fiziki visokih energij moramo za opis konsistentne teorije upo\v stevati simetrije problema. Te se delijo
na notranje in zunanje.

Zunanje simetrije so tiste, ki veljajo v splo\v snem za vse fizikalne probleme: ohranitvi
energije in gibalne koli\v cine sta posledici invariantnosti fizikalnih zakonov na translacije (oz. rotacije) po
prostoru in \v casu. Operatorji teh transformacij tvorijo grupo, ki se imenuje po francoskem fiziku Henriju Poincaré
-- Poincaréjeva grupa. Ta upo\v steva transformacije klasi\v cne fizike in splo\v sne teorije relativnosti.

Notranje simetrije so simetrije med posameznimi polji, ki nastopajo v na\v sem problemu.
Operatorji teh transformacij, na katere je na\v s problem invarianten, so lahko rotacije v ve\v cdimenzionalnem
prostoru in mogo\v ce sestavljajo grupo, lahko pa so tudi kar polja sama -- opi\v semo jih
namre\v c s ti. operatorji polja, ki morajo zado\v s\v cati bodisi bozonski, bodisi fermionski algebri.

V sedemdesetih letih prej\v snjega stoletja, so si ljudje prizadevali, da bi poenotili opis fermionov in bozonov.
Supersimetrija je otrok te zveze in hkrati predstavlja edino mo\v znost cin zdru\v zitve notranjih simetrij z
zunanjimi, tj. naravno raz\v siri Poincaréjevo grupo s simetrijami polj.

Tekom celotnega seminarja bomo imeli v mislih re\v sevanje Schr\" odingerjeve ena\v cbe,

\begin{equation}
	H = -\frac{\hbar^2}{2m}\nabla^2 + mV(\vec{x}).
	\label{s1}
\end{equation}

\ni Zaradi enostavnosti pisave bomo imeli brezdimenzijske koli\v cine $\hbar = m = 1$.
Omejili se bomo na enodimenzionalne probleme, tj. Hamiltonian en.~\eqref{s1} prepi\v semo v

\begin{equation}
	H = -\frac{1}{2}\px^2 + V(x).
	\label{sx}
\end{equation}

\ni Na\v s potencial naj bo neni\v celen in navzdol omejen\footnote{Obstaja tak $V_0$, da $V(x) \geq V_0,\ \forall x$.}. Potem
lahko izmed vezanih stanj poi\v s\v cemo lastne pare $E_n$, $\psi_n (x)$ Hamiltonovega operatorja $H$,

\begin{equation}
	H\psi_n(x) = E_n\psi_n(x) = i\pt\psi_n(x).
\end{equation}

\ni Za $E_0 = 0$ velja $H|\psi_0\rangle = 0$. Od tod dobimo pogoj za potencial

\begin{equation}
	V(x) = \frac{1}{2}\frac{\px^2\psi_0}{\psi_0},
	\label{vpogoj}
\end{equation}

Na\v s Hamiltonian bi radi razcepili na produkt posplo\v senega kreacijskega in anihilacijskega operatorja, tj.
$H = a^\dagger a$. Vidimo, da je~\eqref{sx} oblike
$H \sim (A + B)(A - B)$, tako bomo definirali superpotencial $W(x)$, da bo

\begin{align}
	a &= W(x) + \frac{1}{\sqrt{2}}\px,          \label{anihilator} \\
	a^\dagger &= W(x) - \frac{1}{\sqrt{2}}\px.  \label{kreator}
\end{align}

\ni Taka operatorja sta res drug drugemu hermitsko adjungirana, saj je operator $\px$ anti-hermitski (do totalnega
odvoda natan\v cno).

Pogoje za obstoj $W(x)$ bomo dolo\v cili retrospektivno, kot se to mnogokrat zgodi v fiziki.
En.~\eqref{anihilator} in~\eqref{kreator} vstavimo v Hamiltonov operator, od koder bodo vezi,
katerim mora zado\v s\v cati, o\v citnej\v se. Faktoriziran $H$ na neki funkciji $\phi (x)$ stori

\begin{align}
	H\phi(x) = a^\dagger a\ \phi(x)&= \Big[W(x) - \frac{1}{\sqrt{2}}\px\Big]\Big[W(x) +
		\frac{1}{\sqrt{2}}\px\Big]\phi(x) \notag \\
	&= \Big[-\frac{1}{2}\px^2 + W^2(x)\Big]\phi(x) - \frac{1}{\sqrt{2}}\Big[
		\underbrace{\px \big(W(x)\phi(x)\big) - W(x)\px\phi(x)}_{(\px W(x))\phi(x)} \Big] \notag \\
	&= \bigg\{-\frac{1}{2}\px^2 +
		\underbrace{W^2(x) - \frac{1}{\sqrt{2}}\big[\px W(x)\big]}_{V(x)}\bigg\}\phi(x).
	\label{dokaz.aad}
\end{align}

\ni To je prvotna Schr\" odingerjeva ena\v cba~\eqref{sx} natanko tedaj, kadar $W (x)$ spo\v stuje slede\v ca izraza:

\begin{align}
	V(x) &=\frac{1}{2}\frac{\px^2\psi_0(x)}{\psi_0(x)} = W^2(x) - \frac{1}{\sqrt{2}}\px
		W(x) \label{riccati} \\
	W(x) &= -\koren\frac{\px\psi_0(x)}{\psi_0(x)} = -\koren \px \ln \psi_0 (x)\label{superpotencial},
\end{align}

\ni kjer smo izraz~\eqref{superpotencial} dobili z re\v sevanjem Riccatijeve ena\v cbe~\eqref{riccati}.

Tako smo dobili $H_1 = a^\dagger a$ in ima re\v sitve $\psi_n(x)$ in $E_n$, kot jih poznamo
od prej. Poglejmo, kaj se zgodi, \v ce zamenjamo vrstni red operatorjev -- definirajmo $H_2 = aa^\dagger$.
Dobimo par Hamiltonianov $H_1$ in $H_2$,

\begin{align}
	H_1 &= -\frac{1}{2}\px^2 + V_1(x) = a^\dagger a, \\
	H_2 &= -\frac{1}{2}\px^2 + V_2(x) = aa^\dagger,
\end{align}

\ni kjer sta $V_1(x)$ in $V_2(x)$ povezana prek

\begin{align}
	V_1(x) &= W^2(x) - \frac{1}{\sqrt{2}}\px W(x), \notag \\
	V_2(x) &= W^2(x) + \frac{1}{\sqrt{2}}\px W(x). \label{v2pot}
\end{align}

\ni Ena\v cbo~\eqref{v2pot} lahko doka\v zemo z istim postopkom kot v en.~\eqref{dokaz.aad}.

Pravimo, da je $V_2(x)$ supersimetri\v cni partner $V_1(x)$. Hamiltonian $H_1$ ima lastne pare $\psi^{(1)}_n(x)$,
$E^{(1)}_n$, $H_2$ pa $\psi^{(2)}_n(x)$, $E^{(2)}_n$. Lastne funkcije in energijski spekter $H_2$ lahko dobimo z
re\v sevanjem, ali pa ga uganemo:

\begin{equation}
	\psi^{(2)}_n(x) = a\psi^{(1)}_n(x),
\end{equation}

\ni od koder vidimo da $\psi^{(2)}_0(x)$ ne obstaja, saj anihilacijski operator iz vakuuma po definiciji naredi
ni\v clo. Energijski spekter $H_2$ je enak tistemu iz $H_1$, s tem da nima osnovnega stanja $E_0$, kar lahko
poka\v zemo kot

\begin{align}
	H_2\psi^{(2)}_n(x) &= aa^\dagger (a\psi^{(1)}_n(x)) = a(a^\dagger a)\psi^{(1)}_n(x) = \notag \\
		&= aE^{(1)}_n\psi^{(1)}_n(x) = E_n^{(1)}(a\psi_n^{(1)}) = E_n^{(1)}\psi_n^{(2)},
	\label{degenener}
\end{align}

\ni se pravi

\begin{equation}
	E^{(1)}_n \equiv E^{(2)}_n, \qquad n = 1, 2, 3 \ldots
\end{equation}

\ni Zaradi tega redefiniramo $H_1$, da bo po novem osnovno stanje $\psi_0^{(1)}$ v jedru operatorja $H_1$

\begin{equation}
	H_1 \to H_1^\prime = H_1 - E_0.
\end{equation}

Kadar $E_0 \neq 0$ govorimo o zlomu supersimetrije.

Hamiltoniana $H_1$ in $H_2$ bi radi zdru\v zili v enega, tako da se prostora ne me\v sata. Zato
definiramo

\begin{equation}
	\H \equiv H_1 \oplus H_2 \equiv \begin{bmatrix}H_1 & \\
		& H_2 \end{bmatrix},
\end{equation}

\begin{equation}
	Q^\dagger = \sigma^+ a^\dagger = \begin{bmatrix} 0 & a^\dagger \\
		0 & 0 \end{bmatrix}, \quad
	Q = \sigma^- a = \begin{bmatrix} 0 & 0 \\
		a & 0 \end{bmatrix},
	\label{superop}
\end{equation}

\begin{equation}
	Q^\dagger Q + QQ^\dagger = \{Q, Q^\dagger\} = \H.
\end{equation}

Operatorji $Q$ in $Q^\dagger$ skupaj s generatorji Poincaréjeve grupe tvorijo ti. supersimetri\v cno algebro, ali
na kratko "`superalgebro"'.

\subsection{Me\v sanje bozonskih in fermionskih stanj}

Bozonski in fermionski operatorji tvorijo algebro, ki zado\v s\v ca kanoni\v cnim komutacijskim relacijam

\begin{align}
	[a_i, a_j^\dagger]_\pm &= \delta_{ij}, \\
	[a_i, a_j]_\pm &= 0,
\end{align}

\ni kjer je $[\bullet,\bullet]_- \equiv [\bullet,\bullet]$ komutator in velja za bozone,
$[\bullet,\bullet]_+ \equiv \{\bullet,\bullet\}$ pa je anti-komutator in velja za fermione.

V supersimetri\v cnih teorijah bozonske in fermionske operatorje opi\v semo na identi\v cno enak
na\v cin. To je posebej razvidno iz operatorjev $Q$ in $Q^\dagger$ iz en.~\eqref{superop}. Operatorja
$a$ in $a^\dagger$ tvorita boznonsko algebro

\begin{equation}
	[a, a^\dagger] = (\px W), \qquad [a, a] = 0.
\end{equation}

\ni V splo\v snem je $\px W = 1$ le za harmonski oscilator, za vi\v sje \v clene pa je problem zaradi anharmonske
sklopitve bolj kompliciran, vendar obstaja dokaz za splo\v sen polinomski potencial, v katerem
opis v Fockovem prostoru ni ve\v c tako enostaven, so pa $a$, $a^\dagger$ \v se vedno bozonski.
Definirajmo operatorje $c$ in $c^\dagger$, tako da

\begin{equation}
	c = \sigma^+, \qquad c^\dagger = \sigma^-,
\end{equation}

\ni Poka\v zemo lahko, da operatorji $c$ in $c^\dagger$ zadostijo enostavni fermionski algebri,

\begin{equation}
	\{c, c^\dagger\} = 1, \quad \{c, c\} = 0, \quad n_F = \frac{1 - [c,c^\dagger]}{2},
\end{equation}

\ni kjer je $n_F$ fermionski operator \v stetja. $Q$ in $Q^\dagger$ lahko potem zapi\v semo kot

\begin{equation}
	Q = c^\dagger a, \qquad Q^\dagger = a^\dagger c.
\end{equation}

\ni Sedaj vidimo, kako je pravzaprav treba interpretirati operatorje $Q$ in $Q^\dagger$. Ker namre\v c
velja

\begin{equation}
	[Q, \mathbf{H}] = [Q^\dagger, \mathbf{H}] = 0,
\end{equation}

\ni vidimo, da $Q$ in $Q^\dagger$ menjata bozonska in fermionska stanja, ne da bi pri tem spremenila energijo sistema,
hkrati pa prenesejo stanja iz $H_1$ v $H_2$ oz. $H_2$ v $H_1$. Torej $H_1$ ustreza bozonom, $H_2$ pa fermionom.

