\section{Ponovitev}

Iz predavanj Višje kvantne mehanike dr. prof. Tomaža Prosena se spomnimo, da lahko polja
kvantiziramo prek operatorjev polja, za kar pa nujno potrebujemo kreacijske, $a^\dagger$ in 
anihilacijske $a$ operatorje.

\subsection{Grupa}

\vspace{0.5 cm}

{\em Grupa} je množica elementov z naslednjimi lastnostmi:

\begin{enumerate}
	\item{Med elementi obstaja asociativna operacija.}
	\item{Množica je za to operacijo zaprta.}
	\item{Izmed teh elementov je eden tak, ki je za to operacijo enota.}
	\item{V množici so zajeti tudi vsi inverzni elementi.}
\end{enumerate}

V kolikor ne veljajo vsi pogoji imamo enega izmed nižjih objektov, kot je na primer monoid
ali polgrupa.

Če je operacija komutativna, je ta grupa abelova in operacijo imenujemo seštevanje. Sicer
pa je grupa neabelova in operacijo imenujemo množenje.

\subsection{Algebra}

\vspace{0.5cm}

Pogostokrat srečamo pojem `algebra'. Pa ponovimo, kaj je to.\\

\emph{Kolobar} je množica elementov z naslednjimi lastnostmi:

\begin{enumerate}
	\item{Ta množica je abelova grupa z operacijo seštevanja.}
	\item{Elementi so tudi grupa za operacijo množenja.}
	\item{Za kombinacijo operacij velja distributivnostna relacija.}
\end{enumerate}

\emph{Algebra} je vektorski prostor nad kolobarjem.

Liejeve grupe imajo tudi svojo algebro -- generatorji Liejevih grup nepenjajo vektorski
prostor. Da pokažemo, da zadoščajo pogojem iz algebre zadošča zapis (anti)komutacijskih
relacij.

\begin{equation}
	[A,B]_\pm = AB \pm BA
\end{equation}

Tudi druge končnim grupam lahko priredimo algebre -- za nas relevantna je simetrična grupa
$S_n$, ki jo imenujemo tudi permutacijska grupa. Generatorji so kreacijsko-anihilacijski
operatorji. Lastne vrednosti so $\pm 1$, hkrati pa grupa komutira s Hamiltonianom, kar pomeni,
da je to dobra simetrija. Od tu delce ločimo glede na lihe oz. sode rešitve glede na $S_n$. 
Lihe ustrezajo fermionom, sode pa bozonom. 

Tako od tu dobimo fermionsko in bozonsko algebro:

\begin{align}
	[a_i, a_j^\dagger]_\pm &= \delta_{ij}, \\
	[a_i, a_j]_\pm &= 0,
\end{align}

kjer komutatorji ustrezajo bozonom, antikomutatorji pa fermionom.

\section{Uvod v supersimetrijo}

Obravnavajmo Schr\" odingerjevo enačbo, tj. Hamiltonian

\begin{equation}
	H = -\frac{\hbar^2}{2m}\nabla^2 + mV(\vec{x}).
	\label{schroedinger}
\end{equation}

Vpeljemo enote $\hbar = c_0 = \varepsilon_0 = 1$. $V(\vec{x})$ je brezdimenzijski, vidimo, da
nam $m$ služi zgolj za reskalirenje koordinate $\vec{x}$. Da bo~\eqref{schroedinger} brez
dimenzije naredimo $\vec{x} \to m\vec{x}$. Tako je brezdimenzijska Schr\" odingerjeva enačba

\begin{equation}
	H = -\frac{1}{2}\nabla^2 + V(\vec{x}),
	\label{s1}
\end{equation}

kjer velja še $H = i\pt$. V eni dimenziji se~\eqref{s1} glasi

\begin{equation}
	H = -\frac{1}{2}\px^2 + V(x).
	\label{sx}
\end{equation}

Velja

\begin{equation}
	H\psi_n(x) = E_n\psi_n(x) = i\pt\psi_n(x).
\end{equation}

Za $E_0 = 0$ torej velja $H|\psi_0\rangle = 0$. Od tu dobimo pogoj za potencial

\begin{equation}
	V(x) = \frac{1}{2}\frac{\px^2\psi_0}{\psi_0}
	\label{vpogoj}
\end{equation}

Naš Hamiltonian bi radi zapisali v obliko z operatorjem štetja, $a^\dagger$ in $a$, se pravi
$H = a^\dagger a$. Vidimo, da je~\eqref{sx} oblike $H \sim (A + B)(A - B)$, tako bomo
definirali superpotencial $W(x)$, da bo

\begin{align}
	a &= W(x) + \frac{1}{\sqrt{2}}\px, \\
	a^\dagger &= W(x) - \frac{1}{\sqrt{2}}\px.
\end{align}

Poglejmo, kaj je $a^\dagger a$,

\begin{align}
	a^\dagger a &= \Big[W(x) - \frac{1}{\sqrt{2}}\px\Big]\Big[W(x) +
		\frac{1}{\sqrt{2}}\px\Big] \\
	&= -\frac{1}{2}\px^2 + \underbrace{W^2(x) - \frac{1}{\sqrt{2}}\px W(x)}_{V(x)}
		-\underbrace{\frac{1}{\sqrt{2}}W(x)\px}_{=0}
\end{align}

Se pravi

\begin{align}
	V(x) &=\frac{1}{2}\frac{\px^2\psi_0(x)}{\psi_0(x)} = W^2(x) - \frac{1}{\sqrt{2}}\px
		W(x) \label{riccati} \\
	W(x) &= -\frac{1}{\sqrt{2}}\frac{\px\psi_0(x)}{\psi_0(x)} \label{superpotencial},
\end{align}

kjer smo izraz~\eqref{superpotencial} dobili z reševanjem Riccatijeve enačbe~\eqref{riccati}.

Tako smo dobili $H_1 = a^\dagger a$ in ima rešitve $\psi_n(x)$ in $E_n$, kot jih poznamo
od prej.

Definirajmo še $H_2 = aa^\dagger$. Dobimo ga kot

\begin{align}
	H_1 &= -\frac{1}{2}\px^2 + V_1(x) = a^\dagger a, \\
	H_2 &= -\frac{1}{2}\px^2 + V_2(x) = aa^\dagger,
\end{align}

kjer

\begin{align}
	V_1(x) &= W^2(x) - \frac{1}{\sqrt{2}}\px W(x), \\
	V_2(x) &= W^2(x) + \frac{1}{\sqrt{2}}\px W(x).
\end{align}

$H_1$ ima lastne pare $\psi^{(1)}_n(x)$, $E^{(1)}_n$, $H_2$ pa $\psi^{(2)}_n(x)$, $E^{(2)}_n$.

$V_2(x)$ je supersimetrični partner $V_1(x)$. Lastne funkcije in energijski spekter $H_2$
dobimo lahko z reševanjem, ali pa uganemo

\begin{equation}
	\psi^{(2)}_n(x) = a\psi^{(1)}_n(x),
\end{equation}

od koder vidimo da $\psi^{(2)}_0(x)$ ne obstaja, saj anihilacijski operator iz vakuuma naredi
ničlo po definiciji. Energijski spekter $H_2$ je enak tistemu iz $H_1$, s tem da nima
osnovnega stanja $E_0$, kar lahko pokažemo kot

\begin{equation}
	H_2\psi^{(2)}_n(x) = aa^\dagger (a\psi^{(1)}_n(x)) = a(a^\dagger a)\psi^{(1)}_n(x) =
		aE^{(1)}_n\psi^{(1)}_n(x) = E_n^{(1)}(a\psi_n^{(1)}) = E_n^{(1)}\psi_n^{(2)},
\end{equation}

se pravi

\begin{equation}
	E^{(1)}_n \equiv E^{(2)}_n, \qquad n = 1, 2, 3 \ldots
\end{equation}

Hamiltoniana $H_1$ in $H_2$ bi radi združili v enega, tako da se prostora ne mešata. Zato
definiramo

\begin{equation}
	H \equiv \begin{bmatrix}H_1& & \\ & &H_2 \end{bmatrix},
\end{equation}

\begin{equation}
	Q^\dagger = \begin{bmatrix} 0& &a^\dagger \\ 0& &0 \end{bmatrix}, \quad
	Q = \begin{bmatrix} 0& &0 \\ a& &0 \end{bmatrix},
\end{equation}

ki zadoščajo sledečim kanoničnim (anti)komutacijskim relacijam:

\begin{align}
	[Q_i,Q_j] = \{Q_i,Q_j\} = 0 \\
	\{Q_i,Q_j^\dagger\} = H\delta_{ij}
\end{align}


