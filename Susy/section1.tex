\section{Ponovitev}

Iz predavanj Vi\v sje kvantne mehanike dr. prof. Toma\v za Prosena se spomnimo, da lahko polja
kvantiziramo prek operatorjev polja, za kar pa nujno potrebujemo kreacijske, $a^\dagger$ in 
anihilacijske $a$ operatorje.

\subsection{Grupa}

\vspace{0.5 cm}

{\em Grupa} je mno\v zica elementov z naslednjimi lastnostmi:

\begin{enumerate}
	\item{Med elementi obstaja asociativna operacija.}
	\item{Mno\v zica je za to operacijo zaprta.}
	\item{Izmed teh elementov je natanko eden tak, ki je za to operacijo enota.}
	\item{V mno\v zici so zajeti tudi vsi inverzni elementi.}
\end{enumerate}

V kolikor ne veljajo vsi pogoji imamo enega izmed ni\v zjih objektov, kot je na primer monoid
ali polgrupa.

\v Ce je operacija komutativna, je ta grupa abelova in operacijo imenujemo se\v stevanje. Sicer
pa je grupa neabelova in operacijo imenujemo mno\v zenje.

\subsection{Algebra}

\vspace{0.5cm}

Pogostokrat sre\v camo pojem `algebra'. Pa ponovimo, kaj je to.\\

\emph{Kolobar} je mno\v zica elementov z naslednjimi lastnostmi:

\begin{enumerate}
	\item{Ta mno\v zica je abelova grupa z operacijo se\v stevanja.}
	\item{Elementi so tudi grupa za operacijo mno\v zenja.}
	\item{Za kombinacijo operacij velja distributivnostna relacija.}
\end{enumerate}

\emph{Algebra} je vektorski prostor nad kolobarjem.

Liejeve grupe imajo tudi svojo algebro -- generatorji Liejevih grup nepenjajo vektorski
prostor. Da poka\v zemo, da zado\v s\v cajo pogojem iz algebre zado\v s\v ca zapis (anti)komutacijskih
relacij.

\begin{equation}
	[A,B]_\pm = AB \pm BA
\end{equation}

Tudi druge kon\v cnim grupam lahko priredimo algebre -- za nas relevantna je simetri\v cna grupa
$S_n$, ki jo imenujemo tudi permutacijska grupa. Generatorji so kreacijsko-anihilacijski
operatorji. Lastne vrednosti so $\pm 1$, hkrati pa grupa komutira s Hamiltonianom, kar pomeni,
da je to dobra simetrija. Od tu delce lo\v cimo glede na lihe oz. sode re\v sitve glede na $S_n$. 
Lihe ustrezajo fermionom, sode pa bozonom. 

Tako od tu dobimo fermionsko in bozonsko algebro:

\begin{align}
	[a_i, a_j^\dagger]_\pm &= \delta_{ij}, \\
	[a_i, a_j]_\pm &= 0,
\end{align}

kjer komutatorji ustrezajo bozonom, antikomutatorji pa fermionom. Ti operatorji pripadajo eni izmed
nerazcepnih upodobitev grupe $S_n$ in potemtakem tvorijo bazo za invarinantnega prostora, v katerem
"`\v zivi"' na\v s Hamiltonian (grupa $S_n$ komutira s hamiltonskim operatorjem).

\section{Uvod v supersimetrijo}

Pa si omo\v cimo noge v vodi supersimetrije: obravnavajmo Schr\" odingerjevo ena\v cbo, tj. Hamiltonian

\begin{equation}
	H = -\frac{\hbar^2}{2m}\nabla^2 + mV(\vec{x}).
	\label{schroedinger}
\end{equation}

Vpeljemo enote $\hbar = c_0 = \varepsilon_0 = 1$. $V(\vec{x})$ je brezdimenzijski, vidimo, da
nam $m$ slu\v zi zgolj za reskalirenje koordinate $\vec{x}$. Da bo~\eqref{schroedinger} brez
dimenzije naredimo $\vec{x} \to m\vec{x}$. Tako je brezdimenzijska Schr\" odingerjeva ena\v cba

\begin{equation}
	H = -\frac{1}{2}\nabla^2 + V(\vec{x}),
	\label{s1}
\end{equation}

kjer velja \v se $H = i\pt$. V eni dimenziji se~\eqref{s1} glasi

\begin{equation}
	H = -\frac{1}{2}\px^2 + V(x).
	\label{sx}
\end{equation}

Velja

\begin{equation}
	H\psi_n(x) = E_n\psi_n(x) = i\pt\psi_n(x).
\end{equation}

Za $E_0 = 0$ torej velja $H|\psi_0\rangle = 0$. Od tu dobimo pogoj za potencial

\begin{equation}
	V(x) = \frac{1}{2}\frac{\px^2\psi_0}{\psi_0}
	\label{vpogoj}
\end{equation}

Na\v s Hamiltonian bi radi zapisali v obliko z operatorjem \v stetja, $a^\dagger$ in $a$, se pravi
$H = a^\dagger a$, zato moramo poiskati nek pameten razcep. Vidimo, da je~\eqref{sx} oblike
$H \sim (A + B)(A - B)$, tako bomo definirali superpotencial $W(x)$, da bo

\begin{align}
	a &= W(x) + \frac{1}{\sqrt{2}}\px, \\
	a^\dagger &= W(x) - \frac{1}{\sqrt{2}}\px.
\end{align}

Vse lepo in prav, vendar, a tak $W (x)$ sploh obstaja? V fiziki na to vpra\v sanje odgovorimo
retrospektivno in to bomo storili tudi sedaj. Vse skupaj vstavimo v Hamiltonian in iz njega dolo\v cimo
vezi, katerim mora zado\v s\v cati.

Poglejmo, kaj naredi $a^\dagger a$ na neki funkciji $\phi (x)$:

\begin{align}
	a^\dagger a\ \phi(x)&= \Big[W(x) - \frac{1}{\sqrt{2}}\px\Big]\Big[W(x) +
		\frac{1}{\sqrt{2}}\px\Big]\phi(x) \\
	&= \Big[-\frac{1}{2}\px^2 + W^2(x)\Big]\phi(x) - \frac{1}{\sqrt{2}}\Big[
		\underbrace{\px \big(W(x)\phi(x)\big) - W(x)\px\phi(x)}_{(\px W(x))\phi(x)} \Big] \\
	&= \bigg\{-\frac{1}{2}\px^2 +
		\underbrace{W^2(x) - \frac{1}{\sqrt{2}}\big[\px W(x)\big]}_{V(x)}\bigg\}\phi(x),
\end{align}

Se pravi, \v ce je ta ena\v cba Schr\" odingerjeva, potem mora $W (x)$ spo\v stovati slede\v ca izraza:

\begin{align}
	V(x) &=\frac{1}{2}\frac{\px^2\psi_0(x)}{\psi_0(x)} = W^2(x) - \frac{1}{\sqrt{2}}\px
		W(x) \label{riccati} \\
	W(x) &= -\frac{1}{\sqrt{2}}\frac{\px\psi_0(x)}{\psi_0(x)} \label{superpotencial},
\end{align}

kjer smo izraz~\eqref{superpotencial} dobili z re\v sevanjem Riccatijeve ena\v cbe~\eqref{riccati}.

Tako smo dobili $H_1 = a^\dagger a$ in ima re\v sitve $\psi_n(x)$ in $E_n$, kot jih poznamo
od prej.

Poglejmo, kaj se zgodi, \v ce vrstni red obrnemo. Definirajmo \v se $H_2 = aa^\dagger$. Dobimo ga kot

\begin{align}
	H_1 &= -\frac{1}{2}\px^2 + V_1(x) = a^\dagger a, \\
	H_2 &= -\frac{1}{2}\px^2 + V_2(x) = aa^\dagger,
\end{align}

kjer

\begin{align}
	V_1(x) &= W^2(x) - \frac{1}{\sqrt{2}}\px W(x), \\
	V_2(x) &= W^2(x) + \frac{1}{\sqrt{2}}\px W(x).
\end{align}

$H_1$ ima lastne pare $\psi^{(1)}_n(x)$, $E^{(1)}_n$, $H_2$ pa $\psi^{(2)}_n(x)$, $E^{(2)}_n$.

Pravimo, da je $V_2(x)$ supersimetri\v cni partner $V_1(x)$. Lastne funkcije in energijski spekter $H_2$
dobimo lahko z re\v sevanjem, ali pa uganemo

\begin{equation}
	\psi^{(2)}_n(x) = a\psi^{(1)}_n(x),
\end{equation}

od koder vidimo da $\psi^{(2)}_0(x)$ ne obstaja, saj anihilacijski operator iz vakuuma naredi
ni\v clo po definiciji. Energijski spekter $H_2$ je enak tistemu iz $H_1$, s tem da nima
osnovnega stanja $E_0$, kar lahko poka\v zemo kot

\begin{align}
	H_2\psi^{(2)}_n(x) &= aa^\dagger (a\psi^{(1)}_n(x)) = a(a^\dagger a)\psi^{(1)}_n(x) = \notag \\
		&= aE^{(1)}_n\psi^{(1)}_n(x) = E_n^{(1)}(a\psi_n^{(1)}) = E_n^{(1)}\psi_n^{(2)},
\end{align}

se pravi

\begin{equation}
	E^{(1)}_n \equiv E^{(2)}_n, \qquad n = 1, 2, 3 \ldots
\end{equation}

Zaradi tega, spremenimo definicijo $H_1$ tako, da ne bo imel ve\v c osnovnega stanja
\begin{equation}
	H_1 \to H_1^\prime = H_1 - E_0.
\end{equation}

Hamiltoniana $H_1$ in $H_2$ bi radi zdru\v zili v enega, tako da se prostora ne me\v sata. Zato
definiramo

\begin{equation}
	H \equiv H_1 \oplus H_2 \equiv \begin{bmatrix}H_1 & \\
		& H_2 \end{bmatrix},
\end{equation}

\begin{equation}
	Q^\dagger = \begin{bmatrix} 0 & a^\dagger \\
		0 & 0 \end{bmatrix}, \quad
	Q = \begin{bmatrix} 0 & 0 \\
		a & 0 \end{bmatrix},
\end{equation}

\begin{equation}
	H = Q^\dagger Q + QQ^\dagger,
\end{equation}

ki zado\v s\v cajo slede\v cim kanoni\v cnim (anti)komutacijskim relacijam:

\begin{align}
	[Q_i,Q_j] = \{Q_i,Q_j\} = 0 \\
	\{Q_i,Q_j^\dagger\} = H\delta_{ij}
\end{align}

Tak Hamiltonian ima degeneriran spekter, saj imata $H_1$ in $H_2$ iste lastne vrednosti.


