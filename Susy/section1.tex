\section{Osnove supersimetrije}

V fiziki visokih energij moramo za opis konsistentne teorije upo\v stevati simetrije problema. Te se delijo
na notranje in zunanje.

Zunanje simetrije so tiste, ki veljajo v splo\v snem za vse fizikalne probleme: ohranitvi
energije in gibalne koli\v cine sta posledici invariantnosti fizikalnih zakonov na translacije (oz. rotacije) po
prostoru in \v casu. Operatorji teh transformacij tvorijo grupo, ki se imenuje po francoskem fiziku Henriju Poincaré
-- Poincaréjeva grupa. Ta upo\v steva transformacije klasi\v cne fizike in splo\v sne teorije relativnosti.

Notranje simetrije, so simetrije med individualnimi polji in so lahko skorajda arbitrarne.
Operatorji teh transformacij so lahko rotacije v ve\v cdimenzionalnem prostoru in sestavljajo Liejevo
grupo, lahko so to\v ckovne simetrije, lahko so pa tudi kar polja sama -- opi\v semo jih
namre\v c s ti. operatorji polja, ki morajo zado\v s\v cati bodisi bozonski, bodisi fermionski algebri.

Supersimetrija je nastala iz te\v znje po zdru\v zitvi notranjih simetrij z zunanjimi. Lahko si jo mislimo kot
naravno raz\v siritev Poincaréjeve grupe. Supersimetrija je edini mo\v zen na\v cin kako povezati notranje in zunanje
simetrije.

Tekom celotnega seminarja bomo imeli v mislih re\v sevanje Schr\" odingerjeve ena\v cbe,

\begin{equation}
	H = -\frac{\hbar^2}{2m}\nabla^2 + mV(\vec{x}).
	\label{s1}
\end{equation}

\ni Zaradi enostavnosti pisave bomo imeli brezdimenzijske koli\v cine $\hbar = m = 1$.
Omejili se bomo na enodimenzionalne probleme, tj. Hamiltonian en.~\eqref{s1} prepi\v semo v

\begin{equation}
	H = -\frac{1}{2}\px^2 + V(x).
	\label{sx}
\end{equation}

\ni Predpostavimo, da je na\v s potencial neni\v celen in navzdol omejen. Velja

\begin{equation}
	H\psi_n(x) = E_n\psi_n(x) = i\pt\psi_n(x).
\end{equation}

\ni Za $E_0 = 0$ torej velja $H|\psi_0\rangle = 0$. Od tod dobimo pogoj za potencial

\begin{equation}
	V(x) = \frac{1}{2}\frac{\px^2\psi_0}{\psi_0},
	\label{vpogoj}
\end{equation}

\ni kjer smo predpostavili, da so stanja $\psi_n(x)$ vezana, tj. $\psi_0(x)$ je dobro dolo\v cen in legitimnost
en.~\eqref{vpogoj} ni vpra\v sljiva.

Na\v s Hamiltonian bi radi razcepili na produkt posplo\v senega kreacijskega in anihilacijskega operatorja, tj.
$H = a^\dagger a$, tj. neke vrste operator \v stetja. Vidimo, da je~\eqref{sx} oblike
$H \sim (A + B)(A - B)$, tako bomo definirali superpotencial $W(x)$, da bo

\begin{align}
	a &= W(x) + \frac{1}{\sqrt{2}}\px,          \label{anihilator} \\
	a^\dagger &= W(x) - \frac{1}{\sqrt{2}}\px.  \label{kreator}
\end{align}

\ni Taka operatorja sta res drug drugemu hermitsko adjungirana, saj je operator $\px$ anti-hermitski (do totalnega
odvoda natan\v cno).

Pogoje za obstoj $W(x)$ bomo dolo\v cili retrospektivno, kot se to mnogokrat zgodi v fiziki.
En.~\eqref{anihilator} in~\eqref{kreator} vstavimo v Hamiltonov operator, od koder bodo vezi,
katerim mora zado\v s\v cati, o\v citnej\v se.

Poglejmo si kako faktoriziran $H$ deluje na neko funkcijo $\phi (x)$:

\begin{align}
	H\phi(x) = a^\dagger a\ \phi(x)&= \Big[W(x) - \frac{1}{\sqrt{2}}\px\Big]\Big[W(x) +
		\frac{1}{\sqrt{2}}\px\Big]\phi(x) \notag \\
	&= \Big[-\frac{1}{2}\px^2 + W^2(x)\Big]\phi(x) - \frac{1}{\sqrt{2}}\Big[
		\underbrace{\px \big(W(x)\phi(x)\big) - W(x)\px\phi(x)}_{(\px W(x))\phi(x)} \Big] \notag \\
	&= \bigg\{-\frac{1}{2}\px^2 +
		\underbrace{W^2(x) - \frac{1}{\sqrt{2}}\big[\px W(x)\big]}_{V(x)}\bigg\}\phi(x),
	\label{dokaz.aad}
\end{align}

\ni To je Schr\" odingerjeva ena\v cba, natanko tedaj, kadar $W (x)$ spo\v stuje slede\v ca izraza:

\begin{align}
	V(x) &=\frac{1}{2}\frac{\px^2\psi_0(x)}{\psi_0(x)} = W^2(x) - \frac{1}{\sqrt{2}}\px
		W(x) \label{riccati} \\
	W(x) &= -\koren\frac{\px\psi_0(x)}{\psi_0(x)} = -\koren \px \ln \psi_0 (x)\label{superpotencial},
\end{align}

\ni kjer smo izraz~\eqref{superpotencial} dobili z re\v sevanjem Riccatijeve ena\v cbe~\eqref{riccati}.

Tako smo dobili $H_1 = a^\dagger a$ in ima re\v sitve $\psi_n(x)$ in $E_n$, kot jih poznamo
od prej. Poglejmo, kaj se zgodi, \v ce obrnemo vrstni red operatorjev -- definirajmo \v se $H_2 = aa^\dagger$,
ki ga dobimo kot

\begin{align}
	H_1 &= -\frac{1}{2}\px^2 + V_1(x) = a^\dagger a, \\
	H_2 &= -\frac{1}{2}\px^2 + V_2(x) = aa^\dagger,
\end{align}

\ni kjer

\begin{align}
	V_1(x) &= W^2(x) - \frac{1}{\sqrt{2}}\px W(x), \notag \\
	V_2(x) &= W^2(x) + \frac{1}{\sqrt{2}}\px W(x). \label{v2pot}
\end{align}

\ni Ena\v cbo~\eqref{v2pot} lahko doka\v zemo z istim postopkom kot~\eqref{dokaz.aad}, le da na funkcijo $\phi(x)$
delujemo z operatorjem $aa^\dagger$.

Pravimo, da je $V_2(x)$ supersimetri\v cni partner $V_1(x)$. Hamiltonian $H_1$ ima lastne pare $\psi^{(1)}_n(x)$,
$E^{(1)}_n$, $H_2$ pa $\psi^{(2)}_n(x)$, $E^{(2)}_n$. Lastne funkcije in energijski spekter $H_2$ lahko dobimo z
re\v sevanjem, ali pa ga uganemo:

\begin{equation}
	\psi^{(2)}_n(x) = a\psi^{(1)}_n(x),
\end{equation}

od koder vidimo da $\psi^{(2)}_0(x)$ ne obstaja, saj anihilacijski operator iz vakuuma naredi
ni\v clo po definiciji. Energijski spekter $H_2$ je enak tistemu iz $H_1$, s tem da nima
osnovnega stanja $E_0$, kar lahko poka\v zemo kot

\begin{align}
	H_2\psi^{(2)}_n(x) &= aa^\dagger (a\psi^{(1)}_n(x)) = a(a^\dagger a)\psi^{(1)}_n(x) = \notag \\
		&= aE^{(1)}_n\psi^{(1)}_n(x) = E_n^{(1)}(a\psi_n^{(1)}) = E_n^{(1)}\psi_n^{(2)},
	\label{degenener}
\end{align}

\ni se pravi

\begin{equation}
	E^{(1)}_n \equiv E^{(2)}_n, \qquad n = 1, 2, 3 \ldots
\end{equation}

\ni Zaradi tega, spremenimo definicijo $H_1$ tako, da ne bo imel ve\v c osnovnega stanja
\begin{equation}
	H_1 \to H_1^\prime = H_1 - E_0.
\end{equation}

Hamiltoniana $H_1$ in $H_2$ bi radi zdru\v zili v enega, tako da se prostora ne me\v sata. Zato
definiramo

\begin{equation}
	\H \equiv H_1 \oplus H_2 \equiv \begin{bmatrix}H_1 & \\
		& H_2 \end{bmatrix},
\end{equation}

\begin{equation}
	Q^\dagger = \sigma^+ a^\dagger = \begin{bmatrix} 0 & a^\dagger \\
		0 & 0 \end{bmatrix}, \quad
	Q = \sigma^- a = \begin{bmatrix} 0 & 0 \\
		a & 0 \end{bmatrix},
	\label{superop}
\end{equation}

\begin{equation}
	Q^\dagger Q + QQ^\dagger = \{Q, Q^\dagger\} = \H.
\end{equation}

Operatorji $Q$ in $Q^\dagger$ skupaj s generatorji Poincaréjeve grupe tvorijo ti. supersimetri\v cno algebro, ali
na kratko "`superalgebro"'.

\subsection{Me\v sanje bozonskih in fermionskih stanj}

Bozonski in fermionski operatorji tvorijo algebro, ki zado\v s\v ca kanoni\v cnim komutacijskim relacijam

\begin{align}
	[a_i, a_j^\dagger]_\pm &= \delta_{ij}, \\
	[a_i, a_j]_\pm &= 0,
\end{align}

\ni kjer je $[\bullet,\bullet]_+ \equiv [\bullet,\bullet]$ komutator in velja za bozone,
$[\bullet,\bullet]_- \equiv \{\bullet,\bullet\}$ pa je anti-komutator in velja za fermione.

V supersimetri\v cnih teorijah obstaja direktna linija, ki povezuje bozonske operatorje s fermionskimi.
To je posebej razvidno iz operatorjev $Q$ in $Q^\dagger$ iz en.~\eqref{superop}. Operatorja
$a$ in $a^\dagger$ tvorita boznonsko algebro

\begin{equation}
	[a, a^\dagger] = (\px W), \qquad [a, a] = 0.
\end{equation}

\ni V splo\v snem je $\px W = 1$ le za harmonski oscilator, za vi\v sje \v clene pa je problem bolj
kompliciran, zarade anharmonske sklopitve, vendar obstaja dokaz za splo\v sen polinomski potencial, v katerem
opis v Fockovem prostoru ni ve\v c tako enostaven. Definirajmo operatorje $c$ in $c^\dagger$, tako da

\begin{equation}
	c = \sigma^+, \qquad c^\dagger = \sigma^-,
\end{equation}

\ni Poka\v zemo lahko, da operatorji $c$ in $c^\dagger$ zadostijo enostavni fermionski algebri,

\begin{equation}
	\{c, c^\dagger\} = 1, \quad \{c, c\} = 0, \quad n_F = \frac{1 - [c,c^\dagger]}{2},
\end{equation}

\ni kjer je $n_F$ fermionski operator \v stetja. $Q$ in $Q^\dagger$ lahko potem zapi\v semo kot

\begin{equation}
	Q = c^\dagger a, \qquad Q^\dagger = a^\dagger c.
\end{equation}

\ni Sedaj vidimo, kako je pravzaprav treba interpretirati operatorje $Q$ in $Q^\dagger$. Ker namre\v c
velja

\begin{equation}
	[Q, H] = [Q^\dagger, H] = 0,
\end{equation}

\ni ti operatorji menjajo bozonska in fermionska stanja, ne da bi pri tem spremenili energijo sistema.

\subsection{Zlom supersimetrije}

Hamiltonian $\mathbf{H}$ ima degeneriran spekter, saj imata $H_1$ in $H_2$ iste lastne vrednosti. Kadar velja
$E^{(1)}_0 = 0$ pravimo, da je supersimetrija zlomljena, saj Hamiltoniana nista ve\v c degenerirana
in takih operatorjev $Q$ nimamo ve\v c. V teorijah polja to merimo s tako imenovanim Witten-ovim
indeksom,

\begin{equation}
	\Delta (\beta) = \Tr\big[\e^{-\beta H_1} - \e^{-\beta H_2}\big].
\end{equation}

\ni Za supersimetri\v cne teorije je 
\begin{equation}
	\lim_{\beta \to 0} \Delta(\beta) = 0,
\end{equation}

\ni za teorije z zlomljeno supersimetrijo pa 

\begin{equation}
	\lim_{\beta \to 0} \Delta (\beta) = 1,
\end{equation}

\ni saj osnovno stanje $H_1$ pre\v zivi.

