\section{Ponovitev}

Iz predavanj Vi\v sje kvantne mehanike se spomnimo, da lahko polja
kvantiziramo prek operatorjev polja, za kar pa nujno potrebujemo kreacijske, $a^\dagger$ in 
anihilacijske $a$ operatorje. Na kratko ponovimo nekaj pojmov, s katerimi se bomo sre\v cevali tekom tega
seminarja.

\subsection{Grupa}

\vspace{0.5 cm}

{\em Grupa} je mno\v zica elementov z naslednjimi lastnostmi:

\begin{enumerate}
	\item{Med elementi obstaja asociativna operacija.}
	\item{Mno\v zica je za to operacijo zaprta.}
	\item{Izmed teh elementov je natanko eden tak, ki je za to operacijo enota.}
	\item{V mno\v zici so zajeti tudi vsi inverzni elementi.}
\end{enumerate}

V kolikor ne veljajo vsi pogoji imamo enega izmed ni\v zjih objektov, kot je na primer monoid (nimamo
nujno vseh inverzov) ali polgrupa (nimamo nujno udentitete).

\v Ce je operacija komutativna, je ta grupa abelova in operacijo imenujemo se\v stevanje. Sicer
pa je grupa neabelova in operacijo imenujemo mno\v zenje.

\subsection{Algebra}

\vspace{0.5cm}

Pogostokrat sre\v camo pojem `algebra'. Pa ponovimo, kaj je to.\\

\emph{Kolobar} je mno\v zica elementov z naslednjimi lastnostmi:

\begin{enumerate}
	\item{Ta mno\v zica je abelova grupa z operacijo se\v stevanja.}
	\item{Elementi so hkrati (pol)grupa za operacijo mno\v zenja.}
	\item{Za kombinacijo operacij velja distributivnostna relacija.}
\end{enumerate}

\emph{Algebra} je vektorski prostor nad kolobarjem.

Liejeve grupe imajo tudi svojo algebro -- generatorji Liejevih grup nepenjajo vektorski
prostor na mnogoterosti grupe\footnote{Vsaka Liejeva grupa je hkrati mnogoterost}. Da poka\v zemo,
da zado\v s\v cajo pogojem iz algebre zado\v s\v ca zapis komutacijskih relacij med generatorji, zato
bomo na tem mestu definirali komutator, en.~\eqref{komutator} in anti-komutator, en.~\eqref{antikomutator}:

\begin{align}
	[A,B] &\equiv AB - BA, \label{komutator}\\
	\{A,B\} &\equiv AB + BA. \label{antikomutator}
\end{align}

Vpeljali bomo skupno notacijo, s katero bomo lahko "`me\v sali"' med obema pojmoma:

\begin{align}
	[A,B]_\pm = AB \pm BA.
\end{align}

Tudi kon\v cnim grupam lahko priredimo algebre -- za nas relevantna je simetri\v cna grupa
$S_n$, ki jo imenujemo tudi permutacijska grupa.

\v Ce to grupo raz\v sirimo z operacijo parnosti, ima ta nova grupa dve nerazcepni upodobitvi. Funkcije so lahko
bodisi simetri\v cne na permutacije, ali pa anti-simetri\v cne. Baza za prvo upodobitev so bozonski, za drugo
pa fermionski kreacijsko-anihilacijski operatorji.

Lastne vrednosti so $\pm 1$ (parnost), hkrati pa grupa komutira s Hamiltonianom, kar pomeni,
da je to dobra simetrija in da Hamiltonian lahko zapi\v semo tako, v bazi te upodobitve, oz. druga\v ce:
Hamiltonian lahko zapi\v semo s pomo\v cjo fermionskih ali bozonskih kreacijsko-anihilaijckih operatorjev.

Tako od tu dobimo fermionsko in bozonsko algebro:

\begin{align}
	[a_i, a_j^\dagger]_\pm &= \delta_{ij}, \\
	[a_i, a_j]_\pm &= 0,
\end{align}

kjer komutatorji ustrezajo bozonom, antikomutatorji pa fermionom.

\section{Uvod v supersimetrijo}

Pa si omo\v cimo noge v vodi supersimetrije: obravnavajmo Schr\" odingerjevo ena\v cbo, tj. Hamiltonian

\begin{equation}
	H = -\frac{\hbar^2}{2m}\nabla^2 + mV(\vec{x}).
	\label{schroedinger}
\end{equation}

Vpeljemo enote $\hbar = c_0 = \varepsilon_0 = 1$. Maso $m$ bomo zaenkrat postavili na $1$. Tako je brezdimenzijska
Schr\" odingerjeva ena\v cba

\begin{equation}
	H = -\frac{1}{2}\nabla^2 + V(\vec{x}),
	\label{s1}
\end{equation}

kjer velja \v se $H = i\pt$. V eni dimenziji se~\eqref{s1} glasi

\begin{equation}
	H = -\frac{1}{2}\px^2 + V(x).
	\label{sx}
\end{equation}

Predpostavimo, da je na\v s potencial neni\v celen in navzdol omejen.

Velja

\begin{equation}
	H\psi_n(x) = E_n\psi_n(x) = i\pt\psi_n(x).
\end{equation}

Za $E_0 = 0$ torej velja $H|\psi_0\rangle = 0$. Od tu dobimo pogoj za potencial

\begin{equation}
	V(x) = \frac{1}{2}\frac{\px^2\psi_0}{\psi_0},
	\label{vpogoj}
\end{equation}

kjer smo predpostavili, da so stanja $\psi_n(x)$ vezana stanja, tj. je $\psi_0(x)$ dobro dolo\v cen, in
to lahko storimo.

Na\v s Hamiltonian bi radi zapisali v obliko z operatorjem \v stetja, $a^\dagger$ in $a$, se pravi
$H = a^\dagger a$, zato moramo poiskati nek pameten razcep. Vidimo, da je~\eqref{sx} oblike
$H \sim (A + B)(A - B)$, tako bomo definirali superpotencial $W(x)$, da bo

\begin{align}
	a &= W(x) + \frac{1}{\sqrt{2}}\px, \\
	a^\dagger &= W(x) - \frac{1}{\sqrt{2}}\px.
\end{align}

Taka operatorja sta res drug drugemu hermitsko adjungirana, saj je operator $\px$ anti-hermitski (do totalnega
odvoda natan\v cno).

Vse lepo in prav, vendar, a tak $W (x)$ sploh obstaja? V fiziki na taka vpra\v sanja ponavadi odgovorimo
retrospektivno in to bomo storili tudi sedaj. Vse skupaj vstavimo v Hamiltonian in iz njega dolo\v cimo
vezi, katerim mora zado\v s\v cati.

Poglejmo, kaj naredi $a^\dagger a$ na neki funkciji $\phi (x)$:

\begin{align}
	a^\dagger a\ \phi(x)&= \Big[W(x) - \frac{1}{\sqrt{2}}\px\Big]\Big[W(x) +
		\frac{1}{\sqrt{2}}\px\Big]\phi(x) \notag \\
	&= \Big[-\frac{1}{2}\px^2 + W^2(x)\Big]\phi(x) - \frac{1}{\sqrt{2}}\Big[
		\underbrace{\px \big(W(x)\phi(x)\big) - W(x)\px\phi(x)}_{(\px W(x))\phi(x)} \Big] \notag \\
	&= \bigg\{-\frac{1}{2}\px^2 +
		\underbrace{W^2(x) - \frac{1}{\sqrt{2}}\big[\px W(x)\big]}_{V(x)}\bigg\}\phi(x),
	\label{dokaz.aad}
\end{align}

Se pravi, \v ce je ta ena\v cba Schr\" odingerjeva, potem mora $W (x)$ spo\v stovati slede\v ca izraza:

\begin{align}
	V(x) &=\frac{1}{2}\frac{\px^2\psi_0(x)}{\psi_0(x)} = W^2(x) - \frac{1}{\sqrt{2}}\px
		W(x) \label{riccati} \\
	W(x) &= -\frac{1}{\sqrt{2}}\frac{\px\psi_0(x)}{\psi_0(x)} \label{superpotencial},
\end{align}

kjer smo izraz~\eqref{superpotencial} dobili z re\v sevanjem Riccatijeve ena\v cbe~\eqref{riccati}.

Tako smo dobili $H_1 = a^\dagger a$ in ima re\v sitve $\psi_n(x)$ in $E_n$, kot jih poznamo
od prej.

Poglejmo, kaj se zgodi, \v ce vrstni red obrnemo. Definirajmo \v se $H_2 = aa^\dagger$. Dobimo ga kot

\begin{align}
	H_1 &= -\frac{1}{2}\px^2 + V_1(x) = a^\dagger a, \\
	H_2 &= -\frac{1}{2}\px^2 + V_2(x) = aa^\dagger,
\end{align}

kjer

\begin{align}
	V_1(x) &= W^2(x) - \frac{1}{\sqrt{2}}\px W(x), \\
	V_2(x) &= W^2(x) + \frac{1}{\sqrt{2}}\px W(x). \label{v2pot}
\end{align}

Ena\v cbo~\eqref{v2pot} lahko doka\v zemo z istim postopkom kot~\eqref{dokaz.aad}, le da na funkcijo $\phi(x)$
delujemo z operatorjem $aa^\dagger$.

$H_1$ ima lastne pare $\psi^{(1)}_n(x)$, $E^{(1)}_n$, $H_2$ pa $\psi^{(2)}_n(x)$, $E^{(2)}_n$.

Pravimo, da je $V_2(x)$ supersimetri\v cni partner $V_1(x)$. Lastne funkcije in energijski spekter $H_2$
dobimo lahko z re\v sevanjem, ali pa uganemo

\begin{equation}
	\psi^{(2)}_n(x) = a\psi^{(1)}_n(x),
\end{equation}

od koder vidimo da $\psi^{(2)}_0(x)$ ne obstaja, saj anihilacijski operator iz vakuuma naredi
ni\v clo po definiciji. Energijski spekter $H_2$ je enak tistemu iz $H_1$, s tem da nima
osnovnega stanja $E_0$, kar lahko poka\v zemo kot

\begin{align}
	H_2\psi^{(2)}_n(x) &= aa^\dagger (a\psi^{(1)}_n(x)) = a(a^\dagger a)\psi^{(1)}_n(x) = \notag \\
		&= aE^{(1)}_n\psi^{(1)}_n(x) = E_n^{(1)}(a\psi_n^{(1)}) = E_n^{(1)}\psi_n^{(2)},
	\label{degenener}
\end{align}

se pravi

\begin{equation}
	E^{(1)}_n \equiv E^{(2)}_n, \qquad n = 1, 2, 3 \ldots
\end{equation}

Zaradi tega, spremenimo definicijo $H_1$ tako, da ne bo imel ve\v c osnovnega stanja
\begin{equation}
	H_1 \to H_1^\prime = H_1 - E_0.
\end{equation}

Funkcije hamiltoniana $H_2$ bi morali v dobiti spet z re\v sevanjem 

Hamiltoniana $H_1$ in $H_2$ bi radi zdru\v zili v enega, tako da se prostora ne me\v sata. Zato
definiramo

\begin{equation}
	\H \equiv H_1 \oplus H_2 \equiv \begin{bmatrix}H_1 & \\
		& H_2 \end{bmatrix},
\end{equation}

\begin{equation}
	Q^\dagger = \sigma^+ a^\dagger = \begin{bmatrix} 0 & a^\dagger \\
		0 & 0 \end{bmatrix}, \quad
	Q = \sigma^- a = \begin{bmatrix} 0 & 0 \\
		a & 0 \end{bmatrix},
\end{equation}

\begin{equation}
	Q^\dagger Q + QQ^\dagger = \{Q, Q^\dagger\} = \H.
\end{equation}

\subsection{Me\v sanje bozonskih in fermionskih stanj}

Kot se verjetno kar najbolj povdarja, obstaja v supersimetri\v cnih teorijah nekak\v sna direktna linija,
ki povezuje bozonske operatorje s fermionskimi. Pa poglejmo, kaj to pravzaprav pomeni. Operatorja
$a$ in $a^\dagger$ tvorita boznonsko algebro

\begin{equation}
	[a, a^\dagger] = (\px W), \qquad [a, a] = 0.
\end{equation}

V splo\v snem je $\px W = 1$ le za harmonski oscilator, za vi\v sje \v clene pa je problem bolj
kompliciran, zarade anharmonske sklopitve. Opis v Fockovem prostoru ni ve\v c tako enostaven, vendar
mi verjemite na besedo, da so to \v se vedno bozonski operatorji.

Definirajmo operatorje $c$ in $c^\dagger$, tako da

\begin{equation}
	c = \sigma^+, \qquad c^\dagger = \sigma^-,
\end{equation}

Poka\v zemo lahko, da tile operatorji zadostijo enostavni fermionski algebri,

\begin{equation}
	\{c, c^\dagger\} = 1, \qquad \{c, c\} = 0,
\end{equation}

torej lahko $Q$ in $Q^\dagger$ zapi\v semo kot

\begin{equation}
	Q = c^\dagger a, \qquad Q^\dagger = a^\dagger c.
\end{equation}

Sedaj vidimo, kako je pravzaprav treba interpretirati operatorje $Q$ in $Q^\dagger$. Ker namre\v c
velja

\begin{equation}
	[Q, H] = [Q^\dagger, H] = 0,
\end{equation}

tile operatorji me\v sajo bozonska in fermionska stanje ne da bi pri tem spremenili energijo stanja.

\subsection{Zlom supersimetrije}

Tak Hamiltonian ima degeneriran spekter, saj imata $H_1$ in $H_2$ iste lastne vrednosti. Kadar velja
$E^{(1)}_0 = 0$ pravimo, da je supersimetrija zlomljena, saj Hamiltoniana nista ve\v c degenerirana
in takih operatorjev $Q$ nimamo ve\v c. V teorijah polja to merimo s tako imenovanim Witten-ovim
indeksom,

\begin{equation}
	\Delta (\beta) = \Tr\big[\e^{-\beta H_1} - \e^{-\beta H_2}\big].
\end{equation}

Za supersimetri\v cne teorije je 
\begin{equation}
	\lim_{\beta \to 0} \Delta(\beta) = 0,
\end{equation}
za teorije z zlomljeno supersimetrijo pa 

\begin{equation}
	\lim_{\beta \to 0} \Delta (\beta) = 1,
\end{equation}

saj osnovno stanje $H_1$ pre\v zivi.

